

\subsection{Create a horizontal Grid: {\tt gridCreate}}
\index{gridCreate}
\label{gridCreate}

The function {\tt gridCreate} creates a horizontal Grid.

\subsubsection*{Usage}

\begin{verbatim}
    int gridCreate(int gridtype, int size);
\end{verbatim}

\hspace*{4mm}\begin{minipage}[]{15cm}
\begin{deflist}{\tt gridtype\ }
\item[{\tt gridtype}]
The type of the grid, one of the set of predefined {\CDI} grid types.
                     The valid {\CDI} grid types are {\tt GRID\_GENERIC}, {\tt GRID\_GAUSSIAN},
                     {\tt GRID\_LONLAT}, {\tt GRID\_LCC}, {\tt GRID\_SPECTRAL},
                     {\tt GRID\_GME}, {\tt GRID\_CURVILINEAR} and {\tt GRID\_UNSTRUCTURED} and.
\item[{\tt size}]
Number of gridpoints.

\end{deflist}
\end{minipage}

\subsubsection*{Result}

{\tt gridCreate} returns an identifier to the Grid.


\subsubsection*{Example}

Here is an example using {\tt gridCreate} to create a regular lon/lat Grid:

\begin{lstlisting}[language=C, backgroundcolor=\color{pyellow}, basicstyle=\small, columns=flexible]

    #include "cdi.h"
       ...
    #define  nlon  12
    #define  nlat   6
       ...
    double lons[nlon] = {0, 30, 60, 90, 120, 150, 180, 210, 240, 270, 300, 330};
    double lats[nlat] = {-75, -45, -15, 15, 45, 75};
    int gridID;
       ...
    gridID = gridCreate(GRID_LONLAT, nlon*nlat);
    gridDefXsize(gridID, nlon);
    gridDefYsize(gridID, nlat);
    gridDefXvals(gridID, lons);
    gridDefYvals(gridID, lats);
       ...
\end{lstlisting}


\subsection{Destroy a horizontal Grid: {\tt gridDestroy}}
\index{gridDestroy}
\label{gridDestroy}
\subsubsection*{Usage}

\begin{verbatim}
    void gridDestroy(int gridID);
\end{verbatim}

\hspace*{4mm}\begin{minipage}[]{15cm}
\begin{deflist}{\tt gridID\ }
\item[{\tt gridID}]
Grid ID, from a previous call to {\htmlref{\tt gridCreate}{gridCreate}}.

\end{deflist}
\end{minipage}


\subsection{Duplicate a horizontal Grid: {\tt gridDuplicate}}
\index{gridDuplicate}
\label{gridDuplicate}

The function {\tt gridDuplicate} duplicates a horizontal Grid.

\subsubsection*{Usage}

\begin{verbatim}
    int gridDuplicate(int gridID);
\end{verbatim}

\hspace*{4mm}\begin{minipage}[]{15cm}
\begin{deflist}{\tt gridID\ }
\item[{\tt gridID}]
Grid ID, from a previous call to {\htmlref{\tt gridCreate}{gridCreate}} or {\htmlref{\tt vlistInqVarGrid}{vlistInqVarGrid}}.

\end{deflist}
\end{minipage}

\subsubsection*{Result}

{\tt gridDuplicate} returns an identifier to the duplicated Grid.



\subsection{Get the type of a Grid: {\tt gridInqType}}
\index{gridInqType}
\label{gridInqType}

The function {\tt gridInqType} returns the type of a Grid.

\subsubsection*{Usage}

\begin{verbatim}
    int gridInqType(int gridID);
\end{verbatim}

\hspace*{4mm}\begin{minipage}[]{15cm}
\begin{deflist}{\tt gridID\ }
\item[{\tt gridID}]
Grid ID, from a previous call to {\htmlref{\tt gridCreate}{gridCreate}} or {\htmlref{\tt vlistInqVarGrid}{vlistInqVarGrid}}.

\end{deflist}
\end{minipage}

\subsubsection*{Result}

{\tt gridInqType} returns the type of the grid,
one of the set of predefined {\CDI} grid types.
The valid {\CDI} grid types are {\tt GRID\_GENERIC}, {\tt GRID\_GAUSSIAN},
{\tt GRID\_LONLAT}, {\tt GRID\_LCC}, {\tt GRID\_SPECTRAL}, {\tt GRID\_GME},
{\tt GRID\_CURVILINEAR} and {\tt GRID\_UNSTRUCTURED}.



\subsection{Get the size of a Grid: {\tt gridInqSize}}
\index{gridInqSize}
\label{gridInqSize}

The function {\tt gridInqSize} returns the size of a Grid.

\subsubsection*{Usage}

\begin{verbatim}
    int gridInqSize(int gridID);
\end{verbatim}

\hspace*{4mm}\begin{minipage}[]{15cm}
\begin{deflist}{\tt gridID\ }
\item[{\tt gridID}]
Grid ID, from a previous call to {\htmlref{\tt gridCreate}{gridCreate}} or {\htmlref{\tt vlistInqVarGrid}{vlistInqVarGrid}}.

\end{deflist}
\end{minipage}

\subsubsection*{Result}

{\tt gridInqSize} returns the number of grid points of a Grid.



\subsection{Define the number of values of a X-axis: {\tt gridDefXsize}}
\index{gridDefXsize}
\label{gridDefXsize}

The function {\tt gridDefXsize} defines the number of values of a X-axis.

\subsubsection*{Usage}

\begin{verbatim}
    void gridDefXsize(int gridID, int xsize);
\end{verbatim}

\hspace*{4mm}\begin{minipage}[]{15cm}
\begin{deflist}{\tt gridID\ }
\item[{\tt gridID}]
Grid ID, from a previous call to {\htmlref{\tt gridCreate}{gridCreate}}.
\item[{\tt xsize}]
Number of values of a X-axis.

\end{deflist}
\end{minipage}


\subsection{Get the number of values of a X-axis: {\tt gridInqXsize}}
\index{gridInqXsize}
\label{gridInqXsize}

The function {\tt gridInqXsize} returns the number of values of a X-axis.

\subsubsection*{Usage}

\begin{verbatim}
    int gridInqXsize(int gridID);
\end{verbatim}

\hspace*{4mm}\begin{minipage}[]{15cm}
\begin{deflist}{\tt gridID\ }
\item[{\tt gridID}]
Grid ID, from a previous call to {\htmlref{\tt gridCreate}{gridCreate}} or {\htmlref{\tt vlistInqVarGrid}{vlistInqVarGrid}}.

\end{deflist}
\end{minipage}

\subsubsection*{Result}

{\tt gridInqXsize} returns the number of values of a X-axis.



\subsection{Define the number of values of a Y-axis: {\tt gridDefYsize}}
\index{gridDefYsize}
\label{gridDefYsize}

The function {\tt gridDefYsize} defines the number of values of a Y-axis.

\subsubsection*{Usage}

\begin{verbatim}
    void gridDefYsize(int gridID, int ysize);
\end{verbatim}

\hspace*{4mm}\begin{minipage}[]{15cm}
\begin{deflist}{\tt gridID\ }
\item[{\tt gridID}]
Grid ID, from a previous call to {\htmlref{\tt gridCreate}{gridCreate}}.
\item[{\tt ysize}]
Number of values of a Y-axis.

\end{deflist}
\end{minipage}


\subsection{Get the number of values of a Y-axis: {\tt gridInqYsize}}
\index{gridInqYsize}
\label{gridInqYsize}

The function {\tt gridInqYsize} returns the number of values of a Y-axis.

\subsubsection*{Usage}

\begin{verbatim}
    int gridInqYsize(int gridID);
\end{verbatim}

\hspace*{4mm}\begin{minipage}[]{15cm}
\begin{deflist}{\tt gridID\ }
\item[{\tt gridID}]
Grid ID, from a previous call to {\htmlref{\tt gridCreate}{gridCreate}} or {\htmlref{\tt vlistInqVarGrid}{vlistInqVarGrid}}.

\end{deflist}
\end{minipage}

\subsubsection*{Result}

{\tt gridInqYsize} returns the number of values of a Y-axis.



\subsection{Define the number of parallels between a pole and the equator: {\tt gridDefNP}}
\index{gridDefNP}
\label{gridDefNP}

The function {\tt gridDefNP} defines the number of parallels between a pole and the equator
of a Gaussian grid.

\subsubsection*{Usage}

\begin{verbatim}
    void gridDefNP(int gridID, int np);
\end{verbatim}

\hspace*{4mm}\begin{minipage}[]{15cm}
\begin{deflist}{\tt gridID\ }
\item[{\tt gridID}]
Grid ID, from a previous call to {\htmlref{\tt gridCreate}{gridCreate}}.
\item[{\tt np}]
Number of parallels between a pole and the equator.

\end{deflist}
\end{minipage}


\subsection{Get the number of parallels between a pole and the equator: {\tt gridInqNP}}
\index{gridInqNP}
\label{gridInqNP}

The function {\tt gridInqNP} returns the number of parallels between a pole and the equator
of a Gaussian grid.

\subsubsection*{Usage}

\begin{verbatim}
    int gridInqNP(int gridID);
\end{verbatim}

\hspace*{4mm}\begin{minipage}[]{15cm}
\begin{deflist}{\tt gridID\ }
\item[{\tt gridID}]
Grid ID, from a previous call to {\htmlref{\tt gridCreate}{gridCreate}} or {\htmlref{\tt vlistInqVarGrid}{vlistInqVarGrid}}.

\end{deflist}
\end{minipage}

\subsubsection*{Result}

{\tt gridInqNP} returns the number of parallels between a pole and the equator.



\subsection{Define the values of a X-axis: {\tt gridDefXvals}}
\index{gridDefXvals}
\label{gridDefXvals}

The function {\tt gridDefXvals} defines all values of the X-axis.

\subsubsection*{Usage}

\begin{verbatim}
    void gridDefXvals(int gridID, const double *xvals);
\end{verbatim}

\hspace*{4mm}\begin{minipage}[]{15cm}
\begin{deflist}{\tt gridID\ }
\item[{\tt gridID}]
Grid ID, from a previous call to {\htmlref{\tt gridCreate}{gridCreate}}.
\item[{\tt xvals}]
X-values of the grid.

\end{deflist}
\end{minipage}


\subsection{Get all values of a X-axis: {\tt gridInqXvals}}
\index{gridInqXvals}
\label{gridInqXvals}

The function {\tt gridInqXvals} returns all values of the X-axis.

\subsubsection*{Usage}

\begin{verbatim}
    int gridInqXvals(int gridID, double *xvals);
\end{verbatim}

\hspace*{4mm}\begin{minipage}[]{15cm}
\begin{deflist}{\tt gridID\ }
\item[{\tt gridID}]
Grid ID, from a previous call to {\htmlref{\tt gridCreate}{gridCreate}} or {\htmlref{\tt vlistInqVarGrid}{vlistInqVarGrid}}.
\item[{\tt xvals}]
Pointer to the location into which the X-values are read.
                    The caller must allocate space for the returned values.

\end{deflist}
\end{minipage}

\subsubsection*{Result}

Upon successful completion {\tt gridInqXvals} returns the number of values and
the values are stored in {\tt xvals}.
Otherwise, 0 is returned and {\tt xvals} is empty.



\subsection{Define the values of a Y-axis: {\tt gridDefYvals}}
\index{gridDefYvals}
\label{gridDefYvals}

The function {\tt gridDefYvals} defines all values of the Y-axis.

\subsubsection*{Usage}

\begin{verbatim}
    void gridDefYvals(int gridID, const double *yvals);
\end{verbatim}

\hspace*{4mm}\begin{minipage}[]{15cm}
\begin{deflist}{\tt gridID\ }
\item[{\tt gridID}]
Grid ID, from a previous call to {\htmlref{\tt gridCreate}{gridCreate}}.
\item[{\tt yvals}]
Y-values of the grid.

\end{deflist}
\end{minipage}


\subsection{Get all values of a Y-axis: {\tt gridInqYvals}}
\index{gridInqYvals}
\label{gridInqYvals}

The function {\tt gridInqYvals} returns all values of the Y-axis.

\subsubsection*{Usage}

\begin{verbatim}
    int gridInqYvals(int gridID, double *yvals);
\end{verbatim}

\hspace*{4mm}\begin{minipage}[]{15cm}
\begin{deflist}{\tt gridID\ }
\item[{\tt gridID}]
Grid ID, from a previous call to {\htmlref{\tt gridCreate}{gridCreate}} or {\htmlref{\tt vlistInqVarGrid}{vlistInqVarGrid}}.
\item[{\tt yvals}]
Pointer to the location into which the Y-values are read.
                    The caller must allocate space for the returned values.

\end{deflist}
\end{minipage}

\subsubsection*{Result}

Upon successful completion {\tt gridInqYvals} returns the number of values and
the values are stored in {\tt yvals}.
Otherwise, 0 is returned and {\tt yvals} is empty.



\subsection{Define the bounds of a X-axis: {\tt gridDefXbounds}}
\index{gridDefXbounds}
\label{gridDefXbounds}

The function {\tt gridDefXbounds} defines all bounds of the X-axis.

\subsubsection*{Usage}

\begin{verbatim}
    void gridDefXbounds(int gridID, const double *xbounds);
\end{verbatim}

\hspace*{4mm}\begin{minipage}[]{15cm}
\begin{deflist}{\tt xbounds\ }
\item[{\tt gridID}]
Grid ID, from a previous call to {\htmlref{\tt gridCreate}{gridCreate}}.
\item[{\tt xbounds}]
X-bounds of the grid.

\end{deflist}
\end{minipage}


\subsection{Get the bounds of a X-axis: {\tt gridInqXbounds}}
\index{gridInqXbounds}
\label{gridInqXbounds}

The function {\tt gridInqXbounds} returns the bounds of the X-axis.

\subsubsection*{Usage}

\begin{verbatim}
    int gridInqXbounds(int gridID, double *xbounds);
\end{verbatim}

\hspace*{4mm}\begin{minipage}[]{15cm}
\begin{deflist}{\tt xbounds\ }
\item[{\tt gridID}]
Grid ID, from a previous call to {\htmlref{\tt gridCreate}{gridCreate}} or {\htmlref{\tt vlistInqVarGrid}{vlistInqVarGrid}}.
\item[{\tt xbounds}]
Pointer to the location into which the X-bounds are read.
                    The caller must allocate space for the returned values.

\end{deflist}
\end{minipage}

\subsubsection*{Result}

Upon successful completion {\tt gridInqXbounds} returns the number of bounds and
the bounds are stored in {\tt xbounds}.
Otherwise, 0 is returned and {\tt xbounds} is empty.



\subsection{Define the bounds of a Y-axis: {\tt gridDefYbounds}}
\index{gridDefYbounds}
\label{gridDefYbounds}

The function {\tt gridDefYbounds} defines all bounds of the Y-axis.

\subsubsection*{Usage}

\begin{verbatim}
    void gridDefYbounds(int gridID, const double *ybounds);
\end{verbatim}

\hspace*{4mm}\begin{minipage}[]{15cm}
\begin{deflist}{\tt ybounds\ }
\item[{\tt gridID}]
Grid ID, from a previous call to {\htmlref{\tt gridCreate}{gridCreate}}.
\item[{\tt ybounds}]
Y-bounds of the grid.

\end{deflist}
\end{minipage}


\subsection{Get the bounds of a Y-axis: {\tt gridInqYbounds}}
\index{gridInqYbounds}
\label{gridInqYbounds}

The function {\tt gridInqYbounds} returns the bounds of the Y-axis.

\subsubsection*{Usage}

\begin{verbatim}
    int gridInqYbounds(int gridID, double *ybounds);
\end{verbatim}

\hspace*{4mm}\begin{minipage}[]{15cm}
\begin{deflist}{\tt ybounds\ }
\item[{\tt gridID}]
Grid ID, from a previous call to {\htmlref{\tt gridCreate}{gridCreate}} or {\htmlref{\tt vlistInqVarGrid}{vlistInqVarGrid}}.
\item[{\tt ybounds}]
Pointer to the location into which the Y-bounds are read.
                    The caller must allocate space for the returned values.

\end{deflist}
\end{minipage}

\subsubsection*{Result}

Upon successful completion {\tt gridInqYbounds} returns the number of bounds and
the bounds are stored in {\tt ybounds}.
Otherwise, 0 is returned and {\tt ybounds} is empty.



\subsection{Define the name of a X-axis: {\tt gridDefXname}}
\index{gridDefXname}
\label{gridDefXname}

The function {\tt gridDefXname} defines the name of a X-axis.

\subsubsection*{Usage}

\begin{verbatim}
    void gridDefXname(int gridID, const char *name);
\end{verbatim}

\hspace*{4mm}\begin{minipage}[]{15cm}
\begin{deflist}{\tt gridID\ }
\item[{\tt gridID}]
Grid ID, from a previous call to {\htmlref{\tt gridCreate}{gridCreate}}.
\item[{\tt name}]
Name of the X-axis.

\end{deflist}
\end{minipage}


\subsection{Get the name of a X-axis: {\tt gridInqXname}}
\index{gridInqXname}
\label{gridInqXname}

The function {\tt gridInqXname} returns the name of a X-axis.

\subsubsection*{Usage}

\begin{verbatim}
    void gridInqXname(int gridID, char *name);
\end{verbatim}

\hspace*{4mm}\begin{minipage}[]{15cm}
\begin{deflist}{\tt gridID\ }
\item[{\tt gridID}]
Grid ID, from a previous call to {\htmlref{\tt gridCreate}{gridCreate}} or {\htmlref{\tt vlistInqVarGrid}{vlistInqVarGrid}}.
\item[{\tt name}]
Name of the X-axis. The caller must allocate space for the
                    returned string. The maximum possible length, in characters, of
                    the string is given by the predefined constant {\tt CDI\_MAX\_NAME}.

\end{deflist}
\end{minipage}

\subsubsection*{Result}

{\tt gridInqXname} returns the name of the X-axis to the parameter name.



\subsection{Define the longname of a X-axis: {\tt gridDefXlongname}}
\index{gridDefXlongname}
\label{gridDefXlongname}

The function {\tt gridDefXlongname} defines the longname of a X-axis.

\subsubsection*{Usage}

\begin{verbatim}
    void gridDefXlongname(int gridID, const char *longname);
\end{verbatim}

\hspace*{4mm}\begin{minipage}[]{15cm}
\begin{deflist}{\tt longname\ }
\item[{\tt gridID}]
Grid ID, from a previous call to {\htmlref{\tt gridCreate}{gridCreate}}.
\item[{\tt longname}]
Longname of the X-axis.

\end{deflist}
\end{minipage}


\subsection{Get the longname of a X-axis: {\tt gridInqXlongname}}
\index{gridInqXlongname}
\label{gridInqXlongname}

The function {\tt gridInqXlongname} returns the longname of a X-axis.

\subsubsection*{Usage}

\begin{verbatim}
    void gridInqXlongname(int gridID, char *longname);
\end{verbatim}

\hspace*{4mm}\begin{minipage}[]{15cm}
\begin{deflist}{\tt longname\ }
\item[{\tt gridID}]
Grid ID, from a previous call to {\htmlref{\tt gridCreate}{gridCreate}} or {\htmlref{\tt vlistInqVarGrid}{vlistInqVarGrid}}.
\item[{\tt longname}]
Longname of the X-axis. The caller must allocate space for the
                    returned string. The maximum possible length, in characters, of
                    the string is given by the predefined constant {\tt CDI\_MAX\_NAME}.

\end{deflist}
\end{minipage}

\subsubsection*{Result}

{\tt gridInqXlongname} returns the longname of the X-axis to the parameter longname.



\subsection{Define the units of a X-axis: {\tt gridDefXunits}}
\index{gridDefXunits}
\label{gridDefXunits}

The function {\tt gridDefXunits} defines the units of a X-axis.

\subsubsection*{Usage}

\begin{verbatim}
    void gridDefXunits(int gridID, const char *units);
\end{verbatim}

\hspace*{4mm}\begin{minipage}[]{15cm}
\begin{deflist}{\tt gridID\ }
\item[{\tt gridID}]
Grid ID, from a previous call to {\htmlref{\tt gridCreate}{gridCreate}}.
\item[{\tt units}]
Units of the X-axis.

\end{deflist}
\end{minipage}


\subsection{Get the units of a X-axis: {\tt gridInqXunits}}
\index{gridInqXunits}
\label{gridInqXunits}

The function {\tt gridInqXunits} returns the units of a X-axis.

\subsubsection*{Usage}

\begin{verbatim}
    void gridInqXunits(int gridID, char *units);
\end{verbatim}

\hspace*{4mm}\begin{minipage}[]{15cm}
\begin{deflist}{\tt gridID\ }
\item[{\tt gridID}]
Grid ID, from a previous call to {\htmlref{\tt gridCreate}{gridCreate}} or {\htmlref{\tt vlistInqVarGrid}{vlistInqVarGrid}}.
\item[{\tt units}]
Units of the X-axis. The caller must allocate space for the
                    returned string. The maximum possible length, in characters, of
                    the string is given by the predefined constant {\tt CDI\_MAX\_NAME}.

\end{deflist}
\end{minipage}

\subsubsection*{Result}

{\tt gridInqXunits} returns the units of the X-axis to the parameter units.



\subsection{Define the name of a Y-axis: {\tt gridDefYname}}
\index{gridDefYname}
\label{gridDefYname}

The function {\tt gridDefYname} defines the name of a Y-axis.

\subsubsection*{Usage}

\begin{verbatim}
    void gridDefYname(int gridID, const char *name);
\end{verbatim}

\hspace*{4mm}\begin{minipage}[]{15cm}
\begin{deflist}{\tt gridID\ }
\item[{\tt gridID}]
Grid ID, from a previous call to {\htmlref{\tt gridCreate}{gridCreate}}.
\item[{\tt name}]
Name of the Y-axis.

\end{deflist}
\end{minipage}


\subsection{Get the name of a Y-axis: {\tt gridInqYname}}
\index{gridInqYname}
\label{gridInqYname}

The function {\tt gridInqYname} returns the name of a Y-axis.

\subsubsection*{Usage}

\begin{verbatim}
    void gridInqYname(int gridID, char *name);
\end{verbatim}

\hspace*{4mm}\begin{minipage}[]{15cm}
\begin{deflist}{\tt gridID\ }
\item[{\tt gridID}]
Grid ID, from a previous call to {\htmlref{\tt gridCreate}{gridCreate}} or {\htmlref{\tt vlistInqVarGrid}{vlistInqVarGrid}}.
\item[{\tt name}]
Name of the Y-axis. The caller must allocate space for the
                    returned string. The maximum possible length, in characters, of
                    the string is given by the predefined constant {\tt CDI\_MAX\_NAME}.

\end{deflist}
\end{minipage}

\subsubsection*{Result}

{\tt gridInqYname} returns the name of the Y-axis to the parameter name.



\subsection{Define the longname of a Y-axis: {\tt gridDefYlongname}}
\index{gridDefYlongname}
\label{gridDefYlongname}

The function {\tt gridDefYlongname} defines the longname of a Y-axis.

\subsubsection*{Usage}

\begin{verbatim}
    void gridDefYlongname(int gridID, const char *longname);
\end{verbatim}

\hspace*{4mm}\begin{minipage}[]{15cm}
\begin{deflist}{\tt longname\ }
\item[{\tt gridID}]
Grid ID, from a previous call to {\htmlref{\tt gridCreate}{gridCreate}}.
\item[{\tt longname}]
Longname of the Y-axis.

\end{deflist}
\end{minipage}


\subsection{Get the longname of a Y-axis: {\tt gridInqYlongname}}
\index{gridInqYlongname}
\label{gridInqYlongname}

The function {\tt gridInqYlongname} returns the longname of a Y-axis.

\subsubsection*{Usage}

\begin{verbatim}
    void gridInqXlongname(int gridID, char *longname);
\end{verbatim}

\hspace*{4mm}\begin{minipage}[]{15cm}
\begin{deflist}{\tt longname\ }
\item[{\tt gridID}]
Grid ID, from a previous call to {\htmlref{\tt gridCreate}{gridCreate}} or {\htmlref{\tt vlistInqVarGrid}{vlistInqVarGrid}}.
\item[{\tt longname}]
Longname of the Y-axis. The caller must allocate space for the
                    returned string. The maximum possible length, in characters, of
                    the string is given by the predefined constant {\tt CDI\_MAX\_NAME}.

\end{deflist}
\end{minipage}

\subsubsection*{Result}

{\tt gridInqYlongname} returns the longname of the Y-axis to the parameter longname.



\subsection{Define the units of a Y-axis: {\tt gridDefYunits}}
\index{gridDefYunits}
\label{gridDefYunits}

The function {\tt gridDefYunits} defines the units of a Y-axis.

\subsubsection*{Usage}

\begin{verbatim}
    void gridDefYunits(int gridID, const char *units);
\end{verbatim}

\hspace*{4mm}\begin{minipage}[]{15cm}
\begin{deflist}{\tt gridID\ }
\item[{\tt gridID}]
Grid ID, from a previous call to {\htmlref{\tt gridCreate}{gridCreate}}.
\item[{\tt units}]
Units of the Y-axis.

\end{deflist}
\end{minipage}


\subsection{Get the units of a Y-axis: {\tt gridInqYunits}}
\index{gridInqYunits}
\label{gridInqYunits}

The function {\tt gridInqYunits} returns the units of a Y-axis.

\subsubsection*{Usage}

\begin{verbatim}
    void gridInqYunits(int gridID, char *units);
\end{verbatim}

\hspace*{4mm}\begin{minipage}[]{15cm}
\begin{deflist}{\tt gridID\ }
\item[{\tt gridID}]
Grid ID, from a previous call to {\htmlref{\tt gridCreate}{gridCreate}} or {\htmlref{\tt vlistInqVarGrid}{vlistInqVarGrid}}.
\item[{\tt units}]
Units of the Y-axis. The caller must allocate space for the
                    returned string. The maximum possible length, in characters, of
                    the string is given by the predefined constant {\tt CDI\_MAX\_NAME}.

\end{deflist}
\end{minipage}

\subsubsection*{Result}

{\tt gridInqYunits} returns the units of the Y-axis to the parameter units.



\subsection{Define the reference number for an unstructured grid: {\tt gridDefNumber}}
\index{gridDefNumber}
\label{gridDefNumber}

The function {\tt gridDefNumber} defines the reference number for an unstructured grid.

\subsubsection*{Usage}

\begin{verbatim}
    void gridDefNumber(int gridID, const int number);
\end{verbatim}

\hspace*{4mm}\begin{minipage}[]{15cm}
\begin{deflist}{\tt gridID\ }
\item[{\tt gridID}]
Grid ID, from a previous call to {\htmlref{\tt gridCreate}{gridCreate}}.
\item[{\tt number}]
Reference number for an unstructured grid.

\end{deflist}
\end{minipage}


\subsection{Get the reference number to an unstructured grid: {\tt gridInqNumber}}
\index{gridInqNumber}
\label{gridInqNumber}

The function {\tt gridInqNumber} returns the reference number to an unstructured grid.

\subsubsection*{Usage}

\begin{verbatim}
    int gridInqNumber(int gridID);
\end{verbatim}

\hspace*{4mm}\begin{minipage}[]{15cm}
\begin{deflist}{\tt gridID\ }
\item[{\tt gridID}]
Grid ID, from a previous call to {\htmlref{\tt gridCreate}{gridCreate}} or {\htmlref{\tt vlistInqVarGrid}{vlistInqVarGrid}}.

\end{deflist}
\end{minipage}

\subsubsection*{Result}

{\tt gridInqNumber} returns the reference number to an unstructured grid.


\subsection{Define the position of grid in the reference file: {\tt gridDefPosition}}
\index{gridDefPosition}
\label{gridDefPosition}

The function {\tt gridDefPosition} defines the position of grid in the reference file.

\subsubsection*{Usage}

\begin{verbatim}
    void gridDefPosition(int gridID, const int position);
\end{verbatim}

\hspace*{4mm}\begin{minipage}[]{15cm}
\begin{deflist}{\tt position\ }
\item[{\tt gridID}]
Grid ID, from a previous call to {\htmlref{\tt gridCreate}{gridCreate}}.
\item[{\tt position}]
Position of grid in the reference file.

\end{deflist}
\end{minipage}


\subsection{Get the position of grid in the reference file: {\tt gridInqPosition}}
\index{gridInqPosition}
\label{gridInqPosition}

The function {\tt gridInqPosition} returns the position of grid in the reference file.

\subsubsection*{Usage}

\begin{verbatim}
    int gridInqPosition(int gridID);
\end{verbatim}

\hspace*{4mm}\begin{minipage}[]{15cm}
\begin{deflist}{\tt gridID\ }
\item[{\tt gridID}]
Grid ID, from a previous call to {\htmlref{\tt gridCreate}{gridCreate}} or {\htmlref{\tt vlistInqVarGrid}{vlistInqVarGrid}}.

\end{deflist}
\end{minipage}

\subsubsection*{Result}

{\tt gridInqPosition} returns the position of grid in the reference file.


\subsection{Define the reference URI for an unstructured grid: {\tt gridDefReference}}
\index{gridDefReference}
\label{gridDefReference}

The function {\tt gridDefReference} defines the reference URI for an unstructured grid.

\subsubsection*{Usage}

\begin{verbatim}
    void gridDefReference(int gridID, const char *reference);
\end{verbatim}

\hspace*{4mm}\begin{minipage}[]{15cm}
\begin{deflist}{\tt reference\ }
\item[{\tt gridID}]
Grid ID, from a previous call to {\htmlref{\tt gridCreate}{gridCreate}}.
\item[{\tt reference}]
Reference URI for an unstructured grid.

\end{deflist}
\end{minipage}


\subsection{Get the reference URI to an unstructured grid: {\tt gridInqReference}}
\index{gridInqReference}
\label{gridInqReference}

The function {\tt gridInqReference} returns the reference URI to an unstructured grid.

\subsubsection*{Usage}

\begin{verbatim}
    char *gridInqReference(int gridID, char *reference);
\end{verbatim}

\hspace*{4mm}\begin{minipage}[]{15cm}
\begin{deflist}{\tt gridID\ }
\item[{\tt gridID}]
Grid ID, from a previous call to {\htmlref{\tt gridCreate}{gridCreate}} or {\htmlref{\tt vlistInqVarGrid}{vlistInqVarGrid}}.

\end{deflist}
\end{minipage}

\subsubsection*{Result}

{\tt gridInqReference} returns the reference URI to an unstructured grid.


\subsection{Define the UUID for an unstructured grid: {\tt gridDefUUID}}
\index{gridDefUUID}
\label{gridDefUUID}

The function {\tt gridDefUUID} defines the UUID for an unstructured grid.

\subsubsection*{Usage}

\begin{verbatim}
    void gridDefUUID(int gridID, const char *uuid);
\end{verbatim}

\hspace*{4mm}\begin{minipage}[]{15cm}
\begin{deflist}{\tt gridID\ }
\item[{\tt gridID}]
Grid ID, from a previous call to {\htmlref{\tt gridCreate}{gridCreate}}.
\item[{\tt uuid}]
UUID for an unstructured grid.

\end{deflist}
\end{minipage}


\subsection{Get the UUID to an unstructured grid: {\tt gridInqUUID}}
\index{gridInqUUID}
\label{gridInqUUID}

The function {\tt gridInqUUID} returns the UUID to an unstructured grid.

\subsubsection*{Usage}

\begin{verbatim}
    void gridInqUUID(int gridID, char *uuid);
\end{verbatim}

\hspace*{4mm}\begin{minipage}[]{15cm}
\begin{deflist}{\tt gridID\ }
\item[{\tt gridID}]
Grid ID, from a previous call to {\htmlref{\tt gridCreate}{gridCreate}} or {\htmlref{\tt vlistInqVarGrid}{vlistInqVarGrid}}.

\end{deflist}
\end{minipage}

\subsubsection*{Result}

{\tt gridInqUUID} returns the UUID to an unstructured grid to the parameter uuid.
