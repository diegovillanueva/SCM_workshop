

\subsection{Define a Variable: {\tt vlistDefVar}}
\index{vlistDefVar}
\label{vlistDefVar}

The function {\tt vlistDefVar} adds a new variable to vlistID.

\subsubsection*{Usage}

\begin{verbatim}
    INTEGER FUNCTION vlistDefVar(INTEGER vlistID, INTEGER gridID, INTEGER zaxisID, 
                                 INTEGER tsteptype)
\end{verbatim}

\hspace*{4mm}\begin{minipage}[]{15cm}
\begin{deflist}{\tt tsteptype\ }
\item[{\tt vlistID}]
Variable list ID, from a previous call to {\htmlref{\tt vlistCreate}{vlistCreate}}.
\item[{\tt gridID}]
Grid ID, from a previous call to {\htmlref{\tt gridCreate}{gridCreate}}.
\item[{\tt zaxisID}]
Z-axis ID, from a previous call to {\htmlref{\tt zaxisCreate}{zaxisCreate}}.
\item[{\tt tsteptype}]
One of the set of predefined {\CDI} timestep types.
                     The valid {\CDI} timestep types are {\tt TSTEP\_CONSTANT} and {\tt TSTEP\_INSTANT}.

\end{deflist}
\end{minipage}

\subsubsection*{Result}

{\tt vlistDefVar} returns an identifier to the new variable.


\subsubsection*{Example}

Here is an example using {\tt vlistCreate} to create a variable list
and add a variable with {\tt vlistDefVar}.

\begin{lstlisting}[language=Fortran, backgroundcolor=\color{pyellow}, basicstyle=\small, columns=flexible]

    INCLUDE 'cdi.h'
       ...
    INTEGER vlistID, varID
       ...
    vlistID = vlistCreate()
    varID = vlistDefVar(vlistID, gridID, zaxisID, TIME_INSTANT)
       ...
    streamDefVlist(streamID, vlistID)
       ...
    vlistDestroy(vlistID)
       ...
\end{lstlisting}


\subsection{Get the Grid ID of a Variable: {\tt vlistInqVarGrid}}
\index{vlistInqVarGrid}
\label{vlistInqVarGrid}

The function {\tt vlistInqVarGrid} returns the grid ID of a variable.

\subsubsection*{Usage}

\begin{verbatim}
    INTEGER FUNCTION vlistInqVarGrid(INTEGER vlistID, INTEGER varID)
\end{verbatim}

\hspace*{4mm}\begin{minipage}[]{15cm}
\begin{deflist}{\tt vlistID\ }
\item[{\tt vlistID}]
Variable list ID, from a previous call to {\htmlref{\tt vlistCreate}{vlistCreate}} or {\htmlref{\tt streamInqVlist}{streamInqVlist}}.
\item[{\tt varID}]
Variable identifier.

\end{deflist}
\end{minipage}

\subsubsection*{Result}

{\tt vlistInqVarGrid} returns the grid ID of the variable.



\subsection{Get the Zaxis ID of a Variable: {\tt vlistInqVarZaxis}}
\index{vlistInqVarZaxis}
\label{vlistInqVarZaxis}

The function {\tt vlistInqVarZaxis} returns the zaxis ID of a variable.

\subsubsection*{Usage}

\begin{verbatim}
    INTEGER FUNCTION vlistInqVarZaxis(INTEGER vlistID, INTEGER varID)
\end{verbatim}

\hspace*{4mm}\begin{minipage}[]{15cm}
\begin{deflist}{\tt vlistID\ }
\item[{\tt vlistID}]
Variable list ID, from a previous call to {\htmlref{\tt vlistCreate}{vlistCreate}} or {\htmlref{\tt streamInqVlist}{streamInqVlist}}.
\item[{\tt varID}]
Variable identifier.

\end{deflist}
\end{minipage}

\subsubsection*{Result}

{\tt vlistInqVarZaxis} returns the zaxis ID of the variable.



\subsection{Define the code number of a Variable: {\tt vlistDefVarCode}}
\index{vlistDefVarCode}
\label{vlistDefVarCode}

The function {\tt vlistDefVarCode} defines the code number of a variable.

\subsubsection*{Usage}

\begin{verbatim}
    SUBROUTINE vlistDefVarCode(INTEGER vlistID, INTEGER varID, INTEGER code)
\end{verbatim}

\hspace*{4mm}\begin{minipage}[]{15cm}
\begin{deflist}{\tt vlistID\ }
\item[{\tt vlistID}]
Variable list ID, from a previous call to {\htmlref{\tt vlistCreate}{vlistCreate}}.
\item[{\tt varID}]
Variable identifier.
\item[{\tt code}]
Code number.

\end{deflist}
\end{minipage}


\subsection{Get the Code number of a Variable: {\tt vlistInqVarCode}}
\index{vlistInqVarCode}
\label{vlistInqVarCode}

The function {\tt vlistInqVarCode} returns the code number of a variable.

\subsubsection*{Usage}

\begin{verbatim}
    INTEGER FUNCTION vlistInqVarCode(INTEGER vlistID, INTEGER varID)
\end{verbatim}

\hspace*{4mm}\begin{minipage}[]{15cm}
\begin{deflist}{\tt vlistID\ }
\item[{\tt vlistID}]
Variable list ID, from a previous call to {\htmlref{\tt vlistCreate}{vlistCreate}} or {\htmlref{\tt streamInqVlist}{streamInqVlist}}.
\item[{\tt varID}]
Variable identifier.

\end{deflist}
\end{minipage}

\subsubsection*{Result}

{\tt vlistInqVarCode} returns the code number of the variable.



\subsection{Define the name of a Variable: {\tt vlistDefVarName}}
\index{vlistDefVarName}
\label{vlistDefVarName}

The function {\tt vlistDefVarName} defines the name of a variable.

\subsubsection*{Usage}

\begin{verbatim}
    SUBROUTINE vlistDefVarName(INTEGER vlistID, INTEGER varID, CHARACTER*(*) name)
\end{verbatim}

\hspace*{4mm}\begin{minipage}[]{15cm}
\begin{deflist}{\tt vlistID\ }
\item[{\tt vlistID}]
Variable list ID, from a previous call to {\htmlref{\tt vlistCreate}{vlistCreate}}.
\item[{\tt varID}]
Variable identifier.
\item[{\tt name}]
Name of the variable.

\end{deflist}
\end{minipage}


\subsection{Get the name of a Variable: {\tt vlistInqVarName}}
\index{vlistInqVarName}
\label{vlistInqVarName}

The function {\tt vlistInqVarName} returns the name of a variable.

\subsubsection*{Usage}

\begin{verbatim}
    SUBROUTINE vlistInqVarName(INTEGER vlistID, INTEGER varID, CHARACTER*(*) name)
\end{verbatim}

\hspace*{4mm}\begin{minipage}[]{15cm}
\begin{deflist}{\tt vlistID\ }
\item[{\tt vlistID}]
Variable list ID, from a previous call to {\htmlref{\tt vlistCreate}{vlistCreate}} or {\htmlref{\tt streamInqVlist}{streamInqVlist}}.
\item[{\tt varID}]
Variable identifier.
\item[{\tt name}]
Returned variable name. The caller must allocate space for the
                    returned string. The maximum possible length, in characters, of
                    the string is given by the predefined constant {\tt CDI\_MAX\_NAME}.

\end{deflist}
\end{minipage}

\subsubsection*{Result}

{\tt vlistInqVarName} returns the name of the variable to the parameter name if available,
otherwise the result is an empty string.



\subsection{Define the long name of a Variable: {\tt vlistDefVarLongname}}
\index{vlistDefVarLongname}
\label{vlistDefVarLongname}

The function {\tt vlistDefVarLongname} defines the long name of a variable.

\subsubsection*{Usage}

\begin{verbatim}
    SUBROUTINE vlistDefVarLongname(INTEGER vlistID, INTEGER varID, 
                                   CHARACTER*(*) longname)
\end{verbatim}

\hspace*{4mm}\begin{minipage}[]{15cm}
\begin{deflist}{\tt longname\ }
\item[{\tt vlistID}]
Variable list ID, from a previous call to {\htmlref{\tt vlistCreate}{vlistCreate}}.
\item[{\tt varID}]
Variable identifier.
\item[{\tt longname}]
Long name of the variable.

\end{deflist}
\end{minipage}


\subsection{Get the longname of a Variable: {\tt vlistInqVarLongname}}
\index{vlistInqVarLongname}
\label{vlistInqVarLongname}

The function {\tt vlistInqVarLongname} returns the longname of a variable if available,
otherwise the result is an empty string.

\subsubsection*{Usage}

\begin{verbatim}
    SUBROUTINE vlistInqVarLongname(INTEGER vlistID, INTEGER varID, 
                                   CHARACTER*(*) longname)
\end{verbatim}

\hspace*{4mm}\begin{minipage}[]{15cm}
\begin{deflist}{\tt longname\ }
\item[{\tt vlistID}]
Variable list ID, from a previous call to {\htmlref{\tt vlistCreate}{vlistCreate}} or {\htmlref{\tt streamInqVlist}{streamInqVlist}}.
\item[{\tt varID}]
Variable identifier.
\item[{\tt longname}]
Long name of the variable. The caller must allocate space for the
                    returned string. The maximum possible length, in characters, of
                    the string is given by the predefined constant {\tt CDI\_MAX\_NAME}.

\end{deflist}
\end{minipage}

\subsubsection*{Result}

{\tt vlistInqVaeLongname} returns the longname of the variable to the parameter longname.



\subsection{Define the standard name of a Variable: {\tt vlistDefVarStdname}}
\index{vlistDefVarStdname}
\label{vlistDefVarStdname}

The function {\tt vlistDefVarStdname} defines the standard name of a variable.

\subsubsection*{Usage}

\begin{verbatim}
    SUBROUTINE vlistDefVarStdname(INTEGER vlistID, INTEGER varID, 
                                  CHARACTER*(*) stdname)
\end{verbatim}

\hspace*{4mm}\begin{minipage}[]{15cm}
\begin{deflist}{\tt vlistID\ }
\item[{\tt vlistID}]
Variable list ID, from a previous call to {\htmlref{\tt vlistCreate}{vlistCreate}}.
\item[{\tt varID}]
Variable identifier.
\item[{\tt stdname}]
Standard name of the variable.

\end{deflist}
\end{minipage}


\subsection{Get the standard name of a Variable: {\tt vlistInqVarStdname}}
\index{vlistInqVarStdname}
\label{vlistInqVarStdname}

The function {\tt vlistInqVarStdname} returns the standard name of a variable if available,
otherwise the result is an empty string.

\subsubsection*{Usage}

\begin{verbatim}
    SUBROUTINE vlistInqVarStdname(INTEGER vlistID, INTEGER varID, 
                                  CHARACTER*(*) stdname)
\end{verbatim}

\hspace*{4mm}\begin{minipage}[]{15cm}
\begin{deflist}{\tt vlistID\ }
\item[{\tt vlistID}]
Variable list ID, from a previous call to {\htmlref{\tt vlistCreate}{vlistCreate}} or {\htmlref{\tt streamInqVlist}{streamInqVlist}}.
\item[{\tt varID}]
Variable identifier.
\item[{\tt stdname}]
Standard name of the variable. The caller must allocate space for the
                    returned string. The maximum possible length, in characters, of
                    the string is given by the predefined constant {\tt CDI\_MAX\_NAME}.

\end{deflist}
\end{minipage}

\subsubsection*{Result}

{\tt vlistInqVarName} returns the standard name of the variable to the parameter stdname.



\subsection{Define the units of a Variable: {\tt vlistDefVarUnits}}
\index{vlistDefVarUnits}
\label{vlistDefVarUnits}

The function {\tt vlistDefVarUnits} defines the units of a variable.

\subsubsection*{Usage}

\begin{verbatim}
    SUBROUTINE vlistDefVarUnits(INTEGER vlistID, INTEGER varID, CHARACTER*(*) units)
\end{verbatim}

\hspace*{4mm}\begin{minipage}[]{15cm}
\begin{deflist}{\tt vlistID\ }
\item[{\tt vlistID}]
Variable list ID, from a previous call to {\htmlref{\tt vlistCreate}{vlistCreate}}.
\item[{\tt varID}]
Variable identifier.
\item[{\tt units}]
Units of the variable.

\end{deflist}
\end{minipage}


\subsection{Get the units of a Variable: {\tt vlistInqVarUnits}}
\index{vlistInqVarUnits}
\label{vlistInqVarUnits}

The function {\tt vlistInqVarUnits} returns the units of a variable if available,
otherwise the result is an empty string.

\subsubsection*{Usage}

\begin{verbatim}
    SUBROUTINE vlistInqVarUnits(INTEGER vlistID, INTEGER varID, CHARACTER*(*) units)
\end{verbatim}

\hspace*{4mm}\begin{minipage}[]{15cm}
\begin{deflist}{\tt vlistID\ }
\item[{\tt vlistID}]
Variable list ID, from a previous call to {\htmlref{\tt vlistCreate}{vlistCreate}} or {\htmlref{\tt streamInqVlist}{streamInqVlist}}.
\item[{\tt varID}]
Variable identifier.
\item[{\tt units}]
Units of the variable. The caller must allocate space for the
                    returned string. The maximum possible length, in characters, of
                    the string is given by the predefined constant {\tt CDI\_MAX\_NAME}.

\end{deflist}
\end{minipage}

\subsubsection*{Result}

{\tt vlistInqVarUnits} returns the units of the variable to the parameter units.



\subsection{Define the data type of a Variable: {\tt vlistDefVarDatatype}}
\index{vlistDefVarDatatype}
\label{vlistDefVarDatatype}

The function {\tt vlistDefVarDatatype} defines the data type of a variable.

\subsubsection*{Usage}

\begin{verbatim}
    SUBROUTINE vlistDefVarDatatype(INTEGER vlistID, INTEGER varID, INTEGER datatype)
\end{verbatim}

\hspace*{4mm}\begin{minipage}[]{15cm}
\begin{deflist}{\tt datatype\ }
\item[{\tt vlistID}]
Variable list ID, from a previous call to {\htmlref{\tt vlistCreate}{vlistCreate}}.
\item[{\tt varID}]
Variable identifier.
\item[{\tt datatype}]
The data type identifier.
                    The valid {\CDI} data types are {\tt DATATYPE\_PACK8}, {\tt DATATYPE\_PACK16},
                    {\tt DATATYPE\_PACK24}, {\tt DATATYPE\_FLT32}, {\tt DATATYPE\_FLT64},
                    {\tt DATATYPE\_INT8}, {\tt DATATYPE\_INT16} and {\tt DATATYPE\_INT32}.

\end{deflist}
\end{minipage}


\subsection{Get the data type of a Variable: {\tt vlistInqVarDatatype}}
\index{vlistInqVarDatatype}
\label{vlistInqVarDatatype}

The function {\tt vlistInqVarDatatype} returns the data type of a variable.

\subsubsection*{Usage}

\begin{verbatim}
    INTEGER FUNCTION vlistInqVarDatatype(INTEGER vlistID, INTEGER varID)
\end{verbatim}

\hspace*{4mm}\begin{minipage}[]{15cm}
\begin{deflist}{\tt vlistID\ }
\item[{\tt vlistID}]
Variable list ID, from a previous call to {\htmlref{\tt vlistCreate}{vlistCreate}} or {\htmlref{\tt streamInqVlist}{streamInqVlist}}.
\item[{\tt varID}]
Variable identifier.

\end{deflist}
\end{minipage}

\subsubsection*{Result}

{\tt vlistInqVarDatatype} returns an identifier to the data type of the variable.
The valid {\CDI} data types are {\tt DATATYPE\_PACK8}, {\tt DATATYPE\_PACK16}, {\tt DATATYPE\_PACK24},
{\tt DATATYPE\_FLT32}, {\tt DATATYPE\_FLT64}, {\tt DATATYPE\_INT8}, {\tt DATATYPE\_INT16} and 
{\tt DATATYPE\_INT32}.



\subsection{Define the missing value of a Variable: {\tt vlistDefVarMissval}}
\index{vlistDefVarMissval}
\label{vlistDefVarMissval}

The function {\tt vlistDefVarMissval} defines the missing value of a variable.

\subsubsection*{Usage}

\begin{verbatim}
    SUBROUTINE vlistDefVarMissval(INTEGER vlistID, INTEGER varID, REAL*8 missval)
\end{verbatim}

\hspace*{4mm}\begin{minipage}[]{15cm}
\begin{deflist}{\tt vlistID\ }
\item[{\tt vlistID}]
Variable list ID, from a previous call to {\htmlref{\tt vlistCreate}{vlistCreate}}.
\item[{\tt varID}]
Variable identifier.
\item[{\tt missval}]
Missing value.

\end{deflist}
\end{minipage}


\subsection{Get the missing value of a Variable: {\tt vlistInqVarMissval}}
\index{vlistInqVarMissval}
\label{vlistInqVarMissval}

The function {\tt vlistInqVarMissval} returns the missing value of a variable.

\subsubsection*{Usage}

\begin{verbatim}
    REAL*8 FUNCTION vlistInqVarMissval(INTEGER vlistID, INTEGER varID)
\end{verbatim}

\hspace*{4mm}\begin{minipage}[]{15cm}
\begin{deflist}{\tt vlistID\ }
\item[{\tt vlistID}]
Variable list ID, from a previous call to {\htmlref{\tt vlistCreate}{vlistCreate}} or {\htmlref{\tt streamInqVlist}{streamInqVlist}}.
\item[{\tt varID}]
Variable identifier.

\end{deflist}
\end{minipage}

\subsubsection*{Result}

{\tt vlistInqVarMissval} returns the missing value of the variable.

