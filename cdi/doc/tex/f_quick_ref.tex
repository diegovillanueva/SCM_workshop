\chapter*{Quick Reference}
\addcontentsline{toc}{chapter}{Quick Reference}

This appendix provide a brief listing of the Fortran language bindings of the
CDI library routines:

\section*{\tt \htmlref{cdiClearAdditionalKeys}{cdiClearAdditionalKeys}}

\begin{verbatim}
    SUBROUTINE cdiClearAdditionalKeys
\end{verbatim}

Clear the list of additional GRIB keys..


\section*{\tt \htmlref{cdiDefAdditionalKey}{cdiDefAdditionalKey}}

\begin{verbatim}
    SUBROUTINE cdiDefAdditionalKey (CHARACTER*(*) string)
\end{verbatim}

Register an additional GRIB key which is read when file is opened..


\section*{\tt \htmlref{gridCreate}{gridCreate}}

\begin{verbatim}
    INTEGER FUNCTION gridCreate (INTEGER gridtype, INTEGER size)
\end{verbatim}

Create a horizontal Grid.


\section*{\tt \htmlref{gridDefNP}{gridDefNP}}

\begin{verbatim}
    SUBROUTINE gridDefNP (INTEGER gridID, INTEGER np)
\end{verbatim}

Define the number of parallels between a pole and the equator.


\section*{\tt \htmlref{gridDefNumber}{gridDefNumber}}

\begin{verbatim}
    SUBROUTINE gridDefNumber (INTEGER gridID, INTEGER number)
\end{verbatim}

Define the reference number for an unstructured grid.


\section*{\tt \htmlref{gridDefNvertex}{gridDefNvertex}}

\begin{verbatim}
    SUBROUTINE gridDefNvertex (INTEGER gridID, INTEGER nvertex)
\end{verbatim}

Define the number of vertex of a Gridbox.


\section*{\tt \htmlref{gridDefPosition}{gridDefPosition}}

\begin{verbatim}
    SUBROUTINE gridDefPosition (INTEGER gridID, INTEGER position)
\end{verbatim}

Define the position of grid in the reference file.


\section*{\tt \htmlref{gridDefPrec}{gridDefPrec}}

\begin{verbatim}
    SUBROUTINE gridDefPrec (INTEGER gridID, INTEGER prec)
\end{verbatim}

Define the precision of a Grid.


\section*{\tt \htmlref{gridDefReference}{gridDefReference}}

\begin{verbatim}
    SUBROUTINE gridDefReference (INTEGER gridID, CHARACTER*(*) reference)
\end{verbatim}

Define the reference URI for an unstructured grid.


\section*{\tt \htmlref{gridDefUUID}{gridDefUUID}}

\begin{verbatim}
    SUBROUTINE gridDefUUID (INTEGER gridID, INTEGER*1(16) uuid)
\end{verbatim}

Define the UUID of an unstructured grid.


\section*{\tt \htmlref{gridDefXbounds}{gridDefXbounds}}

\begin{verbatim}
    SUBROUTINE gridDefXbounds (INTEGER gridID, DOUBLEPRECISION xbounds(*))
\end{verbatim}

Define the bounds of a X-axis.


\section*{\tt \htmlref{gridDefXlongname}{gridDefXlongname}}

\begin{verbatim}
    SUBROUTINE gridDefXlongname (INTEGER gridID, CHARACTER*(*) xlongname)
\end{verbatim}

Define the longname of a X-axis.


\section*{\tt \htmlref{gridDefXname}{gridDefXname}}

\begin{verbatim}
    SUBROUTINE gridDefXname (INTEGER gridID, CHARACTER*(*) xname)
\end{verbatim}

Define the name of a X-axis.


\section*{\tt \htmlref{gridDefXsize}{gridDefXsize}}

\begin{verbatim}
    SUBROUTINE gridDefXsize (INTEGER gridID, INTEGER xsize)
\end{verbatim}

Define the size of a X-axis.


\section*{\tt \htmlref{gridDefXunits}{gridDefXunits}}

\begin{verbatim}
    SUBROUTINE gridDefXunits (INTEGER gridID, CHARACTER*(*) xunits)
\end{verbatim}

Define the units of a X-axis.


\section*{\tt \htmlref{gridDefXvals}{gridDefXvals}}

\begin{verbatim}
    SUBROUTINE gridDefXvals (INTEGER gridID, DOUBLEPRECISION xvals(*))
\end{verbatim}

Define the values of a X-axis.


\section*{\tt \htmlref{gridDefYbounds}{gridDefYbounds}}

\begin{verbatim}
    SUBROUTINE gridDefYbounds (INTEGER gridID, DOUBLEPRECISION ybounds(*))
\end{verbatim}

Define the bounds of a Y-axis.


\section*{\tt \htmlref{gridDefYlongname}{gridDefYlongname}}

\begin{verbatim}
    SUBROUTINE gridDefYlongname (INTEGER gridID, CHARACTER*(*) ylongname)
\end{verbatim}

Define the longname of a Y-axis.


\section*{\tt \htmlref{gridDefYname}{gridDefYname}}

\begin{verbatim}
    SUBROUTINE gridDefYname (INTEGER gridID, CHARACTER*(*) yname)
\end{verbatim}

Define the name of a Y-axis.


\section*{\tt \htmlref{gridDefYsize}{gridDefYsize}}

\begin{verbatim}
    SUBROUTINE gridDefYsize (INTEGER gridID, INTEGER ysize)
\end{verbatim}

Define the size of a Y-axis.


\section*{\tt \htmlref{gridDefYunits}{gridDefYunits}}

\begin{verbatim}
    SUBROUTINE gridDefYunits (INTEGER gridID, CHARACTER*(*) yunits)
\end{verbatim}

Define the units of a Y-axis.


\section*{\tt \htmlref{gridDefYvals}{gridDefYvals}}

\begin{verbatim}
    SUBROUTINE gridDefYvals (INTEGER gridID, DOUBLEPRECISION yvals(*))
\end{verbatim}

Define the values of a Y-axis.


\section*{\tt \htmlref{gridDestroy}{gridDestroy}}

\begin{verbatim}
    SUBROUTINE gridDestroy (INTEGER gridID)
\end{verbatim}

Destroy a horizontal Grid.


\section*{\tt \htmlref{gridDuplicate}{gridDuplicate}}

\begin{verbatim}
    INTEGER FUNCTION gridDuplicate (INTEGER gridID)
\end{verbatim}

Duplicate a Grid.


\section*{\tt \htmlref{gridInqNP}{gridInqNP}}

\begin{verbatim}
    INTEGER FUNCTION gridInqNP (INTEGER gridID)
\end{verbatim}

Get the number of parallels between a pole and the equator.


\section*{\tt \htmlref{gridInqNumber}{gridInqNumber}}

\begin{verbatim}
    INTEGER FUNCTION gridInqNumber (INTEGER gridID)
\end{verbatim}

Get the reference number to an unstructured grid.


\section*{\tt \htmlref{gridInqNvertex}{gridInqNvertex}}

\begin{verbatim}
    INTEGER FUNCTION gridInqNvertex (INTEGER gridID)
\end{verbatim}

Get the number of vertex of a Gridbox.


\section*{\tt \htmlref{gridInqPosition}{gridInqPosition}}

\begin{verbatim}
    INTEGER FUNCTION gridInqPosition (INTEGER gridID)
\end{verbatim}

Get the position of grid in the reference file.


\section*{\tt \htmlref{gridInqPrec}{gridInqPrec}}

\begin{verbatim}
    INTEGER FUNCTION gridInqPrec (INTEGER gridID)
\end{verbatim}

Get the precision of a Grid.


\section*{\tt \htmlref{gridInqReference}{gridInqReference}}

\begin{verbatim}
    INTEGER FUNCTION gridInqReference (INTEGER gridID, CHARACTER*(*) reference)
\end{verbatim}

Get the reference URI to an unstructured grid.


\section*{\tt \htmlref{gridInqSize}{gridInqSize}}

\begin{verbatim}
    INTEGER FUNCTION gridInqSize (INTEGER gridID)
\end{verbatim}

Get the size of a Grid.


\section*{\tt \htmlref{gridInqType}{gridInqType}}

\begin{verbatim}
    INTEGER FUNCTION gridInqType (INTEGER gridID)
\end{verbatim}

Get the type of a Grid.


\section*{\tt \htmlref{gridInqUUID}{gridInqUUID}}

\begin{verbatim}
    SUBROUTINE gridInqUUID (INTEGER gridID, INTEGER*1(16) uuid)
\end{verbatim}

Get the UUID of an unstructured grid.


\section*{\tt \htmlref{gridInqXbounds}{gridInqXbounds}}

\begin{verbatim}
    INTEGER FUNCTION gridInqXbounds (INTEGER gridID, DOUBLEPRECISION xbounds(*))
\end{verbatim}

Get the bounds of a X-axis.


\section*{\tt \htmlref{gridInqXlongname}{gridInqXlongname}}

\begin{verbatim}
    SUBROUTINE gridInqXlongname (INTEGER gridID, CHARACTER*(*) xlongname)
\end{verbatim}

Get the longname of a X-axis.


\section*{\tt \htmlref{gridInqXname}{gridInqXname}}

\begin{verbatim}
    SUBROUTINE gridInqXname (INTEGER gridID, CHARACTER*(*) xname)
\end{verbatim}

Get the name of a X-axis.


\section*{\tt \htmlref{gridInqXsize}{gridInqXsize}}

\begin{verbatim}
    INTEGER FUNCTION gridInqXsize (INTEGER gridID)
\end{verbatim}

Get the size of a X-axis.


\section*{\tt \htmlref{gridInqXstdname}{gridInqXstdname}}

\begin{verbatim}
    SUBROUTINE gridInqXstdname (INTEGER gridID, CHARACTER*(*) xstdname)
\end{verbatim}

Get the standard name of a X-axis.


\section*{\tt \htmlref{gridInqXunits}{gridInqXunits}}

\begin{verbatim}
    SUBROUTINE gridInqXunits (INTEGER gridID, CHARACTER*(*) xunits)
\end{verbatim}

Get the units of a X-axis.


\section*{\tt \htmlref{gridInqXval}{gridInqXval}}

\begin{verbatim}
    DOUBLEPRECISION FUNCTION gridInqXval (INTEGER gridID, INTEGER index)
\end{verbatim}

Get one value of a X-axis.


\section*{\tt \htmlref{gridInqXvals}{gridInqXvals}}

\begin{verbatim}
    INTEGER FUNCTION gridInqXvals (INTEGER gridID, DOUBLEPRECISION xvals(*))
\end{verbatim}

Get all values of a X-axis.


\section*{\tt \htmlref{gridInqYbounds}{gridInqYbounds}}

\begin{verbatim}
    INTEGER FUNCTION gridInqYbounds (INTEGER gridID, DOUBLEPRECISION ybounds(*))
\end{verbatim}

Get the bounds of a Y-axis.


\section*{\tt \htmlref{gridInqYlongname}{gridInqYlongname}}

\begin{verbatim}
    SUBROUTINE gridInqYlongname (INTEGER gridID, CHARACTER*(*) ylongname)
\end{verbatim}

Get the longname of a Y-axis.


\section*{\tt \htmlref{gridInqYname}{gridInqYname}}

\begin{verbatim}
    SUBROUTINE gridInqYname (INTEGER gridID, CHARACTER*(*) yname)
\end{verbatim}

Get the name of a Y-axis.


\section*{\tt \htmlref{gridInqYsize}{gridInqYsize}}

\begin{verbatim}
    INTEGER FUNCTION gridInqYsize (INTEGER gridID)
\end{verbatim}

Get the size of a Y-axis.


\section*{\tt \htmlref{gridInqYstdname}{gridInqYstdname}}

\begin{verbatim}
    SUBROUTINE gridInqYstdname (INTEGER gridID, CHARACTER*(*) ystdname)
\end{verbatim}

Get the standard name of a Y-axis.


\section*{\tt \htmlref{gridInqYunits}{gridInqYunits}}

\begin{verbatim}
    SUBROUTINE gridInqYunits (INTEGER gridID, CHARACTER*(*) yunits)
\end{verbatim}

Get the units of a Y-axis.


\section*{\tt \htmlref{gridInqYval}{gridInqYval}}

\begin{verbatim}
    DOUBLEPRECISION FUNCTION gridInqYval (INTEGER gridID, INTEGER index)
\end{verbatim}

Get one value of a Y-axis.


\section*{\tt \htmlref{gridInqYvals}{gridInqYvals}}

\begin{verbatim}
    INTEGER FUNCTION gridInqYvals (INTEGER gridID, DOUBLEPRECISION yvals(*))
\end{verbatim}

Get all values of a Y-axis.


\section*{\tt \htmlref{streamClose}{streamClose}}

\begin{verbatim}
    SUBROUTINE streamClose (INTEGER streamID)
\end{verbatim}

Close an open dataset.


\section*{\tt \htmlref{streamDefByteorder}{streamDefByteorder}}

\begin{verbatim}
    SUBROUTINE streamDefByteorder (INTEGER streamID, INTEGER byteorder)
\end{verbatim}

Define the byteorder.


\section*{\tt \htmlref{streamDefCompLevel}{streamDefCompLevel}}

\begin{verbatim}
    SUBROUTINE streamDefCompLevel (INTEGER streamID, INTEGER complevel)
\end{verbatim}

Define compression level.


\section*{\tt \htmlref{streamDefCompType}{streamDefCompType}}

\begin{verbatim}
    SUBROUTINE streamDefCompType (INTEGER streamID, INTEGER comptype)
\end{verbatim}

Define compression type.


\section*{\tt \htmlref{streamDefTimestep}{streamDefTimestep}}

\begin{verbatim}
    INTEGER FUNCTION streamDefTimestep (INTEGER streamID, INTEGER tsID)
\end{verbatim}

Define time step.


\section*{\tt \htmlref{streamDefVlist}{streamDefVlist}}

\begin{verbatim}
    SUBROUTINE streamDefVlist (INTEGER streamID, INTEGER vlistID)
\end{verbatim}

Define the Vlist for a stream.


\section*{\tt \htmlref{streamInqByteorder}{streamInqByteorder}}

\begin{verbatim}
    INTEGER FUNCTION streamInqByteorder (INTEGER streamID)
\end{verbatim}

Get the byteorder.


\section*{\tt \htmlref{streamInqCompLevel}{streamInqCompLevel}}

\begin{verbatim}
    INTEGER FUNCTION streamInqCompLevel (INTEGER streamID)
\end{verbatim}

Get compression level.


\section*{\tt \htmlref{streamInqCompType}{streamInqCompType}}

\begin{verbatim}
    INTEGER FUNCTION streamInqCompType (INTEGER streamID)
\end{verbatim}

Get compression type.


\section*{\tt \htmlref{streamInqFiletype}{streamInqFiletype}}

\begin{verbatim}
    INTEGER FUNCTION streamInqFiletype (INTEGER streamID)
\end{verbatim}

Get the filetype.


\section*{\tt \htmlref{streamInqTimestep}{streamInqTimestep}}

\begin{verbatim}
    INTEGER FUNCTION streamInqTimestep (INTEGER streamID, INTEGER tsID)
\end{verbatim}

Get time step.


\section*{\tt \htmlref{streamInqVlist}{streamInqVlist}}

\begin{verbatim}
    INTEGER FUNCTION streamInqVlist (INTEGER streamID)
\end{verbatim}

Get the Vlist of a stream.


\section*{\tt \htmlref{streamOpenRead}{streamOpenRead}}

\begin{verbatim}
    INTEGER FUNCTION streamOpenRead (CHARACTER*(*) path)
\end{verbatim}

Open a dataset for reading.


\section*{\tt \htmlref{streamOpenWrite}{streamOpenWrite}}

\begin{verbatim}
    INTEGER FUNCTION streamOpenWrite (CHARACTER*(*) path, INTEGER filetype)
\end{verbatim}

Create a new dataset.


\section*{\tt \htmlref{streamReadVar}{streamReadVar}}

\begin{verbatim}
    SUBROUTINE streamReadVar (INTEGER streamID, INTEGER varID,
                              DOUBLEPRECISION data(*), INTEGER nmiss)
\end{verbatim}

Read a variable.


\section*{\tt \htmlref{streamReadVarSlice}{streamReadVarSlice}}

\begin{verbatim}
    SUBROUTINE streamReadVarSlice (INTEGER streamID, INTEGER varID, INTEGER levelID,
                                   DOUBLEPRECISION data(*), INTEGER nmiss)
\end{verbatim}

Read a horizontal slice of a variable.


\section*{\tt \htmlref{streamSync}{streamSync}}

\begin{verbatim}
    SUBROUTINE streamSync (INTEGER streamID)
\end{verbatim}

Synchronize an Open Dataset to Disk.


\section*{\tt \htmlref{streamWriteVar}{streamWriteVar}}

\begin{verbatim}
    SUBROUTINE streamWriteVar (INTEGER streamID, INTEGER varID,
                               DOUBLEPRECISION data(*), INTEGER nmiss)
\end{verbatim}

Write a variable.


\section*{\tt \htmlref{streamWriteVarSlice}{streamWriteVarSlice}}

\begin{verbatim}
    SUBROUTINE streamWriteVarSlice (INTEGER streamID, INTEGER varID, INTEGER levelID,
                                    DOUBLEPRECISION data(*), INTEGER nmiss)
\end{verbatim}

Write a horizontal slice of a variable.


\section*{\tt \htmlref{subtypeCreate}{subtypeCreate}}

\begin{verbatim}
    INTEGER FUNCTION subtypeCreate (INTEGER subtype)
\end{verbatim}

Create a variable subtype.


\section*{\tt \htmlref{subtypeDefActiveIndex}{subtypeDefActiveIndex}}

\begin{verbatim}
    SUBROUTINE subtypeDefActiveIndex (INTEGER subtypeID, INTEGER index)
\end{verbatim}

Set the currently active index of a subtype (e.g. current tile index)..


\section*{\tt \htmlref{subtypeInqActiveIndex}{subtypeInqActiveIndex}}

\begin{verbatim}
    INTEGER FUNCTION subtypeInqActiveIndex (INTEGER subtypeID)
\end{verbatim}

Get the currently active index of a subtype (e.g. current tile index)..


\section*{\tt \htmlref{subtypeInqSize}{subtypeInqSize}}

\begin{verbatim}
    INTEGER FUNCTION subtypeInqSize (INTEGER subtypeID)
\end{verbatim}

Get the size of a subtype (e.g. no. of tiles)..


\section*{\tt \htmlref{subtypeInqTile}{subtypeInqTile}}

\begin{verbatim}
    INTEGER FUNCTION subtypeInqTile (INTEGER subtypeID, INTEGER tileindex,
                                     INTEGER attribute)
\end{verbatim}

Specialized version of subtypeInqSubEntry looking for tile/attribute pair..


\section*{\tt \htmlref{tableRead}{tableRead}}

\begin{verbatim}
    INTEGER FUNCTION tableRead (CHARACTER*(*) tablefile)
\end{verbatim}

read table of parameters from file in tabular format.


\section*{\tt \htmlref{tableWrite}{tableWrite}}

\begin{verbatim}
    SUBROUTINE tableWrite (CHARACTER*(*) filename, INTEGER tableID)
\end{verbatim}

write table of parameters to file in tabular format.


\section*{\tt \htmlref{tableWriteC}{tableWriteC}}

\begin{verbatim}
    SUBROUTINE tableWriteC (CHARACTER*(*) filename, INTEGER tableID)
\end{verbatim}

write table of parameters to file in C language format.


\section*{\tt \htmlref{taxisCreate}{taxisCreate}}

\begin{verbatim}
    INTEGER FUNCTION taxisCreate (INTEGER timetype)
\end{verbatim}

Create a Time axis.


\section*{\tt \htmlref{taxisDefCalendar}{taxisDefCalendar}}

\begin{verbatim}
    SUBROUTINE taxisDefCalendar (INTEGER taxisID, INTEGER calendar)
\end{verbatim}

Define the calendar.


\section*{\tt \htmlref{taxisDefFdate}{taxisDefFdate}}

\begin{verbatim}
    SUBROUTINE taxisDefFdate (INTEGER taxisID, INTEGER date)
\end{verbatim}

Define the forecast reference date.


\section*{\tt \htmlref{taxisDefFtime}{taxisDefFtime}}

\begin{verbatim}
    SUBROUTINE taxisDefFtime (INTEGER taxisID, INTEGER time)
\end{verbatim}

Define the forecast reference time.


\section*{\tt \htmlref{taxisDefRdate}{taxisDefRdate}}

\begin{verbatim}
    SUBROUTINE taxisDefRdate (INTEGER taxisID, INTEGER date)
\end{verbatim}

Define the reference date.


\section*{\tt \htmlref{taxisDefRtime}{taxisDefRtime}}

\begin{verbatim}
    SUBROUTINE taxisDefRtime (INTEGER taxisID, INTEGER time)
\end{verbatim}

Define the reference time.


\section*{\tt \htmlref{taxisDefVdate}{taxisDefVdate}}

\begin{verbatim}
    SUBROUTINE taxisDefVdate (INTEGER taxisID, INTEGER date)
\end{verbatim}

Define the verification date.


\section*{\tt \htmlref{taxisDefVtime}{taxisDefVtime}}

\begin{verbatim}
    SUBROUTINE taxisDefVtime (INTEGER taxisID, INTEGER time)
\end{verbatim}

Define the verification time.


\section*{\tt \htmlref{taxisDestroy}{taxisDestroy}}

\begin{verbatim}
    SUBROUTINE taxisDestroy (INTEGER taxisID)
\end{verbatim}

Destroy a Time axis.


\section*{\tt \htmlref{taxisInqCalendar}{taxisInqCalendar}}

\begin{verbatim}
    INTEGER FUNCTION taxisInqCalendar (INTEGER taxisID)
\end{verbatim}

Get the calendar.


\section*{\tt \htmlref{taxisInqFdate}{taxisInqFdate}}

\begin{verbatim}
    INTEGER FUNCTION taxisInqFdate (INTEGER taxisID)
\end{verbatim}

Get the forecast reference date.


\section*{\tt \htmlref{taxisInqFtime}{taxisInqFtime}}

\begin{verbatim}
    INTEGER FUNCTION taxisInqFtime (INTEGER taxisID)
\end{verbatim}

Get the forecast reference time.


\section*{\tt \htmlref{taxisInqRdate}{taxisInqRdate}}

\begin{verbatim}
    INTEGER FUNCTION taxisInqRdate (INTEGER taxisID)
\end{verbatim}

Get the reference date.


\section*{\tt \htmlref{taxisInqRtime}{taxisInqRtime}}

\begin{verbatim}
    INTEGER FUNCTION taxisInqRtime (INTEGER taxisID)
\end{verbatim}

Get the reference time.


\section*{\tt \htmlref{taxisInqVdate}{taxisInqVdate}}

\begin{verbatim}
    INTEGER FUNCTION taxisInqVdate (INTEGER taxisID)
\end{verbatim}

Get the verification date.


\section*{\tt \htmlref{taxisInqVtime}{taxisInqVtime}}

\begin{verbatim}
    INTEGER FUNCTION taxisInqVtime (INTEGER taxisID)
\end{verbatim}

Get the verification time.


\section*{\tt \htmlref{vlistCat}{vlistCat}}

\begin{verbatim}
    SUBROUTINE vlistCat (INTEGER vlistID2, INTEGER vlistID1)
\end{verbatim}

Concatenate two variable lists.


\section*{\tt \htmlref{vlistCopy}{vlistCopy}}

\begin{verbatim}
    SUBROUTINE vlistCopy (INTEGER vlistID2, INTEGER vlistID1)
\end{verbatim}

Copy a variable list.


\section*{\tt \htmlref{vlistCopyFlag}{vlistCopyFlag}}

\begin{verbatim}
    SUBROUTINE vlistCopyFlag (INTEGER vlistID2, INTEGER vlistID1)
\end{verbatim}

Copy some entries of a variable list.


\section*{\tt \htmlref{vlistCopyVarName}{vlistCopyVarName}}

\begin{verbatim}
    CHARACTER(80) FUNCTION vlistCopyVarName (INTEGER vlistId, INTEGER varId)
\end{verbatim}

Safe and convenient version of vlistInqVarName.


\section*{\tt \htmlref{vlistCreate}{vlistCreate}}

\begin{verbatim}
    INTEGER FUNCTION vlistCreate
\end{verbatim}

Create a variable list.


\section*{\tt \htmlref{vlistDefAttFlt}{vlistDefAttFlt}}

\begin{verbatim}
    INTEGER FUNCTION vlistDefAttFlt (INTEGER vlistID, INTEGER varID,
                                     CHARACTER*(*) name, INTEGER type, INTEGER len,
                                     DOUBLEPRECISION dp(*))
\end{verbatim}

Define a floating point attribute.


\section*{\tt \htmlref{vlistDefAttInt}{vlistDefAttInt}}

\begin{verbatim}
    INTEGER FUNCTION vlistDefAttInt (INTEGER vlistID, INTEGER varID,
                                     CHARACTER*(*) name, INTEGER type, INTEGER len,
                                     INTEGER ip(*))
\end{verbatim}

Define an integer attribute.


\section*{\tt \htmlref{vlistDefAttTxt}{vlistDefAttTxt}}

\begin{verbatim}
    INTEGER FUNCTION vlistDefAttTxt (INTEGER vlistID, INTEGER varID,
                                     CHARACTER*(*) name, INTEGER len,
                                     CHARACTER*(*) tp_cbuf)
\end{verbatim}

Define a text attribute.


\section*{\tt \htmlref{vlistDefTaxis}{vlistDefTaxis}}

\begin{verbatim}
    SUBROUTINE vlistDefTaxis (INTEGER vlistID, INTEGER taxisID)
\end{verbatim}

Define the time axis of a variable list.


\section*{\tt \htmlref{vlistDefVar}{vlistDefVar}}

\begin{verbatim}
    INTEGER FUNCTION vlistDefVar (INTEGER vlistID, INTEGER gridID, INTEGER zaxisID,
                                  INTEGER tsteptype)
\end{verbatim}

Create a new variable.


\section*{\tt \htmlref{vlistDefVarCode}{vlistDefVarCode}}

\begin{verbatim}
    SUBROUTINE vlistDefVarCode (INTEGER vlistID, INTEGER varID, INTEGER code)
\end{verbatim}

Define the code number of a Variable.


\section*{\tt \htmlref{vlistDefVarDatatype}{vlistDefVarDatatype}}

\begin{verbatim}
    SUBROUTINE vlistDefVarDatatype (INTEGER vlistID, INTEGER varID, INTEGER datatype)
\end{verbatim}

Define the data type of a Variable.


\section*{\tt \htmlref{vlistDefVarDblKey}{vlistDefVarDblKey}}

\begin{verbatim}
    SUBROUTINE vlistDefVarDblKey (INTEGER vlistID, INTEGER varID, CHARACTER*(*) name,
                                  DOUBLEPRECISION value)
\end{verbatim}

Set an arbitrary keyword/double value pair for GRIB API.


\section*{\tt \htmlref{vlistDefVarExtra}{vlistDefVarExtra}}

\begin{verbatim}
    SUBROUTINE vlistDefVarExtra (INTEGER vlistID, INTEGER varID, CHARACTER*(*) extra)
\end{verbatim}

Define extra information of a Variable.


\section*{\tt \htmlref{vlistDefVarIntKey}{vlistDefVarIntKey}}

\begin{verbatim}
    SUBROUTINE vlistDefVarIntKey (INTEGER vlistID, INTEGER varID, CHARACTER*(*) name,
                                  INTEGER value)
\end{verbatim}

Set an arbitrary keyword/integer value pair for GRIB API.


\section*{\tt \htmlref{vlistDefVarLongname}{vlistDefVarLongname}}

\begin{verbatim}
    SUBROUTINE vlistDefVarLongname (INTEGER vlistID, INTEGER varID,
                                    CHARACTER*(*) longname)
\end{verbatim}

Define the long name of a Variable.


\section*{\tt \htmlref{vlistDefVarMissval}{vlistDefVarMissval}}

\begin{verbatim}
    SUBROUTINE vlistDefVarMissval (INTEGER vlistID, INTEGER varID,
                                   DOUBLEPRECISION missval)
\end{verbatim}

Define the missing value of a Variable.


\section*{\tt \htmlref{vlistDefVarName}{vlistDefVarName}}

\begin{verbatim}
    SUBROUTINE vlistDefVarName (INTEGER vlistID, INTEGER varID, CHARACTER*(*) name)
\end{verbatim}

Define the name of a Variable.


\section*{\tt \htmlref{vlistDefVarParam}{vlistDefVarParam}}

\begin{verbatim}
    SUBROUTINE vlistDefVarParam (INTEGER vlistID, INTEGER varID, INTEGER param)
\end{verbatim}

Define the parameter number of a Variable.


\section*{\tt \htmlref{vlistDefVarStdname}{vlistDefVarStdname}}

\begin{verbatim}
    SUBROUTINE vlistDefVarStdname (INTEGER vlistID, INTEGER varID,
                                   CHARACTER*(*) stdname)
\end{verbatim}

Define the standard name of a Variable.


\section*{\tt \htmlref{vlistDefVarTiles}{vlistDefVarTiles}}

\begin{verbatim}
    INTEGER FUNCTION vlistDefVarTiles (INTEGER vlistID, INTEGER gridID,
                                       INTEGER zaxisID, INTEGER tsteptype,
                                       INTEGER tilesetID)
\end{verbatim}

Create a new tile-based variable.


\section*{\tt \htmlref{vlistDefVarUnits}{vlistDefVarUnits}}

\begin{verbatim}
    SUBROUTINE vlistDefVarUnits (INTEGER vlistID, INTEGER varID, CHARACTER*(*) units)
\end{verbatim}

Define the units of a Variable.


\section*{\tt \htmlref{vlistDestroy}{vlistDestroy}}

\begin{verbatim}
    SUBROUTINE vlistDestroy (INTEGER vlistID)
\end{verbatim}

Destroy a variable list.


\section*{\tt \htmlref{vlistDuplicate}{vlistDuplicate}}

\begin{verbatim}
    INTEGER FUNCTION vlistDuplicate (INTEGER vlistID)
\end{verbatim}

Duplicate a variable list.


\section*{\tt \htmlref{vlistHasVarKey}{vlistHasVarKey}}

\begin{verbatim}
    INTEGER FUNCTION vlistHasVarKey (INTEGER vlistID, INTEGER varID,
                                     CHARACTER*(*) name)
\end{verbatim}

returns 1 if meta-data key was read, 0 otherwise..


\section*{\tt \htmlref{vlistInqAtt}{vlistInqAtt}}

\begin{verbatim}
    INTEGER FUNCTION vlistInqAtt (INTEGER vlistID, INTEGER varID, INTEGER attrnum,
                                  CHARACTER*(*) name, INTEGER typep, INTEGER lenp)
\end{verbatim}

Get information about an attribute.


\section*{\tt \htmlref{vlistInqAttFlt}{vlistInqAttFlt}}

\begin{verbatim}
    INTEGER FUNCTION vlistInqAttFlt (INTEGER vlistID, INTEGER varID,
                                     CHARACTER*(*) name, INTEGER mlen,
                                     DOUBLEPRECISION dp(*))
\end{verbatim}

Get the value(s) of a floating point attribute.


\section*{\tt \htmlref{vlistInqAttInt}{vlistInqAttInt}}

\begin{verbatim}
    INTEGER FUNCTION vlistInqAttInt (INTEGER vlistID, INTEGER varID,
                                     CHARACTER*(*) name, INTEGER mlen, INTEGER ip(*))
\end{verbatim}

Get the value(s) of an integer attribute.


\section*{\tt \htmlref{vlistInqAttTxt}{vlistInqAttTxt}}

\begin{verbatim}
    INTEGER FUNCTION vlistInqAttTxt (INTEGER vlistID, INTEGER varID,
                                     CHARACTER*(*) name, INTEGER mlen,
                                     CHARACTER*(*) tp_cbuf)
\end{verbatim}

Get the value(s) of a text attribute.


\section*{\tt \htmlref{vlistInqNatts}{vlistInqNatts}}

\begin{verbatim}
    INTEGER FUNCTION vlistInqNatts (INTEGER vlistID, INTEGER varID, INTEGER nattsp)
\end{verbatim}

Get number of variable attributes assigned to this variable.


\section*{\tt \htmlref{vlistInqTaxis}{vlistInqTaxis}}

\begin{verbatim}
    INTEGER FUNCTION vlistInqTaxis (INTEGER vlistID)
\end{verbatim}

Get the time axis of a variable list.


\section*{\tt \htmlref{vlistInqVarCode}{vlistInqVarCode}}

\begin{verbatim}
    INTEGER FUNCTION vlistInqVarCode (INTEGER vlistID, INTEGER varID)
\end{verbatim}

Get the code number of a Variable.


\section*{\tt \htmlref{vlistInqVarDatatype}{vlistInqVarDatatype}}

\begin{verbatim}
    INTEGER FUNCTION vlistInqVarDatatype (INTEGER vlistID, INTEGER varID)
\end{verbatim}

Get the data type of a Variable.


\section*{\tt \htmlref{vlistInqVarDblKey}{vlistInqVarDblKey}}

\begin{verbatim}
    DOUBLEPRECISION FUNCTION vlistInqVarDblKey (INTEGER vlistID, INTEGER varID,
                                                CHARACTER*(*) name)
\end{verbatim}

raw access to GRIB meta-data.


\section*{\tt \htmlref{vlistInqVarExtra}{vlistInqVarExtra}}

\begin{verbatim}
    SUBROUTINE vlistInqVarExtra (INTEGER vlistID, INTEGER varID, CHARACTER*(*) extra)
\end{verbatim}

Get extra information of a Variable.


\section*{\tt \htmlref{vlistInqVarIntKey}{vlistInqVarIntKey}}

\begin{verbatim}
    INTEGER FUNCTION vlistInqVarIntKey (INTEGER vlistID, INTEGER varID,
                                        CHARACTER*(*) name)
\end{verbatim}

raw access to GRIB meta-data.


\section*{\tt \htmlref{vlistInqVarLongname}{vlistInqVarLongname}}

\begin{verbatim}
    SUBROUTINE vlistInqVarLongname (INTEGER vlistID, INTEGER varID,
                                    CHARACTER*(*) longname)
\end{verbatim}

Get the long name of a Variable.


\section*{\tt \htmlref{vlistInqVarMissval}{vlistInqVarMissval}}

\begin{verbatim}
    DOUBLEPRECISION FUNCTION vlistInqVarMissval (INTEGER vlistID, INTEGER varID)
\end{verbatim}

Get the missing value of a Variable.


\section*{\tt \htmlref{vlistInqVarName}{vlistInqVarName}}

\begin{verbatim}
    SUBROUTINE vlistInqVarName (INTEGER vlistID, INTEGER varID, CHARACTER*(*) name)
\end{verbatim}

Get the name of a Variable.


\section*{\tt \htmlref{vlistInqVarParam}{vlistInqVarParam}}

\begin{verbatim}
    INTEGER FUNCTION vlistInqVarParam (INTEGER vlistID, INTEGER varID)
\end{verbatim}

Get the parameter number of a Variable.


\section*{\tt \htmlref{vlistInqVarStdname}{vlistInqVarStdname}}

\begin{verbatim}
    SUBROUTINE vlistInqVarStdname (INTEGER vlistID, INTEGER varID,
                                   CHARACTER*(*) stdname)
\end{verbatim}

Get the standard name of a Variable.


\section*{\tt \htmlref{vlistInqVarSubtype}{vlistInqVarSubtype}}

\begin{verbatim}
    INTEGER FUNCTION vlistInqVarSubtype (INTEGER vlistID, INTEGER varID)
\end{verbatim}

Return subtype ID for a given variable..


\section*{\tt \htmlref{vlistInqVarUnits}{vlistInqVarUnits}}

\begin{verbatim}
    SUBROUTINE vlistInqVarUnits (INTEGER vlistID, INTEGER varID, CHARACTER*(*) units)
\end{verbatim}

Get the units of a Variable.


\section*{\tt \htmlref{vlistMerge}{vlistMerge}}

\begin{verbatim}
    SUBROUTINE vlistMerge (INTEGER vlistID2, INTEGER vlistID1)
\end{verbatim}

Merge two variable lists.


\section*{\tt \htmlref{vlistNgrids}{vlistNgrids}}

\begin{verbatim}
    INTEGER FUNCTION vlistNgrids (INTEGER vlistID)
\end{verbatim}

Number of grids in a variable list.


\section*{\tt \htmlref{vlistNsubtypes}{vlistNsubtypes}}

\begin{verbatim}
    INTEGER FUNCTION vlistNsubtypes (INTEGER vlistID)
\end{verbatim}

Number of subtypes in a variable list.


\section*{\tt \htmlref{vlistNumber}{vlistNumber}}

\begin{verbatim}
    INTEGER FUNCTION vlistNumber (INTEGER vlistID)
\end{verbatim}

Number type in a variable list.


\section*{\tt \htmlref{vlistNvars}{vlistNvars}}

\begin{verbatim}
    INTEGER FUNCTION vlistNvars (INTEGER vlistID)
\end{verbatim}

Number of variables in a variable list.


\section*{\tt \htmlref{vlistNzaxis}{vlistNzaxis}}

\begin{verbatim}
    INTEGER FUNCTION vlistNzaxis (INTEGER vlistID)
\end{verbatim}

Number of zaxis in a variable list.


\section*{\tt \htmlref{zaxisCreate}{zaxisCreate}}

\begin{verbatim}
    INTEGER FUNCTION zaxisCreate (INTEGER zaxistype, INTEGER size)
\end{verbatim}

Create a vertical Z-axis.


\section*{\tt \htmlref{zaxisDefLevel}{zaxisDefLevel}}

\begin{verbatim}
    SUBROUTINE zaxisDefLevel (INTEGER zaxisID, INTEGER levelID,
                              DOUBLEPRECISION levels)
\end{verbatim}

Define one level of a Z-axis.


\section*{\tt \htmlref{zaxisDefLevels}{zaxisDefLevels}}

\begin{verbatim}
    SUBROUTINE zaxisDefLevels (INTEGER zaxisID, DOUBLEPRECISION levels(*))
\end{verbatim}

Define the levels of a Z-axis.


\section*{\tt \htmlref{zaxisDefLongname}{zaxisDefLongname}}

\begin{verbatim}
    SUBROUTINE zaxisDefLongname (INTEGER zaxisID, CHARACTER*(*) longname_optional)
\end{verbatim}

Define the longname of a Z-axis.


\section*{\tt \htmlref{zaxisDefName}{zaxisDefName}}

\begin{verbatim}
    SUBROUTINE zaxisDefName (INTEGER zaxisID, CHARACTER*(*) name_optional)
\end{verbatim}

Define the name of a Z-axis.


\section*{\tt \htmlref{zaxisDefNlevRef}{zaxisDefNlevRef}}

\begin{verbatim}
    SUBROUTINE zaxisDefNlevRef (INTEGER gridID, INTEGER nhlev)
\end{verbatim}

Define the number of half levels of a generalized Z-axis.


\section*{\tt \htmlref{zaxisDefNumber}{zaxisDefNumber}}

\begin{verbatim}
    SUBROUTINE zaxisDefNumber (INTEGER gridID, INTEGER number)
\end{verbatim}

Define the reference number for a generalized Z-axis.


\section*{\tt \htmlref{zaxisDefPsName}{zaxisDefPsName}}

\begin{verbatim}
    SUBROUTINE zaxisDefPsName (INTEGER zaxisID, CHARACTER*(*) psname_optional)
\end{verbatim}

Define the name of the surface pressure variable of a hybrid sigma pressure Z-axis.


\section*{\tt \htmlref{zaxisDefUUID}{zaxisDefUUID}}

\begin{verbatim}
    SUBROUTINE zaxisDefUUID (INTEGER zaxisID, INTEGER*1(16) uuid)
\end{verbatim}

Define the UUID of a generalized Z-axis.


\section*{\tt \htmlref{zaxisDefUnits}{zaxisDefUnits}}

\begin{verbatim}
    SUBROUTINE zaxisDefUnits (INTEGER zaxisID, CHARACTER*(*) units_optional)
\end{verbatim}

Define the units of a Z-axis.


\section*{\tt \htmlref{zaxisDestroy}{zaxisDestroy}}

\begin{verbatim}
    SUBROUTINE zaxisDestroy (INTEGER zaxisID)
\end{verbatim}

Destroy a vertical Z-axis.


\section*{\tt \htmlref{zaxisDuplicate}{zaxisDuplicate}}

\begin{verbatim}
    INTEGER FUNCTION zaxisDuplicate (INTEGER zaxisID)
\end{verbatim}

Duplicate a Z-axis.


\section*{\tt \htmlref{zaxisInqLevel}{zaxisInqLevel}}

\begin{verbatim}
    DOUBLEPRECISION FUNCTION zaxisInqLevel (INTEGER zaxisID, INTEGER levelID)
\end{verbatim}

Get one level of a Z-axis.


\section*{\tt \htmlref{zaxisInqLevels}{zaxisInqLevels}}

\begin{verbatim}
    SUBROUTINE zaxisInqLevels (INTEGER zaxisID, DOUBLEPRECISION levels(*))
\end{verbatim}

Get all levels of a Z-axis.


\section*{\tt \htmlref{zaxisInqLongname}{zaxisInqLongname}}

\begin{verbatim}
    SUBROUTINE zaxisInqLongname (INTEGER zaxisID, CHARACTER*(*) longname)
\end{verbatim}

Get the longname of a Z-axis.


\section*{\tt \htmlref{zaxisInqName}{zaxisInqName}}

\begin{verbatim}
    SUBROUTINE zaxisInqName (INTEGER zaxisID, CHARACTER*(*) name)
\end{verbatim}

Get the name of a Z-axis.


\section*{\tt \htmlref{zaxisInqNlevRef}{zaxisInqNlevRef}}

\begin{verbatim}
    INTEGER FUNCTION zaxisInqNlevRef (INTEGER gridID)
\end{verbatim}

Get the number of half levels of a generalized Z-axis.


\section*{\tt \htmlref{zaxisInqNumber}{zaxisInqNumber}}

\begin{verbatim}
    INTEGER FUNCTION zaxisInqNumber (INTEGER gridID)
\end{verbatim}

Get the reference number to a generalized Z-axis.


\section*{\tt \htmlref{zaxisInqPsName}{zaxisInqPsName}}

\begin{verbatim}
    SUBROUTINE zaxisInqPsName (INTEGER zaxisID, CHARACTER*(*) psname)
\end{verbatim}

Get the name of the surface pressure variable of a hybrid sigma pressure Z-axis.


\section*{\tt \htmlref{zaxisInqSize}{zaxisInqSize}}

\begin{verbatim}
    INTEGER FUNCTION zaxisInqSize (INTEGER zaxisID)
\end{verbatim}

Get the size of a Z-axis.


\section*{\tt \htmlref{zaxisInqStdname}{zaxisInqStdname}}

\begin{verbatim}
    SUBROUTINE zaxisInqStdname (INTEGER zaxisID, CHARACTER*(*) stdname)
\end{verbatim}

Get the standard name of a Z-axis.


\section*{\tt \htmlref{zaxisInqType}{zaxisInqType}}

\begin{verbatim}
    INTEGER FUNCTION zaxisInqType (INTEGER zaxisID)
\end{verbatim}

Get the type of a Z-axis.


\section*{\tt \htmlref{zaxisInqUUID}{zaxisInqUUID}}

\begin{verbatim}
    SUBROUTINE zaxisInqUUID (INTEGER zaxisID, INTEGER*1(16) uuid)
\end{verbatim}

Get the UUID of a generalized Z-axis.


\section*{\tt \htmlref{zaxisInqUnits}{zaxisInqUnits}}

\begin{verbatim}
    SUBROUTINE zaxisInqUnits (INTEGER zaxisID, CHARACTER*(*) units)
\end{verbatim}

Get the units of a Z-axis.


