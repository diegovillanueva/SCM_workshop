Attributes may be associated with each variable to specify non
CDI standard properties. CDI standard properties as code, name,
units, and missing value are directly associated with each variable by
the corresponding CDI function (e.g. {\htmlref{\tt vlistDefVarName}{vlistDefVarName}}).
An attribute has a variable to which it is assigned, a name, a type,
a length, and a sequence of one or more values.
The attributes have to be defined after the variable is created and 
before the variable list is associated with a stream.
Attributes are only used for netCDF datasets.

It is also possible to have attributes that are not associated with any variable.
These are called global attributes and are identified by using CDI\_GLOBAL as a 
variable pseudo-ID. Global attributes are usually related to the dataset as a whole.

CDI supports integer, floating point and text attributes. The data types are defined 
by the following predefined constants:

\vspace*{3mm}
\hspace*{8mm}\begin{minipage}{15cm}
\begin{deflist}{{\large\tt DATATYPE\_TXT \ \ }}
\item[{\large\tt DATATYPE\_INT16}]   16-bit integer attribute
\item[{\large\tt DATATYPE\_INT32}]   32-bit integer attribute
\item[{\large\tt DATATYPE\_FLT32}]   32-bit floating point attribute
\item[{\large\tt DATATYPE\_FLT64}]   64-bit floating point attribute
\item[{\large\tt DATATYPE\_TXT}]     Text attribute
\end{deflist}
\end{minipage}
