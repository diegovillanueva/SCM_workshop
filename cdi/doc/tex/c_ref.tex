

\section*{\tt 
\ifpdf
\hyperref[gridCreate]{gridCreate}
\else
gridCreate
\fi
}
\begin{verbatim}
    int gridCreate(int gridtype, int size);
\end{verbatim}

Create a horizontal Grid
\ifpdfoutput{}{(\ref{gridCreate})}


\section*{\tt 
\ifpdf
\hyperref[gridDefNP]{gridDefNP}
\else
gridDefNP
\fi
}
\begin{verbatim}
    void gridDefNP(int gridID, int np);
\end{verbatim}

Define the number of parallels between a pole and the equator
\ifpdfoutput{}{(\ref{gridDefNP})}


\section*{\tt 
\ifpdf
\hyperref[gridDefNumber]{gridDefNumber}
\else
gridDefNumber
\fi
}
\begin{verbatim}
    void gridDefNumber(int gridID, const int number);
\end{verbatim}

Define the reference number for an unstructured grid
\ifpdfoutput{}{(\ref{gridDefNumber})}


\section*{\tt 
\ifpdf
\hyperref[gridDefPosition]{gridDefPosition}
\else
gridDefPosition
\fi
}
\begin{verbatim}
    void gridDefPosition(int gridID, const int position);
\end{verbatim}

Define the position of grid in the reference file
\ifpdfoutput{}{(\ref{gridDefPosition})}


\section*{\tt 
\ifpdf
\hyperref[gridDefReference]{gridDefReference}
\else
gridDefReference
\fi
}
\begin{verbatim}
    void gridDefReference(int gridID, const char *reference);
\end{verbatim}

Define the reference URI for an unstructured grid
\ifpdfoutput{}{(\ref{gridDefReference})}


\section*{\tt 
\ifpdf
\hyperref[gridDefUUID]{gridDefUUID}
\else
gridDefUUID
\fi
}
\begin{verbatim}
    void gridDefUUID(int gridID, const char *uuid);
\end{verbatim}

Define the UUID for an unstructured grid
\ifpdfoutput{}{(\ref{gridDefUUID})}


\section*{\tt 
\ifpdf
\hyperref[gridDefXbounds]{gridDefXbounds}
\else
gridDefXbounds
\fi
}
\begin{verbatim}
    void gridDefXbounds(int gridID, const double *xbounds);
\end{verbatim}

Define the bounds of a X-axis
\ifpdfoutput{}{(\ref{gridDefXbounds})}


\section*{\tt 
\ifpdf
\hyperref[gridDefXlongname]{gridDefXlongname}
\else
gridDefXlongname
\fi
}
\begin{verbatim}
    void gridDefXlongname(int gridID, const char *longname);
\end{verbatim}

Define the longname of a X-axis
\ifpdfoutput{}{(\ref{gridDefXlongname})}


\section*{\tt 
\ifpdf
\hyperref[gridDefXname]{gridDefXname}
\else
gridDefXname
\fi
}
\begin{verbatim}
    void gridDefXname(int gridID, const char *name);
\end{verbatim}

Define the name of a X-axis
\ifpdfoutput{}{(\ref{gridDefXname})}


\section*{\tt 
\ifpdf
\hyperref[gridDefXsize]{gridDefXsize}
\else
gridDefXsize
\fi
}
\begin{verbatim}
    void gridDefXsize(int gridID, int xsize);
\end{verbatim}

Define the number of values of a X-axis
\ifpdfoutput{}{(\ref{gridDefXsize})}


\section*{\tt 
\ifpdf
\hyperref[gridDefXunits]{gridDefXunits}
\else
gridDefXunits
\fi
}
\begin{verbatim}
    void gridDefXunits(int gridID, const char *units);
\end{verbatim}

Define the units of a X-axis
\ifpdfoutput{}{(\ref{gridDefXunits})}


\section*{\tt 
\ifpdf
\hyperref[gridDefXvals]{gridDefXvals}
\else
gridDefXvals
\fi
}
\begin{verbatim}
    void gridDefXvals(int gridID, const double *xvals);
\end{verbatim}

Define the values of a X-axis
\ifpdfoutput{}{(\ref{gridDefXvals})}


\section*{\tt 
\ifpdf
\hyperref[gridDefYbounds]{gridDefYbounds}
\else
gridDefYbounds
\fi
}
\begin{verbatim}
    void gridDefYbounds(int gridID, const double *ybounds);
\end{verbatim}

Define the bounds of a Y-axis
\ifpdfoutput{}{(\ref{gridDefYbounds})}


\section*{\tt 
\ifpdf
\hyperref[gridDefYlongname]{gridDefYlongname}
\else
gridDefYlongname
\fi
}
\begin{verbatim}
    void gridDefYlongname(int gridID, const char *longname);
\end{verbatim}

Define the longname of a Y-axis
\ifpdfoutput{}{(\ref{gridDefYlongname})}


\section*{\tt 
\ifpdf
\hyperref[gridDefYname]{gridDefYname}
\else
gridDefYname
\fi
}
\begin{verbatim}
    void gridDefYname(int gridID, const char *name);
\end{verbatim}

Define the name of a Y-axis
\ifpdfoutput{}{(\ref{gridDefYname})}


\section*{\tt 
\ifpdf
\hyperref[gridDefYsize]{gridDefYsize}
\else
gridDefYsize
\fi
}
\begin{verbatim}
    void gridDefYsize(int gridID, int ysize);
\end{verbatim}

Define the number of values of a Y-axis
\ifpdfoutput{}{(\ref{gridDefYsize})}


\section*{\tt 
\ifpdf
\hyperref[gridDefYunits]{gridDefYunits}
\else
gridDefYunits
\fi
}
\begin{verbatim}
    void gridDefYunits(int gridID, const char *units);
\end{verbatim}

Define the units of a Y-axis
\ifpdfoutput{}{(\ref{gridDefYunits})}


\section*{\tt 
\ifpdf
\hyperref[gridDefYvals]{gridDefYvals}
\else
gridDefYvals
\fi
}
\begin{verbatim}
    void gridDefYvals(int gridID, const double *yvals);
\end{verbatim}

Define the values of a Y-axis
\ifpdfoutput{}{(\ref{gridDefYvals})}


\section*{\tt 
\ifpdf
\hyperref[gridDestroy]{gridDestroy}
\else
gridDestroy
\fi
}
\begin{verbatim}
    void gridDestroy(int gridID);
\end{verbatim}

Destroy a horizontal Grid
\ifpdfoutput{}{(\ref{gridDestroy})}


\section*{\tt 
\ifpdf
\hyperref[gridDuplicate]{gridDuplicate}
\else
gridDuplicate
\fi
}
\begin{verbatim}
    int gridDuplicate(int gridID);
\end{verbatim}

Duplicate a horizontal Grid
\ifpdfoutput{}{(\ref{gridDuplicate})}


\section*{\tt 
\ifpdf
\hyperref[gridInqNP]{gridInqNP}
\else
gridInqNP
\fi
}
\begin{verbatim}
    int gridInqNP(int gridID);
\end{verbatim}

Get the number of parallels between a pole and the equator
\ifpdfoutput{}{(\ref{gridInqNP})}


\section*{\tt 
\ifpdf
\hyperref[gridInqNumber]{gridInqNumber}
\else
gridInqNumber
\fi
}
\begin{verbatim}
    int gridInqNumber(int gridID);
\end{verbatim}

Get the reference number to an unstructured grid
\ifpdfoutput{}{(\ref{gridInqNumber})}


\section*{\tt 
\ifpdf
\hyperref[gridInqPosition]{gridInqPosition}
\else
gridInqPosition
\fi
}
\begin{verbatim}
    int gridInqPosition(int gridID);
\end{verbatim}

Get the position of grid in the reference file
\ifpdfoutput{}{(\ref{gridInqPosition})}


\section*{\tt 
\ifpdf
\hyperref[gridInqReference]{gridInqReference}
\else
gridInqReference
\fi
}
\begin{verbatim}
    char *gridInqReference(int gridID, char *reference);
\end{verbatim}

Get the reference URI to an unstructured grid
\ifpdfoutput{}{(\ref{gridInqReference})}


\section*{\tt 
\ifpdf
\hyperref[gridInqSize]{gridInqSize}
\else
gridInqSize
\fi
}
\begin{verbatim}
    int gridInqSize(int gridID);
\end{verbatim}

Get the size of a Grid
\ifpdfoutput{}{(\ref{gridInqSize})}


\section*{\tt 
\ifpdf
\hyperref[gridInqType]{gridInqType}
\else
gridInqType
\fi
}
\begin{verbatim}
    int gridInqType(int gridID);
\end{verbatim}

Get the type of a Grid
\ifpdfoutput{}{(\ref{gridInqType})}


\section*{\tt 
\ifpdf
\hyperref[gridInqUUID]{gridInqUUID}
\else
gridInqUUID
\fi
}
\begin{verbatim}
    void gridInqUUID(int gridID, char *uuid);
\end{verbatim}

Get the UUID to an unstructured grid
\ifpdfoutput{}{(\ref{gridInqUUID})}


\section*{\tt 
\ifpdf
\hyperref[gridInqXbounds]{gridInqXbounds}
\else
gridInqXbounds
\fi
}
\begin{verbatim}
    int gridInqXbounds(int gridID, double *xbounds);
\end{verbatim}

Get the bounds of a X-axis
\ifpdfoutput{}{(\ref{gridInqXbounds})}


\section*{\tt 
\ifpdf
\hyperref[gridInqXlongname]{gridInqXlongname}
\else
gridInqXlongname
\fi
}
\begin{verbatim}
    void gridInqXlongname(int gridID, char *longname);
\end{verbatim}

Get the longname of a X-axis
\ifpdfoutput{}{(\ref{gridInqXlongname})}


\section*{\tt 
\ifpdf
\hyperref[gridInqXname]{gridInqXname}
\else
gridInqXname
\fi
}
\begin{verbatim}
    void gridInqXname(int gridID, char *name);
\end{verbatim}

Get the name of a X-axis
\ifpdfoutput{}{(\ref{gridInqXname})}


\section*{\tt 
\ifpdf
\hyperref[gridInqXsize]{gridInqXsize}
\else
gridInqXsize
\fi
}
\begin{verbatim}
    int gridInqXsize(int gridID);
\end{verbatim}

Get the number of values of a X-axis
\ifpdfoutput{}{(\ref{gridInqXsize})}


\section*{\tt 
\ifpdf
\hyperref[gridInqXunits]{gridInqXunits}
\else
gridInqXunits
\fi
}
\begin{verbatim}
    void gridInqXunits(int gridID, char *units);
\end{verbatim}

Get the units of a X-axis
\ifpdfoutput{}{(\ref{gridInqXunits})}


\section*{\tt 
\ifpdf
\hyperref[gridInqXvals]{gridInqXvals}
\else
gridInqXvals
\fi
}
\begin{verbatim}
    int gridInqXvals(int gridID, double *xvals);
\end{verbatim}

Get all values of a X-axis
\ifpdfoutput{}{(\ref{gridInqXvals})}


\section*{\tt 
\ifpdf
\hyperref[gridInqYbounds]{gridInqYbounds}
\else
gridInqYbounds
\fi
}
\begin{verbatim}
    int gridInqYbounds(int gridID, double *ybounds);
\end{verbatim}

Get the bounds of a Y-axis
\ifpdfoutput{}{(\ref{gridInqYbounds})}


\section*{\tt 
\ifpdf
\hyperref[gridInqYlongname]{gridInqYlongname}
\else
gridInqYlongname
\fi
}
\begin{verbatim}
    void gridInqXlongname(int gridID, char *longname);
\end{verbatim}

Get the longname of a Y-axis
\ifpdfoutput{}{(\ref{gridInqYlongname})}


\section*{\tt 
\ifpdf
\hyperref[gridInqYname]{gridInqYname}
\else
gridInqYname
\fi
}
\begin{verbatim}
    void gridInqYname(int gridID, char *name);
\end{verbatim}

Get the name of a Y-axis
\ifpdfoutput{}{(\ref{gridInqYname})}


\section*{\tt 
\ifpdf
\hyperref[gridInqYsize]{gridInqYsize}
\else
gridInqYsize
\fi
}
\begin{verbatim}
    int gridInqYsize(int gridID);
\end{verbatim}

Get the number of values of a Y-axis
\ifpdfoutput{}{(\ref{gridInqYsize})}


\section*{\tt 
\ifpdf
\hyperref[gridInqYunits]{gridInqYunits}
\else
gridInqYunits
\fi
}
\begin{verbatim}
    void gridInqYunits(int gridID, char *units);
\end{verbatim}

Get the units of a Y-axis
\ifpdfoutput{}{(\ref{gridInqYunits})}


\section*{\tt 
\ifpdf
\hyperref[gridInqYvals]{gridInqYvals}
\else
gridInqYvals
\fi
}
\begin{verbatim}
    int gridInqYvals(int gridID, double *yvals);
\end{verbatim}

Get all values of a Y-axis
\ifpdfoutput{}{(\ref{gridInqYvals})}


\section*{\tt 
\ifpdf
\hyperref[streamClose]{streamClose}
\else
streamClose
\fi
}
\begin{verbatim}
    void streamClose(int streamID);
\end{verbatim}

Close an open dataset
\ifpdfoutput{}{(\ref{streamClose})}


\section*{\tt 
\ifpdf
\hyperref[streamDefByteorder]{streamDefByteorder}
\else
streamDefByteorder
\fi
}
\begin{verbatim}
    void streamDefByteorder(int streamID, int byteorder);
\end{verbatim}

Define the byte order
\ifpdfoutput{}{(\ref{streamDefByteorder})}


\section*{\tt 
\ifpdf
\hyperref[streamDefRecord]{streamDefRecord}
\else
streamDefRecord
\fi
}
\begin{verbatim}
    void streamDefRecord(int streamID, int varID, int levelID);
\end{verbatim}

Define the next record
\ifpdfoutput{}{(\ref{streamDefRecord})}


\section*{\tt 
\ifpdf
\hyperref[streamDefTimestep]{streamDefTimestep}
\else
streamDefTimestep
\fi
}
\begin{verbatim}
    int streamDefTimestep(int streamID, int tsID);
\end{verbatim}

Define time step
\ifpdfoutput{}{(\ref{streamDefTimestep})}


\section*{\tt 
\ifpdf
\hyperref[streamDefVlist]{streamDefVlist}
\else
streamDefVlist
\fi
}
\begin{verbatim}
    void streamDefVlist(int streamID, int vlistID);
\end{verbatim}

Define the variable list
\ifpdfoutput{}{(\ref{streamDefVlist})}


\section*{\tt 
\ifpdf
\hyperref[streamInqByteorder]{streamInqByteorder}
\else
streamInqByteorder
\fi
}
\begin{verbatim}
    int streamInqByteorder(int streamID);
\end{verbatim}

Get the byte order
\ifpdfoutput{}{(\ref{streamInqByteorder})}


\section*{\tt 
\ifpdf
\hyperref[streamInqFiletype]{streamInqFiletype}
\else
streamInqFiletype
\fi
}
\begin{verbatim}
    int streamInqFiletype(int streamID);
\end{verbatim}

Get the filetype
\ifpdfoutput{}{(\ref{streamInqFiletype})}


\section*{\tt 
\ifpdf
\hyperref[streamInqTimestep]{streamInqTimestep}
\else
streamInqTimestep
\fi
}
\begin{verbatim}
    int streamInqTimestep(int streamID, int tsID);
\end{verbatim}

Get time step
\ifpdfoutput{}{(\ref{streamInqTimestep})}


\section*{\tt 
\ifpdf
\hyperref[streamInqVlist]{streamInqVlist}
\else
streamInqVlist
\fi
}
\begin{verbatim}
    int streamInqVlist(int streamID);
\end{verbatim}

Get the variable list
\ifpdfoutput{}{(\ref{streamInqVlist})}


\section*{\tt 
\ifpdf
\hyperref[streamOpenRead]{streamOpenRead}
\else
streamOpenRead
\fi
}
\begin{verbatim}
    int streamOpenRead(const char *path);
\end{verbatim}

Open a dataset for reading
\ifpdfoutput{}{(\ref{streamOpenRead})}


\section*{\tt 
\ifpdf
\hyperref[streamOpenWrite]{streamOpenWrite}
\else
streamOpenWrite
\fi
}
\begin{verbatim}
    int streamOpenWrite(const char *path, int filetype);
\end{verbatim}

Create a new dataset
\ifpdfoutput{}{(\ref{streamOpenWrite})}


\section*{\tt 
\ifpdf
\hyperref[streamReadVar]{streamReadVar}
\else
streamReadVar
\fi
}
\begin{verbatim}
    void streamReadVar(int streamID, int varID, double *data, int *nmiss);
\end{verbatim}

Read a variable
\ifpdfoutput{}{(\ref{streamReadVar})}


\section*{\tt 
\ifpdf
\hyperref[streamReadVarSlice]{streamReadVarSlice}
\else
streamReadVarSlice
\fi
}
\begin{verbatim}
    void streamReadVarSlice(int streamID, int varID, int levelID, double *data, 
                            int *nmiss);
\end{verbatim}

Read a horizontal slice of a variable
\ifpdfoutput{}{(\ref{streamReadVarSlice})}


\section*{\tt 
\ifpdf
\hyperref[streamWriteRecord]{streamWriteRecord}
\else
streamWriteRecord
\fi
}
\begin{verbatim}
    void streamWriteRecord(int streamID, const double *data, int nmiss);
\end{verbatim}

Write a horizontal slice of a variable
\ifpdfoutput{}{(\ref{streamWriteRecord})}


\section*{\tt 
\ifpdf
\hyperref[streamWriteVar]{streamWriteVar}
\else
streamWriteVar
\fi
}
\begin{verbatim}
    void streamWriteVar(int streamID, int varID, const double *data, int nmiss);
\end{verbatim}

Write a variable
\ifpdfoutput{}{(\ref{streamWriteVar})}


\section*{\tt 
\ifpdf
\hyperref[streamWriteVarF]{streamWriteVarF}
\else
streamWriteVarF
\fi
}
\begin{verbatim}
    void streamWriteVarF(int streamID, int varID, const float *data, int nmiss);
\end{verbatim}

Write a variable
\ifpdfoutput{}{(\ref{streamWriteVarF})}


\section*{\tt 
\ifpdf
\hyperref[streamWriteVarSlice]{streamWriteVarSlice}
\else
streamWriteVarSlice
\fi
}
\begin{verbatim}
    void streamWriteVarSlice(int streamID, int varID, int levelID, const double *data, 
                             int nmiss);
\end{verbatim}

Write a horizontal slice of a variable
\ifpdfoutput{}{(\ref{streamWriteVarSlice})}


\section*{\tt 
\ifpdf
\hyperref[streamWriteVarSliceF]{streamWriteVarSliceF}
\else
streamWriteVarSliceF
\fi
}
\begin{verbatim}
    void streamWriteVarSliceF(int streamID, int varID, int levelID, const float *data, 
                              int nmiss);
\end{verbatim}

Write a horizontal slice of a variable
\ifpdfoutput{}{(\ref{streamWriteVarSliceF})}


\section*{\tt 
\ifpdf
\hyperref[taxisCreate]{taxisCreate}
\else
taxisCreate
\fi
}
\begin{verbatim}
    int taxisCreate(int taxistype);
\end{verbatim}

Create a Time axis
\ifpdfoutput{}{(\ref{taxisCreate})}


\section*{\tt 
\ifpdf
\hyperref[taxisDefCalendar]{taxisDefCalendar}
\else
taxisDefCalendar
\fi
}
\begin{verbatim}
    void taxisDefCalendar(int taxisID, int calendar);
\end{verbatim}

Define the calendar
\ifpdfoutput{}{(\ref{taxisDefCalendar})}


\section*{\tt 
\ifpdf
\hyperref[taxisDefRdate]{taxisDefRdate}
\else
taxisDefRdate
\fi
}
\begin{verbatim}
    void taxisDefRdate(int taxisID, int rdate);
\end{verbatim}

Define the reference date
\ifpdfoutput{}{(\ref{taxisDefRdate})}


\section*{\tt 
\ifpdf
\hyperref[taxisDefRtime]{taxisDefRtime}
\else
taxisDefRtime
\fi
}
\begin{verbatim}
    void taxisDefRtime(int taxisID, int rtime);
\end{verbatim}

Define the reference time
\ifpdfoutput{}{(\ref{taxisDefRtime})}


\section*{\tt 
\ifpdf
\hyperref[taxisDefVdate]{taxisDefVdate}
\else
taxisDefVdate
\fi
}
\begin{verbatim}
    void taxisDefVdate(int taxisID, int vdate);
\end{verbatim}

Define the verification date
\ifpdfoutput{}{(\ref{taxisDefVdate})}


\section*{\tt 
\ifpdf
\hyperref[taxisDefVtime]{taxisDefVtime}
\else
taxisDefVtime
\fi
}
\begin{verbatim}
    void taxisDefVtime(int taxisID, int vtime);
\end{verbatim}

Define the verification time
\ifpdfoutput{}{(\ref{taxisDefVtime})}


\section*{\tt 
\ifpdf
\hyperref[taxisDestroy]{taxisDestroy}
\else
taxisDestroy
\fi
}
\begin{verbatim}
    void taxisDestroy(int taxisID);
\end{verbatim}

Destroy a Time axis
\ifpdfoutput{}{(\ref{taxisDestroy})}


\section*{\tt 
\ifpdf
\hyperref[taxisInqCalendar]{taxisInqCalendar}
\else
taxisInqCalendar
\fi
}
\begin{verbatim}
    int taxisInqCalendar(int taxisID);
\end{verbatim}

Get the calendar
\ifpdfoutput{}{(\ref{taxisInqCalendar})}


\section*{\tt 
\ifpdf
\hyperref[taxisInqRdate]{taxisInqRdate}
\else
taxisInqRdate
\fi
}
\begin{verbatim}
    int taxisInqRdate(int taxisID);
\end{verbatim}

Get the reference date
\ifpdfoutput{}{(\ref{taxisInqRdate})}


\section*{\tt 
\ifpdf
\hyperref[taxisInqRtime]{taxisInqRtime}
\else
taxisInqRtime
\fi
}
\begin{verbatim}
    int taxisInqRtime(int taxisID);
\end{verbatim}

Get the reference time
\ifpdfoutput{}{(\ref{taxisInqRtime})}


\section*{\tt 
\ifpdf
\hyperref[taxisInqVdate]{taxisInqVdate}
\else
taxisInqVdate
\fi
}
\begin{verbatim}
    int taxisInqVdate(int taxisID);
\end{verbatim}

Get the verification date
\ifpdfoutput{}{(\ref{taxisInqVdate})}


\section*{\tt 
\ifpdf
\hyperref[taxisInqVtime]{taxisInqVtime}
\else
taxisInqVtime
\fi
}
\begin{verbatim}
    int taxisInqVtime(int taxisID);
\end{verbatim}

Get the verification time
\ifpdfoutput{}{(\ref{taxisInqVtime})}


\section*{\tt 
\ifpdf
\hyperref[vlistCat]{vlistCat}
\else
vlistCat
\fi
}
\begin{verbatim}
    void vlistCat(int vlistID2, int vlistID1);
\end{verbatim}

Concatenate two variable lists
\ifpdfoutput{}{(\ref{vlistCat})}


\section*{\tt 
\ifpdf
\hyperref[vlistCopy]{vlistCopy}
\else
vlistCopy
\fi
}
\begin{verbatim}
    void vlistCopy(int vlistID2, int vlistID1);
\end{verbatim}

Copy a variable list
\ifpdfoutput{}{(\ref{vlistCopy})}


\section*{\tt 
\ifpdf
\hyperref[vlistCopyFlag]{vlistCopyFlag}
\else
vlistCopyFlag
\fi
}
\begin{verbatim}
    void vlistCopyFlag(int vlistID2, int vlistID1);
\end{verbatim}

Copy some entries of a variable list
\ifpdfoutput{}{(\ref{vlistCopyFlag})}


\section*{\tt 
\ifpdf
\hyperref[vlistCreate]{vlistCreate}
\else
vlistCreate
\fi
}
\begin{verbatim}
    int vlistCreate(void);
\end{verbatim}

Create a variable list
\ifpdfoutput{}{(\ref{vlistCreate})}


\section*{\tt 
\ifpdf
\hyperref[vlistDefAttFlt]{vlistDefAttFlt}
\else
vlistDefAttFlt
\fi
}
\begin{verbatim}
    int vlistDefAttFlt(int vlistID, int varID, const char *name, int type, int len, 
                       const double *dp);
\end{verbatim}

Define a floating point attribute
\ifpdfoutput{}{(\ref{vlistDefAttFlt})}


\section*{\tt 
\ifpdf
\hyperref[vlistDefAttInt]{vlistDefAttInt}
\else
vlistDefAttInt
\fi
}
\begin{verbatim}
    int vlistDefAttInt(int vlistID, int varID, const char *name, int type, int len, 
                       const int *ip);
\end{verbatim}

Define an integer attribute
\ifpdfoutput{}{(\ref{vlistDefAttInt})}


\section*{\tt 
\ifpdf
\hyperref[vlistDefAttTxt]{vlistDefAttTxt}
\else
vlistDefAttTxt
\fi
}
\begin{verbatim}
    int vlistDefAttTxt(int vlistID, int varID, const char *name, int len, const char *tp);
\end{verbatim}

Define a text attribute
\ifpdfoutput{}{(\ref{vlistDefAttTxt})}


\section*{\tt 
\ifpdf
\hyperref[vlistDefTaxis]{vlistDefTaxis}
\else
vlistDefTaxis
\fi
}
\begin{verbatim}
    void vlistDefTaxis(int vlistID, int taxisID);
\end{verbatim}

Define the time axis
\ifpdfoutput{}{(\ref{vlistDefTaxis})}


\section*{\tt 
\ifpdf
\hyperref[vlistDefVar]{vlistDefVar}
\else
vlistDefVar
\fi
}
\begin{verbatim}
    int vlistDefVar(int vlistID, int gridID, int zaxisID, int tsteptype);
\end{verbatim}

Define a Variable
\ifpdfoutput{}{(\ref{vlistDefVar})}


\section*{\tt 
\ifpdf
\hyperref[vlistDefVarCode]{vlistDefVarCode}
\else
vlistDefVarCode
\fi
}
\begin{verbatim}
    void vlistDefVarCode(int vlistID, int varID, int code);
\end{verbatim}

Define the code number of a Variable
\ifpdfoutput{}{(\ref{vlistDefVarCode})}


\section*{\tt 
\ifpdf
\hyperref[vlistDefVarDatatype]{vlistDefVarDatatype}
\else
vlistDefVarDatatype
\fi
}
\begin{verbatim}
    void vlistDefVarDatatype(int vlistID, int varID, int datatype);
\end{verbatim}

Define the data type of a Variable
\ifpdfoutput{}{(\ref{vlistDefVarDatatype})}


\section*{\tt 
\ifpdf
\hyperref[vlistDefVarLongname]{vlistDefVarLongname}
\else
vlistDefVarLongname
\fi
}
\begin{verbatim}
    void vlistDefVarLongname(int vlistID, int varID, const char *longname);
\end{verbatim}

Define the long name of a Variable
\ifpdfoutput{}{(\ref{vlistDefVarLongname})}


\section*{\tt 
\ifpdf
\hyperref[vlistDefVarMissval]{vlistDefVarMissval}
\else
vlistDefVarMissval
\fi
}
\begin{verbatim}
    void vlistDefVarMissval(int vlistID, int varID, double missval);
\end{verbatim}

Define the missing value of a Variable
\ifpdfoutput{}{(\ref{vlistDefVarMissval})}


\section*{\tt 
\ifpdf
\hyperref[vlistDefVarName]{vlistDefVarName}
\else
vlistDefVarName
\fi
}
\begin{verbatim}
    void vlistDefVarName(int vlistID, int varID, const char *name);
\end{verbatim}

Define the name of a Variable
\ifpdfoutput{}{(\ref{vlistDefVarName})}


\section*{\tt 
\ifpdf
\hyperref[vlistDefVarStdname]{vlistDefVarStdname}
\else
vlistDefVarStdname
\fi
}
\begin{verbatim}
    void vlistDefVarStdname(int vlistID, int varID, const char *stdname);
\end{verbatim}

Define the standard name of a Variable
\ifpdfoutput{}{(\ref{vlistDefVarStdname})}


\section*{\tt 
\ifpdf
\hyperref[vlistDefVarUnits]{vlistDefVarUnits}
\else
vlistDefVarUnits
\fi
}
\begin{verbatim}
    void vlistDefVarUnits(int vlistID, int varID, const char *units);
\end{verbatim}

Define the units of a Variable
\ifpdfoutput{}{(\ref{vlistDefVarUnits})}


\section*{\tt 
\ifpdf
\hyperref[vlistDestroy]{vlistDestroy}
\else
vlistDestroy
\fi
}
\begin{verbatim}
    void vlistDestroy(int vlistID);
\end{verbatim}

Destroy a variable list
\ifpdfoutput{}{(\ref{vlistDestroy})}


\section*{\tt 
\ifpdf
\hyperref[vlistDuplicate]{vlistDuplicate}
\else
vlistDuplicate
\fi
}
\begin{verbatim}
    int vlistDuplicate(int vlistID);
\end{verbatim}

Duplicate a variable list
\ifpdfoutput{}{(\ref{vlistDuplicate})}


\section*{\tt 
\ifpdf
\hyperref[vlistInqAtt]{vlistInqAtt}
\else
vlistInqAtt
\fi
}
\begin{verbatim}
    int vlistInqAtt(int vlistID, int varID, int attnum, char *name, int *typep, int *lenp);
\end{verbatim}

Get information about an attribute
\ifpdfoutput{}{(\ref{vlistInqAtt})}


\section*{\tt 
\ifpdf
\hyperref[vlistInqAttFlt]{vlistInqAttFlt}
\else
vlistInqAttFlt
\fi
}
\begin{verbatim}
    int vlistInqAttFlt(int vlistID, int varID, const char *name, int mlen, double *dp);
\end{verbatim}

Get the value(s) of a floating point attribute
\ifpdfoutput{}{(\ref{vlistInqAttFlt})}


\section*{\tt 
\ifpdf
\hyperref[vlistInqAttInt]{vlistInqAttInt}
\else
vlistInqAttInt
\fi
}
\begin{verbatim}
    int vlistInqAttInt(int vlistID, int varID, const char *name, int mlen, int *ip);
\end{verbatim}

Get the value(s) of an integer attribute
\ifpdfoutput{}{(\ref{vlistInqAttInt})}


\section*{\tt 
\ifpdf
\hyperref[vlistInqAttTxt]{vlistInqAttTxt}
\else
vlistInqAttTxt
\fi
}
\begin{verbatim}
    int vlistInqAttTxt(int vlistID, int varID, const char *name, int mlen, char *tp);
\end{verbatim}

Get the value(s) of a text attribute
\ifpdfoutput{}{(\ref{vlistInqAttTxt})}


\section*{\tt 
\ifpdf
\hyperref[vlistInqNatts]{vlistInqNatts}
\else
vlistInqNatts
\fi
}
\begin{verbatim}
    int vlistInqNatts(int vlistID, int varID, int *nattsp);
\end{verbatim}

Get number of variable attributes
\ifpdfoutput{}{(\ref{vlistInqNatts})}


\section*{\tt 
\ifpdf
\hyperref[vlistInqTaxis]{vlistInqTaxis}
\else
vlistInqTaxis
\fi
}
\begin{verbatim}
    int vlistInqTaxis(int vlistID);
\end{verbatim}

Get the time axis
\ifpdfoutput{}{(\ref{vlistInqTaxis})}


\section*{\tt 
\ifpdf
\hyperref[vlistInqVarCode]{vlistInqVarCode}
\else
vlistInqVarCode
\fi
}
\begin{verbatim}
    int vlistInqVarCode(int vlistID, int varID);
\end{verbatim}

Get the Code number of a Variable
\ifpdfoutput{}{(\ref{vlistInqVarCode})}


\section*{\tt 
\ifpdf
\hyperref[vlistInqVarDatatype]{vlistInqVarDatatype}
\else
vlistInqVarDatatype
\fi
}
\begin{verbatim}
    int vlistInqVarDatatype(int vlistID, int varID);
\end{verbatim}

Get the data type of a Variable
\ifpdfoutput{}{(\ref{vlistInqVarDatatype})}


\section*{\tt 
\ifpdf
\hyperref[vlistInqVarGrid]{vlistInqVarGrid}
\else
vlistInqVarGrid
\fi
}
\begin{verbatim}
    int vlistInqVarGrid(int vlistID, int varID);
\end{verbatim}

Get the Grid ID of a Variable
\ifpdfoutput{}{(\ref{vlistInqVarGrid})}


\section*{\tt 
\ifpdf
\hyperref[vlistInqVarLongname]{vlistInqVarLongname}
\else
vlistInqVarLongname
\fi
}
\begin{verbatim}
    void vlistInqVarLongname(int vlistID, int varID, char *longname);
\end{verbatim}

Get the longname of a Variable
\ifpdfoutput{}{(\ref{vlistInqVarLongname})}


\section*{\tt 
\ifpdf
\hyperref[vlistInqVarMissval]{vlistInqVarMissval}
\else
vlistInqVarMissval
\fi
}
\begin{verbatim}
    double vlistInqVarMissval(int vlistID, int varID);
\end{verbatim}

Get the missing value of a Variable
\ifpdfoutput{}{(\ref{vlistInqVarMissval})}


\section*{\tt 
\ifpdf
\hyperref[vlistInqVarName]{vlistInqVarName}
\else
vlistInqVarName
\fi
}
\begin{verbatim}
    void vlistInqVarName(int vlistID, int varID, char *name);
\end{verbatim}

Get the name of a Variable
\ifpdfoutput{}{(\ref{vlistInqVarName})}


\section*{\tt 
\ifpdf
\hyperref[vlistInqVarStdname]{vlistInqVarStdname}
\else
vlistInqVarStdname
\fi
}
\begin{verbatim}
    void vlistInqVarStdname(int vlistID, int varID, char *stdname);
\end{verbatim}

Get the standard name of a Variable
\ifpdfoutput{}{(\ref{vlistInqVarStdname})}


\section*{\tt 
\ifpdf
\hyperref[vlistInqVarUnits]{vlistInqVarUnits}
\else
vlistInqVarUnits
\fi
}
\begin{verbatim}
    void vlistInqVarUnits(int vlistID, int varID, char *units);
\end{verbatim}

Get the units of a Variable
\ifpdfoutput{}{(\ref{vlistInqVarUnits})}


\section*{\tt 
\ifpdf
\hyperref[vlistInqVarZaxis]{vlistInqVarZaxis}
\else
vlistInqVarZaxis
\fi
}
\begin{verbatim}
    int vlistInqVarZaxis(int vlistID, int varID);
\end{verbatim}

Get the Zaxis ID of a Variable
\ifpdfoutput{}{(\ref{vlistInqVarZaxis})}


\section*{\tt 
\ifpdf
\hyperref[vlistNgrids]{vlistNgrids}
\else
vlistNgrids
\fi
}
\begin{verbatim}
    int vlistNgrids(int vlistID);
\end{verbatim}

Number of grids in a variable list
\ifpdfoutput{}{(\ref{vlistNgrids})}


\section*{\tt 
\ifpdf
\hyperref[vlistNvars]{vlistNvars}
\else
vlistNvars
\fi
}
\begin{verbatim}
    int vlistNvars(int vlistID);
\end{verbatim}

Number of variables in a variable list
\ifpdfoutput{}{(\ref{vlistNvars})}


\section*{\tt 
\ifpdf
\hyperref[vlistNzaxis]{vlistNzaxis}
\else
vlistNzaxis
\fi
}
\begin{verbatim}
    int vlistNzaxis(int vlistID);
\end{verbatim}

Number of zaxis in a variable list
\ifpdfoutput{}{(\ref{vlistNzaxis})}


\section*{\tt 
\ifpdf
\hyperref[zaxisCreate]{zaxisCreate}
\else
zaxisCreate
\fi
}
\begin{verbatim}
    int zaxisCreate(int zaxistype, int size);
\end{verbatim}

Create a vertical Z-axis
\ifpdfoutput{}{(\ref{zaxisCreate})}


\section*{\tt 
\ifpdf
\hyperref[zaxisDefLevels]{zaxisDefLevels}
\else
zaxisDefLevels
\fi
}
\begin{verbatim}
    void zaxisDefLevels(int zaxisID, const double *levels);
\end{verbatim}

Define the levels of a Z-axis
\ifpdfoutput{}{(\ref{zaxisDefLevels})}


\section*{\tt 
\ifpdf
\hyperref[zaxisDefLongname]{zaxisDefLongname}
\else
zaxisDefLongname
\fi
}
\begin{verbatim}
    void zaxisDefLongname(int zaxisID, const char *longname);
\end{verbatim}

Define the longname of a Z-axis
\ifpdfoutput{}{(\ref{zaxisDefLongname})}


\section*{\tt 
\ifpdf
\hyperref[zaxisDefName]{zaxisDefName}
\else
zaxisDefName
\fi
}
\begin{verbatim}
    void zaxisDefName(int zaxisID, const char *name);
\end{verbatim}

Define the name of a Z-axis
\ifpdfoutput{}{(\ref{zaxisDefName})}


\section*{\tt 
\ifpdf
\hyperref[zaxisDefUnits]{zaxisDefUnits}
\else
zaxisDefUnits
\fi
}
\begin{verbatim}
    void zaxisDefUnits(int zaxisID, const char *units);
\end{verbatim}

Define the units of a Z-axis
\ifpdfoutput{}{(\ref{zaxisDefUnits})}


\section*{\tt 
\ifpdf
\hyperref[zaxisDestroy]{zaxisDestroy}
\else
zaxisDestroy
\fi
}
\begin{verbatim}
    void zaxisDestroy(int zaxisID);
\end{verbatim}

Destroy a vertical Z-axis
\ifpdfoutput{}{(\ref{zaxisDestroy})}


\section*{\tt 
\ifpdf
\hyperref[zaxisInqLevel]{zaxisInqLevel}
\else
zaxisInqLevel
\fi
}
\begin{verbatim}
    double zaxisInqLevel(int zaxisID, int levelID);
\end{verbatim}

Get one level of a Z-axis
\ifpdfoutput{}{(\ref{zaxisInqLevel})}


\section*{\tt 
\ifpdf
\hyperref[zaxisInqLevels]{zaxisInqLevels}
\else
zaxisInqLevels
\fi
}
\begin{verbatim}
    void zaxisInqLevels(int zaxisID, double *levels);
\end{verbatim}

Get all levels of a Z-axis
\ifpdfoutput{}{(\ref{zaxisInqLevels})}


\section*{\tt 
\ifpdf
\hyperref[zaxisInqLongname]{zaxisInqLongname}
\else
zaxisInqLongname
\fi
}
\begin{verbatim}
    void zaxisInqLongname(int zaxisID, char *longname);
\end{verbatim}

Get the longname of a Z-axis
\ifpdfoutput{}{(\ref{zaxisInqLongname})}


\section*{\tt 
\ifpdf
\hyperref[zaxisInqName]{zaxisInqName}
\else
zaxisInqName
\fi
}
\begin{verbatim}
    void zaxisInqName(int zaxisID, char *name);
\end{verbatim}

Get the name of a Z-axis
\ifpdfoutput{}{(\ref{zaxisInqName})}


\section*{\tt 
\ifpdf
\hyperref[zaxisInqSize]{zaxisInqSize}
\else
zaxisInqSize
\fi
}
\begin{verbatim}
    int zaxisInqSize(int zaxisID);
\end{verbatim}

Get the size of a Z-axis
\ifpdfoutput{}{(\ref{zaxisInqSize})}


\section*{\tt 
\ifpdf
\hyperref[zaxisInqType]{zaxisInqType}
\else
zaxisInqType
\fi
}
\begin{verbatim}
    int zaxisInqType(int zaxisID);
\end{verbatim}

Get the type of a Z-axis
\ifpdfoutput{}{(\ref{zaxisInqType})}


\section*{\tt 
\ifpdf
\hyperref[zaxisInqUnits]{zaxisInqUnits}
\else
zaxisInqUnits
\fi
}
\begin{verbatim}
    void zaxisInqUnits(int zaxisID, char *units);
\end{verbatim}

Get the units of a Z-axis
\ifpdfoutput{}{(\ref{zaxisInqUnits})}
