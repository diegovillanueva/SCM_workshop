\section{Aerosols}

Aerosols affect the distribution of radiative energy in the atmosphere
by scattering and absorption of electromagnetic radiation. This is the
direct aerosol effect. Aerosols also modify cloud optical properties
and influence the cloud formation processes and precipitation. Since
this also acts on radiation, but the aerosol particles are involved by
the intermediate of clouds, this is called the indirect aerosol
effect. The radiative transfer calculation of
\echam{} needs time dependent 3--d fields of (1) the extinction $\zeta$
by aerosols, (2) the single scattering albedo $\omega$ (SSA), and (3) the
asymmetry factor $g$ for a detailed consideration of the direct
aerosol effect. These quantities depend on the wavelength of the
electromagnetic radiation and have to be provided for each of the
14~solar radiation bands and the 16~thermal radiation bands used in
\echam. In the thermal radiation range, the knowledge of $\zeta$ and
$\omega$ is
sufficient, since the radiative transfer calculation does not account
for scattering in this wavelength regime.


\subsection{Tropospheric aerosols}\label{sec.tropaerosols}

In this section, we describe the generation of a data set of optical
properties of tropospheric aerosols for the historic period of
1850 to 2000
and for three different future 
emission scenarios (\cite{mos107}) until 2100. 
Tropospheric aerosol is partly anthropogenic and highly diverse in
concentration, size, 
and composition.

We combine complete and consistent background
maps of monthly mean aerosol optical properties from global model
results with high 
quality monthly averages from ground based remote sensing. All
equations are valid for monthly means. We
distinguish between 
fine mode aerosols with 
particle radii smaller than $\rm 0.5\,\mu m$ and coarse mode aerosols
with radii equal or larger than $\rm 0.5\,\mu m$. Coarse mode aerosols
are assumed to be of natural origin comprising dust and sea salt. Fine
mode aerosols consist of sulfate and organic matter including black
carbon.
The general procedure consists of 7~steps: (1) Establishing 2--d maps
of column properties of the aerosol optical depth (AOD), single
scattering albedo, and the 
{\AA}ngstr\"om parameter at 550~nm for the year~2000, (2)~separating the
AOD into one due to fine and one due to coarse mode aerosols for the year~2000,
(3)~defining the microphysical properties of the coarse mode,
(4)~spreading the 
optical properties to all wavelengths, (5)~defining altitude
profiles, (6)~establishing an anthropogenic (fine mode) AOD for the
year~2000, (7)~extending the anthropogenic AOD
back to the year 1850 and forward to the year~2100 using
selected future emission scenarios. 

\subsubsection{(1) 2--d maps
of column aerosol optical properties for the year~2000}

In a first step, monthly mean values of the total aerosol optical
depth $\tau$ of each column together with one value of the single
scatter albedo $\omega$, and {\AA}ngstr\"om parameter $\alpha$ for
that column were 
determined for light of a wavelength of $\rm 550\,nm$. The background
data are based on maps of medians
of an ensemble of up to
15~different global models all applying a complex aerosol module.
All models simulated the respective quantities for present
day conditions in the frame work of the AeroCom project
(\cite{kin065}). The resulting maps were blended by the use of AERONET 
ground based sun--/sky--photometer data (\cite{hol981}): We first
associated a ``factor of influence'' around each AERONET site being
equal to one at the exact location of the site and tending to zero
with increasing distance from the site. The rate of decrease depends
on how well the measurements of this site repesent the aerosol optical
properties of its
surroundings. The original medians were then multiplied by the factor
of influence times the ratio of the AERONET measurement value and the
original median. Thus, we obtain a map of $\tau$, $\omega$, $\alpha$
with AERONET--similar values near the AERONET sites and median values of the
model ensemble at remote places. 
The above method provides the monthly mean optical properties on a
$1^\circ\times1^\circ$ longitude--latitude grid for all
aerosols. 

\subsubsection{(2) Splitting the
AOD into fine and coarse mode contributions for the year~2000}

In order to split the aerosol optical depth into a part for
fine and coarse mode, we use the {\AA}ngstr\"om parameter. It is
assumed that the {\AA}ngstr\"om parameter is zero for the coarse mode
and that it varies between 1.6 and 2.2 for the fine mode depending on
humidity. Details of this procedure are described by~\cite{kinxxx}.

\subsubsection{(3) Definition of the coarse mode microphysical properties}

In order to derive the coarse mode aerosol optical properties 
at 550~nm, the respective composition and particle size has to be known.
It is assumed that the coarse mode aerosols are ten times less absorbing than
the fine mode aerosols. The initial guess of the coarse mode single
scattering albedo $\omega_{\rm c,0}$ depends on the total aerosol
single scattering albedo $\omega$ and the ratio of the aerosol optical
depth of the fine mode $\tau_{\rm f}$ and the total aerosol optical
depth $\tau$:

\begin{displaymath}
\omega_{\rm c,0}=1-\frac{1-\omega}{1+9\frac{\tau_{\rm f}}{\tau}}
\end{displaymath}

This choice determines the mixing ratio of dust and sea salt aerosols
in the coarse mode. For latitudes between 35$^\circ$~S and
45$^\circ$~N, the fraction of dust was assumed to increase with coarse mode
aerosol optical depth.
In the initial guess, it is assumed that
dust particles have an effective radius of $1.5\,\mu m$
and sea salt particles have an effective radius of
$2.5\,\mu m$. 
However, the size of the dust particles can be set to larger values
in order to avoid unrealistic small values for the single scattering
albedo of the fine mode aerosols. In such cases, the single scattering
albedo of the fine mode is kept at values given by: 

\begin{equation}\label{eqssalimit}
\omega_{\rm f,0}=1-\left(\frac{1}{10}{\rm
  e}^{-3\tau}+\frac{1}{4}\frac{\tau_{\rm f}}{\tau}\right).
\end{equation}

\begin{table}[hp]
\caption{Wavelengths of the 14~bands in the short wavelength range and
  the 16~bands in the long wavelength range as they are used in the
  radiation calculation of \echam{} and the refractive
  indices of sulfate, dust (\cite{sok983}), and sea salt
  (\cite{nil797})}\label{tabwavelengths}  
\begin{tabular*}{\textwidth}{c@{\extracolsep\fill}ccc}\hline
$\lambda_{\rm v}/{\rm nm}$ & sulfate & dust & sea salt \\\hline
\multicolumn{4}{c}{solar radiation}\\\hline
\cw{00}200 --   \cw{000}263 &$1.450+i1.0\times10^{-9}$&$1.450+i0.025$
&$1.510+i1\times10^{-5}$\\
\cw{00}263 --   \cw{000}345 &$1.450+i1.0\times10^{-9}$&$1.450+i0.020$
&$1.510+i1\times10^{-6}$\\
\cw{00}345 --   \cw{000}442 &$1.445+i1.0\times10^{-9}$&$1.450+i0.0025$
&$1.500+i2\times10^{-8}$\\
\cw{00}442 --   \cw{000}625 &$1.432+i1.0\times10^{-9}$&$1.450+i0.001$
&$1.490+i1\times10^{-8}$\\
\cw{00}625 --   \cw{000}778&$1.427+i5.2\times10^{-8}$&$1.450+i0.00095$
&$1.480+i1\times10^{-7}$\\
\cw{00}778 --  \cw{00}1242&$1.422+i1.3\times10^{-6}$&$1.450+i0.00075$
&$1.470+i1\times10^{-4}$\\
\cw{0}1242 --  \cw{00}1299&$1.413+i7.9\times10^{-6}$&$1.450+i0.00060$
&$1.470+i3.3\times10^{-4}$\\
\cw{0}1299 --  \cw{00}1626&$1.406+i9.0\times10^{-5}$&$1.450+i0.00080$
&$1.460+i5.5\times10^{-4}$\\
\cw{0}1626 --  \cw{00}1942  &$1.393+i5.1\times10^{-4}$&$1.450+i0.0010$
&$1.450+i1\times10^{-3}$\\
\cw{0}1942 --  \cw{00}2151  &$1.382+i1.3\times10^{-3}$&$1.450+i0.0015$
&$1.450+i1.5\times10^{-3}$\\
\cw{0}2151 --  \cw{00}2500  &$1.364+i2.1\times10^{-3}$&$1.460+i0.0025$
&$1.440+i2.5\times10^{-2}$\\
\cw{0}2500 --  \cw{00}3077  &$1.295+i5.5\times10^{-2}$&$1.460+i0.0060$
&$1.400+i8\times10^{-3}$\\
\cw{0}3077 --  \cw{00}3846  &$1.361+i1.4\times10^{-1}$&$1.460+i0.0118$
&$1.480+i1.3\times10^{-2}$\\
\cw{0}3846 -- \cw{0}12195   &$1.400+i2.6\times10^{-1}$&$1.170+i0.10$
&$1.400+i1.4\times10^{-2}$\\\hline
\multicolumn{4}{c}{thermal radiation}\\\hline
\cw{0}3078 --  \cw{000}3846 &$1.380+i1.5\times10^{-1}$ &$1.468+i0.011$
&$1.480+i0.00156$\\ 
\cw{0}3846 --  \cw{000}4202 &$1.397+i1.3\times10^{-1}$&$1.480+i0.0044$
&$1.478+i0.00175$\\
\cw{0}4202 --  \cw{000}4444 &$1.396+i1.2\times10^{-1}$&$1.487+i0.0053$
&$1.488+i0.00246$ \\
\cw{0}4444 --  \cw{000}4808 &$1.385+i1.2\times10^{-1}$&$1.502+i0.0092$
&$1.483+i0.00251$ \\
\cw{0}4808 --  \cw{000}5556 &$1.348+i1.5\times10^{-1}$&$1.525+i0.0228$
&$1.459+i0.00288$ \\
\cw{0}5556 --  \cw{000}6757 &$1.385+i1.7\times10^{-1}$ &$1.423+i0.054$
&$1.505+i0.0180$ \\
\cw{0}6757 --  \cw{000}7194 &$1.277+i1.5\times10^{-1}$&$1.439+i0.0976$
&$1.450+i0.00543$ \\
\cw{0}7194 --  \cw{000}8474 &$1.180+i4.5\times10^{-1}$ &$1.248+i0.105$
&$1.401+i0.0138$ \\
\cw{0}8474 -- \cw{000}9259 &$1.588+i6.7\times10^{-1}$  &$1.613+i0.439$
&$1.638+i0.0293$ \\
\cw{0}9259 -- \cw{00}10204 &$1.777+i6.0\times10^{-1}$  &$2.739+i0.783$
&$1.563+i0.0179$ \\
10204 -- \cw{00}12195 &$1.799+i3.3\times10^{-1}$ &$1.816+i0.299$
&$1.485+i0.0140$ \\
12195 -- \cw{00}14286 &$1.724+i1.6\times10^{-1}$ &$1.697+i0.189$
&$1.408+i0.0192$ \\
14286 -- \cw{00}15873 &$1.601+i1.9\times10^{-1}$ &$1.518+i0.231$
&$1.447+i0.0344$ \\
15873 -- \cw{00}20000 &$1.758+i4.0\times10^{-1}$ &$1.865+i0.546$
&$1.763+i0.111$ \\
20000 -- \cw{00}28571 &$1.850+i2.7\times10^{-1}$ &$2.552+i0.741$
&$1.754+i0.250$ \\
28571 -- 1000000 &$1.850+i2.7\times10^{-1}$ &$2.552+i0.741$
&$1.628+i0.997$ \\
\hline
\end{tabular*}
\end{table} 

\subsubsection{(4) Spreading the 
optical properties to all wavelengths}

The coarse mode aerosol optical properties for all wavelengths
are then determined by Mie scattering calculations from the knowledge of
the refractive indices (see Tab.~\ref{tabwavelengths}), the size, and the 
composition of the coarse mode. Thus, the fine mode aerosol optical
properties at~$\rm 550\,nm$ is also determined.

The aerosol optical properties of the fine mode aerosols are required
at solar wavelengths only since
the interaction of these small particles with light in the
thermal spectral range is negligible. In the solar spectral range, the
aerosol optical depth is given by

\begin{displaymath}
\tau_{\rm f}(\lambda_i)=\tau_{\rm f}({\rm
  550\,nm})\left(\frac{\lambda_i}{\rm 550\,nm}\right)^{\rm \alpha_f},\quad
  i=1,\dots,14 
\end{displaymath}
for the 14~solar spectral bands. The single scattering albedo at $\rm
550\,nm$ is used for shorter wavelengths and reduced towards longer
wavelengths. 
The wavelength dependent asymmetry
factor $g_{\rm f}$ of the fine mode is parametrized as function of the
{\AA}ngstr\"om parameter $\alpha$ and solar wavelength $\lambda$
by

\begin{displaymath}
g_{\rm f}(\lambda)={\rm max}\left\{0.72-0.14\alpha_{\rm
    f}\sqrt{\frac{\lambda}{\lambda_0}-\frac{1}{4}},0.1\right\},\quad
\lambda_0=1\,{\rm \mu m},\quad \rm 0.25\,\mu m
\le \lambda \le 3\,\mu m 
\end{displaymath}

The aerosol optical properties of the year~2000 are now defined.
These data serve as a basis for temporal extension.

\subsubsection{(5) Definition of altitude profiles}

To obtain an altitude profile of the aerosol optical depth,
data from global model studies with ECHAM5--HAM
were adopted. The single scattering and the asymmetry factor of fine
and coarse mode do not change with altitude.
Since the model distinguishes between fine mode and
coarse mode aerosol, local monthly altitude distributions were separately
described for fine mode aerosol and for coarse mode aerosol.

\subsubsection{(6) Establishing an anthropogenic (fine mode) AOD for the
year~2000}

We assume that all anthropogenic aerosols belong to the fine mode
aerosols. The Laboratoire de
M\'et\'eorologie Dynamique (LMD) model simulated the aerosol optical
properties for the year 2000 and the pre--industrial period (\cite{bou028}).
It is assumed that the emissions are all of natural origin for the
pre--industrial period. The pre--industrial fine mode
aerosol optical depths $\tau_{\rm f,pre}^{\rm (LMD)}$ are everywhere lower than
those for the 
fine mode of the year 2000 
$\tau_{\rm f,2000}^{\rm (LMD)}$.
The anthropogenic
aerosol optical depth of the year~2000 $\tau_{\rm a,2000}$ is then
derived from the fine mode aerosol optical depth of the year~2000
$\tau_{\rm a,2000}$:

\begin{displaymath}
\tau_{\rm a,2000}:=\tau_{\rm
  f,2000}\times f_{\rm a,2000},\quad\text{with}\quad f_{\rm
  a,2000}:=\frac{\tau_{\rm f,2000}^{\rm(LMD)}-\tau_{\rm f,pre}^{\rm(LMD)}}{\tau_{\rm 
    f,2000}^{\rm (LMD)}}
\end{displaymath}

We assume that the natural aerosols do not change over time, neither
in concentration nor size or composition. The composition of the anthropognic
aerosols is also kept constant in time but its concentration
changes. 

\subsubsection{(7) Extending the anthropogenic AOD
back and forward in time}


The only quantity that is allowed to vary with time is the
anthropogenic aerosol optical depth. This means that pre--industrial
fine mode and coarse mode aerosol optical properties are constant in
time. The altitude profile associated with the aerosol optical depth
of coarse and fine mode are also kept constant in time. This implies that
we assume the same composition and altitude distribution of
pre--industrial fine mode aerosol and fine mode aerosol for the year
2000. Nevertheless, note that all aerosol optical properties can
change their altitude profile with time because of the weighted mean
values over the various aerosol types that will be discussed in
the section about the implementation. 


\paragraph{Historic}
The contribution of the anthropogenic aerosols to the
aerosol optical depth due to fine mode aerosols is estimated in the
following way: First, an ECHAM5--HAM (\cite{sti055}) hindcast simulation using
National Institute for Environmental Studies (NIES) emissions was
performed for the years~1850--2000. 
The output of the fine mode aerosol optical depth was
interpolated to a $1^\circ\times1^\circ$ longitude--latitude grid and
monthly 10--year means were calculated. From these, anthropogenic
fractions for the year $j$ can be determined by

\begin{displaymath}
f_{{\rm a},j}^{\rm (HAM)}=\frac{\tau_{{\rm f},j}^{\rm (HAM)} -
  \tau_{\rm f,pre}^{\rm (HAM)}}{\tau_{{\rm f},2000}^{\rm (HAM)}-\tau_{\rm
    f,pre}^{\rm (HAM)}}.  
\end{displaymath}

The ratios were then linearly interpolated to all years 1860 to 2000.
Then, the anthropogenic contribution to the aerosol optical depth 
$\tau_{\rm a,2000}$
was multiplied with this ratio resulting in the anthropogenic part of
the 
aerosol optical depth for year $j$:

\begin{displaymath}
\tau_{{\rm a},j}=\tau_{\rm a,2000}\times f_{{\rm a},j}^{\rm (HAM)}.
\end{displaymath}

\subsubsection{Scenarios}

%--- K. Zhang, S. Rast, H. Schmidt

Anthropogenic aerosols have been predicted to be 
an important forcing for the future climate.
For an ideal climate projection, one would utilize a fully coupled
Earth system model  
with online aerosol computations, 
but such experiments are computationally very expensive.
An alternative could be to use an atmosphere--only model  
with a detailed aerosol submodel, to perform simulations with prescribed
emission scenarios  
through the projection period (with sea surface temperatures
from coupled model runs without interactive aerosols), 
to store the simulated aerosol properties, and to use them as a forcing
for the Earth system model. In this case, aerosol model simulations would 
need to be performed for each scenario, which would still be expensive. 
As a flexible and inexpensive alternative, 
we generate future scenarios of aerosol radiative properties 
using the Kinne aerosol climatology of the year 2000, 
the CMIP5 emission scenarios (RCP2.6, RCP4.5,
      and RCP8.5, see the description by~\cite{mos107}), and ECHAM5--HAM
(\cite{sti055}) model 
simulations. 
This approach is based on the following assumptions: 

\begin{enumerate}
\item Vertical profiles of the aerosol optical depth will not change
  significantly in the future. 
\item Changes in the vertically integrated aerosol optical depth (column AOD)
      are approximately linear functions of the changes in emissions.
\item When forced by fixed emissions (with seasonal cycle but without 
      interannual variability or trend), the 10--year average of
      column AOD simulated by ECHAM5--HAM is a good representation of 
      the steady state.
\item Contributions of anthropogenic emissions to the total AOD from
  the 10 regions  
      identified by the CMIP5 emission scenarios
      are largely independent and do not interact with each other strongly.
\end{enumerate}

The first step for constructing the AOD scenarios is to build a
``database'' that quantifies the impact of a change in the emissions in
one of the 10~regions on the global column AOD.
For each of the 10~regions we performed a 10--year 
simulation using the aerosol climate model ECHAM5--HAM but with 
the regional emissions reduced to 50\% of the year 2000 level.
For this purpose, we used ECHAM5--HAM
in the T42L19 resolution. As boundary conditions of the atmosphere, 
the climatological sea surface temperature and sea ice concentration
are used. We only consider the aerosol optical depth caused by fine
mode aerosols since fine mode aerosols are dominated
by the anthropogenic aerosols whereas coarse mode aerosols are of
mostly natural origin. The AOD of the fine mode aerosols is not
affected by the statistical variability of coarse mode
particles. Thus, we reduce the statistical noise in our results.
The resulting 10--year mean column fine mode AOD is then compared with 
a control experiment with the year 2000 emissions.
The grid points at which the fine mode AOD changes are statistically
significant  
are identified, and a map of differences $\Delta \tau^{\rm (f)}_l$ of the
fine mode AOD under a 50\% emission reduction in
region $l$ ($\tau_{l,50\%}^{\rm (f)}$) and the reference AOD of the year 2000
($\tau_{2000}^{\rm (f)}$) for each 
region $l=1,\dots,10$ 
is stored:

\begin{displaymath}
\Delta\tau_l^{\rm (f)}=\tau_{l,50\%}^{\rm (f)}-\tau_{2000}^{\rm (f)}
\end{displaymath}

Only the sulfur emissions were reduced in the simulations,  
because sensitivity simulations indicated that a 
change of carbonaceous aerosol emissions does not contribute to 
the AOD change as significantly as changes in sulfur emissions do. 

In the second step, for any prescribed global emission at any time
instance $j$
we compute the difference between each regional mean emission
$q_{l,j}$ of region $l$ and the corresponding 
year 2000 value $q_{l,2000}$, and then the ratio $\xi_{l,j}$ between this
difference and the difference between
50\% regional mean of the year 2000 given by $q_{l,50\%}$ and the
corresponding emission for the year~2000: 

\begin{displaymath}
\xi_{l,j}:=\frac{q_{l,j}-q_{l,2000}}{q_{l,50\%}-q_{l,2000}}. 
\end{displaymath}

The global fine mode AOD resulting from the emissions $q_{l,j}$ is
then calculated by superposition:

\begin{displaymath}
\tau_{\rm j}^{\rm (f)}={\rm
  max}\left\{\sum_{l=1}^{10}\xi_{l,j}\Delta\tau_l^{\rm
    (f)}+\tau_{2000}^{\rm (f)},\tau_{\rm pre}^{\rm (f)}\right\}, 
\end{displaymath}

where $\tau_{\rm pre}^{\rm (f)}$ is the pre--industrial fine mode
aerosol optical depth. It serves as a lower limit of the fine mode
aerosol optical depth. 
Fine mode AOD data from the year 2000 to 2100 have been created by
using this method. These data can then be used to define a scaling
factor with respect to the AOD of the year~2000.

The main advantage of this method is that one can easily estimate future change 
of aerosol radiative properties for any given emission scenario without 
having to repeat the simulation with the global aerosol model, 
as long as the assumptions listed above are valid.
On the other hand, the method also has some limitations.  
For example, at a certain gridbox the reponse of the fine mode AOD
to emission changes might be non--linear, so the assumption mentioned above    
may cause some error. Furthermore, in both the historical and scenario aerosol 
climatology, aerosol compositions are assumed to be fixed. This will
also bring some 
error into the estimation. 

Nevertheless, it is believed that in terms of global patterns and
regional average,  
the estimated fine mode AOD projections under various scenarios are
comparable to  
those predicted by an online global aerosol model. 
The estimated data for the year 2050 and 2100 were compared with
ECHAM5--HAM simulations  
using emissions for the same year. In terms of main features of the
global distribution  
and regional mean values, the estimated fine mode AOD agrees well with
the predicted ones.   

\subsubsection{Summary of assumptions}

{\bf Natural aerosol is assumed to remain constant over time.}
However, natural aerosol loads depend strongly on meteorological,
synoptical and surface conditions so that at least locally strong
inter--annual variations for natural aerosols can be expected. This
is especially relevant since the AOD of natural aerosol dominates
anthropogenic aerosol in many regions of the world.

{\bf Anthropogenic aerosol is only found in the fine--mode.}
This is not completely true, since dust is partly
of anthropogenic origin, mainly due to man--made changes to
land--cover. However, 
since this anthropogenic contribution is generally a minor fraction
of the dust and in addition highly speculative and uncertain, this
anthropogenic coarse mode contribution has been ignored.
  
{\bf Fine mode aerosol composition does not change with time.}
It should be noted that in both historical extrapolations and
future--scenario simulations only changes to the AOD are considered. 
However, the BC aerosol type had
relatively strong contributions in the 1930ies, while the sulfate
aerosols gain more importance in the 1960ies and 1970ies.
Similarly, changes in the fine--mode composition for future anthropogenic
aerosol can be expected. 

{\bf Future scenarios are based on changes in sulfate emissions and
  the superposition principle is applied.}
However, other aerosol types from anthropogenic sources 
may become more important in the future. Because of the non--linearity
of aerosol physics, the superposition principle may become inaccurate.


\subsubsection{Implementation}\label{secimpl}

For the shortwave (SW) or solar spectral bands, coarse and fine mode
aerosol optical properties (column AOD $\tau_{\rm sw}^{\rm (f,c)}$,
single scattering albedo $\omega_{\rm sw}^{\rm (f,c)}$, and asymmetry
factor $g_{\rm sw}^{\rm (f,c)}$) have to be combined. 
Since fine and coarse mode are assigned 
different normed extinction profiles, $\zeta^{\rm (f)}$ and
$\zeta^{\rm (c)}$, any changes in the ratio between coarse--mode and fine--mode
column AOD will modify the vertical profiles of the single scattering
albedo and the asymmetry factor.

For the longwave (LW) or IR spectral bands only the coarse mode aerosol
contributes, as fine--mode aerosol is too small to play a significant
role at these wavelengths. Since the IR radiative transfer code does not
account for scattering the required properties are the spectrally
resolved column AOD 
$\tau_{\rm lw}^{\rm (c)}$, the column single scattering albedo
$\omega_{\rm lw}^{\rm (c)}$ and coarse--mode altitude distribution via
the normed extinction profile $\zeta^{\rm (c)}$.


The altitude dependent optical depth is calculated in the following
way. Let $(\Delta z_l)_{l=1,L}$ be the geometrical layer thickness of
the \echam{} layers $1,\dots,L$. Let the normed $\zeta^{\rm (f,c)}$ extinction of the
climatology be given for layers $1,\dots,K$ and 

\begin{displaymath}
k:\left\{\begin{array}{ccc}
\{1,\dots,L\} & \rightarrow & \{1,\dots,K\}\\
l & \mapsto & k_l
\end{array}\right.
\end{displaymath}

be the function that gives the layer $k_l$ of the climatology inside
of which the 
mid point of a given layer $l$ of \echam{} is located. For simplicity, we
attribute to this \echam{} layer $l$ the normed extinction $\zeta^{\rm
  (f,c)}_{k_l}$. In general, 

\begin{displaymath}
Z:=\sum\limits_{l=1}^{L}\zeta^{\rm (f,c)}_{k_l}\Delta z_l\neq 1
\end{displaymath}

even if $\sum_{k=1}^K \zeta^{\rm (f,c)}_{k}\Delta y_k=1$ for the layer
thickness $(y_k)_{k=1,K}$ of the climatology. We want to have the same total
optical depth in the simulation with \echam{} as in the climatology. Thus,
we introduce renormalized extinctions

\begin{displaymath}
\tilde{\zeta}_{k_l}^{\rm (f,c)}:=\zeta_{k_l}^{\rm (f,c)}/Z
\end{displaymath}

With these renormalized extinctions, we can calculate the optical
depths $\tau_{{\rm sw,lw},l}^{{\rm (f,c)}}$ for each layer $l=1,L$ of
\echam:

\begin{equation}
\tau_{{\rm sw,lw},l}^{{\rm (f,c)}}=\tau_{\rm sw,lw}^{\rm
  (f,c)}\tilde{\zeta}^{\rm (f,c)}_{k_l} 
\end{equation}

The total column optical depth is then exactly the given optical depth
$\tau_{\rm sw,lw}^{\rm (c,f)}$ of the climatology.

For the SW bands, the optical properties of the combined fine and
coarse aerosol modes are obtained by the usual mixing rules.
This results in the layer dependent optical depth
$\tau_{{\rm sw},l}$, the layer dependent single scattering albedo
$\omega_{{\rm sw},l}$, and the layer dependent asymmetry factor
$g_{{\rm sw},l}$ for each \echam{} layer $l=1,L$:

\begin{align}
\label{eqswtau}
\tau_{{\rm sw},l} & = \tau_{{\rm sw},l}^{\rm (f)}+ \tau_{{\rm
  sw},l}^{\rm (c)}\\
\label{eqswomega}
\omega_{{\rm sw},l} & = \frac{\tau_{{\rm sw},l}^{\rm (f)}\omega_{\rm
    sw}^{\rm (f)}+\tau_{{\rm sw},l}^{\rm (c)}\omega_{\rm
    sw}^{\rm (c)}}{\tau_{{\rm sw},l}}\\
\label{eqswg} 
g_{{\rm sw},l} & = \frac{\tau_{{\rm sw},l}^{\rm (f)}\omega_{\rm
    sw}^{\rm (f)}g_{\rm
    sw}^{\rm (f)}+\tau_{{\rm sw},l}^{\rm (c)}\omega_{\rm
    sw}^{\rm (c)}g_{\rm
    sw}^{\rm (c)}}{\omega_{{\rm sw},l}}
\end{align}

For the LW bands, the absorption optical depth is defined by:

\begin{equation}\label{eqlwtau}
\tau^{\rm (abs)}_{{\rm lw},l}= \tau_{\rm lw}\tilde{\zeta}_{k_l}^{\rm
  (c)}(1-\omega_{\rm lw}) 
\end{equation}

The fine mode aerosols of anthropogenic origin have an effect in the
solar spectrum only and are the sole time dependent
quantities. Tab.~\ref{tabaerofiles} gives an overview of the files
used in \echam.

\begin{table}
\caption{File names of files containing tropospheric aerosol optical
  properties 
  for a year {\tt yyyy} and scenario {\tt
    rcpzz}. The resolution of echam is {\tt \{RES\}}.}\label{tabaerofiles} 
\begin{tabular*}{\textwidth}{l@{\extracolsep\fill}p{5.1cm}}\\\hline
 File name & Explanation  \\\hline
 {\tt T\{RES\}\_aeropt\_kinne\_sw\_b14\_coa.nc} &
  Aerosol optical properties of coarse mode aerosols in the solar
  range of the spectrum. The aerosols are of natural origin (dust, sea
  salt) and independent of the year for historic times.\\
 {\tt T\{RES\}\_aeropt\_kinne\_lw\_b16\_coa.nc } & Aerosol optical
 properties of coarse mode aerosols in the thermal range of the
 spectrum. The aerosols are of natural origin (dust, sea salt) and
 independent of the year for historic times.\\
 {\tt  T\{RES\}\_aeropt\_kinne\_sw\_b14\_fin[\_rcpzz]\_yyyy.nc} & Aerosol
 optical properties of fine mode aerosols in the solar range of the
 spectrum. These aerosols are of anthropogenic origin and therefore
 depend on the year.\\\hline
\end{tabular*}
\end{table}

\subsection{Stratospheric aerosols}

Stratospheric aerosols modify the heating in the
stratosphere and have some influence on the radiation budget in the
troposphere. The optical properties of these aerosols are mainly
determined by the size and concentration of sulfuric acid droplets
that form from SO$_2$ gas in the stratosphere. 
The SO$_2$ gas is either of volcanic origin or is formed from sulfur
containing species from other sources at the surface of the earth.
Ash aerosols from volcanic eruptions are of minor importance
and play a role on short time scales of a few days to weeks only. 
Consequently, the data set of optical properties of stratospheric aerosols
only accounts for the effect of volcanic aerosols released by
eruptions reaching the stratosphere. 
Since the concentration and size distribution of sulfuric acid
droplets in the stratosphere are determined by complex chemical
processes, advective transport, and sedimentation processes in the
stratosphere, the resulting aerosol optical properties are highly
variable in space and time. Nevertheless, due to fast transport in
East--West direction, the optical properties exhibit small variations
for different longitudes at the same latitude but vary strongly with
latitude. Therefore, zonal mean values of the optical properties may
describe the effect of volcanic aerosols on the radiation budget with
sufficient accuracy. 


\subsubsection{Volcanic aerosols from 1850 until 1999}\label{secstenchikov}

The data set of volcanic forcing for the historic period from 1850 to
1999 has been provided by G.~Stenchikov. It is an extended version of
the Pinatubo aerosol data set (PADS) derived by~\cite{ste987} from
stellite measurements of aerosol extinction and effective radii after
the Pinatubo eruption and successfully applied in climate model
studies (\cite{ste049,ste093,tho097,tho091}).
This data set contains monthly mean zonal
averages of the aerosol extinction $\zeta_{\rm v}$, the single scattering
albedo $\omega_{\rm v}$, and the asymmetry factor $g_{\rm v}$ as a function of
altitude, wavelength, and time. Furthermore, the integral aerosol
optical depth of a column $\tau_{\rm v}$ is given as a function of wavelength
and time. The data set comprises the years 1850 to 1999. The data are
given at 40 different mid-level pressures listed in
table~\ref{tab_pres} together with the corresponding interface
pressures. The interpolation with respect to altitude is performed in
a similar way as for the tropospheric aerosols.
 
\begin{table}[hb]
\caption{Pressure levels, mid level pressures (top), pressure at
  interfaces (bottom) in Pa}\label{tab_pres}
\begin{tabular*}{\textwidth}{c@{\extracolsep\fill}cccc}\\\hline
1& 3& 7& 13& 22\\
0\hspace{0.3cm} 2&2\hspace{0.3cm} 4&4\hspace{0.3cm} 10&10\hspace{0.3cm}
16&16\hspace{0.3cm} 28 \\
\rule{0cm}{0.7cm} 35& 52& 76& 108& 150 \\
28\hspace{0.3cm}
42&42\hspace{0.3cm}62&62\hspace{0.3cm} 90&90\hspace{0.3cm} 126&
126\hspace{0.3cm} 174\\ 
 \rule{0cm}{0.7cm}207& 283& 383& 516& 692\\
174\hspace{0.3cm}240&240\hspace{0.3cm} 326&326\hspace{0.3cm}
440&440\hspace{0.3cm} 592&592\hspace{0.3cm} 
    792 \\
\rule{0cm}{0.7cm} 922&    1224& 1619& 2133& 2802\\
792\hspace{0.3cm}1052&1052\hspace{0.3cm}
    1396&1396\hspace{0.3cm} 1842&1842\hspace{0.3cm}
    2424&2424\hspace{0.3cm} 3180\\ 
 \rule{0cm}{0.7cm}3670& 4793& 6236& 8066& 10362\\
3180\hspace{0.3cm} 4160&4160\hspace{0.3cm} 5426&5426\hspace{0.3cm}
7046&7046\hspace{0.3cm} 9086& 9086\hspace{0.3cm}
    11638 \\
\rule{0cm}{0.7cm} 13220& 16748&
    21059& 26192& 32082\\
11638\hspace{0.3cm} 14802&14802\hspace{0.3cm}
18694&18694\hspace{0.3cm} 23424&28960\hspace{0.3cm}
28960&28960\hspace{0.3cm} 35204\\ 
\rule{0cm}{0.7cm} 38675& 45908& 53672& 61799& 70056 \\
35204\hspace{0.3cm} 42146&42146\hspace{0.3cm}
49670&49670\hspace{0.3cm} 57674&
 57674\hspace{0.3cm}65924&65924\hspace{0.3cm}
    74188\\
\rule{0cm}{0.7cm} 78139& 85673&
    92219& 97287& 100368\\
74188\hspace{0.3cm} 82090 &82090\hspace{0.3cm} 89256&89256\hspace{0.3cm} 95182&95182\hspace{0.3cm} 99392&99392\hspace{0.3cm} 101344\\\hline
\end{tabular*}
\end{table}

The aerosol optical properties are provided at 30 wavelength bands
which are listed in 
table~\ref{tab_bands}. We give the index of the corresponding spectral
bands in the \echam{}
radiation code in column three and four of the table. The definition
of wavelength band~30 is different for the new data set and
the radiation code of \echam. 

\begin{table}
\caption{Wavelength bands for optical properties of volcanic aerosols
  in nm}\label{tab_bands}
\begin{tabular*}{\textwidth}{c@{\extracolsep\fill}ccc}\hline
band index & $\lambda_{\rm v}/{\rm nm}$ & \multicolumn{2}{c}{\echam{} band} \\\hline
\cw{1}1  &   \cw{00}200 --   \cw{000}263 &  solar 13 & \\
\cw{1}2  &   \cw{00}263 --   \cw{000}345 &  solar 12 & \\
\cw{1}3  &   \cw{00}345 --   \cw{000}442 &  solar 11 & \\
\cw{1}4  &   \cw{00}442 --   \cw{000}625 &  solar 10 & \\
\cw{1}5  &   \cw{00}625 --   \cw{000}778 &  solar \cw{1}9 & \\
\cw{1}6  &   \cw{00}778 --  \cw{00}1242 &  solar \cw{1}8 & \\
\cw{1}7  &  \cw{0}1242 --  \cw{00}1299 &  solar \cw{1}7 & \\
\cw{1}8  &  \cw{0}1299 --  \cw{00}1626 &  solar \cw{1}6 & \\
\cw{1}9  &  \cw{0}1626 --  \cw{00}1942 &  solar \cw{1}5 & \\
     10  &  \cw{0}1942 --  \cw{00}2151 &  solar \cw{1}4 & \\
     11  &  \cw{0}2151 --  \cw{00}2500 &  solar \cw{1}3 & \\
     12  &  \cw{0}2500 --  \cw{00}3077 &  solar \cw{1}2 & \\
     13  &  \cw{0}3077 --  \cw{00}3846 &  solar \cw{1}1 & thermal 16 \\
     14  &  \cw{0}3846 -- \cw{0}12195 &  solar 14 &\\
     15  &  \cw{0}3333 --  \cw{00}3846 &  \multicolumn{2}{c}{---} \\
     16  &  \cw{0}3846 --  \cw{00}4202 &                & thermal 15 \\
     17  &  \cw{0}4202 --  \cw{00}4444 &                & thermal 14 \\
     18  &  \cw{0}4444 --  \cw{00}4808 &                & thermal 13 \\
     19  &  \cw{0}4808 --  \cw{00}5556 &                & thermal 12 \\
     20  &  \cw{0}5556 --  \cw{00}6757 &                & thermal 11 \\
     21  &  \cw{0}6757 --  \cw{00}7194 &                & thermal 10 \\
     22  &  \cw{0}7194 --  \cw{00}8474 &                & thermal \cw{1}9\\
     23  &  \cw{0}8474 -- \cw{00}9259 &                & thermal \cw{1}8\\
     24  &  \cw{0}9259 -- \cw{0}10204 &                & thermal \cw{1}7\\
     25  & 10204 -- \cw{0}12195 &                & thermal \cw{1}6\\
     26  & 12195 -- \cw{0}14286 &                & thermal \cw{1}5\\
     27  & 14286 -- \cw{0}15873 &                & thermal \cw{1}4\\
     28  & 15873 -- \cw{0}20000 &                & thermal \cw{1}3\\
     29  & 20000 -- \cw{0}40000 &                & thermal \cw{1}2\\
     30  & 40000 -- 250000 &                & thermal \cw{1}1\\
\hline
\end{tabular*}
\end{table}

The aerosol optical properties of the volcanic aerosols in the solar
wavelength range ($\tau_{\rm v}$, $\omega_{\rm v}$, $g_{\rm v}$) and
  the thermal wavelength 
  range ($\tau^{\rm (lw)}_{\rm v}$, 
  $\omega^{\rm (lw)}_{\rm v}$) are added to
the given aerosol optical properties in the solar wavelength range
($\tau$, $\omega$, $g$) and the thermal wavelength range ($\tau^{\rm
  (lw)}$, $\omega^{\rm (lw)}$ ) by the 
usual mixing rules (see equations~(\ref{eqmixing1})--(\ref{eqmixing4})) although the
tropospheric and stratospheric aerosols are separated by
region. Nevertheless, this procedure assures generality and allows for
an overlap of these regions.

\begin{eqnarray}
\label{eqmixing1}
\tau_{\rm total}&:=&\tau+\tau_{\rm v}\\
\omega_{\rm total}&:=&\frac{\tau\omega+\tau_{\rm
  v}\omega_{\rm v}}{\tau_{\rm total}}\\ 
g_{\rm total}&:=&\frac{\tau\omega g+\tau_{\rm
  v}\omega_{\rm v}g_{\rm v}}{\tau_{\rm total}\omega_{\rm total}}\\
\tau^{\rm (lw)}_{\rm total}&:=&\tau^{\rm (lw)}+\tau^{(\rm lw)}_{\rm
  v}(1-\omega^{\rm (lw)}_{\rm v})\label{eqmixing4}
\end{eqnarray}

\subsubsection{Volcanic aerosols from 790 until 2010}

The historic record of stratospheric volcanic aerosols by
G.~Stenchikov (see the previous section) comprises the period
from 1850 until 1999 only. Volcanic eruptions prior to this period can
be taken into account by the use of the longterm data set by
T.~Crowley that gives information about the volcanic forcing in terms
of total aerosol optical
depth and the effective radius  since 790. 
In that
case, no information about the height distribution of the aerosols is
available. T.~Crowley estimated the total aerosol optical depth at 550~nm
for four
latitude bands ($30^\circ{\rm N}$ -- $90^\circ{\rm N}$, $0^\circ{\rm
  N}$ -- $30^\circ{\rm N}$, $30^\circ{\rm S}$ -- $0^\circ{\rm N}$,
$90^\circ{\rm N}$ -- $30^\circ{\rm S}$). For each of these latitude
bands, he also gives an estimate of the effective radius of the aerosols.
These original values for the aerosol optical depth and the effective
radius are linearily interpolated for latitudes
in $\left[15^\circ{\rm N},45^\circ{\rm N}\left[\right.\right.$
(between the values for the latitude bands $30^\circ{\rm N}$ --
$90^\circ{\rm N}$, $0^\circ{\rm 
  N}$ -- $30^\circ{\rm N}$),
$\left[15^\circ{\rm S},15^\circ{\rm N}\left[\right.\right.$ (between
the values for the latitude bands $0^\circ{\rm
  N}$ -- $30^\circ{\rm N}$, $30^\circ{\rm S}$ -- $0^\circ{\rm N}$),
and $\left[45^\circ{\rm S},15^\circ{\rm S}\left[\right.\right.$
(between the values for the latitude bands $30^\circ{\rm S}$ --
$0^\circ{\rm N}$, 
$90^\circ{\rm N}$ -- $30^\circ{\rm S}$).

For the radiation calculation, it is assumed that the volcanic
aerosols consist of 75\% sulfate 
aerosols. Certain 
wavelength and radius dependence tables prepared by S.~Kinne are used
to estimate the aerosol optical properties from the the aerosol
optical depth at 550~nm and the effective radius $r_{\rm err}$
assuming a logarithmic normal distribution. 
For each particle radius
$r$ and wavelength 
$\lambda$ a table provides the ratio
$(r,\lambda)\mapsto\xi(r,\lambda):=\zeta(r,\lambda)/\zeta(r,550{\rm})$
where $\zeta$ is the extinction coefficient, the single scattering
albedo $(r,\lambda)\mapsto\omega(r,\lambda)$, and the
asymmetry factor $(r,\lambda)\mapsto g(r,\lambda)$. Since
$\zeta$ is assumed 
to be constant in a model layer, the extinction is proportional to the
aerosol optical depth in one layer. Therefore, the space, time, and
wavelength dependent volcanic aerosol optical properties $\tau_{\rm
  v}$ are given for any position 
$\vec{x}$ in the atmosphere and time $t$ by:

\begin{eqnarray}\label{eqtau}
\tau_{\rm v}(\vec{x},t,\lambda)&=&\xi(r_{\rm
eff}(\vec{x},t),\lambda)\times\tau_{550}(\vec{x},r) \\\label{eqssa}
\omega_{\rm v}(\vec{x},t,\lambda)&=&\omega(r_{\rm
eff}(\vec{x},t),\lambda) \\\label{eqasy}
g_{\rm v}(\vec{x},t,\lambda)&=&g(r_{\rm
eff}(\vec{x},t),\lambda) 
\end{eqnarray}

The aerosol optical properties of the volcanic or stratospheric
aerosols are linearly interpolated in time and then added to the
aerosol optical properties according to the common mixing rules
resulting in the following overall aerosol optical properties
$\tau_{\rm total}$, $\omega_{\rm total}$, $g_{\rm total}$. In the case of solar
wavelenghts, the full mixing rules are applied:

\begin{eqnarray}\label{eqm1}
\tau_{\rm total}&=&\sum\limits_{i=1}^l \tau_i\\\label{eqm2}
\omega_{\rm total}&=&\frac{\sum\limits_{i=1}^l \omega_i\tau_i}{\tau_{\rm total}}\\\label{eqm3}
g_{\rm total}&=&\frac{\sum\limits_{i=1}^l g_i\omega_i\tau_i}{\tau_{\rm
    a}\omega_{\rm total}}
\end{eqnarray}

In the case of thermal wavelengths, only the aerosol optical depth has
to be provided, but only the ``absorbence'' is taken into account:

\begin{equation}\label{eqm4}
\tau_{\rm total}=\sum\limits_{i=1}^l \tau_i(1-\omega_i)
\end{equation}


For historic volcanic eruptions, there is no information available
about the vertical distribution of 
volcanic aerosols from measurements, but we know that
the altitude distribution depends on the neutral buoyancy height
of the volcanic plume at which the 
aerosols form.
Furthermore, we know that the neutral buoyancy height is also
limited because of the gravity effect on the plume as described
by~\cite{her106, tim093} and by personal communication of H.--F.~Graf~2005. From
this, we conclude that the aerosols are located mainly in the
stratosphere. The exact altitude position is not of first order
relevance for the radiation budget in the troposphere provided that the total
aerosol optical depth is correct. On the other hand, the influence on
the dynamics of the stratosphere depends on the exact altitude but is
not so relevant for simulations with a focus on the climate. We
therefore decided to use an altitude profile that is similar to the
injection height of SO$_2$ as it was observed from satellite after the Pinatubo
eruption, see~\cite{spa1997}. The following pressure dependent weight
function $w$ is 
used at all geographical locations:

\begin{equation}
p\mapsto w(p)=\frac{1}{4}\times{\bf 1}_{[30{\rm hPa},40{\rm hPa}[}(p) +
\frac{1}{2}\times{\bf 1}_{[40{\rm hPa},50{\rm hPa}[}(p)+\frac{1}{4}\times{\bf 1}_{[50{\rm hPa},60{\rm hPa}[}(p)
\end{equation}

where ${\bf 1}_{A}$ is the characteristic function of set $A$. Let
$\vec{y}$ represent a location on the surface of the Earth and be
$(\vec{y},r)\mapsto\tau_{\rm crow}(\vec{y},r)$ the aerosol optical depth at
550~nm and a certain effective radius $r$ provided by T.~Crowley, then 
$\tau_{550}=w\times\tau_{\rm crow}$. The time, space and wavelength
dependent optical properties of the volcanic aerosols are then given
by equations~(\ref{eqtau}--\ref{eqasy}). As in the case of the HAM
derived volcanic aerosol properties, the full mixing rules are applied
in the case of the solar radiation according to
equations~(\ref{eqm1}--\ref{eqm3}). In the case of the thermal
radiation, the simplified
equation~(\ref{eqm4}) is used.
