\newpage

\section{Water Vapor in the Middle Atmosphere}

In the stratosphere and mesosphere the major source of water vapor, besides transport from the troposphere, is the oxidation of methane. 
In order not to underestimate middle atmospheric water vapor, \echam{} includes a submodel that parameterizes methane oxidation as well as the
photolysis of water vapor. By default this submodel is however switched off. The formulation of the processes follows the respective 
implementation in the ECMWF IFS (see IFS documentation Cy36r1, part IV, chapter 9 for details). 
The scheme adds the two tendency terms $Q_\mathrm{methox}$ and $Q_\mathrm{photo}$ to the right hand side of the prognostic equation for the grid-cell 
mean mass mixing ratio of water vapor  $\bar{r_{v}}$ (eq.~\ref{9.1}).

$Q_\mathrm{methox}$ describes the parameterization of methane oxidation and is defined as
\begin{equation}
Q_\mathrm{methox} = ( r_{v,\mathrm{lim}} - \bar{r_{v}} ) / \tau_\mathrm{methox}.
\end{equation}
$r_{v,\mathrm{lim}}$ is set 
to the value of 4.25 ppm and hence independent of the methane mixing ratio acting on radiation.
The dependence of $\tau_\mathrm{methox}$ in days on atmospheric pressure $p$ in hPa is given by
\begin{equation}
\tau_\mathrm{methox} = \left\{ 
\begin{array}{ll} 
    100 & p \le 0.5\\
    100 \left[ 1 + \alpha_1 \frac{\left\{ \ln(p/0.5) \right\}^{4}} { \ln (100/p) } \right] & 0.5 < p < 100\\
    \infty   & p \ge 100
\end{array} \right.
\end{equation}
with
\begin{equation}
 \alpha_1 = \frac{19 \ln 10}{(\ln 20)^4}.
\end{equation}
This means that water vapor production by methane is strongest in the mesosphere and not considered in the troposphere.


$Q_\mathrm{photo}$ describes the photolysis of water vapor and is defined as
\begin{equation}
Q_\mathrm{photo} = - \bar{r_{v}} / \tau_\mathrm{photo}.
\end{equation}
The dependence of $\tau_\mathrm{photo}$ in days on pressure in hPa given by
\begin{equation}
\tau_\mathrm{photo} = \left\{
\begin{array}{ll}
    3 & p \le 0.001\\
    \left[ \exp \left\{ \alpha_2 - 0.5 ( \ln 100 + \alpha_2 ) \left( 1 + \cos \frac{\pi \ln (p/0.2)}{\ln 0.005} \right) \right\} - 0.01 \right]^{-1} & 0.001 < p < 0.2\\
    \infty   & p \ge 0.2
\end{array} \right.
\end{equation}
with
\begin{equation}
 \alpha_2 = \ln \left(\frac{1}{3} +0.01 \right).
\end{equation}
This definition ensures that the photolysis of water vapor contributes significantly to the water vapor budget only above about 0.1 hPa
where
$\tau_\mathrm{photo}$ reaches values of below 100 days.

The vertical profile of the photochemical lifetime of water vapor $(1/\tau_\mathrm{methox} + 1/\tau_\mathrm{photo})^{-1}$ combined 
from the two parameterized processes above agrees reasonably to a respective profile given in Fig. 5.23 of \citet{brasseur05}.


