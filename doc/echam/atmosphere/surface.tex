\chapter{Surface processes}\label{c6}

\section{Heat budget of the soil}\label{s6.1}

\subsection{Land surface temperature}\label{su6.1.1.}

The surface energy balance is the link between the atmosphere and the
underlying surface, with the surface temperature $T_s$ as the key variable. The
interface between the land surface and the atmosphere can be understood as
a `layer' at the surface which is in contact with the atmosphere. The
surface energy balance for this layer can then be written as

\begin{equation}
C_L \frac{\partial T_s}{\partial t}=R_{net}+LE+H+G\label{6.1}
\end{equation}

where $C_L$ is the heat capacity of the layer [Jm$^{-2}$K$^{-1}$], $H$ is
the sensible heat flux, $LE$ the latent heat flux ($L$ is the latent heat
of vaporization or sublimation of water, respectively), $G$ is the ground
heat flux and $R_{net}$ the net radiation consisting of the following
components,

\begin{equation}
R_{net}=(1-\alpha_s)R_{sd}+ \epsilon R_{ld}-\epsilon\sigma T_s^4\label{6.2}
\end{equation}

where $\alpha_s$ is the surface albedo, $R_{sd}$ the downwelling solar
radiation, $R_{ld}$ the downwelling longwave radiation, $\epsilon$ the
surface emissivity, and $\alpha$ the Stefan-Boltzmann constant. Note
that in \eref{6.1} downward (upward) fluxes are positive
(negative). Due to the strong coupling between the surface and the
atmosphere, the numerical solution of \eref{6.1} is closely linked to
the vertical heat transfer within the atmosphere (c.f. section
\ref{sec:para1}). In ECHAM5, an implicit coupling scheme is used
\citep{schulz01} which is unconditionally stable and allows to 
synchronously calculate the prognostic variables and the surface
fluxes. In order to avoid iterations, the nonlinear terms appearing in
the flux formulations are replaced by truncated Taylor
expansions. These terms are the upward longwave radiation (i.e. the
last term in \eref{6.2})

\begin{equation}
\epsilon\sigma(T_s^{n+1})^4\approx \epsilon\sigma (T_s^n)^4+ 4
\epsilon\sigma(T_s^n)^3(T_s^{n+1}-T_{s}^n)\equiv 
\epsilon\sigma(T_{rad}^{n+1})^4,
\label{6.3}\end{equation}

where $T_{rad}^{n+1}$ is an effective `radiative temperature' used in the
radiation scheme to close the energy balance, and the saturated specific
humidity

\begin{equation}
q_s(T_s^{n+1})
\approx q_s(T_s^n)+\left.\frac{\partial q_s}{\partial T_s}\right|_{T_s^n}
(T_s^{n+1}-T_s^n)\label{6.4}
\end{equation}

where $n$ is the time level. With these linearizations, an expression for
$T_s^{n+1}$ is obtained which implicitly includes the surface fluxes at
time level $n+1$, and $T_s^{n+1}$, on the other hand, is used in the
vertical diffusion scheme for calculating new fluxes and atmospheric
profiles of temperature and humidity. This ensures energy conservation in
the coupled system because the atmosphere receives the same surface fluxes
as applied in the land surface scheme.

\subsection{Soil temperatures}\label{su6.1.2}

The temperature profile within the soil is calculated from the thermal
diffusion equation

\begin{equation}
C_s\frac{\partial T}{\partial t}=-\frac{\partial G}{\partial
z}=-\frac{\partial}{\partial z}\left(-\lambda_s\frac{\partial T}{\partial
z}\right)\label{6.5}
\end{equation}

where $C_s=\rho_s c_s$ is the volumetric heat capacity of the soil,
[Jm$^{-3}$K$^{-1}$], $\rho_s$ the soil density [kgm$^{-3}$], $c_s$ the
soil specific heat [Jkg$^{-1}$K$^{-1}$], $G$ is the thermal heat flux
(positive downward), $\lambda_s=\rho_s c_s k_s$ is the thermal
conductivity [Wm$^{-1}$K$^{-1}$], $\kappa_s$ the thermal diffusivity
[m$^2$s$^{-1}$] and $z$ the depth [m]. For the numerical solution
of \eref{6.5}, the upper 10 m of the soil are divided into 5 unevenly
spaced layers with thicknesses, from top to bottom, according to
0.065, 0.254, 0.913, 2.902, and 5.700 [m]. An implicit time
integration scheme is applied where the finite difference form of
\eref{6.5} is reduced to a system of the type:

\begin{equation}
T^{n+1}_{k+1}=A^n_k+B^n_kT^{n+1}_k\label{6.6}
\end{equation}

where $n$ is the time level and $k$ the layer index. The coefficients
$A^n_k$ and $B^n_k$  include the vertical increments $\Delta z$, the time
step $\Delta t$, the volumetric heat capacity $C_s$, the thermal
conductivity $\lambda_s$ and the temperature at time level $n$ in $A^n_k$.
The coefficients $A$ and $B$ are obtained by integrating upward from
bottom to top, assuming vanishing heat flux at the bottom of the lowest
layer. The solution of \eref{6.6} proceeds from top to bottom by using

\begin{equation}
T^{n+1}_1= T^{n+1}_s\label{6.7}
\end{equation}

as the upper boundary condition. For snow-free land, spatially varying
$C_s$ and $\kappa_s$ are prescribed for ice sheets/glaciers and for
five soil types according to the FAO soil map. For snow covered land, the
solution method remains unchanged, except that a mass-weighted mixture of
soil and snow is applied to determine the properties $\rho_s, c_s,
\kappa_s$. For example, if the snow fills the top soil layer completely,
and the next one partially, the respective properties for snow are used
for the top layer and a mass-weighted mixture is used for the next one.

\section{Water budget}\label{s6.2}

Budget equations are formulated for the following water reservoirs (depth
in m)

\begin{itemize}
\item Snow $h_{snc}$ (water equivalent) intercepted by the canopy
\item Snow $h_{sn}$ (water equivalent) at the surface
\item Rain water $h_{wc}$ intercepted by the canopy
\item Soil water $h_{ws}$ \end{itemize}

\subsection{Interception of snow by the canopy}\label{su6.2.1}

The snow interception scheme \citep{roesch01} accounts for the effects
of snowfall ($c_v S$, where $c_{v}=0.25$ is an interception parameter
and $S$ the snowfall rate), sublimation of snow from the canopy
$(E_{snc} < 0)$, or deposition $(E_{snc}> 0)$, and unloading of snow
due to slipping and melting $U(T_c)$ and wind $U(v_c)$, where $T_c$
and $v_c$ are temperature and wind speed in the canopy, respectively,

\begin{equation}
\rho_w\dnd{h_{snc}}{t}=c_{v}S+E_{snc}-\rho_wh_{snc}[U(T_c)+U(v_c)].\label{6.8}
\end{equation}

The accumulation of snow on the canopy is limited by the capacity of the
interception reservoir, $h_{snc} \le h_{cin}$, where the interception
reservoir, $h_{cin}$, is a function of the leaf area index, $LAI$,

\begin{equation}
h_{cin} = h_{0}LAI\label{6.9}
\end{equation}

with $h_{0}=2\cdot 10^{-4}$ m.
The unloading processes are parameterized according to

\begin{equation}
U(T_c)=(T_c-c_1)/c_2\ge 0\label{6.10}
\end{equation}

\begin{equation}
U(v_c)=v_c/c_3\ge 0\label{6.11}
\end{equation}

with $c_1 = (T_0 - 3)$, where $T_0$ is the freezing temperature of
water [K], $c_2 = 1.87 \cdot 10^5$ Ks and $c_3 = 1.56 \cdot
10^5$ m.  Because $T_c$ and $v_c$ are not available in the model, the
respective values at the lowest model level are used
instead. Consistent with this assumption, the snow melt parameterized
according to \eref{6.10} results in a cooling of the lowest model
layer

\begin{equation}
\dnd{T_c}{t}=-\rho_wh_{snc}U(T_{c}) \frac{L_f}{c_p}\frac{g}{\Delta 
p}\label{6.12}
\end{equation}

where $\Delta p$ is the pressure thickness of the lowest model layer.

\subsection{Snow at the surface}\label{su6.2.2}

The snow budget at the surface is given by

\begin{equation}
\rho_w\dnd{h_{sn}}{t}=(1-c_{v})S+E_{sn}-M_{sn}+\rho_wh_{snc}U(v_c)\label{6.13}.
\end{equation}

Here, $E_{sn}<0$ is the sublimation of snow, $E_{sn} > 0$  the deposition,
and $M_{sn}$ is the snow melt

\begin{equation}
M_{sn}=\frac{C_s}{L_f}\left ( \frac{T^*-T_0}{\Delta t}\right)\label{6.14}
\end{equation}

where $C_s$ is the heat capacity of the snow layer, $\rho_w$ the density
of water, $L_f$ the latent heat of fusion, and $T^*$ is the
surface temperature obtained from the surface energy balance equation 
\eref{6.1}
without considering snow melt. The `final' surface temperature, i.e.
including the cooling due to snow melt, is given by

\begin{equation}
T_s=T^*-\frac{L_f}{C_s}(M_{sn}\Delta t)=T_0\label{6.15}.
\end{equation}

For the special case of complete melting during one time step $M_{sn}\Delta t$
is limited by the available snow amount $\rho_wh_{sn}$ so
that $T^* >T_s \ge T_0$. Over ice sheets and glaciers, snow processes 
are neglected,
i.e. $h_{sn}=h_{snc}=0$. However, a melting term analogous to
\eref{6.14} is
diagnosed for $T^* > T_0$ and $T_s$ is set to $T_0$ in this case.

\subsection{Interception of rain by the canopy}\label{su6.2.3}

Analogous to snowfall, a fraction $c_{v}R$ of the incoming rain $R$ is
intercepted in the canopy with an upper limit defined by the capacity of
the interception reservoir, $h_{cin}$, analogous to the interception of
snowfall

\begin{equation}
\rho_w\dnd{h_{wc}}{t}=c_{v}R+E_{wc}+\rho_w h_{snc}U(T_c)\label{6.16}
\end{equation}

where $E_{wc}< 0$  is the evaporation and $E_{wc}> 0$ the dew deposition.
Melting of $h_{snc}$ according to \eref{6.8} and \eref{6.10} 
contributes to $h_{wc}$
unless the capacity of the interception reservoir is exceeded. In that
case, the excess water would contribute to the soil water budget through
the term $M_{snc}$  in \eref{6.17}.

\subsection{Soil water}\label{su6.2.4}

Changes in soil water $h_{ws}$ due to rainfall, evaporation, snow melt,
surface runoff and drainage are calculated for a single bucket with
geographically varying maximum field capacity $h_{cws}$

\begin{equation}
\rho_w\dnd{h_{ws}}{t}=(1-c_{v})R+E_{ws}+M_{sn}+M_{snc}-R_s-D\label{6.17}
\end{equation}

where $E_{ws}<0$ includes the effects of bare-soil evaporation and
evapotranspiration, and $E_{ws}>0$ the dew deposition. $M_{sn}$ is the
snow melt at the surface according to \eref{6.14}, $M_{snc}$ the
excess snow melt in the canopy, $R_{s}$ the surface runoff and $D$ the
drainage.  Runoff and drainage are calculated from a scheme which
takes into account the heterogeneous distribution of field capacities
within a grid-cell \citep{duemnil92}. The storage capacity of the soil
is not represented by a unique value, as in the traditional bucket
scheme, but by a set of values with a probability density function
$F(h_{ws})$. Then, a `storage capacity distribution curve' can be
defined, which represents the fraction $f_{ws}$ of the grid-cell in
which the storage capacity is less or equal to $h_{ws}$

\begin{equation}
f_{ws}=1-(1-h_{ws}/h_{cws})^b\label{6.18}
\end{equation}

where $b$  is a shape parameter that defines the sub-grid scale
characteristics of the basin or grid-cell. By integrating over one time
step all fluxes in \eref{6.17} contributing to the total water
input

\begin{equation}
Q=\frac{\Delta t}{\rho_w} [(1-c_{v})R+M_{sn}+M_{snc}+\max
(0,E_{ws})]\label{6.19},
\end{equation}

the surface runoff can be expressed as

\begin{equation}
\frac{\Delta 
t}{\rho_w}R_s=Q-(h_{cws}-h_{ws})+\left[\left(1-\frac{h_{ws}}{h_{cws}}\right)^{1/(1+b)}
-\frac{Q}{(1+b) h_{cws}}\right]^{1+b}\label{6.20}
\end{equation}

provided that the soil is not brought to saturation during one time step
(i.e. [$\cdots] > 0$). In this case, $Q>0$  will always result in $R_s>0$
due to the contribution from sub-grid scale areas that become saturated.
If, on the other hand, $Q$ and $h_{ws}$ are large enough for generating
saturation of the whole grid-cell so that $[ \cdots ] \le 0$, the
traditional bucket approach is applied, i.e. the surface runoff is given
by the excess of water above the maximum value

\begin{equation}
\frac{\Delta t}{\rho_w}R_s=Q-(h_{cws}-h_{ws})\label{6.21}.
\end{equation}

The infiltration of water into the soil during one time step can be
expressed as $I=Q-R_s \Delta t/\rho_w$. No infiltration is allowed in
frozen soil, so that $R_s \Delta t/\rho_w=Q$ in this case. The shape
parameter $b$ in \eref{6.18} and \eref{6.20} is parameterized as a 
function of the
steepness of the sub-grid scale terrain, or standard deviation of height
$\sigma_z$ in the grid-cell, according to

\begin{equation}
b=\frac{\sigma_z-\sigma_0}{\sigma_z+\sigma_{max}}\label{6.22}
\end{equation}

where $\sigma_0= 100$ m is a threshold value below which a small constant,
$b = 0.01$, is applied, and $\sigma_{max}$  is a resolution dependent
maximum value so that $b \to 0.5$ for $\sigma_z \to \sigma_{max}$.
Drainage occurs independently of the water input $Q$ if the soil water
$h_{ws}$ is between 5\% and 90\% of the field capacity $h_{cws}$ (`slow
drainage') or between 90\% and 100\% of $h_{cws}$ (`fast drainage'):

\begin{eqnarray}D&=&0 \mbox{ for} \; \left\{h_{ws}\le 
h_{min}\right\}\label{6.23}\\
D&=&d_{min}\left(\frac{h_{ws}}{h_{cvs}}\right) \, \mbox{for} \,
\left\{h_{min} < h_{ws}\le h_{dr}\right\}\label{6.24}\\
D&=&d_{min}\left(\frac{h_{ws}}{h_{cws}}\right)+(d_{max}-d_{min})
\left[\frac{h_{ws}-h_{dr}}{h_{cws}-h_{dr}}\right]^d
\,\mbox{for} \, \{h_{ws} > h_{dr}\}\label{6.25}
\end{eqnarray}


with

\begin{center}
\begin{tabular}{lll}
$d_{min}$&=&$2.8 \cdot 10^{-7}$kg/(m$^2$s)\\
$d_{max}$&=&$2.8 \cdot 10^{-5}$kg/(m$^2$s)\\
$d$&=&1.5\\
$h_{min}$&=&$0.05 \cdot h_{cws}$ \\
$h_{dr}$&=&$0.9 \cdot h_{cws}$.
\end{tabular}
\end{center}

\section{Lake model\label{sec:lake}}

A simple scheme is used for calculating the water temperature, ice
thickness and ice temperature of lakes, with water/ice fraction of 100\%
assigned to all non-coastal land points with land fraction $f_{l}\le 50\%$.
For $f_{l} > 50\%$, the water/ice fraction is set to zero. The 
temperature $T_w$
of a constant-depth mixed layer ($h_m=10$ m) is estimated from the heat
budget equation by neglecting both horizontal heat fluxes and those at the
bottom of the layer. Thus, $T_w$ is uniquely determined by the net surface
heat flux, $H$, which represents the sum of all radiative and turbulent
flux components,

\begin{equation}
C_w\dnd{T_w}{t} = H\label{6.26}
\end{equation}

where $C_w = c_w\rho_wh_m$ is the heat capacity of the layer and $c_p$
the specific heat of water at constant pressure. Ice formation may
occur if $T_w$ falls below the freezing point $T_0$. However, for
numerical reasons, ice formation is suppressed until the cooling,
during a time step, becomes large enough to form a slab of ice with
thickness $h_i \ge h_{min}$ and $h_{min}=0.1$ m. Therefore, the finite
difference form of (\ref{6.26}) includes a residual term $(R_f \le 0$;
see below) which takes into account `unrealized freezing' from the
previous time step in the case $0<h_i \le h_{min}$

\begin{equation}
\frac{C_w(T^*-T^n_w)}{\Delta t} =H+R_f\label{6.27}
\end{equation}

where $n$  is the time level and $T^*$ a preliminary water temperature.

\renewcommand{\labelenumi}{\roman{enumi}.}

\paragraph*{Cases:}
\begin{enumerate}

\item $T^* \ge T_{0}$. Here, the new temperature is given by 
$T^{n+1}_w=T^*$. Moreover,
$h_i^{n+1}=0$ and $R_f=0$.

\item $T^* < T_{0}$ but $C_w(T^*-T_{0})/\Delta t>
-(\rho_iL_{f}h_{min})/\Delta t$, where $\rho_{i}$
is the ice density
and $L_{f}$ the latent heat of fusion. In this  case of  `unrealized
ice formation', the cooling is not large enough to form a slab of ice
with $h_i\ge h_{min}$, so that $h_i^{n+1}=0$ and $T_w^{n+1}=T_{0}$, while

\begin{equation}
R_{f}=\frac{C_w(T^*-T_0)}{\Delta t} <0\label{6.28}
\end{equation}

is used in \eref{6.27} at the next time step. Assuming, for example, a
constant $H<0$ in \eref{6.27}, $R_{f}$ will steadily decrease in the course
of the integration until the following conditions are fulfilled:

\item  $T^* < T_{0}$ and $R_f\le-(\rho_{i}L_{f}h_{min})/\Delta t$. In 
this case,
the cooling is large enough to form a slab of ice with
$h_{i}\ge h_{min}$ so that $T^{n+1}_{w}=T_{0}$, $R_{f}=0$ and

\begin{equation}
h_{i}^{n+1}=-\left(\frac{C_w}{\rho_{i}L_{f}}\right)
(T^*-T_{0})\label{6.29}.
\end{equation}
\end{enumerate}

Once a slab of ice is formed (note that partial ice cover is not taken
into accout, i.e. a lake is either ice-free or totally covered with ice),
changes in ice thickness may
result from freezing at the bottom, through conductive heat fluxes
($H_{c}$), melting ($M_{i}$) at the surface, sublimation of ice
$(E_{i}<0$) or deposition $(E_{i}>0$):

\begin{equation}
\rho_{i}\dnd{h_{i}}{t}=\frac{(H_{c}-M_{i})}{L_{f}} + E_{i}.\label{6.30}
\end{equation}

$M_{i}$ is calculated together with the ice temperature, and $H_{c}$ is defined
as

\begin{equation}
H_{c}=-\frac{\kappa_{i}}{h_{eff}}(T_{i}-T_{0})\label{6.31}
\end{equation}

where $T_{i}\le T_{0}$ is the surface temperature of the ice layer, 
$\kappa_{i}$ the
thermal conductivity of ice and $h_{eff}$ an effective ice thickness 
which takes into
account the effect of a snow layer (with depth $h_{sni}$) on the 
conductive heat
flux

\begin{equation}
h_{eff}=h_{i}+\left(\frac{\kappa_{i}}{\kappa_{s}}\right)h_{sni}\label{6.32}
\end{equation}

where $\kappa_{s}$ is the thermal conductivity of snow.
The finite difference form of \eref{6.30} involves the calculation of 
a preliminary
ice thickness $h^{*}$, analogous to $T^{*}$ in \eref{6.27},

\begin{equation}
\rho_{i}\frac{(h^*-h^n_{i})}{\Delta 
t}=\frac{(H_{c}-M_{i})}{L_{f}}+E_{i}\label{6.33}
\end{equation}

\paragraph*{Cases:}

\begin{enumerate}
\item $h^* \ge h_{min}$ so that  $h^{n+1}_{i}=h^*$.

\item $0<h^*<h_{min}$. In this case of `unrealized melting', part of 
the available
heat is used to decrease the ice thickness to the minimum value
$h_{i}^{n+1}=h_{min}$, while the residual flux

\begin{equation}
R_{m}=\frac{\rho_{i}L_{f}(h_{min}-h^*)}{\Delta t}>0\label{6.34}
\end{equation}

is used for the calculation of the surface temperature (see below).


\item $h^*\le 0$. In this case of complete melting, $h^{n+1}_{i}=0$, 
and the residual heat
(if $h^*<0$) is used to increase the water temperature according to


\begin{equation}
C_w\frac{(T^{n+1}_{w}-T^n_{w})}{\Delta t}=\frac{-\rho_{i}L_{f}h^*}{\Delta
t}\label{6.35}
\end{equation}

with $T^{n}_{w}=T_{0}$.
\end{enumerate}

The
surface temperature is calculated from the heat budget of a thin slab of
ice with thickness $h_{0}$= 0.1 m

\begin{equation}
C_i\dnd{T_{i}}{t}=H+H_{c}\label{6.36}
\end{equation}

where $C_i=\rho_{i}c_{i}h_{0}$ is the heat capacity of the slab
and $c_i$ the specific heat of ice. $H$ is the net surface heat flux (sum
of all radiative and turbulent flux components) and  the conductive heat
flux is defined in \eref{6.31}. The finite difference form of 
\eref{6.36} includes the
residual term  defined in \eref{6.34}, which is non-zero (positive) 
for the case
of `unrealized melting'

\begin{equation}
C_i\frac{(T^{*}-T^n_{i})}{\Delta
t}=H+R_{m}-\frac{\kappa_{i}}{h_{eff}}(T^{*}-T_{0})\label{6.37}
\end{equation}

or, equivalently,

\begin{equation}
\left(\frac{C_i}{\Delta t}+\frac{\kappa_{i}}{h_{eff}}\right)T^{*}
=H+R_{m}+\frac{C_{i}}{\Delta t}T^{n}_{i}+\frac{K_{i}}{h_{eff}}T_{0}\label{6.38}
\end{equation}

where $T^*$ is a
preliminary temperature.


\paragraph*{Cases:}

\begin{enumerate}
\item $T^* \le T_{0}$. In this case, the new surface temperature is 
given by $T^{n+1}_{i}=T^*$.

\item $T^* > T_{0}$. In this case, the new surface temperature is set to the
freezing point $T^{n+1}_{i}=T_{0}$ and the residual heat flux, i.e. the ice
melt

\begin{equation}
M_{i}=\left(\frac{C_i}{\Delta 
t}+\frac{\kappa_{i}}{h_{eff}}\right)(T^{*}-T_{0}), \label{6.39}
\end{equation}

is used in \eref{6.33} for calculating  $h_{i}$ at the next time step.
\end{enumerate}
\renewcommand{\labelenumi}{\arabic{enumi}.}

The snow depth on lake-ice ($h_{sni}$ in m) is obtained from the 
following budget
equation

\begin{equation}
\rho_{w}\dnd{h_{sni}}{t}=S+E_{sni}-M_{sni}\label{6.40}
\end{equation}

where $S$ is the snowfall rate, $E_{sni}<0$ the sublimation,  $E_{sni}>0$ the
deposition, and $M_{sni}$ the snow melt calculated analogous to
\eref{6.39}. Note that no
melting of ice is allowed unless the snow has vanished completely.


\section{Sea-ice}\label{s6.4}

Sea-ice concentration, $f_{si}$, is interpolated in time from
monthly observations while sea-ice thickness, $h_{si}$, is prescribed according
to $h_{si}$ = 2 m (1 m)  for the northern (southern) hemisphere, 
respectively, if
$f_{si} > 0.01$. Otherwise, $f_{si}$ = 0 and $h_{si}$ = 0. Sea-ice 
temperature is
calculated analogous to lake-ice temperature (c.f. \eref{6.36} to 
\eref{6.38}). The only
difference is that the freezing temperature of freshwater ($T_0$ =
0\grad) is
replaced by the freezing temperature of saline water, $T_{0s}$ = -1.77\grad.
Furthermore, a residual heat flux analogous to $M_{i}$ in \eref{6.39} 
is calculated
for diagnostic purposes (note that $h_{si}$ is not calculated but prescribed)
and snow on sea-ice is not taken into account.


\section{Coupling to mixed layer ocean}\label{s6.5}

As an option, the sea surface temperature, $SST$, and sea-ice
thickness, $h_{i}$, can be calculated in a way analogous to the lake
model (see section \ref{sec:lake}). The main differences are (i) that the
mixed layer depth is assumed to be larger ($h$ = 50 m) and (ii) that
the ocean heat transport has to be taken into account. The heat budget
of a constant-depth mixed layer can be written as

\begin{equation}
C_{m}\dnd{T_{m}}{t}=H-Q\label{6.41}
\end{equation}

where $T_{m}= SST$, $H$ = net surface heat flux (including all radiative and
turbulent fluxes), $Q$ = ocean heat transport and $C_{m}$ is the heat 
capacity of
the slab. While $Q$ is unkown, its monthly climatology, $Q_{clim}$, can be
derived from \eref{6.41} by replacing $T_{m}$ by the observed $SST$ 
climatology, $T_{clim}$,
and $H$ by its climatology, $H_{clim}$. For consistency, $H_{clim}$ 
has to be computed
from the uncoupled AGCM with $T_{clim}$ used as lower boundary forcing,
resulting in

\begin{equation}
Q_{clim}=H_{clim}-C_{m}\dnd{T_{clim}}{t}\label{6.42}.
\end{equation}

This simple approach of approximating $Q$ in \eref{6.41} by its
monthly climatology has the main advantage that systematic errors in
simulated SST are practically avoided while $SST$ variability is captured
through the variability of $H$. The main limitation is that variability of
ocean heat transports is neglected. The flux \eref{6.42} is applied also, with
appropriate area weighting, in those regions where sea-ice is observed,
according to climatology, except when the observed ice fraction is larger
than 0.9. In order to avoid excessive ice formation in the southern ocean,
a constant heat flux of 20 W/m$^2$ is applied to the sea-ice equation in
those regions.

\section{Surface albedo}\label{su6.5.1}

For snow-free land surfaces, an annual mean background
albedo $\alpha_{bg}$ has been derived from satellite data and values 
allocated to a
high-resolution global distribution of major ecosystem complexes
(Hagemann, 2002).
For most surfaces, the albedo of snow and ice is
assumed to be a linear function of surface temperature, ranging between a
minimum value at the melting point $T_{s}=T_{0}$  and a maximum value 
for `cold'
temperatures $T_{s}\le T_{0}-T_{d}$:

\begin{equation}
\alpha_{sn}=\alpha_{sn,min}+(\alpha_{sn,max}-\alpha_{sn,min}) 
f(T_{s})\label{6.43}
\end{equation}

where

\begin{equation}
f(T_{s})=\min\left\{\max\left[\left(\frac{T_{0}-T_{s}}{T_{0}-T_{d}}\right),0\right],1\right\}
\label{6.44}\end{equation}

with $T_{d}=T_{0}-1$ for sea-ice and lake-ice (with or
without snow) and $T_{d}=T_{0}-5$ for snow on land and ice sheets.

The minimum and maximum values of snow/ice albedos are assigned as given in table \ref{tab:suralb}.

\begin{table}[h]
\begin{center}
\begin{tabular}{|l|c|c|}\hline
Surface type &$a_{sn,min}$&$a_{sn,max}$\\ \hline\hline
land &	0.30&0.80\\ \hline
canopy&	0.20&	0.20\\ \hline
  land ice&	0.60&	0.80\\ \hline
  sea ice &	0.50&	0.75\\ \hline
lake ice&	0.50&	0.75\\ \hline
snow on lake ice&	0.60&	0.80\\ \hline
\end{tabular}
\end{center}
\caption{Snow and ice albedos for different surface types\label{tab:suralb}}
\end{table}

Over snow-covered land, the grid-mean albedo depends on a number of
parameters such as the fractional forest area, the leaf area index, the
bare-soil albedo, the snow albedo according to \eref{6.43}, and the fractional
snow cover of both the ground and the canopy (Roesch et al., 2001). The
fractional snow cover at the ground is a function of snow depth
($h_{sn}$ in m water equivalent) and slope of terrain approximated by 
the subgrid-scale
standard deviation of height $\sigma_{z}$

\begin{equation}
f_{sn}=\gamma_{1} \tanh (100 h_{sn})\sqrt{
\frac{1000 h_{sn}}{1000 h_{sn}+\gamma_{2}\sigma_{z}
+\epsilon}}\label{6.45}
\end{equation}

where $\gamma_{1}$=0.95, $\gamma_{2}$= 0.15 and $\epsilon$ is a
small number used to avoid division by zero for totally flat and snow-free
grid-cells.
The fractional snow cover of the canopy is defined as

\begin{equation}
f_{snc}=\min \left(1,\frac{h_{snc}}{h_{cin}}\right)\label{6.46}
\end{equation}
where $h_{snc}$ is the snow depth at the canopy and $h_{cin}$ the
interception capacity.  As in the Canadian Land Surface Scheme (CLASS;
\cite{verseghy91}), the albedo of snow-covered forests is related to a
sky view factor ($SVF$) which describes the degree of canopy
closure. The $SVF$ is related to the leaf area index $LAI$ by an
exponential function according to

\begin{equation}
SVF=e^{-(r LAI)}\label{6.47}
\end{equation}

with $r=1$ for both needleleaf and broadleaf trees. The total forest
albedo is computed as

\begin{equation}
\alpha_{for}=SVF \alpha_{g}+(1-SVF) \alpha_{can}\label{6.48}
\end{equation}

where $\alpha_{g}$ is the albedo of the ground

\begin{equation}
\alpha_{g}=f_{sn} \alpha_{sn}+(1-f_{sn}) \alpha_{bg}\label{6.49}
\end{equation}

and $\alpha_{can}$ the albedo of the canopy

\begin{equation}
\alpha_{can}=f_{snc}\alpha_{snc}+(1-f_{snc}) \alpha_{sfr}\label{6.50}
\end{equation}

  where $\alpha_{snc}=0.2$ is the albedo of the snow-covered part of
the canopy and $\alpha_{sfr}$ the albedo of the snow-free canopy. The
grid-mean albedo is finally given by

\begin{equation}
\alpha_{surf}=f_{for} \alpha_{for}+(1-f_{for}) \alpha_{g}\label{6.51}
\end{equation}

where $f_{for}$ is the fractional forest area. To be consistent with
the definition of the background albedo $\alpha_{bg}$
\citep{hagemann02}, we set
$\alpha_{sfr}=\alpha_{bg}$ so that $\alpha_{surf}=\alpha_{bg}$ for a
snow-free grid-cell.

Over water surfaces (lake, ocean), the albedo is set to a constant
value $\alpha_{sea}=0.07$. When a grid-cell is partially covered with
sea ice, the grid-mean albedo is defined as a weighted average

\begin{equation}
\alpha_{surf}=f_{sea} \alpha_{sea}+f_{ice} \alpha_{sn}\label{6.52}
\end{equation}


where $f_{sea}$ and $f_{ice}$ are the respective fractional areas and
$\alpha_{sn}$ is given by \eref{6.43}.  Note that the land fraction is
either 1 (100\% land) or zero, so that \eref{6.51} applies to all land
points while \eref{6.52} is used for all grid points covered with
water and/or ice.

