\section{Cumulus convection}\label{c8}

As in ECHAM4 and ECHAM5, a mass flux scheme \cite[]{tiedtke89} is applied for
cumulus convection with modifications for penetrative convection
according to \cite{nordeng94}. The contribution of cumulus convection
to the large scale budgets of heat, moisture and momentum is
represented by an ensemble of clouds consisting of updrafts and
donwdrafts in a steady state. The bulk equations for mass, heat,
moisture, cloud water and momentum for an ensemble of cumulus
updrafts are

\begin{eqnarray}
\dnd{M_{u}}{z}&=&E_{u}-D_{u}\label{8.1}\\
\dnd{}{z}(M_{u}s_{u})&=&E_{u}\overline{s}-D_{u}s_{u}+L\overline{\rho}c_{u}\label{8.2}\\
\dnd{}{z}(M_{u}q_{u})&=&E_{u}\overline{q}-D_{u}q_{u}-\overline{\rho}c_{u}\label{8.3}\\
\dnd{}{z}(M_{u}l_{u})&=&-D_{u}l_{u}+\overline{\rho}c_{u}-\overline{\rho}P_{u}\label{8.4}\\
\dnd{}{z}(M_{u}u_{u})&=&E_{u}\overline{u}-D_{u}u_{u}\label{8.5}\\
\dnd{}{z}(M_{u}v_{u})&=&E_{u}\overline{v}-D_{u}v_{u}\label{8.6}
\end{eqnarray}

where the subscript $u$ denotes updraft variables and the overbar
denotes large-scale variables. $E$ is entrainment, $D$ is detrainment,
$s=c_{p}T+gz$ the dry static energy, $\rho$ the air density, $q$ is specific
humidity, $l$ the cloud water mixing ratio, $c_{u}$ the release of latent
heat from condensation,  $P_{u}$ the conversion of cloud water to
precipition, and $ u $ and $ v $ are the components of the horizontal wind
vector. A corresponding set of equations is used for the cumulus
downdrafts which are assumed to originate from mixing of cloud air
with environmental air which has been cooled to its wet bulb
temperature by evaporation of precipitation generated in the
updrafts. The cloud water detrainment in \eref{8.4} is used as a source term
in the stratiform cloud water/ice equations (\ref{9.2}, \ref{9.3}).

\subsection{Organized entrainment}\label{c8.1}
In \cite{tiedtke89}, organized entrainment is consistent with the
closure and is based on a moisture convergence
hypothesis. \cite{nordeng94}, on the other hand, assumes organized
entrainment to take place as inflow of air into the cloud when cloud
parcels accelerate upwards, i.e. when the buoyancy is
positive. Organized detrainment takes place where the air decelerates,
i.e. when the buoyancy becomes negative. Organized entrainment and
detrainment are therefore related to the cloud activity
itself. Fractional entrainment and detrainment rates, $\epsilon_{i}$
and $\delta_{i}$ \cite[]{turner63}, are introduced so that for an
individual updraft $i$, $E_{i}=M_{i}\epsilon_{i}$ and
$D_{i}=M_{i}\delta_{i}$, and for the cloud ensemble

\begin{equation}
	E=M\epsilon=\sum_{i} M_{i}\epsilon_{i}=\sum_{i}E_{i}
	\label{8.7}
\end{equation}

\begin{equation}
	D=M\delta=\sum_{i} M_{i}\delta_{i}=\sum_{i}D_{i}
	\label{8.8}
\end{equation}

where the cloud ensemble mass flux is defined as

\begin{equation}
	M=\sum_{i} M_{i}=\sum_{i}\overline{\rho}\sigma_{i}w_{i}
	\label{8.9}
\end{equation}

with fractional area $\sigma_{i}$ and vertical velocity $w_{i}$. Equation
\eref{8.1} can then
be expressed as

\begin{equation}
\frac{1}{M} \dnd{M}{z}=\epsilon - \delta \label{8.10}
\end{equation}

where the subscript $u$ denoting the updraft has been omitted for
convenience (in the following as well). According to \cite{simpson69},
the steady state vertical momentum equation for an individual updraft
is given by

\begin{equation}
w_{i}\dnd{w_{i}}{z}=b_{i}-\epsilon_{i}w^2_{i}\label{8.11}
\end{equation}


where $b_{i}$ is the buoyancy term which may include water loading and
non-hydrostatic effects. Assuming that the fractional area
$\sigma_{i}$ of each individual updraft is constant with height
(except in the ouflow part, see later), organized entrainment,
according to \eref{8.9} and
\eref{8.10},
can be written as

\begin{equation}
\epsilon_{i}=\frac{1}{M_{i}}\dnd{M_{i}}{z}=\frac{1}{w_{i}}\dnd{w_{i}}{z}+
\frac{1}{\overline{\rho}}\dnd{\overline{\rho}}{z}
\label{8.12}
\end{equation}


whenever the buoyancy is positive ($\delta_{i} = 0$). By integrating 
\eref{8.11} upwards, starting at cloud base ($z = 0$), and using \eref{8.7}, 
\eref{8.11} and
\eref{8.12}, the organized entrainment rate of the cloud ensemble becomes

\begin{equation}
\epsilon=\frac{b}{2\left(w^2_{0}+\int_{o}^z
bdz\right)}+\frac{1}{\overline{\rho}}\dnd{\overline{\rho}}{z}
\label{8.13}
\end{equation}

with the ensemble buoyancy
$b=\frac{g}{\overline{T}_{v}}(T_{v}-\overline{T}_{v})-gl$.

\subsection{Organized detrainment}\label{s8.2}

Organized detrainment is defined as the loss of total massflux due to
detrainment of those clouds which are losing their buoyancy, i.e.

\begin{equation}
D=E-\dnd{M}{z}=\sum_{i}\overline{\rho}\sigma_{i}w_{i}\left(\frac{1}{w_{i}}
\dnd{w_{i}}{z}+\frac{1}{\overline{\rho}}\dnd{\overline{\rho}}{z}\right)
-\dnd{}{z}\sum_{i}\overline{\rho}\sigma_{i}w_{i}=
-\sum_{i}\overline{\rho}w_{i}\dnd{\sigma_{i}}{z}
\label{8.14}
\end{equation}

Since the fractional area of each individual member of the ensemble
is assumed to be constant with height, except for the detrainment
level, the only contribution to the sum in \eref{8.14} comes from those
members of the ensemble which detrain at this level ($k$), i.e.,

\begin{equation}
D=-\sum_{k}\overline{\rho}w_{k}\dnd{\sigma_{k}}{z}\approx
\frac{\overline{\rho}\sigma_{k}w_{k}}{\Delta z}=\frac{M_{k}}{\Delta z}
\label{8.15}
\end{equation}

where $\Delta z$ is the depth over which the detrainment takes place. Thus,
organized detrainment is equal to the change of mass flux with
height. Since the in-cloud vertical velocities are primarily a
function of the height above cloud base and, hence, $w_{k}\approx w$, 
and due to the
assumption that individual clouds do not change their area fraction
before they start to detrain, the individual cloud cover change is
equal to the total, i.e.,

\begin{equation}
\dnd{\sigma_{k}}{z}=\dnd{\sigma}{z}
\label{8.16}
\end{equation}

so that, according to \eref{8.15} and \eref{8.16}, the organized 
detrainment may be
parameterized as

\begin{equation}
D=-\frac{M}{\sigma}\dnd{\sigma}{z}
\label{8.17}
\end{equation}

It remains to determine the variation of cloud cover with height.
Having obtained the level where clouds start to detrain ($z_{d}$), an
analytical function $\sigma = \sigma(z)$ is specified with boundary values
  $\sigma (z_{d})= \sigma_{0}$ and $\sigma = (z_{t})= \sigma_{0}$,
  where $z_{t}$ is the highest possible cloud level
obtained from undiluted ascent starting at cloud base. In the
parameterization, the spectrum of clouds detraining at different
levels is realized through the following function

\begin{equation}
\sigma (z)=\sigma_{0}\cos\left[\frac{\pi}{2}\frac{(z-z_{d})}{(z_{t}-z_{d})}
\right]
\label{8.18}
\end{equation}

Except for being continuous at $z = z_{d}$, and satisfying the boundary
conditions specified above, there is no physical reason for chosing
this particular function.

\subsection{Adjustment closure}\label{s8.3}

The adjustment-type closure suggested by \cite{nordeng94} relates the
cloud base mass flux to convective instability. The dominant part of
convective heating and drying, respectively, is due to compensating
subsidence in the environment \cite{fritsch80}

\begin{eqnarray}
\dnd{\overline{T}}{t}&\approx&\frac{1}{\overline{\rho}
c_{p}}M\dnd{\overline{s}}{z}\label{8.19}\\
\dnd{\overline{q}}{t}&\approx&\frac{1}{\overline{\rho}}M\dnd{\overline{q}}{z}
\label{8.20}
\end{eqnarray}

where $M$ is the massflux.

Convective activity is expressed in terms of $CAPE$ (convective
available potential energy) which is estimated from the parcel ascent
incorporating the effects of water loading,

\begin{equation}
CAPE =
\int_{base}^{top}\left(\frac{g}{\overline{T}_{v}}\left[T_{v}-\overline{T}_{v}\right]
-gl\right)dz
\label{8.21}
\end{equation}

where cloud ensemble values are used for $T_{v}$ and $l$. The change of $CAPE$
due to convective heating/moistening is approximated by

\begin{equation}
\dnd{}{t}CAPE \approx -
\int_{base}^{top}\frac{g}{\overline{T}_{v}}\dnd{\overline{T}_{v}}{t}dz=-M_{B}
\int_{base}^{top}\left(\frac{\left[1+\delta \overline{q}\right]}{c_{p}
\overline{T}_{v}}\dnd{\ovl{s}}{z}+\delta\dnd{\ovl{q}}{z}\right)\eta\frac{g}{\ovl{\rho}}dz
\label{8.22}
\end{equation}

with normalized mass flux $\eta$ defined as $M=M_{B}\cdot \eta(z)$  where
$M_{B}$ is the cloud base mass
flux. By assuming that convection acts to reduce $CAPE$ towards zero
over a specified time scale $\tau$, the time rate of change is
approximated by

\begin{equation}
\dnd{}{t}CAPE \approx - \frac{CAPE}{\tau}
\label{8.23}
\end{equation}

so that the cloud base mass flux can be obtained from \eref{8.22} and
\eref{8.23}
according to

\begin{equation}
M_{B}=\frac{CAPE}{\tau} \left\{
\int_{base}^{top}\left[\frac{(1+\delta \overline{q})}{c_{p}
\overline{T}_{v}}\dnd{\ovl{s}}{z}+\delta\dnd{\ovl{q}}{z}\right]
\eta\frac{g}{\ovl{\rho}}dz\right\}^{-1}.
\label{8.24}
\end{equation}

Since $\eta$  is not known before the total mass flux is known, $CAPE$ is
estimated through a first guess $M_{B}=M_{B}^*$ obtained from first 
applying the
moisture convergence scheme. Thus, the cloud base mass can finally be
written as

\begin{equation}
M_{B}=\frac{CAPE}{\tau}=M_{B}^* \left\{
\int_{base}^{top}\left[\frac{(1+\delta \overline{q})}{c_{p}
\overline{T}_{v}}\dnd{\ovl{s}}{z}+\delta\dnd{}{z}\ovl{q}\right]M^*
\frac{g}{\ovl{\rho}}dz\right\}^{-1}.
\label{8.25}
\end{equation}

Following \cite{nordeng94}, who argued that $\tau$ should be smaller
(larger) with increasing (decreasing) horizontal resolution, we apply
an algorithm similar to that used in the ECMWF model, $\tau[s]=\min
(3\cdot 3600, 2\cdot 3600\cdot 63/N$), where $N$
denotes the spectral resolution. 






