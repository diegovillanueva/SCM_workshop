
\section{Slab ocean}

As an option, the sea surface temperatures, $SSTs$, and sea-ice
thickness, $h_{i}$, can be calculated using a slab ocean model. 
The $SSTs$ are computed from the ocean surface heat balance. The slab ocean model 
represents the surface ocean as a slab of specified depth $h$. 
In the model the mixed layer depth $h$ is set to $50m$.
The slab ocean model is a thermodynamical model that does not contain any 
explicit computation of ocean dynamics. The ocean dynamics are represented 
by a prescribed ocean heat transport divergence which may also be interpreted as a heat-flux correction.
The prognostic variable in the ocean is the mixed layer temperature $T_m$.
The heat budget of a constant-depth mixed layer $h$ can be written as:
\begin{equation}
C_{m}\frac{\partial T_{m}}{\partial t}=H-F_O
\label{eqn:energy_equation_so}
\end{equation}
where $T_{m}$ is the sea surface temperature $SST$, $H$ = net surface heat flux (including all radiative and
turbulent fluxes), $F_O$ = the divergence of ocean heat transport and $C_{m}=\rho_{sea} h c_{sea}$ is the heat 
capacity of the slab. The sea water density $\rho_{sea}$ is set to $1025\,kg\,m^{-3}$ and the heat capacity of sea water $c_{sea}$
to $3994\,J\,kg^{-1}\,K^{-1}$. 
While $F_O$ is unknown, its monthly climatology, $F_{clim}$, can be
derived from equation \ref{eqn:energy_equation_so} by replacing $T_{m}$ by the observed $SST$ 
climatology, $T_{clim}$,
and $H$ by its climatology, $H_{clim}$. For consistency, $H_{clim}$ 
has to be computed
from the uncoupled AGCM with $T_{clim}$ used as lower boundary forcing,
resulting in
\begin{equation}
F_{clim}=H_{clim}-C_{m}\frac{\partial T_{clim}}{\partial t}.
\label{eqn:energy_equation_clim_so}
\end{equation}
This simple approach of approximating $F_O$ in equation \ref{eqn:energy_equation_so} by its
monthly climatology has the main advantage that systematic errors in
simulated $SSTs$ are practically avoided while $SST$ variability is captured
through the variability of $H$. The main limitation is that variability of
ocean heat transport is neglected. 
The heat flux divergence $F_{clim}$ is applied also, with
appropriate area weighting, in those regions where sea-ice is observed,
according to climatology, except when the observed ice fraction is larger than 0.9.
Then the heat flux divergence $F_{clim}$ is set to zero.
To compute the net climatological surface heat flux $H_{clim}$ 
a 'long' (10-20 years) standalone ECHAM6 simulation has to be performed forced by
prescribed climatological SSTs. Please note that if changes are made to ECHAM6 physical or dynamical parameters, a new heat-flux climatology
has to be computed.\\  
Equations \ref{eqn:energy_equation_so} and \ref{eqn:energy_equation_clim_so} give:
\begin{equation}
C_{m}\frac{\partial T_{m}}{\partial t}=H-F_O\approx H-F_{clim}=H-H_{clim}+C_{m}\frac{\partial T_{clim}}{\partial t}.
\end{equation}
This may be written as:
\begin{equation}
C_{m}\frac{\partial (T_{m}-T_{clim})}{\partial t}=H-H_{clim}.
\end{equation}
Thus, the model predicts only the deviations from the observed seasonal cycle of SST, forced by
anomalies in the net surface heat fluxes. Therefore, the mean (long-term averaged) climatological
seasonal cycle of SST as simulated by the slab ocean model is similar to the observed one. 
\\
The term $F_{clim}$ in equation  \ref{eqn:energy_equation_clim_so} may also be interpreted as a "heat-flux correction" $Q$ of the slab ocean model, i.\,e.\,
\begin{equation}
C_{m}\frac{\partial T_{m}}{\partial t}=H-Q
\end{equation}
with 
\begin{equation}
Q=F_{clim}=H_{clim}-C_{m}\frac{\partial T_{clim}}{\partial t}.
\end{equation}
The heat-flux correction $Q$ has to be provided to the slab ocean model in every month (each year the same)
and is then interpolated in time onto the respective time step (in the same way as for all other surface
variables that are prescribed as monthly means).\vspace{2cm}

\noindent For more details see subroutines {\em ml\_ocean.f90} and {\em ml\_flux.f90}.


