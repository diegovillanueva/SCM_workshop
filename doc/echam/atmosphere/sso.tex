\section{Parameterized gravity wave drag from subgrid scale orography}

The Subgrid Scale Orographic Parameterization (hereafter SSOP)
developed by \cite{lott97} and \cite{lott99} that has been implemented
in the \echam{} model is aimed at representing the effects of orographic
variations on scales smaller than the typical horizontal resolution of
a climate model.

The orography may affect the atmospheric flow in many ways. The SSOP
considered in \echam{} takes into account two main mechanisms of
interaction between the orography and the atmospheric flow:

\begin{enumerate}
\item momentum transfer from the earth to the atmosphere accomplished by
      orographic gravity waves and 
\item the drag exerted by the subgrid scale mountains when the air flow 
      is blocked at low levels. 
\end{enumerate}

The part of the SSOP concerning the propagation and dissipation of the
orographic gravity waves follows the formulation of \cite{palmer86}
and \cite{miller89}. In addition, the SSOP has options for including
the effects of low level trapped lee waves and of subgrid scale
orographic lift \cite[]{lott99}.

Concerning the specification of the gravity wave forcing, the SSOP
includes a relatively detailed description of the subgrid scale
orography (based on the work of \cite{baines90}) in order to
take into account anisotropic orography and directional effects.


\subsection{\label{suboro}Representation of the subgrid scale orography}
At one gridpoint, it is assumed that the subgrid scale orography can
be described by seven parameters: the standard deviation $\mu$, the
anisotropy $\gamma$, the slope $\sigma$, the orientation $\theta$, the
minimum $Z_{min}$, the maximum $Z_{max}$, and the mean $Z_{mea}$
elevation of the orography.

These parameters are evaluated offline for each gridpoint from the US
Navy (10'x10') topographic dataset. The last three parameters are
taken directly from the US Navy data set (for each horizontal model
resolution considered), while the first four parameters are derived
from topographic gradients relationships as formulated by
\cite{baines90}. In order to derive relationships between the low
level flow and the orientation of the orography, it is assumed that
the subgrid scale orography has the shape of an elliptical
mountain. Thereafter, in each gridbox a typical number of elliptical
mountains is considered.  For a brief description and additional
references see \cite{lott97}.

Concerning the large scale flow, it is assumed that the model mean
orography is the optimal representation (e.g. the so-called envelope
orography used sometimes in low resolution models is excluded).


\subsection{Gravity wave drag from subgrid-scale orography}

In case that the subgrid scale mountains are high enough, the vertical
motion of the air is limited and part of the low level flow (below the
mountain top) is blocked and a drag should be provided at the model
levels that intersect the subgrid scale orography (the so-called low
level drag). Given a non-dimensional height of the mountain:
$H_{n}=NH/U$, where $H$ is the maximum elevation of the mountain, $U$
the wind speed and $N$ the Brunt-V\"is\"la frequency, it can be shown
theoretically that part of the low level flow is blocked for
$H_{n}>>1$. For $H_{n}<<1$ all the flow goes over the mountain and
gravity waves are generated by the vertical motion of the air
\cite[]{lott97}.


In the parameterization, it is distinguished between the incident
flow, flowing above the mountain and forcing the gravity waves, and
the blocked flow, associated with the low level drag. The incident
flow is defined as the average of the wind speed, the Brunt-V\"is\"la
frequency, and the fluid density between the model ground, $Z_{mea}$,
and the mountain peak, $Z_{max}$. This flow is referenced as $U_{H}$,
$N_{H}$, and $\rho _{H}$, respectively. Concerning the blocked flow,
the parameter of interest is the height of the blocked flow, $Z_{b}$
defined as the highest level that satisfies the condition:

\begin{equation}
\int ^{Z_{max}}_{Z_{b}}\frac{N}{U_{p}}dz\leq H_{NC}
\end{equation}

where the wind speed $U_{p}$ is calculated by resolving the wind
$\overrightarrow{U}$ in the direction of the incident flow.  The
parameter $H_{NC}$ tunes the depth of the blocked flow layer and is of
order one.

The low level drag for each layer below $Z_{b}$ is given by:

\begin{equation}
\overrightarrow{D}_{b}(z)=-\rho C_{d}max\left( 2-\frac{1}{r},0\right) 
\frac{\sigma }{2\mu }\left( \frac{Z_{b}-z}{Z_{b}-Z_{mea}}\right) 
\left( B\cos ^{2}\psi _{H}+C\sin ^{2}\psi _{H}\right) 
\frac{\overrightarrow{U}\left| \overrightarrow{U}\right| }{2}
\end{equation}


where $\psi _{H}$ is the angle between the incident flow and the
normal orographic ridge direction, the constants $B$ and $C$ are
functions of the anisotropy, and $r$ is the aspect ratio of the ridge
as seen by the incident flow (see \cite{lott97} and \cite{lott99}).

If there is low level flow blocking, it is therefore assumed that the
effective height, $Z_{eff}$, of the orography seen by the atmospheric
flow is reduced to $Z_{max}-Z_{b}$. In  case that there is no low
level flow blocking, $Z_{max}-Z_{min}$ is instead used as effective
height.

Taking into account the difference in orientation between a orographic
ridge and the incident flow and the typical number of ridges within a
gridbox, the gravity wave stress at the source level is given by:

\begin{equation}
\tau =\rho _{H}GU_{H}N_{H}Z_{eff}^{2}\frac{\sigma }{4\mu }\left|
\overrightarrow{P}\right| 
\end{equation} 

where the parameter $G$ tunes the gravity wave stress amplitude and is
of order one. The directional vector $\overrightarrow{P}$ is given by:

\begin{equation} 
\overrightarrow{P}=(B\cos
^{2}\psi _{H}+C\sin ^{2}\psi _{H};\left( B-C\right) \sin \psi _{H}\cos
\psi _{H})
\end{equation}

\subsection{Gravity wave drag}

Above the source level the gravity wave stress is constant, except
when the waves encounter a critical level or when they break. Given
that the gravity wave drag is the vertical derivative of the gravity
wave stress, the gravity waves produce a drag on the resolved flow
only when a critical level is reached or when they become unstable and
break, in agreement with wave-mean flow theory.

A critical level is encountered when the background wind turns with
height so that it becomes zero in the plane of the low level stress.
If this happens, the gravity wave stress is set to zero at that level.

The part of the SSOP that concerns gravity wave breaking follows the
original formulation of \cite{palmer86}, that uses a breaking
condition based on the Richardson number and the \cite{lindzen81a}
saturation hypothesis to determine the stress at the breaking level.

At each model level a minimum Richardson number that includes the
gravity wave influence on the static stability and wind shear is
evaluated:

\begin{equation}
Ri_{min}=Ri\frac{1-(N\delta h/U)}{\left\{ 1+Ri^{1/2}(N\delta hU)\right\} ^{2}}
\end{equation}

where $Ri=(N/(dU/dz))^{2}$ is the background (resolved) flow
Richardson number, $N$ the background static stability, $U$ the
background wind speed (derived from the projection of the wind vector
in the plane of the low level stress), and $\delta $h is the amplitude
of the vertical displacement induced by the gravity waves. $\delta h$
is derived following a steady two dimensional model of gravity waves
and is given by:

\begin{equation}
\delta h^{2}=G\frac{\rho _{H}N_{H}U_{H}}{\rho NU}Z^{2}_{eff}
\end{equation}

$Ri_{min}$ is a lower bound (hence 'minimum') to the Richardson
number, in the sense that it is the minimum value that can be
anticipated from a steady two dimensional model of gravity wave
propagation \cite[]{palmer86}.

It is assumed that instability occurs if $Ri_{min}<Ri_{c}$, where
$Ri_{c}$ is the critical Richardson number equal to 0.25. This
condition takes into account the occurrence of both convective
overturning and shear instability.

If the critical Richardson number is reached, the waves are assumed to
saturate: their amplitude is limited to the value at which instability
occurs \cite[]{lindzen81a}. The wave vertical displacement is therefore
computed from the $Ri_{min}$ equation with $Ri_{min}=Ri_{c}$.  This
vertical displacement, $\varepsilon $, is thereafter used in the
gravity wave stress at the breaking height, the saturation stress:

\begin{equation}
\tau _{s}=\rho \frac{U^{3}}{N}\varepsilon ^{2}\frac{\sigma }{2\mu }
\end{equation}

The equation for the saturation stress is obtained following
\cite{lindzen81a}. Thereafter, above the breaking level, the gravity
wave stress remains constant and equal to the saturation stress, if
the condition for instability is not reached again.

In the parameterization, the calculation of the gravity wave stress
proceeds from the bottom to the top of the vertical column. The
procedure of evaluating the Richardson number and the search for
instability is therefore applied from the bottom up, and can produce
more than one breaking level.

