In the course of evaluating the MPI--ESM and ECHAM6 simulations
a number of  bugs have been identified which impact the
simulations. We list these bugs here and give a description of their
effects.

\section{Bug in anthropogenic aerosol data set}
In implementing the new aerosol climatology (see
Section~\ref{sec.tropaerosols}) a data formatting error
led to a somewhat weaker anthropogenic aerosol forcing than was
foreseen in the original data set, with the effect most pronounced
over the heavily populated regions of the northern hemispheric
continents. The adjusted all sky aerosol forcing for the AMIP
period, calculated as the difference between the top--of--atmosphere
fluxes for the AMIP period with the aerosol load for this period
(including the formatting bug) and a run for the same period but with
the pre--industrial tropospheric aerosol loading is \unit[-0.34]{W
  m$^{-2}$}.  If the calculation is repeated but with the formatting
bug removed the adjusted forcing increases to \unit[-0.50]{W
  m$^{-2}$}.  For reference, the difference in the clear sky shortwave
adjusted forcing between the two simulations is nearly three times as
large (\unit[0.42]{W m$^{-2}$} ) suggesting that much of the missing
forcing attributable to the formatting error is offset by additional
adjustments, compensating effects in the long wave, and cloud masking
effects. Use of the correct aerosol only has a small impact in the
representation of the clear sky reflected solar irradiance, decreasing
the root mean square error relative to CERES from 6.6 to \unit[6.5]{W
  m$^{-2}$}. This error is present in the external data set version
{\tt r0001} only, the newer versions have corrected aerosol data.

\section{Energy conservation violation}
In preparing \echam.2 emphasis was given to achieving energy
conservation in the physics and surface coupling. This was achieved
through a number of bug-fixes, including inconsistent use of specific
heats, the coupling between convection and cloud schemes,
discretization errors in the convection and turbulent diffusion
schemes. After these corrections the atmospheric physics conserve
energy, and the surface energy budget is consistent. However, the
atmosphere still leaks a small amount of energy, presumably due to
issues in the dynamical core.  
