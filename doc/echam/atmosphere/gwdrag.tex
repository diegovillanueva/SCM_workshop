\section{Parameterization of the momentum flux deposition due to a gravity
         wave spectrum}\label{c7}


The parameterization of the effects of a gravity wave spectrum is
based on the Doppler spread theory of propagation and dissipation of a
gravity wave spectrum as formulated by
\cite{hines91a,hines91b,hines91c,hines93}. The simplifications to the
Doppler spread theory necessary for developing the parameterization
are discussed in \cite{hines97a,hines97b}. Here the version of the
parameterization formulated following \cite{hines97a,hines97b} that
has been implemented in the middle atmosphere (MA) version of
ECHAM5 (\cite{manzini2006,giorgetta2006}) is presented. 
The impact of the Doppler spread parameterization in the
middle atmosphere of the previous model cycle (MAECHAM4) has been
discussed by \cite{manzini97} and \cite{manzini98}.

\subsection{Hines Doppler spread theory}\label{s7.2}

During the last decades, an increasing number of observations by a
variety of techniques has contributed to the current characterization
of atmospheric gravity waves (among others,
\cite{allen95,eckermann94,hamilton91,smith87,dewan86,hirota84}). 
The forcing mechanisms generating the gravity waves are most likely
located in the troposphere, and may include convective activity, shear
instabilities, frontal systems, transient flow over topography. In the
middle atmosphere, the gravity waves appear to propagate predominantly
upward from their tropospheric source regions and they appear to form
a broad background spectrum of waves.


On the basis of these observations, the Hines Doppler spread theory
(henceforth, HDST) assumes an interacting and upward propagating
gravity wave spectrum with tropospheric sources. The HDST does not
directly deal with the forcing mechanisms of the gravity waves.  The
HDST considered in the current parameterization only assumes that the
variety of the forcing mechanisms gives rise to a broad band and
continuous spectrum. Within this framework, quasi monochromatic waves
cannot be take into account, although extensions are possible \citep{hines97b}.


Evidence that the vertically propagating gravity waves are dissipating
and therefore interacting with the large scale flow has been derived
form the observations that the amplitude of the spectrum at high
vertical wavenumbers tends not to change with height, in spite of the
growth expected in response to the decrease in atmospheric density, a
behavior usually termed saturation \citep{smith87,dewan86}. A variety
of processes can be responsible for saturation (see
\cite{fritts84,fritts89} for reviews). According to the HDST, as the
gravity waves propagate upwards from the troposphere to the
mesosphere, an essential role in the saturation process is played be
the nonlinear advective interaction exerted on each wave component of
the spectrum by the gravity wave wind field induced by the other
waves.


The HDST builds upon gravity wave theory for small amplitude waves
propagating in a mean flow that is uniform horizontally and temporally
and slowly varying in altitude (for a review of gravity wave theory see
\cite{andrews87}). The core aspects of the Hines Doppler spread
theory \citep{hines91a,hines91b,hines91c,hines93} are here briefly
summarized:


\begin{enumerate}
\item The effects of the wind field induced by the other waves on each
      wave component of the spectrum at a given height are assumed to
      be approximately determined by treating the wave induced wind
      field as a background horizontal wind field, slowly varying in
      altitude while horizontally and temporally uniform. As the waves
      propagate upward, their vertical wavenumber spectrum is
      therefore modified, and in turn the spectral characteristics of
      the wave induced wind field are also modified. In a statistical
      sense, the modification induced by the advective nonlinear
      interaction is found to be a Doppler spreading of the vertical
      wavenumber spectrum towards high wavenumbers.\label{i7.1}

\item In conformity with wave theory, wave action density is conserved
      as the waves propagate upwards, until the gravity wave spectrum
      as a whole becomes unstable and the waves at high vertical
      wavenumbers break down into turbulence. For parameterization
      purposes the transition to turbulence is taken to occur at a
      specific vertical wavenumber, the \emph{maximum permissible
      vertical wavenumber $m_{M}$}, and the breaking waves (with $m >
      m_{M}$) are removed from the spectrum. Within the HDST it is
      possible to assume a more complex and smooth transition to
      turbulence. The momentum carried by the waves that have been
      removed is deposited into the large scale background
      flow.\label{i7.2}


\item In agreement with wave-mean flow interaction theory \citep{andrews87},
      within the HDST formulation the background large
      scale flow has the effect of producing Doppler shifts of the
      vertical wavenumber. In the presence of a background large scale
      flow, differential momentum flux deposition (hence forcing of
      the large scale flow) can therefore occur also for an isotropic
      gravity wave spectrum. Consequently, it becomes necessary to
      take into account the variations in the azimuth of wave
      propagation. Note that in the absence of a background large
      scale flow and for an isotropic gravity wave spectrum, the
      momentum flux would be deposited isotropically and no \emph{net}
      deposition of momentum flux would occur.\label{i7.3}


\item An aspect of the Hines formulation crucial to the practical
      development of a parameterization is that at any given height,
      the spectral characteristics of the gravity waves are determined
      by a modification of the gravity wave spectrum at a specified
      low altitude. The calculation of the spectral evolution with
      altitude is therefore by-passed in the parameterization, by
      keeping track of the portion of the gravity wave spectrum at the
      specified low altitude which continue to propagate upward. The
      largest vertical wavenumber of the spectrum at the specified low
      altitude that continues to propagate upward at the current
      height of interest is called the \emph{cutoff vertical
      wavenumber $m_{C}$}.  The vertical evolution of $m_{C}$ is the
      key computation of the parameterization. A drastic reduction of
      the gravity wave quantities describing the vertical evolution of
      the gravity wave spectrum is therefore achieved, a requirement
      for any parameterization of practical use in a general
      circulation model.\label{i7.4}
\end{enumerate}


\subsection{The Hines Doppler Spread Parameterization (HDSP)}\label{s7.3}

The quantity that has to be evaluated is the deposition of the
horizontal momentum transferred by the vertically propagating gravity
waves (what is referred to as momentum flux deposition).  As commonly
done in a general circulation model, only vertical propagation is
considered, assuming that a gridbox is large enough that oblique
propagation (outside the vertical column) can be neglected. For each
gridbox, the dependence in the azimuth of wave propagation must be
discretized: the total number of azimuths considered is defined to be
$J$. It is assumed (although not necessary) that the $J$ azimuths are
equally spaced around the azimuth circle.


Within this framework, the HDSP requires the specification of the
input gravity wave spectrum at some low altitude (within the forcing
region). Thereafter, the momentum flux deposition is determined in
function of the large scale flow and buoyancy frequency, the input
gravity wave spectrum at a specified low altitude, and a limited
number of height varying gravity wave related quantities, the most
important being the horizontal wind variance and the cutoff vertical
wavenumber. These quantities are defined and derived below.

At any given height, the broad band gravity wave spectrum is
characterized by the power spectral density $H_{j}^2$ of the
horizontal winds associated with the gravity waves at that height in
the $j$-th azimuth. The power spectral density is a function of
horizontal wavenumber $k$ (a directional wavenumber in the
$j$-azimuth) and vertical wavenumber $m$. The spectrum is assumed to
be separable in $k$ and $m$. For convenience, $k$ and $m$ are made
positive for upward propagating waves.


The \emph{horizontal wind variance} at the height of interest that is
contributed by the waves propagating in the $j$-th azimuth is the
integral over all positive horizontal and vertical wavenumbers of the
power spectral density:

\begin{equation}
s_j^2=\int_{0}^{m_{M}}\int_{0}^{\infty}H_{j}^{2}
dkdm\label{7.1}
\end{equation}

The integral in the vertical wavenumber is limited by the maximum
permissible vertical wavenumber $m_{M}$ (see point \ref{i7.2} in
section \ref{s7.2}). The $s^2_{j}$ are derived in section
\ref{s7.3.2}.


At any given height, the \emph{total rms horizontal wind speed}
$\sigma_{T}$ is contributed by gravity waves propagating in all
azimuths:

\begin{equation}
\sigma_{T}=\left(\sum_{j=1}^J s_{j}^2\right)^{1/2}
\label{7.2}
\end{equation}

At any given height, the \emph{total rms horizontal wind speed
$\sigma_{j}$ in the $j$-th azimuth} depends on the variance from waves
in the $j$-th azimuth and in all other azimuths, non-orthogonal to the
$j$-th direction. These contributions must be added up, and they are
found by projecting the $s^2_{j}$ wind variances on the azimuth of
interest. The total rms horizontal wind speed $\sigma_{j}$ in the
$j$-th azimuth is:

\begin{equation}
\sigma_{j}=\left(\sum_{p=1}^J s_{p}^2\cos 
(\alpha_{p}-\alpha_{j})^{2}\right)^{1/2}
\label{7.3}
\end{equation}

where $\alpha_{p}$ and $\alpha_{j}$ are respectively the $p$- and
$j$-th azimuth.

The height where the gravity wave spectrum is specified is defined to
be the \emph{initial (or launching) height}, and any gravity wave
quantity at the initial height is given the subscript $I$.


\subsubsection{Cutoff vertical wavenumber}\label{su7.3.1}

At any given height, the dispersion relation for an individual gravity
wave with azimuth $j$ in a background flow that is horizontally and
temporally uniform, is:

\begin{equation}
\omega/k=N/m+V_{j}+v_{j}
\label{7.4}
\end{equation}


where $k$ is the horizontal wavenumber, $ m $ the vertical wavenumber,
$\omega$ the ground based frequency, $N$ the buoyancy frequency, and
$V_{j}$ and $ v_{j}$ are respectively the large scale background flow
and the wave induced wind field in the $j$-th azimuth. For
convenience, $ k$, $\omega$ and $ m $ are made positive for upward
propagating waves. Given that $\omega$ and $k$ are height independent,
from the combination of equation \eref{7.4} as written for the initial
height and for some overlying height of interest, it is obtained:

\begin{equation}
N/m= N_{I}/m_{I}+ V_{jI}- V_{j} - v_{j}
\label{7.5}
\end{equation}


assuming that the induced wave field at the initial height is
negligible.  Equation \eref{7.5} expresses the mapping between the
vertical wavenumber $m$ at the height of interest and the
corresponding vertical wavenumber $m_{I}$ at the initial height. In
equation \eref{7.5}, $N_{I}$ and $V_{jI}$ are respectively the
buoyancy frequency and the $j$-directed large scale background flow at
the initial height.


Equation \eref{7.5} shows that as $V_{j}+v_{j}$ increases, the
vertical wavenumber $m$ is Doppler shifted to infinity and into
negative values. Before reaching negative values, at sufficiently
large vertical wavenumbers, the spectrum is likely to become unstable
and dissipative processes are likely to take place (the vertical
wavelength is reduced, critical level conditions are approached). In
practice, it is assumed that this transition occurs at a specific
vertical wavenumber $m_{M}$ (large, positive and less than infinity),
the maximum permissible vertical wavenumber of the spectrum at the
height of interest (already introduced, see point \ref{i7.2} in
section \ref{7.2}). The waves with wavenumbers equal or larger than
$m_{M}$ are supposed to be dissipated and are removed from the
spectrum. The vertical wavenumber $m_{M}$ may be reached by a wave
when the wave induced wind field $ v_{j} $ increases to the value:

\begin{equation}
  v_{jM}=N_{I}/m_{I}-N/m_{M}+ V_{jI}- V_{j}
\label{7.6}
\end{equation}

The probability for the induced wind field to meet condition
\eref{7.6} was first derived in \cite{hines93} for the case of no large
scale background flow and $m_M$ equal to infinity, in order to
determine the cutoff vertical wavenumber $m_{C}$ (see point \ref{i7.4}
in section \ref{s7.2}). In \cite{hines93} it was found that the
probability for a wave to survive to some height decreases rapidly as
$m_I$ enters a particular critical range. On the basis of this rapid
transition and further approximations \citep{hines97a}, an expression
for $v_{jM}$ is found in order to evaluate a provisional (i.e.,
subject to two conditions expressed below) cutoff vertical wavenumber
in the $j$-th azimuth for the general case of a positive and finite
$m_M$ and a nonzero background flow:

\begin{equation}
\{m_{j}\}_{TRIAL} = N_{I} (N/m_{M} + V_{j} - V_{jI} +
\Phi_{1}\sigma_{j})^{-1}
\label{7.7}
\end{equation}


with $ v_{jM}$ expressed in terms of the total rms wind speed in the
$j$-th azimuth $\sigma_{i}$. The coefficient $\Phi_{1}$ that appears
in \eref{7.7} is a nondimensional factor that lays in the range: $ 1.2
< \Phi_{1} < 1.9$, deduced in \cite{hines93,hines97a}. In \eref{7.7}
the cutoff vertical wavenumber $m_{C}$ is a function of azimuth and is
denoted $m_{j}$.

The maximum permissible wavenumber $m_{M}$ was determined in
\cite{hines91b} by the condition of marginal instability of the total
wave system. In \cite{hines97a} the derivation is extended by
approximation to the case of a nonzero background flow, so that:

\begin{equation}
N/m_{M}=\Phi_{2}\sigma_{T}
\label{7.8}
\end{equation}


where $\Phi_{2}$ is a second nondimensional factor that lays in the
range: $0.1 < \Phi_{2} < 0.4$ deduced in \cite{hines91b,hines97a}. The
limits of $\Phi_{2}$ are intended to roughly correspond to 17\% or 8\%
of space time being convectively unstable, with and additional 10\% or
4 \% being dynamically unstable. Inserting \eref{7.8} in \eref{7.7},
the provisorial cutoff wavenumber in the $j$-th azimuth becomes:

\begin{equation}
\{m_{j}\}_{TRIAL} = N_{I}
(\Phi_{2}\sigma_{T}+V_{j}-V_{jI}+\Phi_{1}\sigma_{j})^{-1}
\label{7.9}
\end{equation}

Equation \eref{7.9} is the fundamental equation of the HDSP. The first
term on the right-hand side of equation \eref{7.9} represents the
effect of instability of the spectrum as a whole at the height of
interest. The $V_{j}-V_{jI}$ term represents the effect of Doppler
shifting by the background winds, common for instance also to
parameterizations based on \cite{lindzen81a}. The $\Phi_{1}\sigma_{j}$
term (unique to this theory) represents the nonlinear effect of
localized Doppler shifting on individual waves by all the other waves.

The two above mentioned conditions to be imposed on
$\{m_{j}\}_{TRIAL}$ are: (1) the cutoff wavenumber must be
monotonically non increasing with height, (2) the cutoff wavenumber
must be positive.  Equation \eref{7.9} shows that these conditions can
be achieved, because there always exists a positive $m_{j}$ at the
initial height, where $V_{j}- V_{jI}$ is zero.

\subsubsection{Horizontal wind variance}\label{s7.3.2}

At the height of interest and in the $j$-th azimuth, an elementary
contribution of the power spectral density $H^2_{j}$ of the horizontal
winds associated with the gravity waves to the horizontal wind
variance is written:

\begin{equation}
H^2_{j} dk dm = \rho^{-1} \rho_{I} s^{2}_{jI} K_{j}(k) M_{j}(m) dk dm
\label{7.10}
\end{equation}


where $\rho$ is the atmospheric density and $\rho_{I}$ is its value at
the initial height. $K_{j}$ and $M_{j}$ are respectively the
horizontal and vertical wavenumber spectra in the $j$-th azimuth (the
spectrum is assumed to be separable in $ k $ and $ m)$. The integrals
of $K_{j}dk $ and $M_{j}dm$ over all positive values are taken to be
normalized to $1$ at the initial height. As required by the definition
of spectral density, the integral of the horizontal wind power
spectral density at the initial height is therefore equal to $
s^2_{jI}$, the horizontal wind variance at the initial height.


The theory and the parameterization as developed to date consider that
the $K_{j}$ spectrum is unchanging with height, while the $M_{j}$
spectrum evolves in response to the background large scale flow,
buoyancy frequency, and nonlinear interactions.

Thereafter, the conservation of the vertical flux of the horizontal
momentum (or equivalently wave action, see point \ref{i7.2} in section
\ref{s7.2}) is used to compute the horizontal wind variance. Given
that the vertical flux of the horizontal momentum transported by the
waves that are not yet removed from the spectrum is conserved, the
portion of the spectra not removed at the height of interest and that
at the initial height are related by:

\begin{equation}
(HW)_{j} dm =\rho^{-1} \rho_{I} (HW)_{jI} dm_{I}
\label{7.11}
\end{equation}

where an elemental range $dm_{I}$ of the initial spectrum is mapped
into the range $dm$ at the height of interest. $(HW)_{j}$ represents
the covariance spectrum of the horizontal and vertical velocity
fluctuations associated with the gravity waves, the vertical flux of
horizontal momentum transferred by the waves, at the height of
interest. $(HW)_{jI}$ is the covariance at the initial height.


Following gravity wave theory, the vertical velocity perturbation is
in phase with the horizontal velocity perturbation and is given by $
k/m $ times the horizontal velocity, hence:

  \begin{equation}
H^2_{j} dm = \rho^{-1} \rho_{I} H^2_{jI} (m/m_{I})dm_{I}
\label{7.12}
\end{equation}


the horizontal wavenumber $k$ being constant with height.

The determination of the horizontal wind variance in the $j$-th
azimuth at the height of interest can therefore be achieved by
integration of the right-hand side of equation \eref{7.12} over all
positive $m _{I}$ up to the cut off vertical wavenumber $m_{j}$. For
this purpose, $ m $ on the right must be written as a function of
$m_{I}$. This can be done by means of \eref{7.5}, with the induced
wind field contribution $v_{j}$ ignored, under the approximation that
the spreading effect is significant only for waves at large vertical
wavenumber, and that the contribution of those waves to the total wind
variance is small \cite{hines91a}:


\begin{equation}
m/m_{I}=N/N_{I}(1-(V_{j}-V_{jI})m_{I}/N_{I})^{-1}
\label{7.13}
\end{equation}

Substituting \eref{7.13} into \eref{7.12} and integrating over all
positive $ k $ and $ m$, the horizontal wind variance at any height is
obtained:

\begin{equation}
s^2_{j} = \rho^{-1} \rho_{I}NN_{I}^{-1}s^2_{jI}
\int_{0}^{m_{j}}M_{jI}(m_{I})(1-
N_{I}^{-1}(V_{j}-V_{Ji})m_{I})^{-1}dm_{I}
\label{7.14}
\end{equation}


The determination of the evoluted $M_{j}$ spectrum is therefore
by-passed by the mapping between the spectrum at the current height
and the initial spectrum. The initial spectrum $M_{jI}$ and the cutoff
vertical wavenumber $m_{j}$ are all what is needed to compute the
horizontal wind variance.



\subsubsection{Momentum flux deposition}\label{su7.3.3}

At any given height, the vertical flux density of the $j$-directed
horizontal momentum that is transferred upward by the $j$-directed
waves is:

\begin{equation}
F_{j} = \rho\int_{0}^{m_{M}}\int_{0}^{\infty}(HW)_{j}dk dm
\label{7.15}
\end{equation}


where $(HW)$ is the covariance spectrum of the horizontal and vertical
velocity fluctuations associated with the gravity waves already
introduced. Using again the conservation of horizontal momentum for
the portion of the spectra not removed, the $j$-directed flux density
at the height of interest can be written in terms of the spectrum at
the initial height:


\begin{equation}
F_{j} = \rho_{1}s_{jI}^2K^*\int_{0}^{m_{j}}M_{jI}(m_{I})m_{I}^{-1}dm_{I}
\label{7.16}
\end{equation}

Where $K^*$ is obtained by the integration of $kK_{j}dk $ over all
positive $k$, and by neglecting the dependence on azimuth for
simplicity (although not necessary). $K^*$ can be considered a
weighted average of the directional horizontal wavenumber, and is
called the \emph{characteristic horizontal wavenumber}. In \eref{7.16}
the integral in vertical wavenumber $m_{I}$ is limited by the $m_{j}$
cutoff vertical wavenumber. Height variations in $ F_{j}$ are
therefore expressed by the dependence in height of the $m_{j}$ cutoff
vertical wavenumber.

In order to compute the rate of horizontal momentum flux deposition at
each gridpoint of the general circulation model, the momentum flux
must be expressed in the cardinal eastward and northward azimuths,
respectively. The rate of horizontal momentum flux deposition is
thereafter given by the vertical convergence of the momentum flux in
the cardinal directions.



\subsection{Summary}\label{s7.4}


The parameters that must be specified at the initial (launching)
height are the total rms gravity wave wind speed $\sigma_{TI}$, the
initial vertical wavenumber spectrum $M_{jI}$, and the $s_{jI}^2$
variances, which sum over the azimuths must be $\sigma_{TI}^2$, as
defined in \eref{7.2}. In addition, the location of the initial
height, the characteristic horizontal wavenumber $K^*$, and the
nondimensional factors $\Phi_{1}$ and $\Phi_{2}$ must be specified.

Given that the current knowledge about the global and seasonal
distributions of these gravity wave parameters is very limited, simple
choices have been made so far, based on the generalization of
observations of gravity wave variances and spectra, for instance 
\cite{allen95,fritts92,vincent97}.

The vertical wavenumber spectrum at the initial height is assumed to
follow a power law form in the initial vertical wavenumber, extending
from $ m_{I} = 0$ to the cutoff vertical wavenumber $m_{jI}$ at the
initial height. Its integral must be normalized to 1 at the initial
height, therefore:

\begin{equation}
M_{jI}(m_{I}) = (s+1)m_{jI}^{-s-1}m_{I}^s
\label{7.17}
\end{equation}


where $ s $ is the slope. The cutoff vertical wavenumber $ m_{jI}$ at
the initial height is computed from \eref{7.9} with $V_{j} = V_{jI}$:

\begin{equation}
m_{jI} = N_{I}(\Phi_{2}\sigma_{TI}+\Phi_{1}\sigma_{jI})^{-1}
\label{7.18}
\end{equation}


The computation of the cutoff vertical wavenumber $m_{j}$ thereafter
proceeds upward. At the first step upward and above the $m_{j}$ is
obtained by \eref{7.9}, subjected to the conditions of being
monotonically non increasing with height and positive. In principle,
above the initial height the horizontal wind variance at that vertical
level should be used in \eref{7.9}. However, the horizontal wind
variance at that vertical level depends in turn on the cutoff vertical
wavenumber that has to be evaluated. An iteration procedure would
therefore be required. As \cite{hines97a} has suggested, in case the
vertical resolution of the general circulation model is sufficiently
high, the iteration can be avoided by using in the computation of the
$m_{j}$ at any vertical level above the initial height the horizontal
wind variance at the level immediately below. This approach is used in
the parameterization implemented in the ECHAM model.


\subsection{Implementation Details}\label{s7.5}

The Hines gravity wave drag parameterization is by default activated in ECHAM6
at the standard resolutions T63L47 and T63L95, which both resolve the atmosphere 
up to 0.01 hPa. The setup generally depends on the model resolution, and is 
described below for the standard resolutions T63L47 and T63L95. However, the setup can
be modified through the control parameters of the \textit{gwsctl} Fortran namelist. 

The launching height $H_{I}$ of the parameterized gravity 
waves source spectrum is set to the model level, which above sea is at 680 hPa. 
For the L47 as well as the L95 vertical grid this is the 10th level above the surface.

The total root-mean-square gravity wave wind speed 
$\sigma_{TI}$ is prescribed as a global constant, if \textit{lrmscon\_lat} = .FALSE.
This is the default setting for ECHAM6 T63 L47.

Alternatively, if \textit{lrmscon\_lat} = .TRUE., $\sigma_{TI}$ can be prescribed by a simple 
function of latitude allowing to distinguish the source strength near the equator 
from that in other latitudes. 
$\sigma_{TI}(\phi)$ is set to $\sigma_{TI,hi}$ for $|\phi| >= \phi_{hi}$ and $\sigma_{TI,lo}$ for $|\phi| <= \phi_{lo}$
Between latitudes $\phi_{hi}$ and $\phi_{lo}$ $\sigma_{TI}$ is 
linearly interpolated between $\sigma_{TI,hi}$ and $\sigma_{TI,lo}$. This degree of freedom to 
specify an equatorial source strengths different from that in mid and high latitudes 
has been introduced to allow the tuning of the period of the QBO simulated in ECHAM6 
configurations with high vertical resolution. For ECHAM6 T63 L95 the following parameters 
are used: $\phi_{lo} = 5 deg$, $\phi_{hi} = 10 deg$, $\sigma_{TI,lo} = 1.2 m/s$, 
and $\sigma_{TI,hi} = 1.0 m/s$.

The source spectrum is assumed to be isotropic with respect 
to the wind at the launch level. The  $s^{2}_{jI}$ variances are 
distributed equally over 8 azimuth angles. 
Further the characteristic horizontal wavenumber $K^{*}$ is set to 126 km.

\begin{table}[htdp]
\caption{Parameters, symbols and control parameters of namelist GWSCTL}
\begin{center}
\begin{tabular}{|l|l|l|}\hline
Parameter                                & Symbol           & Namelist parameters                                            \\ \hline
Initial height                           & $H_{I}$          & \textit{emiss\_lev}                                            \\
Global rms wind at $H_{I}$               & $\sigma_{TI}$    & \textit{rmscon}, used if \textit{lrmscon\_lat=.FALSE.}         \\
Low latitude rms wind at $H_{I}$         & $\sigma_{TI,lo}$ & \textit{rmscon\_lo}, used if \textit{lrmscon\_lat=.TRUE.}      \\
High latitude rms wind at $H_{I}$        & $\sigma_{TI,hi}$ & \textit{rmscon\_hi}, used if \textit{lrmscon\_lat=.TRUE.}      \\
Low latitude limit for $\sigma_{TI,lo}$  & $\phi_{lo}$      & \textit{lat\_rmscon\_lo}, used if \textit{lrmscon\_lat=.TRUE.} \\
High latitude limit for $\sigma_{TI,hi}$ & $\phi_{hi}$      & \textit{lat\_rmscon\_hi}, used if \textit{lrmscon\_lat=.TRUE.} \\
Typical horizontal wavenumber            & $K^{*}$          & \textit{kstar}                                                 \\ \hline
\end{tabular}
\end{center}
\label{default}
\end{table}%
