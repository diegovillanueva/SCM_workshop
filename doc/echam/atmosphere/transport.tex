\section{Transport}

The flux form semi-Lagrangian scheme employed in ECHAM6 for passive
tracer transport has been introduced by \cite{lin96}.  This
type of advection scheme combines typical features of Eulerian, flux
form schemes (i.e., exact mass conservation to machine precision) with
the unconditional stability for all Courant numbers typical of
standard (non conservative) semi-Lagrangian schemes. For Courant
numbers smaller than one, the Lin-Rood schemes reverts to a
multidimensional flux form scheme which takes properly into account
transverse fluxes, such as those developed by Colella, LeVeque,
Leonard and others (see references in \cite{lin96}).  In the constant
velocity case at Courant number smaller than one, it is in fact
identical with the Colella {\it Corner Transport Upwind} scheme.  The
scheme is described here for application to incompressible flows, its
generalization to compressible fluids is described in \cite{lin96}.

Consider the conservative formulation of passive  advection 
in an incompressible fluid

\begin{equation}
\frac {\partial Q}{\partial t} + \nabla \cdot ( {\bf v}Q) =0, 
\label{cons}
\end{equation}

where $Q$ is the tracer concentration and the continuity equation is
given by

\begin{equation}
\nabla \cdot  {\bf v} =0.
\label{cont}
\end{equation}

It is to be remarked that there is an inherent coupling of
(\ref{cons}) to the continuity equation, since in the case of constant
tracer concentration (\ref{cons}) reduces to (\ref{cont}). This
property should be also guaranteed by the discretization of
(\ref{cons}).


Assuming a C-type grid staggering in which normal velocity components
are defined at the grid sides and scalar quantities (to be interpreted
as cell averages) are defined at the cell center, a flux form
discretization of (\ref{cons}) is given by


\begin{equation}
Q_{i,j}^{n+1} = Q_{i,j}^{n} 
- \Big ( {\cal X}_{i+\frac 12,j} -{\cal X}_{i-\frac 12,j} \Big ) 
-\Big (  {\cal Y}_{i,j+\frac 12} -{\cal Y}_{i,j-\frac 12} \Big )
\label{disc1}
\end{equation}

where ${\cal X}_{i+\frac 12,j} {\cal Y}_{i,j+\frac 12}$ and ${\cal
X}_{i-\frac 12,j} {\cal Y}_{i,j-\frac 12}$ are approximations of the
$Q$ fluxes in the E-W and N-S directions, respectively, integrated in
time over the time step $\Delta t .$ In order to achieve unconditional
stability, in the Lin-Rood scheme the fluxes are computed as the sum
of an {\it integer } and a {\it fractional} flux $$ {\cal X}_{i-\frac
12,j}={\cal X}^{int}_{i-\frac 12,j}+{\cal X}^{fr}_{i-\frac 12,j} .$$
The integer fluxes represent the contribution to the flux that arises
in case of Courant numbers larger than one at the cell side $i-\frac
12.$ More specifically, defining $$ C^x_{i-\frac 12,j}=\frac{\Delta t
u^{n+\frac 12}_{i-\frac 12,j}}{\Delta x}= K^x_{i-\frac
12,j}+c^x_{i-\frac 12,j}
$$

$$ 
K^x_{i-\frac 12,j}=INT(C^x_{i-\frac 12,j})  \ \ \ \ \ \ \ \ \
I=INT(i - C^x_{i-\frac 12,j})
$$
(where $INT$ has the same meaning as the corresponding
{\it Fortran95} intrinsic function) 
and assuming e.g. a positive velocity, the integer flux is defined 
as $${\cal X}^{int}_{i-\frac 12,j}=\sum_{k=1}^{K^x_{i-\frac 12,j}}Q^n_{i-k,j}. $$

Thus, the integer flux represents the mass transport through all the
cells crossed completely by a Lagrangian trajectory ending at
$(i-\frac 12,j)$ at timestep $n+1$ during a time interval $\Delta t.$

The fractional flux is defined as the Van Leer flux  

\begin{equation}
{\cal X}^{fr}_{i-\frac 12,j}= 
c^x_{i-\frac 12,j}\Bigg [Q^g_{I,j} 
+\frac{ Q^g_{I+1,j}-Q^g_{I-1,j} }4    
\Big ( SIGN(1,c^x_{i-\frac 12,j}) -c^x_{i-\frac 12,j} \Big ) \Bigg ]
\label{disc2}
\end{equation}

where $SIGN$ has the same meaning as the corresponding {\it Fortran95}
intrinsic function.

The intermediate value $Q^g_{i,j}$ used in the computation of the Van
Leer flux can be interpreted as a first order finite difference
approximation of
 
$$
\frac{\partial Q}{\partial t} +v\frac{\partial Q}{\partial y}=0,
$$

advanced in time $\Delta t / 2$ from timestep $ n $ along the
Lagrangian trajectory ending at $(i-\frac 12,j)$ at timestep $ n+1.$

More precisely,

$$
Q^g_{i,j} = \frac{ \Big (  Q^n_{i,J} +Q^n_{i,j}  \Big )}2
+\frac{\vert c^y_{i,j}\vert}2  \Big ( Q^n_{i,J^*} -Q^n_{i,J} \Big )
$$

where

$$
C^y_{i,j}=\frac{\Delta t}{2\Delta y}\Big ( v^{n+\frac 12}_{i,j-\frac 12}+v^{n+\frac 12}_{i,j+\frac 12}
\Big )
$$

$$
c^y_{i,j}= C^y_{i,j} - INT(C^y_{i,j}) \ \ \ \ \  J=j-INT(C^y_{i,j}) \ \ \ \ \ 
J^*=J-SIGN(1,C^y_{i,j}).
$$

The Lin and Rood scheme satisfies some fundamental requirements
for a tracer advection algorithm:

\begin{itemize}

\item mass conservation: by construction, since it is formulated in
flux form;

\item consistency with the discretization of the continuity equation:
setting $q=1$ yields a discretization  of (\ref{cont}) by the same scheme,

\item monotonicity of the 1D advection schemes: if a flux limiter is
applied to the centered difference $ Q^g_{I+1,j}-Q^g_{I-1,j} $ in
(\ref{disc2}) (see references in \cite{lin96}), the one dimensional flux
operators are guaranteed to be monotonic, although this in general
does not ensure that the full multidimensional scheme is monotonic as
well;

\item preservation of linear tracer correlations: if $q_1, q_2$ are
the concentrations of two different tracers such that $q^n_2=\alpha
q^n_1 + \beta, $ with $\alpha, \beta $ two constants, then the values
$q^{n+1}_1, q^{n+1}_2$ obtained by time update with the Lin and Rood
scheme still satisfy $q^{n+1}_2=\alpha q^{n+1}_1 + \beta. $

\end{itemize}
