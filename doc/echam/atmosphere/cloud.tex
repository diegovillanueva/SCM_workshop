\section{Large-scale cloud scheme\label{sec:cloud}}\label{c9}

The scheme for the respresentation of stratiform clouds consists of
prognostic equations for the vapor, liquid, and ice phase,
respectively, a cloud microphysical scheme (\cite{lohmann96}; with some
revisions), and a diagnostic cloud cover scheme \cite[]{sundqvist89}. 
The statistical cloud cover scheme 
\cite[]{tompkins02} used in ECHAM5 is available on request (see
\cite{roeckner2003a} for details).


\subsection{Governing equations}\label{s9.1}

The governing equations for the grid-cell mean mass mixing ratios of
water vapor, $\bar{r_{\mathrm{v}}}$, cloud liquid water, $\bar{r_{\mathrm{l}}}$, and
cloud ice, $\bar{r_{\mathrm{i}}}$, are written in symbolic form as follows
(units are kgkg$^{-1}$s$^{-1}$)

\begin{multline}
\frac{\bar{\partial r_{\mathrm{v}}}}{\partial t}
 = Q_{\mathrm{Tv}}+Q_{\mathrm{evr}}+Q_{\mathrm{evl}}+ Q_{\mathrm{sbs}}+Q_{\mathrm{sbis}}+Q_{\mathrm{sbi}}-Q_{\mathrm{cnd}}
  -Q_{\mathrm{dep}}-Q_{\mathrm{tbl}}-Q_{\mathrm{tbi}}\label{9.1}
\end{multline}
\begin{multline}
\frac{\bar{\partial r_{\mathrm{l}}}}{\partial t}
= Q_{\mathrm{Tl}}+Q_{\mathrm{mli}}+Q_{\mathrm{mlis}}+Q_{\mathrm{cnd}}+Q_{\mathrm{tbl}}
 -Q_{\mathrm{evl}}-Q_{\mathrm{frh}}-Q_{\mathrm{frs}}-Q_{\mathrm{frc}}\\
 -Q_{\mathrm{aut}}-Q_{\mathrm{racl}}-Q_{\mathrm{sacl}}\label{9.2}
\end{multline}
\begin{multline}
\frac{\bar{\partial r_{\mathrm{i}}}}{\partial t}
=Q_{\mathrm{Ti}}+Q_{\mathrm{sed}}+Q_{\mathrm{dep}}+Q_{\mathrm{tbi}}
-Q_{\mathrm{mli}}-Q_{\mathrm{sbi}}+Q_{\mathrm{frh}}+Q_{\mathrm{frs}}+Q_{\mathrm{frc}}
-Q_{\mathrm{agg}}-Q_{\mathrm{saci}}\label{9.3}
\end{multline}

with
\begin{description}
\item[$Q_{\mathrm{Tv}}$]    Transport of $\overline{r}_{\mathrm{v}}$ by advection ($Q_{\mathrm{Av}}$), diffusion $(Q_{\mathrm{Dv}})$ and changes due to convection
\item[$Q_{\mathrm{Tl}}$]    Transport of $\overline{r}_{\mathrm{l}}$ by advection $(Q_{\mathrm{Al}})$, diffusion $(Q_{\mathrm{Dl}})$ and convective detrainment $(Q_{\mathrm{Cl}})$
\item[$Q_{\mathrm{Ti}}$]    Transport of $\overline{r}_{\mathrm{i}}$ by advection $(Q_{\mathrm{Ai}})$, diffusion $(Q_{\mathrm{Di}})$ and convective detrainment $(Q_{\mathrm{Ci}})$
\item[$Q_{\mathrm{ev} \mathrm{r}}$]  Evaporation of rain falling into the respective layer 
\item[$Q_{\mathrm{ev} \mathrm{l}}$]  Instantaneous evaporation of $r_{\mathrm{l}}$ transported into the cloud-free part of the grid cell
\item[$Q_{\mathrm{sbs}}$]   Sublimation of snow
\item[$Q_{\mathrm{sbis}}$]  Sublimation of $r_{\mathrm{i}}$ in the sedimentation flux 
\item[$Q_{\mathrm{sbi}}$]   Instantaneous sublimation of $r_{\mathrm{i}}$ transported into  the cloud-free part of the grid cell
\item[$Q_{\mathrm{cnd}}$]   Condensation of $r_{\mathrm{v}}$ (if $Q_{\mathrm{cnd}}>0$), or evaporation of $r_{\mathrm{l}}$ (if $Q_{\mathrm{cnd}}<0$)
\item[$Q_{\mathrm{dep}}$]   Deposition of $r_{\mathrm{v}}$ (if $Q_{\mathrm{dep}}>0$), or sublimation of $r_{\mathrm{i}}$ (if $Q_{\mathrm{dep}}<0$)
\item[$Q_{\mathrm{tbl}}$]   Generation ($Q_{\mathrm{tbl}} >0$) or dissipation ($Q_{\mathrm{tbl}} <0$) of $r_{\mathrm{l}}$ through turbulent fluctuations
\item[$Q_{\mathrm{tbi}}$]   Generation ($Q_{\mathrm{tbi}} >0$) or dissipation ($Q_{\mathrm{tbi}} <0$) of $r_{\mathrm{i}}$ through turbulent fluctuations
\item[$Q_{\mathrm{mli}}$]   Instantaneous melting of $r_{\mathrm{i}}$ if the temperature exceeds the freezing point
\item[$Q_{\mathrm{mlis}}$]  Melting of $r_{\mathrm{i}}$ in the sedimentation flux
\item[$Q_{\mathrm{frh}}$]   Homogeneous freezing of  $r_{\mathrm{l}}$
\item[$Q_{\mathrm{frs}}$]   Stochastical and heterogeneous freezing of $r_{\mathrm{l}}$
\item[$Q_{\mathrm{frc}}$]   Contact freezing of $r_{\mathrm{l}}$
\item[$Q_{\mathrm{aut}}$]   Autoconversion of $r_{\mathrm{l}}$
\item[$Q_{\mathrm{racl}}$]  Accretion of $r_{\mathrm{l}}$ by rain
\item[$Q_{\mathrm{sacl}}$]  Accretion of $r_{\mathrm{l}}$ by snow
\item[$Q_{\mathrm{sed}}$]   Sedimentation of $\overline{r}_{\mathrm{i}}$, including losses due to $Q_{\mathrm{sbis}}$ and $Q_{\mathrm{mlis}}$
\item[$Q_{\mathrm{agg}}$]   Aggregation of $r_{\mathrm{i}}$
\item[$Q_{\mathrm{saci}}$]  Accretion of $r_{\mathrm{i}}$ by snow
\end{description}

Note that the transport terms as well as the sedimentation of cloud ice
is calculated from the respective grid-cell mean values (denoted
by an overbar), while the microphysical processes are calculated
from in-cloud values (without overbar). The phase changes sketched
above result in the following temperature change

\begin{equation}
\left(\dnd{T}{t}\right)_{\mathrm{ph}}=\left(\dnd{T}{t}\right)_{\mathrm{vapor}\leftrightarrow 
\mathrm{liquid}}+
\left(\dnd{T}{t}\right)_{\mathrm{vapor} \leftrightarrow 
\mathrm{solid}}+\left(\dnd{T}{t}\right)_{\mathrm{liquid} \leftrightarrow  \mathrm{solid}}
\label{9.4}
\end{equation}

with

\begin{eqnarray}
\left(\dnd{T}{t}\right)_{\mathrm{vapor} \leftrightarrow
\mathrm{liquid}}&=&\frac{L_{\mathrm{v}}}{c_{\mathrm{p}}}(Q_{\mathrm{cnd}}+Q_{\mathrm{tbl}}-Q_{\mathrm{evr}}-Q_{\mathrm{evl}})
\label{9.5}\\
\left(\dnd{T}{t}\right)_{\mathrm{vapor} \leftrightarrow
\mathrm{solid}}&=&\frac{L_{\mathrm{s}}}{c_{\mathrm{p}}}(Q_{\mathrm{dep}}+Q_{\mathrm{tbi}}-Q_{\mathrm{sbs}}-Q_{\mathrm{sbi}}-Q_{\mathrm{sbis}})
\label{9.6}\\
\left(\dnd{T}{t}\right)_{\mathrm{liquid} \leftrightarrow
\mathrm{solid}}&=&\frac{L_{\mathrm{f}}}{c_{\mathrm{p}}}(Q_{\mathrm{frh}}+Q_{\mathrm{frs}}+Q_{\mathrm{frc}}+Q_{\mathrm{sacl}}-Q_{\mathrm{mli}}-Q_{\mathrm{mlis}}-Q_{\mathrm{mls}})
\label{9.7}
\end{eqnarray}

where $L_{\mathrm{v}}$, $L_{\mathrm{s}}$, $L_{\mathrm{f}}$ is the latent heat of vaporization,
sublimation, and fusion, respectively, $c_{\mathrm{p}}$ is the specific heat of
moist air at constant pressure, $Q_{\mathrm{mls}}$ is the melting of snow falling
into the respective layer, and `solid' refers to both cloud ice
and snow.

\subsection{Cloud cover}\label{s9.2}\label{sec:cloudcover}

Fractional cloud cover $C$ is parameterized as a non-linear function of grid-mean 
relative humidity $r$ \cite[]{sundqvist89}. For $r > r_{0}$, where $r_{0} < r_{\mathrm{sat}}$ is a 
subgrid-scale condensation threshold and $r_{\mathrm{sat}}$ ( =1 in general) is the saturation value, 

\begin{equation}
C=1-\sqrt{1-C_{0}}
\label{9.8}
\end{equation}

\begin{equation}
C_{0}=\frac{r-r_{0}}{r_{\mathrm{sat}}-r_{0}}
\label{9.9}
\end{equation}

and $C = 0$ otherwise. Condensational growth of cloud droplets occurs 
if $r > r_{0}$. Oppositely, an existing cloud is diluted by evaporation 
if $r < r_{0}$.
The condensation threshold $r_{0}$ is specified as a function of height 
(or pressure), fitted to the results obtained by \cite{xu91} 
from experiments with a high-resolution cumulus ensemble model, 

\begin{equation}
r_{0}(p)=r_{0,\mathrm{top}} +(r_{0,\mathrm{surf}}-r_{0,\mathrm{top}}) \e{[1-(p_{\mathrm{s}}/p)^\mathrm{n}]}
\label{9.10}
\end{equation}

where $p$ is pressure, $p_{\mathrm{s}}$ is surface pressure, $r_{0,\mathrm{top}} = 0.7$ 
and $r_{0,\mathrm{surf}} = 0.9$ are the upper and lower values of $r_{0}$, and 
$n = 4$ is a fitting parameter. The function \eref{9.10} is used for 
all cloud types except for marine stratus under a low-level inversion. 
If such an inversion is detected below about 700 hPa, $r_{0}$ is reduced 
to $0.7$ below the inversion and $r_{\mathrm{sat}} = 0.9$ so that, according to 
\eref{9.8} and \eref{9.9} $C = 1$ is reached already before the whole layer 
is saturated. This ad hoc parameter setting allows formation of 
thin stratus clouds under a subsidence inversion which otherwise 
would not be captured due to the insufficient vertical resolution 
of the model. The lack of marine stratus clouds is one of the most 
persistent problems in GCMs.


\subsection{Sedimentation and cloud microphysics}\label{s9.3}

\subsubsection{Condensation/evaporation and deposition/sublimation}\label{su9.3.1}

Condensational growth of cloud droplets $(Q_{\mathrm{cnd}} > 0)$ occurs if the grid-mean 
relative humidity $r$ exceeds the specified threshold $r_{0}$ so that
the fractional cloud cover $C > 0$:

\begin{equation}
Q_{\mathrm{cnd}} = C(\Delta q_{\mathrm{f}} - \Delta q_{\mathrm{s}})
\label{9.11}
\end{equation}

where $\Delta q_{\mathrm{f}} > 0$ is the \textit{humidity forcing}, i.e. convergence of
$q=r_{\mathrm{v}}$ resulting from all previously calculated transport processes 
(advection, vertical diffusion, convection), and
 
\begin{equation}
\Delta q_{\mathrm{s}} =  \frac{\left(\Delta T_{\mathrm{f}} + C \frac{L_{\mathrm{v}}}{c_{\mathrm{p}}} \Delta q_{\mathrm{f}} \right)
\dnd{q_{\mathrm{s}}}{T}}
{1+C \frac{L_{\mathrm{v}}}{c_{\mathrm{p}}} \dnd{q_{\mathrm{s}}}{T}}
\label{9.12}
\end{equation}

is the change of saturation water vapor mixing ratio resulting 
from all previously calculated processes (temperature advection, 
adiabatic cooling, radiation, vertical diffusion, convection). 
Oppositely, dissipation $(Q_{\mathrm{cnd}} < 0)$ of an existing cloud $(C > 0)$
is caused by moisture divergence $(\Delta q_{\mathrm{f}} < 0)$ and/or
net heating of the grid-box $(\Delta q_{\mathrm{s}} > 0)$.

Note that humidity changes due to evaporation of rain/cloud water 
and sublimation of snowfall/cloud ice are not included in
$\Delta q_{\mathrm{f}}$. These phase changes ($\Delta q_{\mathrm{e}}$) 
are limited to the clear-sky part 
of the grid-box and do not affect cloud formation/dissipation at the 
respective timestep. However, since these processes change the grid-mean 
humidity, they are able to modify the cloud cover and, hence, the 
amount of condensation at the next timestep. 

The total changes per timestep of humidity and temperature are given by 

\begin{equation}
\Delta q =  \Delta q_{\mathrm{f}} + (1-C) \Delta q_{\mathrm{e}} - Q_{\mathrm{cnd}}
\label{9.13}
\end{equation}

\begin{equation}
\Delta T = \Delta T_{\mathrm{f}} + \frac{L_{\mathrm{v}}}{c_{\mathrm{p}}} Q_{\mathrm{cnd}}
\label{9.14}
\end{equation}

Depositional growth ($Q_{\mathrm{dep}}>0$) and sublimation ($Q_{\mathrm{dep}}<0$) of cloud 
ice is calculated in a 
way analogous to condensation/evaporation if one of the 
following conditions apply

\begin{enumerate}
\renewcommand{\labelenumi}{(\roman{enumi})}
\item  $T<-35$\grad$\;$  or
\item  $T< 0$\grad$\;$   and $x_{\mathrm{i}} >$ threshold value $\gamma_{\mathrm{thr}}$  
where $x_{\mathrm{i}}$ is the in-cloud ice water mixing ratio
\renewcommand{\labelenumi}{\arabic{enumi}}
\end{enumerate}
with $Q_{\mathrm{cnd}}$ and $L_{\mathrm{v}}$ in Eqs. \eref{9.11} - \eref{9.14} replaced by
$Q_{\mathrm{dep}}$ and $L_{\mathrm{s}}$, respectively.

Condition (ii) can be considered as a simple parameterization 
of the Bergeron-Findeisen process describing the fact that the 
equilibrium vapor pressure over water is greater than the saturation 
vapor pressure over ice, at the same temperature. Therefore, 
in mixed phase clouds, the water droplets tend to move to the lower 
pressure over the ice. The vapor will be condensed and freeze onto 
the ice crystal, causing it to grow larger.


\subsubsection{Sedimentation of cloud ice}\label{su9.3.2}

Sedimentation of cloud ice is formally treated like
vertical advection, i.e. the algorithm is applied to grid-cell mean
values so that the flux divergence is given by

\begin{equation}
\left(\dnd{\overline{r}_{\mathrm{i}}}{t}\right)_{\mathrm{sed}}=\frac{1}{\rho}\dnd{\overline{F}_{\mathrm{i}}}{z}
=\frac{1}{\rho}
\dnd{\left(\rho\overline{v}_{\mathrm{i}}\overline{r}_{\mathrm{i}}\right)}{z}\label{9.15}
\end{equation}

where the fall
velocity is parameterized as $\overline{v}_{\mathrm{i}}=\alpha \left(\rho
\overline{r}_{\mathrm{i}}\right)^{\beta}$ with $\alpha = 3.29$ and $\beta = 
0.16$ \cite[]{heymsfield90}. Equation \eref{9.15} can be expressed in
discrete form as

\begin{equation}
Q_{\mathrm{sed}}\equiv\left(\dnd{\overline{r}_{\mathrm{i}}}{t}\right)_{\mathrm{sed}}\approx\frac{1}{\rho\Delta 
z}
\left(\overline{F}_{\mathrm{i}}^{\mathrm{top}}-\rho \overline{v}_{\mathrm{i}}
\overline{r}_{\mathrm{i}}\right)
\label{9.16}
\end{equation}

where $\overline{F}_{\mathrm{i}}^{\mathrm{top}}$ is the incoming sedimentation flux
which has already been subject to sublimation and melting. By keeping
$\overline{F}_{\mathrm{i}}^{\mathrm{top}}$ as well as $\overline{v}_{\mathrm{i}}$ constant during
a time step interval, \eref{9.16} can be solved analytically
(\cite{rotstayn97}; see also section \ref{su9.3.8}). The flux leaving the
respective layer, $\overline{F}_{\mathrm{i}}^{\mathrm{bot}}$, is obtained by integrating
\eref{9.16} over the layer, giving


\begin{equation}
\overline{F}_{\mathrm{i}}^{\mathrm{bot}}=\overline{F}_{\mathrm{i}}^{\mathrm{top}}-\rho\Delta z Q_{\mathrm{sed}}.
\label{9.17}
\end{equation}

As the integration proceeds from the top of the model down to the
surface, the flux at the bottom of a layer can be used as incoming
flux for the layer beneath. In the lowest model layer $(k = N)$, the
flux $\overline{F}_{\mathrm{i}}^{\mathrm{bot}}(k=N)\equiv\overline{F}_{\mathrm{i}}(p_{\mathrm{s}})$,
representing the ice sedimentation at the surface, where $p_{\mathrm{s}}$ is
surface pressure, is added to the snow flux according to \eref{9.40}.



\subsubsection{Freezing of cloud liquid water and melting of cloud
ice}\label{su9.3.3}

At temperatures $T < -35$\grad, the total amount of cloud liquid water
freezes homogeneously and instantaneously, during one time step
$\Delta t$, to cloud ice \cite{levkov92} so that

\begin{equation}
Q_{\mathrm{frh}}=\frac{\overline{r}_{\mathrm{l}}}{\Delta t}.
\label{9.18}
\end{equation}

For stochastical and heterogeneous freezing in the temperature range
-35\grad $\;\le T < $ 0\grad, we use the extrapolated equation by
\cite{bigg53} down to the cloud droplet size \cite[]{levkov92,murakami90}.

\begin{equation}
Q_{\mathrm{frs}} = C a_{1} \{
                     \exp [b_{1}(T_{0}-T)]-1
                  \}
\frac{\rho r^2_{\mathrm{l}}}{\rho_{\mathrm{w}}N_{\mathrm{l}}}
\label{9.19}
\end{equation}

where the constants $a_{1}= 100$ m$^3$s$^{-1}$ and $b_{1}$=0.66
K$^{-1}$ are taken from laboratory experiments, $T_{0}$ = 273.15 K is
the freezing point, $\rho_{\mathrm{w}}$ = 1000 kgm$^{-3}$ is the density of
water, $\rho$ the air density, $T$ the grid-cell mean temperature,
$r_{\mathrm{l}}$ the in-cloud liquid water mixing ratio, $N_{\mathrm{l}}$ is the cloud
droplet number concentration, and $C$ the fractional cloud cover.
$N_{\mathrm{l}}$ is prescribed within the
atmospheric boundary layer ($= 220 \cdot 10^6$ m$^{-3}$ over land and
80 $\cdot$ 10$^6$ m$^{-3}$ over sea, respectively). Above the boundary
layer, $N_{\mathrm{l}}$ decreases exponentially to 50 m$^{-3}$ in the upper
troposphere over both land and ocean.

Brownian diffusion contact nucleation results
from random collisions of aerosol particles with supercooled cloud
droplets. It may be written as (e.g. \cite{levkov92})

\begin{equation}
Q_{\mathrm{frc}}=C m_{\mathrm{io}}F_{1}DF_{\mathrm{ar}}
\label{9.20}
\end{equation}

where $m_{\mathrm{io}}$ = 10$^{-12}$ kg is the initial mass of a nucleated ice
crystal, $DF_{\mathrm{ar}}$ = 1.4 $\cdot$ 10$^{-8}$ m$^{-2}$s$^{-1}$ the
aerosol diffusivity \cite[]{pruppacher78}, and $F_{1}=(4\pi
R_{\mathrm{vl}}N_{\mathrm{l}}N_{\mathrm{a}})/\rho$. The concentration of active contact nuclei is
approximated as $N_{\mathrm{a}}=\max [N_{\mathrm{a}0}(T_{0}-T-3), 0]$, with $N_{\mathrm{a}0}$ = 2
$\cdot$ 10$^5$ m$^{-3}$, and the mean volume droplet radius, $R_{\mathrm{vl}}$,
is obtained from

\begin{equation}
\frac{4}{3}\pi R^3_{\mathrm{vl}}N_{\mathrm{l}}\rho_{\mathrm{w}}=r_{\mathrm{l}}\rho.
\label{9.21}
\end{equation}

Following \cite{levkov92}, cloud ice is assumed to melt completely
when $T > T_{0}$, giving

\begin{equation}
Q_{\mathrm{mli}}=\frac{\overline{r}_{\mathrm{i}}}{\Delta t}.
\label{9.22}
\end{equation}

\subsubsection{Precipitation formation in warm clouds, cold
clouds and in mixed-phase clouds}\label{su9.3.4}

In warm clouds ($T > 0$\grad) and also in mixed phase clouds (-35\grad
$\; \le T < 0$\grad), the cloud liquid water content can be diminished
by autoconversion of cloud droplets, $Q_{\mathrm{aut}}$, growth of rain drops
by accretion of cloud droplets, $Q_{\mathrm{racl}}$, and growth of snow
crystals by accretion of cloud droplets, $Q_{\mathrm{sacl}}$.  The
autoconversion rate is derived from the stochastic collection equation
which describes the time evolution of a droplet spectrum changing by
collisions among droplets of different size \cite[]{beheng94} which
gives


\begin{equation}
Q_{\mathrm{aut}}=C
\gamma_{1}\left[a_{2}n^{-b_{2}}\left(10^{-6}N_{\mathrm{l}}\right)^{-b_{3}}\left(10^{-3}\rho 
r_{\mathrm{l}}\right)^{b_{4}}\right]/\rho
\label{9.23}
\end{equation}


where $a_{2} = 6\cdot 10^{28}$, $n=10$ is the width
parameter of the initial droplet spectrum described by a gamma
distribution, $b_{2}= 1.7$, $b_{3} = 3.3$,  $b_{4} = 4.7$, and
$\gamma_{1}$ is a tunable
parameter which determines the efficiency of the autoconversion
process and, hence, cloud lifetime.

Raindrops, once formed,
continue to grow by accretion of cloud droplets. The accretion rate
is derived from the stochastic collection equation \cite[]{beheng94}

\begin{equation}
Q_{\mathrm{racl}}=\min (C, C_{\mathrm{pr}}) a_{3}r_{\mathrm{l}}\rho r_{\mathrm{rain}}+\gamma_{2}\rho
Q_{\mathrm{aut}}\Delta t
\label{9.24}
\end{equation}

where $r_{\mathrm{rain}}$ is the mass mixing ratio of rain falling into a
fraction $C_{\mathrm{pr}}$ of the respective grid-cell, and $a_{3} = 6$
m$^3$kg$^{-1}$s$^{-1}$.  The second term in the bracket is the local
rainwater production during a time step by autoconversion, and
$\gamma_{2}$ is a tunable parameter. The remaining precipitation
process occurring in the cloud liquid water equation, $Q_{\mathrm{sacl}}$, will
be discussed below together with the analogous process for cloud ice,
$Q_{\mathrm{saci}}$.

The
conversion rate from cloud ice to snow by aggregation of ice
crystals has been adopted from \cite{levkov92}, based on the
work of \cite{murakami90}

\begin{equation}
Q_{\mathrm{agg}}= C \gamma_{3}\frac{\rho 
r^2_{\mathrm{i}}a_{4}E_{\mathrm{ii}}X\left(\frac{\rho_{0}}{\rho}\right)^{1/3}}
{-2\rho_{\mathrm{i}}\log\left(\frac{R_{\mathrm{vi}}}{R_{\mathrm{s}0}}\right)^3}
\label{9.25}
\end{equation}

where $a_{4} = 700$ s$^{-1}$ is an empirical constant, $E_{\mathrm{ii}}=0.1$ is
the collection efficiency between ice crystals, $X = 0.25$ is the
dispersion of the fall velocity spectrum of cloud ice, $\rho_{0} =
1.3$ kgm$^{-3}$ is a reference density of air, $\rho_{\mathrm{i}} = 500$
kgm$^{-3}$ is the density of cloud ice, $R_{\mathrm{vi}}$ is the mean volume
ice crystal radius, $R_{\mathrm{s}0} = 10^{-4}$ m is the smallest radius of a
particle in the snow class, and $\gamma_{3}$ is a tunable
parameter. From simultaneous measurements of $\rho r_{\mathrm{i}}$, $R_{\mathrm{vi}}$
and the effective radius of ice crystals, $R_{\mathrm{ei}}$, Moss (1996;
personal communication) derived the following relationships

\begin{align}
R_{\mathrm{ei}}&= a_{5}\left(10^{3}\rho r_{\mathrm{i}}\right)^{b_5}\label{9.26}\\
R_{\mathrm{ei}}^3&= R_{\mathrm{vi}}^3 \left( a_{6}+b_{6}R_{\mathrm{vi}}^3\right)\label{9.27}
\end{align}

with $R_{\mathrm{ei}}$, $R_{\mathrm{vi}}$ in $\mu$m, $a_5 = 83.8$, $b_5 = 0.216$, $a_6 =
1.61$, $b_6 = 3.56 \cdot 10^{-4}$, so that after solving for
\eref{9.27}

\begin{equation}
R_{\mathrm{vi}}[m]=10^{-6} \left(\sqrt{2809
R_{\mathrm{ei}}^3+5113188}-2261\right)^{1/3}.
\label{9.28}
\end{equation}


The accretional growth of snow through riming and collecting of ice
crystals is based on \cite{lin83} and \cite{levkov92}.  Snow crystals
are assumed to be exponentially distributed \cite[]{gunn58}

\begin{equation}
n_{\mathrm{s}}(D_{\mathrm{s}})=n_{0s}\exp (-\lambda_{\mathrm{s}}D_{\mathrm{s}})
\label{9.29}
\end{equation}

where $n_{\mathrm{s}}(D_{\mathrm{s}})$ is the concentration of
particles of diameters $D_{\mathrm{s}}$ per unit size interval, $D_{\mathrm{s}}$ is the
diameter of the snow particle, $n_{0\mathrm{s}} = 3 \cdot 10^6$ m$^{-4}$ is 
the intercept
parameter obtained from measurements \cite[]{gunn58},
and $\lambda_{\mathrm{s}}$ is the slope of the particle size distribution and is
written as \cite[]{potter91}

\begin{equation}
\lambda_{\mathrm{s}}=\left(\frac{\pi\rho_{\mathrm{s}}n_{0\mathrm{s}}}{\rho r_{\mathrm{snow}}}\right)^{1/4}
\label{9.30}
\end{equation}

where $\rho_{\mathrm{s}} =100$ kgm$^{-3}$ is the bulk
density of snow and $r_{\mathrm{snow}}$ is the mass mixing ratio of snow. Snow
crystals settle through a population of supercooled cloud
droplets, colliding and coalescing with them (riming). The rate of
change in the snow mixing ratio is based on geometric sweep-out
concept integrated over the size distribution \eref{9.29}

\begin{equation}
Q_{\mathrm{sacl}}=\min(C, C_{\mathrm{pr}})\gamma_{4}\frac{\pi
E_{\mathrm{sl}}n_{0\mathrm{s}}k_{\mathrm{s}}r_{\mathrm{l}}\Gamma(3+b_{7})}{4\lambda_{\mathrm{s}}^{3+b_{7}}}\left(\frac{\rho_{0}}{\rho}\right)^{1/2}
\label{9.31}
\end{equation}


where $k_{\mathrm{s}} = 4.83$ m$^2$s$^{-1}$, $b_{7}= 0.25$, $E_{\mathrm{sl}} = 1$ is 
the collection efficiency of
snow for cloud droplets \cite[]{lin83} and $\gamma_{4}$ is a tunable
parameter. The accretion rate of ice crystals by snow is similar
to \eref{9.31} and is expressed as

\begin{equation}
Q_{\mathrm{saci}}=\min(C, C_{\mathrm{pr}})
\frac{\pi E_{\mathrm{si}}n_{0\mathrm{s}}k_{s}r_{\mathrm{i}}\Gamma(3+b_{7})}
{4\lambda_{\mathrm{s}}^{3+b_{7}}}
\left(\frac{\rho_{0}}{\rho}\right)^{1/2}
\label{9.32}
\end{equation}

where the collection efficiency of snow for cloud ice is assumed to be
temperature dependent according to $E_{\mathrm{si}} =\exp [-a_{7}(T_{0}-T)]$
with $a_{7} = 0.025$. Note that $r_{\mathrm{i}} = 0$ for $T > T_{0}$
(c.f.,\eref{9.22}) so that $Q_{\mathrm{saci}} = 0$ in this case. Analogous to
\eref{9.24}, $\rho r_{\mathrm{snow}}$ used in \eref{9.31} and \eref{9.32}
through \eref{9.30} consists of two parts. The first one is a
contribution from the snow flux into the respective grid-cell (c.f.,
section \ref{su9.3.7}), and the second one, $\gamma_{2}\rho Q_{\mathrm{agg}}
\Delta t$, is due to local snow generation through aggregation of
ice crystals \eref{9.25}.

\subsubsection{Evaporation of rain and
sublimation of snow and ice}\label{su9.3.5}

The evaporation of rain is obtained by integration of the evaporation
for a single rain drop of diameter $D_{\mathrm{r}}$ over the Marshall-Palmer
distribution \cite[]{marshall48}. The rate of change can then be
expressed as

\begin{equation}
Q_{evr}=C_{\mathrm{pr}}
\frac{2\pi n_{0\mathrm{r}}S_{\mathrm{l}}}{\rho(A'+B')}\left[
\frac{a_{8}}{\lambda^2_{\mathrm{r}}}+\frac{b_{8}S^{1/3}_{\mathrm{c}}}{\lambda_{\mathrm{r}}^{\delta_{\mathrm{r}}/2}}\Gamma\left(\frac{\delta_{\mathrm{r}}}{2}\right)\left(
\frac{k_{\mathrm{r}}\rho}{\mu}\right)^{1/2}\left(
\frac{\rho_{0}}{\rho}\right)^{1/4}
\right]
\label{9.33}
\end{equation}

where $A'=L^2_{\mathrm{v}}/(K_{\mathrm{a}}R_{\mathrm{v}}T^2)$, $B'=1/(\rho r_{\mathrm{sl}}D_{\mathrm{v}})$,
$K_{\mathrm{a}}$ is the thermal conductivity of air, $R_{\mathrm{v}}$ is the gas constant for
water vapor, $D_{\mathrm{v}}$ is the diffusivity of vapor in the air, $n_{0\mathrm{r}} =
8 \cdot 10^6$ m$^{-4}$ is the intercept parameter, $r_{\mathrm{sl}}$ is the 
saturation water
vapor mixing ratio with respect to liquid water, $S_{\mathrm{l}} = 1 - r_{\mathrm{v}}/r_{\mathrm{sl}}$
is the respective sub-saturation, $S_{\mathrm{c}} = \mu/(\rho D_{\mathrm{v}})$ is the Schmidt
number, $\mu$ is the dynamic viscosity of air, $\delta_{\mathrm{r}} = 5.5$,
$a_{8} = 0.78$, $b_{8} = 0.31$, $k_{\mathrm{r}} = 141.4$ m$^2$s$^{-1}$, and 
the slope of the size distribution
is defined as

\begin{equation}
\lambda_{\mathrm{r}}=\left(\frac{\pi\rho_{\mathrm{w}}n_{0\mathrm{r}}}{\rho r_{\mathrm{rain}}}\right)^{1/4}.
\label{9.34}
\end{equation}

Instead of \eref{9.33} we use a simplified form
obtained after minor simplifications and evaluation of parameters
after \cite{rotstayn97}


\begin{equation}
Q_{\mathrm{evr}}=C_{\mathrm{pr}}
\frac{a_{9}S_{\mathrm{l}}}{\rho^{1/2}(A'+B')}\left(\frac{P_{\mathrm{r}}}{C_{\mathrm{pr}}}\right)^{b_{9}}
\label{9.35}
\end{equation}


where $P_{\mathrm{r}}$ is the rain flux [kgm$^{-2}$s$^{-1}$], $a_{9} = 870$ and
$b_{9} = 0.61$.

Analogously, the sublimation of snow is obtained by integrating the 
sublimation for a
single particle of diameter $D_{\mathrm{s}}$ over the Gunn-Marshall distribution
\eref{9.29}. The time rate of change can then be expressed as


\begin{equation}
Q_{\mathrm{sbs}} =C_{\mathrm{pr}}
\frac{2\pi n_{0\mathrm{s}}S_{\mathrm{i}}}{\rho(A''+B'')}
\left[\frac{a_{8}}{\lambda_{\mathrm{s}}^2}+
\frac{b_{8}S^{1/3}_{\mathrm{c}}}{\lambda_{\mathrm{s}}^{\delta_{\mathrm{s}}/2}}\Gamma\left(\frac{\delta_{\mathrm{s}}}{2}\right)\left(
\frac{k_{\mathrm{s}}\rho}{\mu}\right)^{1/2}\left(
\frac{\rho_{0}}{\rho}\right)^{1/4}\right],
\label{9.36}
\end{equation}

where $A''=L_{\mathrm{s}}^{2}/K_{\mathrm{a}}R_{\mathrm{v}}T^2)$, $B''=1/(\rho_{\mathrm{si}}D_{\mathrm{v}})$,
$r_{\mathrm{si}}$ is the saturation water vapor mixing ratio with respect to
ice, $S_{\mathrm{i}} = 1 - r_{\mathrm{v}}/r_{\mathrm{si}}$ is the respective sub-saturation,
$\delta_{\mathrm{s}}= 5.25$ and $k_{\mathrm{s}}= 4.83$ m$^2$s$^{-1}$
\cite[]{levkov92}. The expression \eref{9.36} is used for sublimation
of both snowfall, $Q_{\mathrm{sbs}}$, and falling ice, $Q_{\mathrm{sbis}}$. Note that in
$Q_{\mathrm{sbis}}$ the slope parameter $\lambda_{\mathrm{s}}$ \eref{9.30} includes the
ice mixing ratio, $r_{\mathrm{ised}}$, instead of $r_{\mathrm{snow}}$ (see section
\ref{su9.3.7}).

\subsubsection{Precipitation}\label{su9.3.6}

The total amount of
non-convective precipitation at a certain pressure level, $p$, is
obtained by integrating the relevant processes from the top of the
model $(p = 0)$ to the respective pressure level. The fluxes of rain
and snow [kgm$^{-2}$s$^{-1}$] can then be expressed as

\begin{align}
P_{\mathrm{rain}}(p)&=\frac{1}{g}\int_{0}^p\left(Q_{\mathrm{aut}}+Q_{\mathrm{racl}}-Q_{\mathrm{evr}}+Q_{\mathrm{mls}}\right)dp
\label{9.37}\\
P_{\mathrm{snow}}(p)&=\frac{1}{g}\int_{0}^p\left(Q_{\mathrm{agg}}+Q_{\mathrm{sacl}}+Q_{\mathrm{saci}}-Q_{\mathrm{sbs}}-Q_{\mathrm{mls}}\right)dp
\label{9.38}
\end{align}

with the snow melt, $Q_{\mathrm{mls}}$, defined in \eref{9.43}. The
sedimentation (see section \ref{su9.3.2}) is given by

\begin{equation}
\bar{F}_{\mathrm{i}}(p)=-\frac{1}{g}\int_{0}^p Q_{\mathrm{sed}} dp \ge 0
\label{9.39}
\end{equation}

where $Q_{\mathrm{sed}}$ includes the effects of sublimation $Q_{\mathrm{sbis}}$, and
melting, $Q_{\mathrm{mlis}}$. At the surface ($p = p_{\mathrm{s}}$), the sedimentation
is added to the snow fall so that the total snow flux is given by

\begin{equation}
P_{\mathrm{snow}}(p_{\mathrm{s}})=\frac{1}{g}\int_{0}^{p_{\mathrm{s}}}\left(Q_{\mathrm{agg}}+Q_{\mathrm{sacl}}+Q_{\mathrm{saci}}-Q_{\mathrm{sbs}}-Q_{\mathrm{mls}}\right) 
dp+
\bar{F}_{\mathrm{i}}(p_{\mathrm{s}}).
\label{9.40}
\end{equation}

Melting of falling ice and
snow is calculated from the heat budget in case the air temperature
exceeds the freezing point. The excess heat in the respective model
layer with pressure thickness, $\Delta p$, is then used for melting all or
part of the snow and/or ice sedimentation according to

\begin{equation}
\frac{c_{\mathrm{p}}\left(\tilde{T}-T_{0}\right)}{\Delta t}\frac{\Delta
p}{g}=\max (L_{\mathrm{f}}\hat{M},0)
\label{9.41}
\end{equation}

where  $\tilde{T}$
includes all processes except melting, and $\hat{M}$ is the preliminary amount of melting. The actual amount of melting depends not only on the
excess heat, according to \eref{9.41}, but also on the available snow
fall, $P_{\mathrm{snow}}$, and/or incoming sedimentation flux, 
$\bar{F}_{\mathrm{i}}^{\mathrm{top}}$:

\begin{eqnarray}
M_{\mathrm{snow}}&=&min(P_{\mathrm{snow}},\hat{M}) \label{9.42a}\\
M_{\mathrm{ice}}&=&min( \bar{F}_{\mathrm{i}}^{\mathrm{top}},\hat{M})
\label{9.42b}
\end{eqnarray}

The temperature change associated
with melting (c.f., \eref{9.7}) can be written as

\begin{equation}
\left(\dnd{T}{t}\right)_{\mathrm{melt}}=-\frac{L_{\mathrm{f}}}{c_{\mathrm{p}}}(Q_{\mathrm{mls}}+Q_{\mathrm{mlis}})
\label{9.43}
\end{equation}


with $Q_{\mathrm{mls}}\equiv M_{\mathrm{snow}}\cdot g/\Delta p$ and $Q_{\mathrm{mlis}}\equiv
M_{\mathrm{ice}}\cdot g/\Delta p$.

The precipitation fluxes \eref{9.37} -
\eref{9.40} represent grid-cell averages, while the accretion processes
\eref{9.24}, \eref{9.31} and \eref{9.32} as well as evaporation of 
rain \eref{9.35} and
sublimation of snow \eref{9.36} depend on the fractional area, $C_{\mathrm{pr}}$, of a
grid-cell covered with precipitation. Our approach for estimating
$C_{\mathrm{pr}}$ is a slight modification of that employed by \cite{tiedtke93},
as defined in \cite{jakob99}

\begin{equation}
C_{\mathrm{pr}}^k=\max
\left(\hat{C}_{\mathrm{pr}}, \frac{C^k\Delta Pr^k +\hat{C}_{\mathrm{pr}}Pr^{k-1}}{\Delta
Pr^k + Pr^{k-1}}\right)
\label{9.44}
\end{equation}


where $Pr^{k-1}$ is the total
precipitation flux, $P_{\mathrm{rain}} + P_{\mathrm{snow}}$, at model level $k - 1$, 
$\Delta Pr^k$ is the
amount of precipitation produced locally in the layer beneath, $C^k$
is the fractional cloud cover in layer $k$ and

\begin{equation}
\hat{C}_{\mathrm{pr}}=
\left(
\begin{array}{c}
C^k\mbox{ for }\left(\Delta P r^k\ge Pr^{k-1}\right)\\
C^{k-1}_{\mathrm{pr}} \mbox{ for }\left(\Delta P r^k < Pr^{k-1}\right)
\end{array}
\right).\label{9.45}
\end{equation}

According to \eref{9.44} and \eref{9.45}, the
vertical profile of $C_{\mathrm{pr}}$ is related to the profiles of both
fractional cloud cover and precipitation. In case the local
precipitation production exceeds the incoming flux, the
precipitation fraction is given by $C^{k}_{\mathrm{pr}}=C^{k}$. Note also that
$C^{k}_{\mathrm{pr}} =0$ for $\Delta Pr^k + Pr^{k-1}=0$.

\subsubsection{Mixing ratios of rain, falling ice and snow\label{su9.3.7}}

The mass mixing ratio
of rain, $r_{\mathrm{rain}}$, is related to the rain flux by

\begin{equation}
\rho r_{\mathrm{rain}}=P_{\mathrm{rain}}/(C_{\mathrm{pr}}v_{\mathrm{r}})
\label{9.46}
\end{equation}

where $P_{\mathrm{rain}}/C_{\mathrm{pr}}$ is the rain flux within the fraction of the 
grid-cell covered
with rain, and $v_{\mathrm{r}}$ is the mass-weighted fall velocity of rain drops
parameterized according to \cite{kessler69}

\begin{equation}
v_{\mathrm{r}}=a_{10}\left(\frac{\rho r_{\mathrm{rain}}}{n_{0\mathrm{r}}}\right)^{1/8} 
\left(\frac{\rho_{0}}{\rho}\right)^{1/2}
\label{9.47}
\end{equation}

with the intercept
parameter $n_{0\mathrm{r}} = 8 \cdot 10^6$ m$^{-4}$ and $ a_{10} = 90.8$. By using
\eref{9.47} in \eref{9.46}
we obtain



\begin{equation}
\rho r_{\mathrm{rain}}=\left(\frac{P_{\mathrm{rain}}}{C_{\mathrm{pr}} 
a_{10}(n_{0\mathrm{r}})^{-1/8}\sqrt{\rho_{0}/\rho}}\right)^{8/9}.
\label{9.48}
\end{equation}


According to \eref{9.15}, the mass mixing ratio of falling ice can be 
obtained from

\begin{equation}
r_{\mathrm{ised}}=\overline{F}^{\mathrm{top}}_{\mathrm{i}}/(\rho v_{\mathrm{i}})
\label{9.49}
\end{equation}

where $v_{\mathrm{i}}$ is
the fall velocity of cloud ice, $\overline{F}^{\mathrm{top}}_{\mathrm{i}}$is the 
grid-cell mean sedimentation
flux and $v_{\mathrm{i}}$ parameterized as in \eref{9.15} by employing the 
\cite{heymsfield90} approach

\begin{equation}
v_{\mathrm{i}}=a_{11}(\rho r_{\mathrm{ised}})^{b_{10}}\label{9.50}
\end{equation}


with $a_{11}= 3.29$ and $b_{10}= 0.16$. By using \eref{9.50} in
\eref{9.49} we obtain

\begin{equation}
\rho
r_{\mathrm{ised}}=\left(\frac{\overline{F}^{\mathrm{top}}_{\mathrm{i}}}{a_{11}}\right)^{1/(1+b_{10})}.\label{9.51}
\end{equation}

Analogously, the mass mixing ratio of snow within the fraction 
$C_{\mathrm{pr}}$ of the grid-cell
covered with snow is obtained from the snow fall rate according to


\begin{equation}
\rho r_{\mathrm{snow}}=\left(\frac{P_{\mathrm{snow}}}{C_{\mathrm{pr}} 
a_{11}}\right)^{1/(1+b_{10})}.\label{9.52}
\end{equation}

\subsubsection{Solution method and parameter choice}\label{su9.3.8}


The cloud microphysical terms are solved in a
split manner, i.e. sequentially. In the following, a subscript $n$
denotes the value of a variable before application of the
respective process, while $n+1$ denotes the updated value after
application of the process. A major part of the microphysics is
solved analytically ($Q_{\mathrm{frs}}$, $Q_{\mathrm{aut}}$, $Q_{\mathrm{agg}}$, $Q_{\mathrm{racl}}$, 
$Q_{\mathrm{sacl}}$, and
$Q_{\mathrm{saci}}$), and these terms can formally be written as

\begin{equation}
\dnd{\Psi}{t}=-F_{\Psi}\Psi^z\label{9.53}
\end{equation}

where $F_{\Psi}>0$  is
kept constant during the respective time interval, $\Delta t$, and $z
\ge 1$.
In the linear case, i.e. for all accretion processes ($Q_{\mathrm{racl}}$, $Q_{\mathrm{sacl}}$,
$Q_{\mathrm{saci}}$), the solution of \eref{9.53} is given by

\begin{equation}
\Psi_{\mathrm{n+1}}=\Psi_{\mathrm{n}} \exp (-F_{\Psi}\Delta t).
\label{9.54}
\end{equation}

For $ z > 1$, i.e. for
$Q_{\mathrm{frs}} (z = 2)$, $Q_{\mathrm{aut}} (z = 4.7)$ and $Q_{\mathrm{agg}} (z = 2)$, the 
solution of \eref{9.53}
is given by

\begin{equation}
\Psi_{\mathrm{n+1}}=\Psi_{\mathrm{n}}\left[1+F_{\Psi}\Delta
t(z-1)\Psi_{\mathrm{n}}^{\mathrm{z-1}}\right]^{1/(1-z)}.
\label{9.55}
\end{equation}


An analytical solution can also be obtained for
the ice sedimentation equation \eref{9.16}
which can be written in the form

\begin{equation}
\dnd{\Psi}{t}=A-B\Psi\label{9.56}
\end{equation}


where $A$ and $B$ are constants. The solution after one time step interval,
$\Delta t$, is given
by

\begin{equation}
\Psi_{\mathrm{n+1}}=\Psi_{\mathrm{n}}\exp (-B\Delta t)+\frac{A}{B} [1-\exp (-B\Delta t)].
\label{9.57}
\end{equation}


In the sedimentation equation,  $\overline{F}^{\mathrm{top}}_{\mathrm{i}}$ is included 
in $A$, while
the fall velocity $v_{\mathrm{i}}$ is included in $B$. Both are assumed constant
during the respective time step interval.

The microphysics scheme includes a large number of parameters,
$(a_{1}, a_{2}, ... , a_{11})$ and $(b_{1}, b_{2}, ... , b_{10})$,
which are kept constant as part of the parameterizations. On the other
hand, ($\gamma_{1}$, $\gamma_{2}$, $\gamma_{3}$, $\gamma_{4}$), in
\eref{9.23}, \eref{9.24}, \eref{9.25}, and \eref{9.31}, respectively,
are used as `tuning' parameters. This can be justified to some extent
because these parameterizations are based on detailed microphysical
models and cannot be applied to large-scale models without
adjustment.
The following values are used in \echam: $\gamma_{1}$ =
15; $0 \le \gamma_{2} \le 0.5$ depending on model resolution;
$\gamma_{3}$ = 95; $\gamma_{4}$ = 0.1; $\gamma_{\mathrm{thr}} = 5 \cdot
10^{-7}$ kgkg$^{-1}$.

