\section{Introduction}\label{sA.1}

To derive the governing equations given by
\eref{(2.2.1)}--\eref{(2.2.7)} and \eref{(2.2.12)}--\eref{(2.2.15)},
we take start from the unparameterized equations for a mixture of dry
air, water vapour, liquid water and ice, and work for convenience in a
Cartesian coordinate system. An individual component is denoted by a
subscript $i$, where $i = d, v, l,$ or $i$ for dry air, water vapour,
liquid water or ice, respectively. The specific mass of component $k$,
denoted by $q_k$, is defined by

\begin{equation}
\label{A.1.1}
q_k = \frac{m_k}{m} = \frac{\rho_k}{\rho}
\end{equation}

where

\begin{tabular}{lp{10cm}}
$m_k$ & is the mass of component of $k$ in a small
material volume moving with
the local velocity of the atmosphere, \\
$m =\sum m_k$ &is the total mass of the material volume, \\
$\rho_k$& is the density of component  $k$, and \\
$\rho = \sum \rho_k$ &is the density of the atmosphere.
\end{tabular}

The rate of change of $m_k$ is denoted by $\dot{m}_k$. This change
occurs because of

\renewcommand{\labelenumi}{\alph{enumi}.}
\begin{enumerate}
\item\label{a} internal phase changes,
\item\label{b} rainfall, snowfall, and surface exchanges.
\end{enumerate}

The rate of change due to (a) alone is denoted by
$\dot{m}_{ki}$, and that due to (b) by $\dot{m}_{ke}$. Then

\begin{eqnarray}
\label{A.1.2}\dot{m}_k & = & \dot{m}_{ki} + \dot{m}_{ke} \\
\label{A.1.3}\dot{m}_{di} & = & \dot{m}_{de} = 0
\end{eqnarray}

\begin{equation}
\label{A.1.4}\sum_i  \dot{m}_{ki} = 0
\end{equation}

The rate of change of total mass is given by

\begin{equation}
\label{A.1.5}\dot{m} = \sum_i  \dot{m}_k = \sum_i \dot{m}_{ke}
\end{equation}

The rate of change of density of component $k$ satisfies the equation

\begin{equation}
\label{A.1.6}\dot{\rho}_{k}=\frac{\rho}{m}{\dot{m}_{k}}
\end{equation}

provided (as is reasonable) volume changes due to precipitation or
phase changes are neglected. The net rate of change of density,
$\dot{\rho}$, is
then given by

\begin{equation}
\label{A.1.7}\dot{\rho}=\frac{\rho}{m}\sum_{k}\dot{m}_{k}=\frac{\rho}{m}\dot{m}
\end{equation}



\section{The advective form of the unparameterized equations}\label{sA.2}
\subsection{The material derivative}\label{suA.2.1}

The material derivative is denoted by $\frac{d}{dt}$.
Its definition is

\begin{equation}
\label{A.2.1.1}
\frac{d}{dt}\equiv \dnd{}{t}+\vec{v} \cdot\nabla
\end{equation}

where $\vec{v}$ here denotes the three-dimensional velocity vector,
and $\nabla$ the ususal three-dimensional vector operator. Horizontal vectors
and operators will subsequently be denoted by a subscript $h$.

\subsection{The equation of state}\label{suA.2.2}

We consider a volume $V$ of atmosphere, of
which dry air and water vapour occupy a volume $V_{d+v}$. The equations of
state for dry air and water vapour are

\begin{equation}
\label{A.2.2.1}
p_{d}V_{d+v}=m_{d}R_{d}T
\end{equation}

and

\begin{equation}
\label{A.2.2.2}
p_{v}V_{d+v}=m_{v}R_{v}T
\end{equation}


where $ p_{d}$ and $ p_{v}$ are partial pressures. Dalton's Law then 
shows that the
total pressure $ p $ is given from \ref{A.2.2.2} by

\begin{equation}
\label{A.2.2.3}
p=\frac{m_{d}R_{d}T + m_{v}R_{v}T}{V_{d+v}}.
\end{equation}

Introducing the specific volumes of liquid water $ v_{l}$, and ice $ v_{i}$,

\begin{equation}
\label{A.2.2.4}V_{d+v}=V-m_{l}v_{l}-m_{i}v_{i}=\frac{m}{\rho}(1-\rho
(q_{l}v_{l}+q_{i}v_{i}))
\end{equation}

and \ref{A.2.2.3} becomes

\begin{equation}
\label{A.2.2.5}
p=\rho T \frac{R_{d}q_{d} + R_{v}q_{v}}{1-\rho
(q_{l}v_{l}+q_{i}v_{i})}.
\end{equation}


or

\begin{equation}
\label{A.2.2.6}
p=\rho T R_{d} \frac{1+\left(\frac{1}{\epsilon}-1\right)q_{v} - q_{l}-q_{i}}
{1-\rho (q_{l}v_{l}+q_{i}v_{i})}.
\end{equation}

where

\begin{equation}
\label{A.2.2.7}
\epsilon=R_{d}/R_{v}
\end{equation}

\subsection{Mass conservation}\label{suA.2.3}

  Conservation of mass for element $k$
leads to the equation

\begin{equation}
\label{A.2.3.1}
\frac{d\rho_{k}}{dt}+\rho_{k}(\nabla\cdot \vec{v})=\dot{\rho}_{k}=
\frac{\rho\dot{m}_{k}}{m}
\end{equation}

Summing over $k$ then gives

\begin{equation}
\label{A.2.3.2}
\frac{d\rho}{dt}+\rho(\nabla\cdot \vec{v})=\frac{\rho\dot{m}}{m}=
\dot{\rho}
\end{equation}

In addition, by definition

\begin{equation}
\label{A.2.3.3}
\frac{dm_{k}}{dt}=\dot{m}_{k}
\end{equation}

which gives

\begin{equation}
\label{A.2.3.4}
\frac{dq_{k}}{dt}=\frac{\dot{m}_{k}}{m}-\frac{m_{k}\dot{m}}{m^2}=\frac{1}{m}
(\dot{m}_{k}-q_{k}\dot{m})
\end{equation}

\subsection{The velocity equation}\label{suA.2.4}

The advective form of the
equations for the horizontal components of velocity is unaltered by
mass changes. The horizontal velocity components thus satisfy the
equation

\begin{equation}
\label{A.2.4.1}
\frac{d \vec{v}_{h}}{dt}=-\frac{1}{\rho}\nabla_{h}p-2 (\vec{\Omega} \times
\vec{v}_{h})_h
\end{equation}

where $\vec{\Omega}$ is the earth's rotation vector. Changes due to
molecular stresses are neglected.

\subsection{The thermodynamic equation}\label{suA.2.5}

As discussed by Dufour
and Van Mieghem (1975, Eq. 5.21), the first law of thermodynamics may
be written

\begin{equation}
\label{A.2.5.1}
\delta Q +\alpha dp=d_{i}H=d_{i}\left(\sum m_{k}h_{k}\right)
\end{equation}

where the $h_{k}$ are specific enthalpies, $\alpha=1/\rho$ is the 
specific volume
and the subscript $i $ denotes changes independent of the mass changes due
to precipitation. As molecular diffusion is neglected, $\delta Q$ 
represents the
heat received by the atmospheric element due to radiation and to heat
exchange with falling rain or snow.

Under the usual assumptions of perfect gas behaviour for dry air and
water vapour, and neglect of variations of the specific enthalpies of
water and ice with pressure, we can write

\begin{equation}
\label{A.2.5.2}
h_{k}=h_{k}^0+C_{pk}T
\end{equation}

and \eref{A.2.5.1} becomes

\begin{equation}
\label{A.2.5.3}
mC_{p}dT=\alpha dp +
\delta Q -\sum_{k}h_{k}d_{i}m_{k}
\end{equation}

where

\begin{equation}
\label{A.2.5.4}
C_{p}=\sum_{k} C_{pk}q_{k}
\end{equation}

Thus considering a material volume of the atmosphere, we obtain the
thermodynamic equation

\begin{equation}
\label{A.2.5.5}
C_{p}\frac{dT}{dt}=\frac{1}{\rho}\frac{dp}{dt} +Q_{R}+Q_{M}-\sum_{k}
h_{k}\frac{\dot{m}_{ki}}{m}
\end{equation}

where $Q_{R}$ and $Q_{M}$ are the heating rates due to respectively
radiation and the heat transferred from falling rain or snow.

\section{The flux forms of the equations}\label{sA.3}

It is convenient to define the differential operator $\frac{D}{Dt}$ by

\begin{equation}
\label{A.3.1}
\frac{DX}{Dt}=\frac{dX}{dt}+X(\nabla \cdot \vec{v})=\dnd{X}{t}+\nabla \cdot (X \vec{v})
\end{equation}

Note that

\begin{equation}
\label{A.3.2}
\rho=\frac{dx}{dt}=\frac{D\rho x}{Dt} \mbox{ if } \dot{\rho} =0
\end{equation}

Equations \eref{A.2.3.4}, \eref{A.2.4.1} and \eref{A.2.5.5} may then be
written

\begin{eqnarray}
\label{A.3.3}
\frac{D\rho}{Dt}&=&\frac{\rho}{m}\dot{m}=\dot{\rho}\\
\label{A.3.4}
\frac{D\rho q_{k}}{Dt}&=&\frac{\rho}{m}\dot{m}_{k}=\dot{\rho}_{k}\\
\label{A.3.5}
\frac{D\rho 
\vec{v}_{h}}{Dt}&=&\dot{\rho}\vec{v}_{h}-\nabla_{h}p-2\rho(\vec{\Omega} 
\times \vec{v}_{h})_{h}\\
\label{A.3.6}
C_{p}\frac{D\rho 
T}{Dt}&=&C_{p}\dot{\rho}T+\frac{dp}{dt}+\rho(Q_{R}+Q_{M})-\rho\sum_{k}
h_{k}\frac{\dot{m}_{ki}}{m}
\end{eqnarray}

 From the definition \eref{A.2.5.4} of $C_{p}$ we obtain

\begin{equation}
\label{A.3.7}
\frac{DC_{p}\rho T}{DT}=
C_{p}\frac{D\rho T}{DT}+\rho T\frac{d}{dt}\sum_{k}C_{pk}q_{k}
\end{equation}

and using \eref{A.2.5.4} and \eref{A.3.6} gives

\begin{equation}
\begin{split}
\label{A.3.8}
\frac{DC_{p} \rho
T}{Dt}=C_{p}\dot{\rho}T+\frac{dp}{dt}&+\rho\left(Q_{R}+Q_{M}\right)-\rho\sum_{k} 
\left(h_{k}^0+C_{pk}T\right)\frac{\dot{m}_{ki}}{m}\\&+\rho T \sum_{k}
C_{pk}\left(\frac{\dot{m}_{k}}{m}-\frac{q_{k}\dot{m}}{m}\right)
\end{split}
\end{equation}

Using \eref{A.1.2}, \eref{A.1.7} and \eref{A.2.5.4}, we obtain from
\eref{A.3.8}:

\begin{equation}
\label{A.3.9}
\frac{DC_{p} \rho 
T}{Dt}=\frac{dp}{dt}+\rho\left(Q_{R}+Q_{M}\right)-\rho\sum_{k}
h^0_{k}\frac{\dot{m}_{ki}}{m}+\rho
T\sum_{k}C_{pk}\frac{\dot{m}_{ke}}{m}
\end{equation}

\section{The introduction of diffusive fluxes}\label{sA.4}

We now introduce
a separation of dependent variables into components that will be
explicitly resolved in the model and components the effect of which
will require parameterization.

If the bar operator represents an average over unresolved scales in
space and time, then we write:

\begin{eqnarray*}
X&=&\ovl{X}+X' \mbox{ with } \ovl{X}' = 0\\
\mbox{ and }X&=&\ovl{\ovl{X}}+X'' \mbox{ with } \ovl{\ovl{X}}''= 0\\
\mbox{ where }\ovl{\ovl{X}}&=&\frac{\ovl{\rho X}}{\ovl{\rho}} \mbox{
is a mass weighted average}.
\end{eqnarray*}

It follows that

\begin{eqnarray*}
\frac{\ovl{\ovl{D}}\ovl{\rho}\ovl{\ovl{X}}}{Dt}&=&\frac{\ovl{D\rho
X}}{Dt}-(\nabla \cdot \ovl{\rho \vec{v}''X''})\\
\frac{\ovl{\ovl{d}}\ovl{X}}{dt}&=&\frac{\ovl{d
X}}{dt}-\left(\ovl{\vec{v}'' \cdot \nabla X}\right)\\
\ovl{\rho} XY&=&\ovl{\rho XY} = \ovl{\rho}\ovl{\ovl{X}}\,\ovl{\ovl{Y}}+
\ovl{\rho X''Y''}
\end{eqnarray*}

Using these results, equations \eref{A.3.2} - \eref{A.3.4} and 
\eref{A.3.8} become

\begin{eqnarray}
\frac{\ovl{\ovl{D}}\ovl{\rho}}{Dt}&=&\ovl{\dot{\rho}}=\ovl{\rho}
\left(\frac{\ovl{\ovl{\dot{m}}}}{m}\right)
\label{A.4.1}\\
\frac{\ovl{\ovl{D}}\ovl{\rho}\,\ovl{\ovl{q_{k}}}}{Dt}&=&\ovl{\dot{\rho}_{k}}-
\left(\nabla \cdot
\overline{\rho\vec{v}''q_{k}''}\right)=\ovl{\rho}\left(
\frac{\ovl{\ovl{\dot{m}_{k}}}}{m}\right)-
\left(\nabla\cdot\overline{\rho\vec{v}''p_{k}''}\right)
\label{A.4.2}\\
\frac{\ovl{\ovl{D}}\ovl{\rho}\ovl{\ovl{\vec{v}_{h}}}}{Dt}&=&\ovl{\dot{\rho}\vec{v}_{h}}-
\nabla_{h}\ovl{p}-2\ovl{\rho}\left(\vec{\Omega} \times
\ovl{\ovl{\vec{v}_{h}}}\right)_{h}-\left(\nabla \cdot
\overline{\rho\vec{v}''\vec{v}''_{h}}\right)\label{A.4.3}\\
&=&\ovl{\rho}\left(\frac{\ovl{\ovl{\dot{m}}}}{m}\right)\ovl{\ovl{\vec{v}_{h}}}-\nabla_{h}
\ovl{\rho}-2\ovl{\rho}\left(\vec{\Omega} \times \ovl{\ovl{\vec{v}_{h}}}\right)_{h}
-\left(\nabla \cdot \overline{\rho\vec{v}''\vec{v}''_{h}}\right)
-\ovl{\rho\left(
\frac{\dot{m}}{m}\right)''\vec{v}''_{h}}\nonumber
\end{eqnarray}

and

\begin{eqnarray}
\frac{\ovl{\ovl{D}}}{Dt}\left(\ovl{\rho}\ovl{\ovl{C_{p}}}\,\ovl{\ovl{T}}+
\ovl{\rho C_{p}''T''}\right)&=&\frac{\ovl{\ovl{d}}\ovl{p}}{dt}+
\ovl{\rho}\left(\ovl{\ovl{Q_{R}}}
+\ovl{\ovl{Q_{M}}}\right)
-\ovl{\rho}\sum_{k}
h_{k}^0\left(\frac{\ovl{\ovl{\dot{m}_{ki}}}}{m}\right)\nonumber\\
&+&\ovl{\rho}\ovl{\ovl{T}}\sum_{k}C_{pk}\left(\frac{\ovl{\ovl{\dot{m}_{ke}}}}{m}\right)
+\ovl{\vec{v}'' \cdot \nabla p}-
\left(\nabla \cdot \ovl{\rho\vec{v}''C_{p}T''}\right)\label{A.4.4}\\
&+&\sum_{k} C_{pk}\ovl{\rho T''\left( \frac{\dot{m}_{ke}}{m} 
\right)''}\nonumber
\end{eqnarray}


The equation of state \ref{A.2.2.5} gives

\begin{equation}
p=\rho RT\label{A.4.5}
\end{equation}


where $R=(R_{d}q_{d}+R_{v}q_{v})/\{1-\rho (q_{l}v_{l}+q_{i}v_{i})\}$

whence

\begin{equation}
\ovl{\rho}= \ovl{\rho RT}= \ovl{\rho}\ovl{\ovl{R}}\,\ovl{\ovl{T}}+
\ovl{\rho R''T''}\label{A.4.6}
\end{equation}

Using $\ovl{\ovl{C_{p}}}=\sum C_{pk}\ovl{\ovl{q_{k}}}$, \eref{A.4.2}
and \eref{A.4.4} may be written

\begin{eqnarray}
\ovl{\ovl{C_{p}}} \frac{\ovl{\ovl{D}}\ovl{\rho}\ovl{\ovl{T}}}{Dt}&=&
\frac{\ovl{\ovl{d}}\ovl{p}}{dt}+\ovl{\rho}
\left(\ovl{\ovl{Q_{R}}}+\ovl{\ovl{Q_{M}}}\right)-\ovl{\rho}\sum_{k}\ovl{\ovl{h_{k}}}
\left(\frac{\ovl{\ovl{\dot{m}_{ki}}}}{m}\right)+\ovl{\rho}\ovl{\ovl{C_p}}\,
\ovl{\ovl{T}}\left(\frac{\ovl{\ovl{\dot{m}}}}{m}\right)\nonumber\\
&+&\ovl{\vec{v}'' \nabla p}-\nabla \cdot
\ovl{\rho\vec{v}''(C_{p}T)''}+\ovl{\ovl{T}}\sum_{k}C_{pk}\nabla \cdot
\ovl{\rho\vec{v}''q_{k}''}\\
&-&\frac{\ovl{\ovl{D}}}{Dt} (\ovl{\rho C_{p}''T'')}+\sum_{k}C_{pk}
\ovl{\rho T''\left(\frac{\dot{m}_{ke}}{m}\right)}\nonumber
\end{eqnarray}


\section{Approximations and definitions}\label{sA.5}

At this stage, we make two
approximations. The first is to neglect the higher-order correlations

\begin{equation*}
\ovl{\rho
T''\left(\frac{\dot{m}_{ke}}{m}\right)''},\quad\frac{\ovl{\ovl{D}}}{Dt}
\ovl{\left(\rho C_{p}''T''\right)}, \quad \ovl{\rho T'' R''}\mbox{
and } \ovl{\rho\left(\frac{\dot{m}}{m}\right)''\vec{v}_{h}}.
\end{equation*}

This is equivalent to assuming higher-order terms are important only
when eddy velocities and derivatives are involved. The second is to
neglect the term in the equation of state, or equivalently to neglect
the volume occupied by liquid water and ice compared with that
occupied by dry air and water vapour.

In addition we introduce the following notation:
\begin{enumerate}
\item The vertical flux of a variable $X$, $\ovl{\rho w''X''}$, is
denoted by $J_{X}$. Here $w$ is the
vertical velocity component.

\item The term $\ovl{v''\cdot \nabla p}$ is added to the term
$\dnd{}{z} \ovl{\rho w''(C_{p}T)''}$ and the resulting sum is
expressed as the derivative $\dnd{J_{S}}{z}$ of the vertical flux of
dry static
energy, plus a term which is written $\ovl{\rho} \ovl{\ovl{Q_{D}}}$ 
and regarded as representing
unorgnized transfers between enthalpy and sub-grid scale kinetic
energy. The latter is parameterized by the heating implied by the
dissipation of kinetic energy due to the parameterized vertical
momentum fluxes $J_{\vec{v}_{h}}$.

\item The net effect of horizontal fluxes is represented only
by their contribution $K_{X}$ to the tendency of variable $X$.

\item The term $-\ovl{\rho} \sum_{k}
\ovl{\ovl{h_{k}}}\left(\frac{\ovl{\ovl{\dot{m}_{ki}}}}{m}\right)$ representing
the latent heat release associated with
internal phase changes is written $\rho\ovl{\ovl{Q_{L}}}$
\end{enumerate}

\section{Return to the advective form}\label{sA.6}

With the above approximations and definitions, we obtain from the
equations of Appendix \ref{sA.4}, on dropping the bar operators

\begin{equation}
\frac{d\rho}{dt}+\rho\nabla\cdot\vec{v}=\rho\frac{\dot{m}}{m}\label{A.6.1}
\end{equation}

\begin{equation}
\frac{dq_{k}}{dt}=S_{q_{k}}-\frac{1}{\rho}\dnd{J_{q_{k}}}{z}+K_{q_{k}}\label{A.6.2}
\end{equation}

\begin{equation}
\frac{d\vec{v}_{h}}{dt}=-\frac{1}{\rho}\nabla_{h}p-2\left(\vec{\Omega}\times
\vec{v}_{h}\right)_{h}-\frac{1}{\rho}\dnd{J_{\vec{v}_{h}}}{z}+K_{\vec{v}_{h}}\label{A.6.3}
\end{equation}


\begin{equation}
\frac{dT}{dt}=\frac{1}{\rho 
C_{p}}\frac{dp}{dt}+\frac{1}{C_{p}}\left(Q_{R}+Q_{L}
+Q_{M}+Q_{D}-\frac{1}{\rho}\left[\dnd{J_{S}}{z}-T\sum_{k}C_{pk}\dnd{J_{q_{k}}}{z}
\right ]\right)+K_{T}\label{A.6.4}
\end{equation}

where

\begin{equation}
S_{q_{k}}=\frac{\dot{m}_{k}}{m}-q_{k}\frac{\dot{m}}{m}\label{A.6.5}.
\end{equation}

In addition we have the equation of state

\begin{equation}
p=\rho T\left(R_{d}q_{d}+R_{v}q_{v}\right)\label{A.6.6}.
\end{equation}

and the hydrostatic equation

\begin{equation}
\dnd{p}{z}=-g\rho\label{A.6.7}.
\end{equation}

\section{The model equations}\label{sA.7}

The model equations \eref{(2.2.1)}--\eref{(2.2.7)} and
\eref{(2.2.12)}--\eref{(2.2.15)} are finally obtained by neglecting
density changes due to precipitation or evaporation, setting
$\dot{m}=0$ in \eref{A.6.1}. This approximation is traditionally made,
although it is open to question.

In addition, $Q_{M}$ is set to zero, an approximation of the same
order as the assumption of no variation of latent heat with
temperature that is made in the parameterizations.

The governing equations are

\begin{eqnarray}
\frac{d\vec{v}_{h}}{dt}
&=&-\frac{1}{\rho}\nabla_{h}p-2\left(\vec{\Omega}\times\vec{v_{h}}\right)_{h}-
\frac{1}{\rho}\dnd{J_{\vec{v}_{h}}}{z}+K_{\vec{v}_{h}}\label{A.7.1}\\
\frac{dT}{dt}
&=&\frac{R_{d}T_{v}}{pC_{p}}\frac{dp}{dt}+\frac{1}{C_{p}}\left(Q_{R}+
Q_{L}+Q_{D}-\frac{1}{\rho}\left[\dnd{J_{s}}{z}-C_{pd}T(\delta
-1)\dnd{J_{q_{v}}}{z}\right]\right)+K_{T}\label{A.7.2}\\
\frac{dq_{i}}{dt}
&=&S_{q_{i}}-\frac{1}{\rho}\dnd{J_{q_{i}}}{z}\label{A.7.3}\\
p&=&\rho R_{d}T_{v}\label{A.7.5}\\
\dnd{p}{z}&=&-g\rho\label{A.7.6}
\end{eqnarray}

with

\begin{equation}
T_{v}=T\left(1+\left(\frac{1}{\epsilon}-1\right)q_{v}\right)\label{A.7.7}
\end{equation}

In this case

\begin{equation*}
C_{p}=C_{pd}\left(1-q_{v}\right)+C_{pv}q
\end{equation*}
which is written

\begin{equation}
C_{p}=C_{pd}\left(1+\left(\delta -1\right)q_{v}\right)\label{A.7.8}
\end{equation}

where $\delta=\frac{C_{pv}}{C_{pd}}$.

The model equations then follow from a change from $z$ -
to $\eta$-coordinates, the formalism for which is given by Kasahara (1974),
and from rewriting the adiabatic terms in their usual form for a
spherical geometry.

