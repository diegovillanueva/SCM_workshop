\subsection{Introduction}

\echam{} allows the implementation of so--called submodels. A submodel
can describe any additional physical processes that will either be linked in
a one--way coupling to echam or a two--way coupling. A one--way
coupling in this context means that the additional physical processes
are such that they need input from the \echam{} base model but do not
change the general circulation. One could also say that the results of
such a model are derived from the \echam{} base model in a
``diagnostic'' way. If the base model is linked by a two--way coupling
to a submodel, the submodel interacts with \echam{} and modifies the
general circulation. An example for the one--way coupling would be
diagnostic chemistry implemented in such a way that the chemical
species are transported by the winds given by \echam{} and the
chemical reactions are driven by the pressure, temperature, humidity
and radiation simulated by \echam. Nevertheless, the concentration of
the chemical species would not be allowed to influence these
quantities. A two--way coupling would be introduced if the
concentration of the chemical species influences the radiation by
absorption of radiation for example. 

The implementation of such submodels needs an interface to the submodel
that provides a certain set of variables to the submodel routines.
In fact, the submodel interface is a collection of dummy subroutines
in \echam{} inside which the special 
subroutines of a submodel can be called. These special subroutines
will not be a part of \echam{} but will perform all submodel specific tasks
as the solution of the chemical kinetic equations for example.
In addition to this submodel interface, many submodels need the
introduction of tracers that are transported with the air flow like
water waper is transported. These tracers are often associated with
certain chemical species having specific physico--chemical
properties. In general, it may occur that a certain species is
represented by several tracers (e.g.~various CO tracers depending on
the region of emission of CO, so--called ``tagged'' tracers) so that
every tracer has the same physico--chemical properties. Conceptually,
it is better to separate the tracer properties from a list of
physico--chemical species properties so that this information is
present only once in the program. This avoids inconsistent definition
of species properties and is therefore more user friendly. This
separation is not yet finished in the current \echam{} version and the
species data structure will therefore not be described here although
it is present. As soon as this species concept has settled, this
description will be added.

\subsection{Submodel Interface}

The submodel interface consists of the subroutines listed in
Tab.~\ref{tabsubmodeli} that are all collected in module {\tt mo\_submodel\_interface.f90}. 

\setlength{\LTcapwidth}{\textwidth}
\setlength{\LTleft}{0pt}\setlength{\LTright}{0pt}

\begin{longtable}{l@{\extracolsep\fill}p{4cm}p{7.0cm}}
\hline\hline\caption[Submodel interface]{Submodel interface
 subroutines. The subroutines are listed in the same order as they are
 called in \echam.}\\\hline\label{tabsubmodeli}
\endfirsthead
\caption[]{{\tt Submodel interface} --- continued}\\\hline
\endhead
\hline\multicolumn{3}{r}{\slshape table continued on next page}\\
\endfoot
\hline %\multicolumn{3}{r}{end of table}
\endlastfoot

Subroutine & Called in & Explanation \\\hline
{\tt init\_subm} & {\tt initialize.f90} & Initialization of
submodel. This comprises reading of specific submodel data. However,
this is not the right place to read gridded fields.\\
{\tt init\_subm\_memory} & {\tt init\_memory} of {\tt
  mo\_memory\_streams.f90} & Allocation of memory for submodel either
in streams or 2-- and 3--dimensional fields.\\
{\tt stepon\_subm} & {\tt stepon.f90} & Called at the beginning of a
new time step. Good for reading data at
regular time intervals.\\
{\tt physc\_subm\_1} & {\tt physc.f90} & Call in the ``physics'' part
of calculation. The ``physics'' processes are processes in one column
over a grid cell. This subroutine is called before the radiation
calculation.\\
{\tt radiation\_subm\_1} & {\tt rrtm\_interface} of {\tt
  mo\_radiation.f90} & Submodels can modify the optical properties of
the atmosphere here. It is called before the radiation fluxes are
calculated. \\
{\tt radiation\_subm\_2} & {\tt rrtm\_interface} of {\tt
  mo\_radiation.f90} & Good for radiation diagnostics performed by
submodels.\\ 
{\tt vdiff\_subm} & {\tt vdiff.f90} & In this subroutine, net surface 
fluxes can be calculated that will be used as boundary 
conditions in the vertical diffusion equation. Good for surface
emission fluxes and dry deposition fluxes. \\
{\tt rad\_heat\_subm} & {\tt radheat.f90} & Diagnostic of heating
rates. \\
{\tt physc\_subm\_2} & {\tt physc.f90} & First interface that is good
for calculation of 
physical processes of submodels like chemical kinetics or aerosol
physics. It is called before cloud physics but after {\tt vdiff} and {\tt radheat}\\
{\tt cuflx\_subm} & {\tt cuflux.f90} & Submodels can interfere with
convection here. E.g.~wet deposition of convective clouds has to be
implemented here.\\
{\tt cloud\_subm} & {\tt cloud.f90} & Implement interaction between
cloud physics and submodels here. E.g.~ ``wet chemistry'' should be
implemented here. Wet deposition of large scale precipitation has to
be implemented here.\\
{\tt physc\_subm\_3} & {\tt physc.f90} & Second interface that is good
for calculation of 
physical processes of submodels like chemical kinetics or aerosol
physics. It is called after cloud physics. \\
{\tt physc\_subm\_4} & {\tt physc.f90} & This is the right place for
submodel diagnostics after all physics processes are calculated.\\
{\tt free\_subm\_memory} & {\tt free\_memory} of {\tt
  mo\_memory\_streams.f90} & Deallocation of allocated submodel memory
here is mandatory, otherwise the internal rerun process will fail. In
addition, it is very important to set back all submodel switches to
their default values. In particular switches that indicate that
certain fields are allocated or certain data are read.\\
\end{longtable}

Inside these interface routines, the submodel specific routines should
be called. These calls have to be implemented all into {\tt
  mo\_submodel\_interface.f90} and the calls have to be effective if
and only if the respective submodel is switched on. Since {\tt
  mo\_submodel\_interface.f90} is part of the \echam{} code but the
submodel routines are not, the calls should be switched off/on by
compiler directives. In that case, the calls can be included in the
standard version of {\tt
  mo\_submodel\_interface.f90}. Neither an extra version of this module has
to be kept by the submodel users nor any update has to be done ``by
hand''. 

The parameter lists of the submodel interface routines are described
in the following subsections.

\subsubsection{Interface of {\tt init\_subm}}

\begin{lstlisting}[caption=init\_subm]
SUBROUTINE init_subm
\end{lstlisting}

This subroutine has no parameter list.\vspace{1cm}

\subsubsection{Interface of {\tt init\_subm\_memory}}

\begin{lstlisting}[caption=init\_subm\_memory]
SUBROUTINE init_subm_memory
\end{lstlisting}

This subroutine has no parameter list. In general, the fields
allocated here belong to the submodel. Since the submodel is supposed to be
organized in modules, global submodel fields should be defined as
module variables and can be brought to any submodel subroutine by use
statements. Streams are easily accessible by their
names. Nevertheless, subroutines of the kind {\tt get\_stream} or {\tt
  get\_stream\_element} are slow and should not be used
repeatedly. Instead, pointers to the stream elements can be stored as
global submodel variables and used later in the program.\vspace{1cm}

\subsubsection{Interface of {\tt stepon\_subm}}

\begin{lstlisting}[caption=stepon\_subm]
SUBROUTINE stepon_subm (current_date, next_date)
  TYPE(time_days)     :: current_date
  TYPE(time_days)     :: next_date
\end{lstlisting}

\setlength{\LTcapwidth}{\textwidth}
\setlength{\LTleft}{0pt}\setlength{\LTright}{0pt}

\begin{longtable}{l@{\extracolsep\fill}llp{7.0cm}}
\hline\hline\caption[Parameters of {\tt stepon\_subm}]{Parameter list
  of arguments passed to {\tt stepon\_subm}}\\\hline\label{tabpstepon_subm}
\endfirsthead
\caption[]{{\tt Parameters of {\tt stepon\_subm}} --- continued}\\\hline
\endhead
\hline\multicolumn{4}{r}{\slshape table continued on next page}\\
\endfoot
\hline %\multicolumn{4}{r}{end of table}
\endlastfoot
name & type & intent & description \\\hline
{\tt current\_date} & {\tt time\_days} &  & time and date of current
time step\\
{\tt next\_date} & {\tt time\_days} &  & time and date of prognostic time step  
\end{longtable}

\subsubsection{Interface of {\tt physc\_subm\_1}}

\begin{lstlisting}[caption=physc\_subm\_1]
SUBROUTINE physc_subm_1 (kproma, kbdim, klev, &
                         klevp1, ktrac, krow, &
                         papm1,  paphm1,      &
                         ptm1,   ptte,        &
                         pxtm1,  pxtte,       &  
                         pqm1,   pqte          ) 
  INTEGER,  INTENT(in)    :: kproma 
  INTEGER,  INTENT(in)    :: kbdim
  INTEGER,  INTENT(in)    :: klev 
  INTEGER,  INTENT(in)    :: klevp1
  INTEGER,  INTENT(in)    :: ktrac 
  INTEGER,  INTENT(in)    :: krow  
  REAL(dp), INTENT(in)    :: papm1 (kbdim,klev)
  REAL(dp), INTENT(in)    :: paphm1(kbdim,klevp1)
  REAL(dp), INTENT(in)    :: ptm1  (kbdim,klev)
  REAL(dp), INTENT(in)    :: ptte  (kbdim,klev)
  REAL(dp), INTENT(inout) :: pxtm1 (kbdim,klev,ktrac)
  REAL(dp), INTENT(inout) :: pxtte (kbdim,klev,ktrac)
  REAL(dp), INTENT(in)    :: pqm1  (kbdim,klev)
  REAL(dp), INTENT(in)    :: pqte  (kbdim,klev)
\end{lstlisting}

\begin{longtable}{l@{\extracolsep\fill}llp{7.0cm}}
\hline\hline\caption[Parameters of {\tt physc\_subm\_1}]{Parameter list
  of arguments passed to {\tt physc\_subm\_1}}\\\hline\label{tabpphysc_subm_1}
\endfirsthead
\caption[]{{\tt Parameters of {\tt physc\_subm\_1}} --- continued}\\\hline
\endhead
\hline\multicolumn{4}{r}{\slshape table continued on next page}\\
\endfoot
\hline %\multicolumn{4}{r}{end of table}
\endlastfoot
name & type & intent & description \\\hline
{\tt kproma} & integer & in & actual length of
block of geographical longitudes (one longitude block can contain grid
cells of various geographical latitudes)\\
{\tt kbdim} &  integer & in & maximum length of block of
geographical longitudes (one longitude block can contain grid cells of
various geographical latitudes)\\
{\tt klev} &  integer & in & number of model levels (layers)\\
{\tt klevp1} &  integer & in & number of layers plus one \\
{\tt ktrac} &  integer & in & number of tracers \\
{\tt krow} &  integer & in & index number of block of geographical
longitudes\\ 
{\tt papm1(kbdim,klev)} & double prec. & in & pressure of dry air at center of model
layers at time step $t-\Delta t$\\
{\tt paphm1(kbdim,klevp1)} & double prec. & in & pressure of dry air at
interfaces between model layers at time step $t-\Delta t$\\
{\tt ptm1(kbdim,klev)} & double prec. & in & temperature at center of
model layers at time step $t-\Delta t$\\
{\tt ptte(kbdim,klev)}& double prec. & in & temperature tendency at
center of model layers accumulated over all processes of actual time
step until call of this subroutine\\
{\tt pxtm1(kbdim,klev,ktrac)} & double prec. & inout & tracer mass or
molar mixing ratio with respect to dry air at center of model layers at
time step $t-\Delta t$\\
{\tt pxtte(kbdim,klev,ktrac)} & double prec. & inout & tendency of tracer
mass or molar mixing ratio with respect to dry air at center of model
layers accumulated over all processes of actual time
step until call of this subroutine \\
{\tt pqm1(kbdim,klev)} & double prec. & in & specific humidity (with
respect to dry air) at
center of model layers at time step $t-\Delta t$ \\
{\tt pqte(kbdim,klev)} & double prec. & in & tendency of specific
humidity (with respect to dry air) at center of model layers
accumulated over all processes of actual time 
step until call of this subroutine \\
\end{longtable}

\subsubsection{Interface of {\tt radiation\_subm\_1}}

\begin{lstlisting}[caption=radiation\_subm\_1]
SUBROUTINE radiation_subm_1 &
   (kproma           ,kbdim            ,klev         ,krow  ,&
    ktrac            ,kaero            ,kpband       ,kb_sw ,&
    aer_tau_sw_vr    ,aer_piz_sw_vr    ,aer_cg_sw_vr        ,&
    aer_tau_lw_vr                                           ,&
    ppd_hl           ,pxtm1                                  )
  INTEGER, INTENT(in) :: kproma
  INTEGER, INTENT(in) :: kbdim 
  INTEGER, INTENT(in) :: klev  
  INTEGER, INTENT(in) :: krow  
  INTEGER, INTENT(in) :: ktrac 
  INTEGER, INTENT(in) :: kaero 
  INTEGER, INTENT(in) :: kpband
  INTEGER, INTENT(in) :: kb_sw 
  REAL(dp), INTENT(inout) :: aer_tau_sw_vr(kbdim,klev,kb_sw), &
                             aer_piz_sw_vr(kbdim,klev,kb_sw), &
                             aer_cg_sw_vr(kbdim,klev,kb_sw),  &
                             aer_tau_lw_vr(kbdim,klev,kpband),&
  REAL(dp), INTENT(in):: ppd_hl(kbdim,klev)     
  REAL(dp), INTENT(in):: pxtm1(kbdim,klev,ktrac)
\end{lstlisting}

\begin{longtable}{p{4.5cm}@{\extracolsep\fill}llp{7.0cm}}
\hline\hline\caption[Parameters of {\tt radiation\_subm\_1}]{Parameter list
  of arguments passed to {\tt radiation\_subm\_1}}\\\hline\label{tabpradiation_subm_1}
\endfirsthead
\caption[]{{\tt Parameters of {\tt radiation\_subm\_1}} --- continued}\\\hline
\endhead
\hline\multicolumn{4}{r}{\slshape table continued on next page}\\
\endfoot
\hline %\multicolumn{4}{r}{end of table}
\endlastfoot
name & type & intent & description \\\hline
{\tt kproma} & integer & in & actual length of
block of geographical longitudes (one longitude block can contain grid
cells of various geographical latitudes)\\
{\tt kbdim} &  integer & in & maximum length of block of
geographical longitudes (one longitude block can contain grid cells of
various geographical latitudes)\\
{\tt klev} &  integer & in & number of model levels (layers)\\
{\tt krow} &  integer & in & index number of block of geographical 
longitudes\\
{\tt ktrac} &  integer & in & number of tracers \\
{\tt kaero} & integer & in & switch for aerosol radiation coupling\\
{\tt kpband} & integer & in & number of bands in the thermal radiation
wavelength range \\
{\tt kb\_sw} & integer & in & number of bands in the solar radiation
wavelength range \\
{\tt aer\_tau\_sw\_vr (kbdim,klev,kb\_sw)} & double prec. & inout &
aerosol optical depth of model layers for solar radiation wavelength
bands. Here, the model layers are ordered from the Earth's surface (level
index 1) to the
top of the atmosphere (level index {\tt klev}) as indicated by
{\tt \_vr} $=$ vertically reversed\\
{\tt aer\_piz\_sw\_vr (kbdim,klev,kb\_sw)} & double prec. & inout &
aerosol single scattering albedo for solar radiation wavelength
bands. Here, the model layers are ordered from the Earth's surface (level
index 1) to the
top of the atmosphere (level index {\tt klev}) as indicated by
{\tt \_vr} $=$ vertically reversed\\
{\tt aer\_cg\_sw\_vr (kbdim,klev,kb\_sw)} & double prec. & inout &
aerosol asymmetry factor for solar radiation wavelength
bands. Here, the model layers are ordered from the Earth's surface (level
index 1) to the
top of the atmosphere (level index {\tt klev}) as indicated by
{\tt \_vr} $=$ vertically reversed\\
{\tt aer\_tau\_lw\_vr (kbdim,klev,kpband)} & double prec. & inout &
aerosol optical depth of model layers for thermal radiation wavelength
bands. Here, the model layers are ordered from the Earth's surface (level
index 1) to the
top of the atmosphere (level index {\tt klev}) as indicated by
{\tt \_vr} $=$ vertically reversed\\
{\tt ppd\_hl(kbdim,klev)} & double prec. & in & absolute value of dry
air pressure difference between upper and lower limit of model layers
at time $t-\Delta t$\\
{\tt pxtm1(kbdim,klev,ktrac)} & double prec. & in & tracer mass or
molar mixing ratio with respect to dry air at center of model layers at
time step $t-\Delta t$\\
\end{longtable}

\subsubsection{Interface of {\tt radiation\_subm\_2}}

\begin{lstlisting}[caption=radiation\_subm\_2]
SUBROUTINE radiation_subm_2(kproma, kbdim, krow, klev, &
                            ktrac, kaero,              &
                            pxtm1                      )
  INTEGER,  INTENT(in) :: kproma
  INTEGER,  INTENT(in) :: kbdim 
  INTEGER,  INTENT(in) :: krow  
  INTEGER,  INTENT(in) :: klev  
  INTEGER,  INTENT(in) :: ktrac 
  INTEGER,  INTENT(in) :: kaero 
  REAL(dp), INTENT(in) :: pxtm1 (kbdim,klev,ktrac)
\end{lstlisting}

\begin{longtable}{p{4.5cm}@{\extracolsep\fill}llp{6.7cm}}
\hline\hline\caption[Parameters of {\tt radiation\_subm\_2}]{Parameter list
  of arguments passed to {\tt radiation\_subm\_2}}\\\hline\label{tabpradiation_subm_2}
\endfirsthead
\caption[]{{\tt Parameters of {\tt radiation\_subm\_2}} --- continued}\\\hline
\endhead
\hline\multicolumn{4}{r}{\slshape table continued on next page}\\
\endfoot
\hline %\multicolumn{4}{r}{end of table}
\endlastfoot
name & type & intent & description \\\hline
{\tt kproma} & integer & in & actual length of
block of geographical longitudes (one longitude block can contain grid
cells of various geographical latitudes)\\
{\tt kbdim} &  integer & in & maximum length of block of
geographical longitudes (one longitude block can contain grid cells of
various geographical latitudes)\\
{\tt krow} &  integer & in & index number of block of geographical\\
{\tt klev} &  integer & in & number of model levels (layers)\\
{\tt ktrac} &  integer & in & number of tracers \\
{\tt kaero} & integer & in & switch for aerosol radiation coupling\\
{\tt pxtm1(kbdim,klev,ktrac)} & double prec. & in & tracer mass or
molar mixing ratio with respect to dry air at center of model layers at
time step $t-\Delta t$\\
\end{longtable}


\subsubsection{Interface of {\tt vdiff\_subm}}

\begin{lstlisting}[caption=vdiff\_subm]
SUBROUTINE vdiff_subm(kproma, kbdim,  klev,   klevp1,        &
                      ktrac,  krow,                          &
                      ptm1,   pum1,   pvm1,   pqm1,          &
                      papm1,  paphm1, paphp1, pgeom1, ptslm1,&
                      pxtm1,  pseaice,pforest,               &
                      pfrl,   pfrw,   pfri,   pcvs,   pcvw,  &
                      pvgrat, ptsw,   ptsi,                  &
                      pu10,   pv10,                          &
                      paz0,   paz0l,  paz0w,  paz0i,         &
                      pcfm,   pcfnc,  pepdu2, pkap,          &
                      pri,    ptvir1, ptvl,                  &
                      psrfl,  pcdn,  pqss,   pvlt,           &
                      loland,                                &
                      pxtte,  pxtems,                        &
                      pxlm1,  pxim1                          )
  INTEGER,  INTENT(in) :: kproma
  INTEGER,  INTENT(in) :: kbdim                      
  INTEGER,  INTENT(in) :: klev                       
  INTEGER,  INTENT(in) :: klevp1                     
  INTEGER,  INTENT(in) :: ktrac                      
  INTEGER,  INTENT(in) :: krow                       
  REAL(dp), INTENT(in) :: ptm1     (kbdim,klev)      
  REAL(dp), INTENT(in) :: pum1     (kbdim,klev)      
  REAL(dp), INTENT(in) :: pvm1     (kbdim,klev)      
  REAL(dp), INTENT(in) :: pqm1     (kbdim,klev)      
  REAL(dp), INTENT(in) :: papm1    (kbdim,klev)      
  REAL(dp), INTENT(in) :: paphm1   (kbdim,klev+1)    
  REAL(dp), INTENT(in) :: paphp1   (kbdim,klev+1)    
  REAL(dp), INTENT(in) :: pgeom1   (kbdim,klev)      
  REAL(dp), INTENT(in) :: ptslm1   (kbdim)           
  REAL(dp), INTENT(inout) :: pxtm1    (kbdim,klev,ktrac)
  REAL(dp), INTENT(in) :: pseaice  (kbdim)         
  REAL(dp), INTENT(in) :: pforest  (kbdim)         
  REAL(dp), INTENT(in) :: pfrl     (kbdim)         
  REAL(dp), INTENT(in) :: pfrw     (kbdim)         
  REAL(dp), INTENT(in) :: pfri     (kbdim)         
  REAL(dp), INTENT(in) :: pcvs     (kbdim)         
  REAL(dp), INTENT(in) :: pcvw     (kbdim)         
  REAL(dp), INTENT(in) :: pvgrat   (kbdim)         
  REAL(dp), INTENT(in) :: ptsw     (kbdim)         
  REAL(dp), INTENT(in) :: ptsi     (kbdim)         
  REAL(dp), INTENT(in) :: pu10     (kbdim)         
  REAL(dp), INTENT(in) :: pv10     (kbdim)         
  REAL(dp), INTENT(in) :: paz0     (kbdim)         
  REAL(dp), INTENT(in) :: paz0l    (kbdim)         
  REAL(dp), INTENT(in) :: paz0w    (kbdim)         
  REAL(dp), INTENT(in) :: paz0i    (kbdim)         
  REAL(dp), INTENT(in) :: pcfm     (kbdim,klev)    
  REAL(dp), INTENT(in) :: pcfnc    (kbdim)         
  REAL(dp), INTENT(in) :: pepdu2                   
  REAL(dp), INTENT(in) :: pkap                     
  REAL(dp), INTENT(in) :: pri      (kbdim)         
  REAL(dp), INTENT(in) :: ptvir1   (kbdim,klev)    
  REAL(dp), INTENT(in) :: ptvl     (kbdim)         
  REAL(dp), INTENT(in) :: psrfl    (kbdim)         
  REAL(dp), INTENT(in) :: pcdn     (kbdim)         
  REAL(dp), INTENT(in) :: pqss     (kbdim,klev)    
  REAL(dp), INTENT(in) :: pvlt     (kbdim)         
  LOGICAL,  INTENT(in) :: loland   (kbdim)         
  REAL(dp), INTENT(inout) :: pxtte    (kbdim,klev,ktrac)
  REAL(dp), INTENT(inout) :: pxtems   (kbdim,ktrac)     
  REAL(dp), INTENT(in) :: pxlm1    (kbdim,klev)      
  REAL(dp), INTENT(in) :: pxim1    (kbdim,klev)      
\end{lstlisting}

\begin{longtable}{l@{\extracolsep\fill}llp{7.0cm}}
\hline\hline\caption[Parameters of {\tt vdiff\_subm}]{Parameter list
  of arguments passed to {\tt vdiff\_subm}}\\\hline\label{tabpvdiff_subm}
\endfirsthead
\caption[]{{\tt Parameters of {\tt vdiff\_subm}} --- continued}\\\hline
\endhead
\hline\multicolumn{4}{r}{\slshape table continued on next page}\\
\endfoot
\hline %\multicolumn{4}{r}{end of table}
\endlastfoot
name & type & intent & description \\\hline
{\tt kproma} & integer & in & actual length of
block of geographical longitudes (one longitude block can contain grid
cells of various geographical latitudes)\\
{\tt kbdim} &  integer & in & maximum length of block of
geographical longitudes (one longitude block can contain grid cells of
various geographical latitudes)\\
{\tt klev} &  integer & in & number of model levels (layers)\\
{\tt klevp1} &  integer & in & number of layers plus one \\
{\tt ktrac} &  integer & in & number of tracers \\
{\tt krow} &  integer & in & index number of block of geographical
longitudes\\ 
{\tt ptm1(kbdim,klev)} & double prec. & in & temperature at center of
model layers at time step $t-\Delta t$\\
{\tt pum1(kbdim,klev)} & double prec. & in & zonal wind component at
center of model layers at time step $t-\Delta t$ \\
{\tt pvm1(kbdim,klev)} & double prec. & in & meridional wind component at
center of model layers at time step $t-\Delta t$ \\
{\tt pqm1(kbdim,klev)} & double prec. & in & specific humidity (with
respect to dry air) at
center of model layers at time step $t-\Delta t$ \\
{\tt papm1(kbdim,klev)} & double prec. & in & pressure of dry air at center of model
layers at time step $t-\Delta t$\\
{\tt paphm1(kbdim,klevp1)} & double prec. & in & pressure of dry air at
interfaces between model layers at time step $t-\Delta t$\\
{\tt paphp1(kbdim,klevp1)} & double prec. & in & pressure of dry air at
interfaces between model layers at prognostic time step $t+\Delta t$\\
{\tt pgeom1(kbdim,klev)} & double prec. & in & geopotential at center
of model layers at time step $t-\Delta t$\\
{\tt ptslm1(kbdim)} & double prec. & in & surface temperature at time
step $t-\Delta t$ \\
{\tt pxtm1(kbdim,klev,ktrac)} & double prec. & inout & tracer mass or
molar mixing ratio with respect to dry air at center of model layers at
time step $t-\Delta t$\\
{\tt pseaice(kbdim)} & double prec. & in & sea ice fraction \\
{\tt pforest(kbdim)} & double prec. & in & forest fraction \\
{\tt pfrl(kbdim)} & double prec. & in & land fraction \\
{\tt pfrw(kbdim)} & double prec. & in & surface water fraction \\
{\tt pfri(kbdim)} & double prec. & in & surface ice fraction \\
{\tt pcvs(kbdim)} & double prec. & in & snow cover fraction \\
{\tt pcvw(kbdim)} & double prec. & in & wet skin fraction \\
{\tt pvgrat(kbdim)} & double prec. & in & vegetation ratio \\
{\tt ptsw(kbdim)} & double prec. & in & surface temperature over water \\
{\tt ptsi(kbdim)} & double prec. & in & surface temperature over ice\\
{\tt pu10(kbdim)} & double prec. & in & zonal wind component 10~m
above the surface \\
{\tt pv10(kbdim)} & double prec. & in & meridional wind component 10~m
above the surface \\
{\tt paz0(kbdim)} & double prec. & in & roughness length \\
{\tt paz0l(kbdim)} & double prec. & in & roughness length over land \\
{\tt paz0w(kbdim)} & double prec. & in & roughness length over water \\
{\tt paz0i(kbdim)} & double prec. & in & roughness length over ice\\
{\tt pcfm(kbdim,klev)} & double prec. & in & stability dependent
momentum transfer coefficient at center of model layers \\
{\tt pcfnc(kbdim)} & double prec. & in & function of heat transfer
coefficient; not set? \\
{\tt pepdu2} & double prec. & in & a constant set in {\tt
  vdiff.f90}. It is used e.g.~in {\tt mo\_surface\_land} as the allowed
minimum of the square of the absolute wind velocity\\
{\tt pkap} & double prec. & in & von Karman constant\\
{\tt pri(kbdim)} & double prec. & in & Richardson number for moist air\\
{\tt ptvir1(kbdim,klev)} & double prec. & in & potential density temperature \\
{\tt ptvl(kbdim)} & double prec. & in & virtual temperature over land \\
{\tt psrfl(kbdim)} & double prec. & in & net surface solar radiation
flux at time (?) $t$ \\
{\tt pcdn(kbdim)} & double prec. & in & heat transfer coefficient averaged over land, water and ice cover fraction of a grid box \\
{\tt pqss(kbdim,klev)} & double prec. & in  & specific humidity at
which the air is saturated at time (?) $t$ \\
{\tt pvlt(kbdim)} & double prec. & in & obsolete, will be removed \\
{\tt loland(kbdim)} & double prec. & in & logical land mask including
glaciers \\
{\tt pxtte(kbdim,klev,ktrac)} & double prec. & inout & tendency of tracer
mass or molar mixing ratio with respect to dry air at center of model
layers accumulated over all processes of actual time
step until call of this subroutine \\
{\tt pxtems(kbdim,ktrac)} & double prec. & inout & surface emission
flux \\
{\tt pxlm1} & double prec. & in & cloud liquid water content at center
of model layers at time step $t-\Delta t$ \\
{\tt pxim1} & double prec. & in & cloud water ice content at center
of model layers at time step $t-\Delta t$ \\
\end{longtable}





\subsubsection{Interface of {\tt rad\_heat\_subm}}

\begin{lstlisting}[caption=rad\_heat\_subm]
SUBROUTINE radheat_subm
   (kproma       ,kbdim           ,klev             ,&
    klevp1       ,krow            ,pconvfact        ,&
    pflxs        ,pflxt)

  INTEGER, INTENT(in)  :: kproma     
  INTEGER, INTENT(in)  :: kbdim      
  INTEGER, INTENT(in)  :: klev       
  INTEGER, INTENT(in)  :: klevp1     
  INTEGER, INTENT(in)  :: krow       
  REAL(dp), INTENT(in) :: pconvfact(kbdim,klev) 
  REAL(dp), INTENT(in) :: pflxs(kbdim,klevp1), pflxt(kbdim,klevp1) 
\end{lstlisting}

\begin{longtable}{l@{\extracolsep\fill}llp{7.0cm}}
\hline\hline\caption[Parameters of {\tt rad\_heat\_subm}]{Parameter list
  of arguments passed to {\tt rad\_heat\_subm}}\\\hline\label{tabrad_heat_subm}
\endfirsthead
\caption[]{{\tt Parameters of {\tt rad\_heat\_subm}} --- continued}\\\hline
\endhead
\hline\multicolumn{4}{r}{\slshape table continued on next page}\\
\endfoot
\hline %\multicolumn{4}{r}{end of table}
\endlastfoot
name & type & intent & description \\\hline
{\tt kproma} & integer & in & actual length of
block of geographical longitudes (one longitude block can contain grid
cells of various geographical latitudes)\\
{\tt kbdim} &  integer & in & maximum length of block of
geographical longitudes (one longitude block can contain grid cells of
various geographical latitudes)\\
{\tt klev} &  integer & in & number of model levels (layers)\\
{\tt klevp1} &  integer & in & number of layers plus one \\
{\tt krow} &  integer & in & index number of block of geographical
longitudes\\ 
{\tt pconvfact(kbdim,klevp1)} & double prec. & in & conversion factor
for conversion of energy flux differences between upper and lower
layer boundary to heating rate of the air in this layer. The factor is
calculated for the time at time step $t-\Delta t$.\\
{\tt pflxs(kbdim,klevp1)} & double prec. & in & net energy flux of solar
radiation integrated over all solar radiation bands at the layer
interfaces for time $t$\\
{\tt pflxt(kbdim,klevp1)} & double prec. & in & net energy flux of thermal
radiation integrated over all thermal radiation bands at the layer
interfaces for time $t$\\
\end{longtable}

\subsubsection{Interface of {\tt physc\_subm\_2}}

\begin{lstlisting}[caption=physc\_subm\_2]
SUBROUTINE physc_subm_2                               &
          (kproma, kbdim, klev,  klevp1, ktrac, krow, &
           itrpwmo, itrpwmop1,                        &
           paphm1, papm1, paphp1, papp1,              &
           ptm1,   ptte,  ptsurf,                     &
           pqm1,   pqte,  pxlm1, pxlte, pxim1, pxite, &
           pxtm1,  pxtte,                             &
           paclc,  ppbl,                              &
           loland, loglac                            )   
  INTEGER, INTENT(in)  :: kproma                     
  INTEGER, INTENT(in)  :: kbdim                      
  INTEGER, INTENT(in)  :: klev                       
  INTEGER, INTENT(in)  :: klevp1                     
  INTEGER, INTENT(in)  :: ktrac                      
  INTEGER,  INTENT(in) :: krow                       
  INTEGER,  INTENT(in) :: itrpwmo  (kbdim)           
  INTEGER,  INTENT(in) :: itrpwmop1(kbdim)           

  REAL(dp), INTENT(in) :: paphm1   (kbdim,klev+1)    
  REAL(dp), INTENT(in) :: papm1    (kbdim,klev)      
  REAL(dp), INTENT(in) :: paphp1   (kbdim,klev+1)    
  REAL(dp), INTENT(in) :: papp1    (kbdim,klev)      
  REAL(dp), INTENT(in) :: ptm1     (kbdim,klev)      
  REAL(dp), INTENT(in) :: ptte     (kbdim,klev)      
  REAL(dp), INTENT(in) :: ptsurf   (kbdim)           
  REAL(dp), INTENT(in) :: pqm1     (kbdim,klev)      
  REAL(dp), INTENT(in) :: pqte     (kbdim,klev)      
  REAL(dp), INTENT(in) :: pxlm1    (kbdim,klev)      
  REAL(dp), INTENT(in) :: pxlte    (kbdim,klev)      
  REAL(dp), INTENT(in) :: pxim1    (kbdim,klev)      
  REAL(dp), INTENT(in) :: pxite    (kbdim,klev)      
  REAL(dp), INTENT(in) :: paclc    (kbdim,klev)      
  REAL(dp), INTENT(in) :: ppbl     (kbdim)           
  REAL(dp), INTENT(inout) :: pxtm1 (kbdim,klev,ktrac)
  REAL(dp), INTENT(inout) :: pxtte (kbdim,klev,ktrac)
  LOGICAL,  INTENT(in) :: loland   (kbdim)           
  LOGICAL,  INTENT(in) :: loglac   (kbdim)           
\end{lstlisting}

\begin{longtable}{l@{\extracolsep\fill}llp{7.0cm}}
\hline\hline\caption[Parameters of {\tt physc\_subm\_2}]{Parameter list
  of arguments passed to {\tt physc\_subm\_2}}\\\hline\label{tabphysc_subm_2}
\endfirsthead
\caption[]{{\tt Parameters of {\tt physc\_subm\_2}} --- continued}\\\hline
\endhead
\hline\multicolumn{4}{r}{\slshape table continued on next page}\\
\endfoot
\hline %\multicolumn{4}{r}{end of table}
\endlastfoot
name & type & intent & description \\\hline
{\tt kproma} & integer & in & actual length of
block of geographical longitudes (one longitude block can contain grid
cells of various geographical latitudes)\\
{\tt kbdim} &  integer & in & maximum length of block of
geographical longitudes (one longitude block can contain grid cells of
various geographical latitudes)\\
{\tt klev} &  integer & in & number of model levels (layers)\\
{\tt klevp1} &  integer & in & number of layers plus one \\
{\tt ktrac} &  integer & in & number of tracers \\
{\tt krow} &  integer & in & index number of block of geographical
longitudes\\ 
{\tt itrpwmo(kbdim)} & integer & in & index of model level at which
meteorological tropopause was detected at time $t$ \\
{\tt itrpwmop1(kbdim)} & integer & in & index of model level at which
meteorological tropopause was detected plus 1 at time $t$ \\
{\tt paphm1(kbdim,klevp1)} & double prec. & in & pressure of dry air at
interfaces between model layers at time step $t-\Delta t$\\
{\tt papm1(kbdim,klev)} & double prec. & in & pressure of dry air at center of model
layers at time step $t-\Delta t$\\
{\tt paphp1(kbdim,klevp1)} & double prec. & in & pressure of dry air at
interfaces between model layers at prognostic time step $t+\Delta t$\\
{\tt papp1(kbdim,klev)} & double prec. & in & pressure of dry air at center of model
layers at time step $t+\Delta t$\\
{\tt ptm1(kbdim,klev)} & double prec. & in & temperature at center of
model layers at time step $t-\Delta t$\\
{\tt ptte(kbdim,klev)}& double prec. & in & temperature tendency at
center of model layers accumulated over all processes of actual time
step until call of this subroutine\\
{\tt ptsurf(kbdim)} & double prec. & in & surface temperature at time
step $t$\\
{\tt pqm1(kbdim,klev)} & double prec. & in & specific humidity (with
respect to dry air) at
center of model layers at time step $t-\Delta t$ \\
{\tt pqte(kbdim,klev)} & double prec. & in & tendency of specific
humidity (with respect to dry air) at center of model layers
accumulated over all processes of actual time 
step until call of this subroutine \\
{\tt pxlm1} & double prec. & in & cloud liquid water content at center
of model layers at time step $t-\Delta t$ \\
{\tt pxlte} & double prec. & in & cloud liquid water tendency at
center of model layers accumulated over all processes of actual time
step until call of this subroutine \\
{\tt pxim1} & double prec. & in & cloud water ice content at center
of model layers at time step $t-\Delta t$ \\
{\tt pxite} & double prec. & in & cloud water ice tendency at
center of model layers accumulated over all processes of actual time
step until call of this subroutine \\
{\tt pxtm1(kbdim,klev,ktrac)} & double prec. & inout & tracer mass or
molar mixing ratio with respect to dry air at center of model layers at
time step $t-\Delta t$\\
{\tt pxtte(kbdim,klev,ktrac)} & double prec. & inout & tendency of tracer
mass or molar mixing ratio with respect to dry air at center of model
layers accumulated over all processes of actual time
step until call of this subroutine \\
{\tt paclc(kbdim,klev)} & double prec. & in & cloud fraction at center
of model layers at time step $t$ \\
{\tt ppbl(kbdim)} & double prec. & in & model layer index of geometrically
highest model layer of planetary boundary layer converted to a real
number at time $t$\\
{\tt loland(kbdim)} & double prec. & in & logical land mask including
glaciers \\
{\tt loglac(kbdim)} & double prec. & in & logical glacier mask \\
\end{longtable}

\subsubsection{Interface of {\tt cuflx\_subm}}

\begin{lstlisting}[caption=cuflx\_subm]
SUBROUTINE cuflx_subm(kbdim,  kproma,  klev,    ktop,   krow, &
                      pxtenh, pxtu,    prhou,                 &
                      pmfu,   pmfuxt,                         &
                      pmlwc,  pmiwc,   pmratepr,pmrateps,     &
                      pfrain, pfsnow,  pfevapr, pfsubls,      &
                      paclc,  pmsnowacl,                      &
                      ptu,    pdpg,                           &
                      pxtte                                   )
  INTEGER, INTENT(in)     :: kbdim, kproma, klev, ktop, &
                             krow
  REAL(dp), INTENT(in)    :: pdpg(kbdim,klev),          &
                             pmratepr(kbdim,klev),      &
                             pmrateps(kbdim,klev),      &
                             pmsnowacl(kbdim,klev),     &
                             ptu(kbdim,klev),           &
                             pfrain(kbdim,klev),        &
                             pfsnow(kbdim,klev),        &
                             pfevapr(kbdim,klev),       &
                             pfsubls(kbdim,klev),       &
                             pmfu(kbdim,klev),          &
                             paclc(kbdim,klev),         &
                             prhou(kbdim,klev)           
  REAL(dp), INTENT(inout) :: pxtte(kbdim,klev,ntrac),   &
                             pmlwc(kbdim,klev),         &
                             pmiwc(kbdim,klev),         &
                             pxtenh(kbdim,klev,ntrac),  &
                             pxtu(kbdim,klev,ntrac),    &
                             pmfuxt(kbdim,klev,ntrac)
\end{lstlisting}

\begin{longtable}{l@{\extracolsep\fill}llp{7.0cm}}
\hline\hline\caption[Parameters of {\tt cuflx\_subm}]{Parameter list
  of arguments passed to {\tt cuflx\_subm}}\\\hline\label{tabcuflx_subm}
\endfirsthead
\caption[]{{\tt Parameters of {\tt cuflx\_subm}} --- continued}\\\hline
\endhead
\hline\multicolumn{4}{r}{\slshape table continued on next page}\\
\endfoot
\hline %\multicolumn{4}{r}{end of table}
\endlastfoot
name & type & intent & description \\\hline
{\tt kbdim} &  integer & in & maximum length of block of
geographical longitudes (one longitude block can contain grid cells of
various geographical latitudes)\\
{\tt kproma} & integer & in & actual length of
block of geographical longitudes (one longitude block can contain grid
cells of various geographical latitudes)\\
{\tt klev} &  integer & in & number of model levels (layers)\\
{\tt ktop} & integer & in & Could be the minimum model layer index of
cloud top layers over one block. In fact, it is set to 1 in {\tt
  cuflx} \\
{\tt pxtenh(kbdim,klev,ntrac)} & double prec. & inout & tracer mass or
molar mixing ratio with respect to dry air at center of model layers at
time step $t+\Delta t$ \\
{\tt pxtu(kbdim,klev,ntrac)} & double prec. & inout & tracer mass
mixing ratio with respect to cloud water at center of model layers in
the liquid or solid cloud water phase at 
time step $t+\Delta t$ \\
{\tt prhou(kbdim,klev)} & double prec. & in & dry air density at center of
model layers at time step $t+\Delta t$ \\
{\tt pmfu(kbdim,klev)} & double prec. & in & convective air mass flux
at center of model layers at time $t$ \\
{\tt pmfuxt(kbdim,klev,ntrac)} & double prec. & inout & net tracer
mass flux due to convective transport and wet deposition at center of
model layers at time step $t+\Delta t$ on exit (in mass mixing ratio
per time)\\
{\tt pmlwc(kbdim,klev)} & double prec. & inout & liquid water content
(mass of liquid water per mass of dry air)
at center of model layers at time $t+\Delta t$ on exit \\
{\tt pmiwc(kbdim,klev)} & double prec. & inout & ice water content
(mass of water ice per mass of dry air)
at center of model layers at time $t+\Delta t$ on exit \\
{\tt pmratepr(kbdim,klev)} & double prec. & in & rain formation rate 
in mass water per mass dry air converted to rain at center of model
layers at time step $t$\\
{\tt pmrateps(kbdim,klev)} & double prec. & in & ice formation rate 
in mass water per mass dry air converted to snow at center of model
layers at time step $t$\\
{\tt pfrain(kbdim,klev)} & double prec. & in & rain flux at centers of
model layers per grid box area at time $t$, evaporation not taken into account \\
{\tt pfsnow(kbdim,klev)} & double prec. & in & snow flux at centers of
model layers per grid box area at time $t$, evaporation not taken into account \\
{\tt pfevapr(kbdim,klev)} & double prec. & in & evaporation of rain at
centers of model layers per grid box area at time $t$ \\
{\tt pfsubls(kbdim,klev)} & double prec. & in & sublimation of snow at
centers of model layers per grid box area at time $t$ \\
{\tt paclc(kbdim,klev)} & double prec. & in & cloud cover at center of
model layer at time step $t$\\
{\tt pmsnowaclc(kbdim,klev)} & double prec. & in & accretion rate of
snow at center of
model layer at time step $t$\\
{\tt ptu(kbdim,klev)} & double prec. & in & temperature at center of
model layer at time step $t-\Delta t$\\
{\tt pdpg(kbdim,klev)} & double prec. & in & geopotential height at
center of model level \\
{\tt pxtte(kbdim,klev,ktrac)} & double prec. & inout & tendency of tracer
mass or molar mixing ratio with respect to dry air at center of model
layers accumulated over all processes of actual time
step until call of this subroutine \\
\end{longtable}

\subsubsection{Interface of {\tt cloud\_subm}}

\begin{lstlisting}[caption=cloud\_subm]
SUBROUTINE cloud_subm(                                   &
           kproma,     kbdim,      klev,       ktop,     &
           krow,                                         &
           pmlwc,      pmiwc,      pmratepr,   pmrateps, &
           pfrain,     pfsnow,     pfevapr,    pfsubls,  &
           pmsnowacl,  paclc,      ptm1,       ptte,     &
           pxtm1,      pxtte,      paphp1,     papp1,    &
           prhop1,     pclcpre)

  INTEGER,  INTENT(in)    :: kproma                      
  INTEGER,  INTENT(in)    :: kbdim                       
  INTEGER,  INTENT(in)    :: klev                        
  INTEGER,  INTENT(in)    :: ktop                        
  INTEGER,  INTENT(in)    :: krow                        
  REAL(dp), INTENT(in)    :: pclcpre  (kbdim,klev)       
  REAL(dp), INTENT(in)    :: pfrain   (kbdim,klev)       
  REAL(dp), INTENT(in)    :: pfsnow   (kbdim,klev)       
  REAL(dp), INTENT(in)    :: pfevapr  (kbdim,klev)       
  REAL(dp), INTENT(in)    :: pfsubls  (kbdim,klev)       
  REAL(dp), INTENT(in)    :: pmsnowacl(kbdim,klev)       
  REAL(dp), INTENT(in)    :: ptm1     (kbdim,klev)       
  REAL(dp), INTENT(in)    :: ptte     (kbdim,klev)       
  REAL(dp), INTENT(in)    :: prhop1   (kbdim,klev)       
  REAL(dp), INTENT(in)    :: papp1    (kbdim,klev)       
  REAL(dp), INTENT(in)    :: paphp1   (kbdim,klev+1)     
  REAL(dp), INTENT(inout) :: paclc    (kbdim,klev)       
  REAL(dp), INTENT(inout) :: pmlwc    (kbdim,klev)       
  REAL(dp), INTENT(inout) :: pmiwc    (kbdim,klev)       
  REAL(dp), INTENT(inout) :: pmratepr (kbdim,klev)       
  REAL(dp), INTENT(inout) :: pmrateps (kbdim,klev)       
  REAL(dp), INTENT(in)    :: pxtm1    (kbdim,klev,ntrac) 
  REAL(dp), INTENT(inout) :: pxtte    (kbdim,klev,ntrac) 
\end{lstlisting}

\begin{longtable}{l@{\extracolsep\fill}llp{7.0cm}}
\hline\hline\caption[Parameters of {\tt cloud\_subm}]{Parameter list
  of arguments passed to {\tt cloud\_subm}}\\\hline\label{tabcloud_subm}
\endfirsthead
\caption[]{{\tt Parameters of {\tt cloud\_subm}} --- continued}\\\hline
\endhead
\hline\multicolumn{4}{r}{\slshape table continued on next page}\\
\endfoot
\hline %\multicolumn{4}{r}{end of table}
\endlastfoot
name & type & intent & description \\\hline
{\tt kproma} & integer & in & actual length of
block of geographical longitudes (one longitude block can contain grid
cells of various geographical latitudes)\\
{\tt kbdim} &  integer & in & maximum length of block of
geographical longitudes (one longitude block can contain grid cells of
various geographical latitudes)\\
{\tt klev} &  integer & in & number of model levels (layers)\\
{\tt ktop} & integer & in & Could be the minimum model layer index of
cloud top layers over one block. In fact, it is set to 1 in {\tt
  cuflx} \\
{\tt krow} &  integer & in & index number of block of geographical
longitudes\\ 
{\tt pmlwc(kbdim,klev)} & double prec. & inout & liquid water content
(mass of liquid water per mass of dry air)
at center of model layers at time $t+\Delta t$ on exit \\
{\tt pmiwc(kbdim,klev)} & double prec. & inout & ice water content
(mass of water ice per mass of dry air)
at center of model layers at time $t+\Delta t$ on exit \\
{\tt pmratepr(kbdim,klev)} & double prec. & inout & rain formation rate 
in mass water per mass dry air converted to rain at center of model
layers at time step $t$\\
{\tt pmrateps(kbdim,klev)} & double prec. & inout & ice formation rate 
in mass water per mass dry air converted to snow at center of model
layers at time step $t$\\
{\tt pfrain(kbdim,klev)} & double prec. & in & rain flux at centers of
model layers per grid box area at time $t$, evaporation not taken into account \\
{\tt pfsnow(kbdim,klev)} & double prec. & in & snow flux at centers of
model layers per grid box area at time $t$, evaporation not taken into account \\
{\tt pfevapr(kbdim,klev)} & double prec. & in & evaporation of rain at
centers of model layers per grid box area at time $t$ \\
{\tt pfsubls(kbdim,klev)} & double prec. & in & sublimation of snow at
centers of model layers per grid box area at time $t$ \\
{\tt pmsnowaclc(kbdim,klev)} & double prec. & in & accretion rate of
snow at center of
model layer at time step $t$\\
{\tt paclc(kbdim,klev)} & double prec. & inout & cloud cover at center of
model layer at time step $t$\\
{\tt ptm1(kbdim,klev)} & double prec. & in & temperature at center of
model layers at time step $t-\Delta t$\\
{\tt ptte(kbdim,klev)}& double prec. & in & temperature tendency at
center of model layers accumulated over all processes of actual time
step until call of this subroutine\\
{\tt pxtm1(kbdim,klev,ntrac)} & double prec. & in & tracer mass or
molar mixing ratio with respect to dry air at center of model layers at
time step $t-\Delta t$\\
{\tt pxtte(kbdim,klev,ntrac)} & double prec. & inout & tendency of tracer
mass or molar mixing ratio with respect to dry air at center of model
layers accumulated over all processes of actual time
step until call of this subroutine \\
{\tt paphp1(kbdim,klev+1)} & double prec. & in & pressure of dry air at
interfaces between model layers at prognostic time step $t+\Delta t$\\
{\tt papp1(kbdim,klev)} & double prec. & in & pressure of dry air at center of model
layers at time step $t+\Delta t$\\
{\tt prhop1(kbdim,klev)} & double prec. & in & dry air density at center of
model layers at time step $t+\Delta t$ \\
{\tt pclcpre(kbdim,klev)} & double prec. & in & fraction of grid box
covered by precipitation at time step $t$ \\
\end{longtable}

\subsubsection{Interface of {\tt physc\_subm\_3}}

\begin{lstlisting}[caption=physc\_subm\_3]
SUBROUTINE physc_subm_3                               &
          (kproma, kbdim, klev,  klevp1, ktrac, krow, &
           paphm1, papm1, paphp1, papp1,              &
           ptm1,   ptte,  ptsurf,                     &
           pqm1,   pqte,                              &
           pxlm1,  pxlte, pxim1, pxite,               &
           pxtm1,  pxtte,                             &
           pgeom1, pgeohm1,                           &
           paclc,                                     &
           ppbl,   pvervel,                           &
           loland, loglac                             )
  INTEGER,  INTENT(in)    :: kproma                   
  INTEGER,  INTENT(in)    :: kbdim                    
  INTEGER,  INTENT(in)    :: klev                     
  INTEGER,  INTENT(in)    :: klevp1                   
  INTEGER,  INTENT(in)    :: ktrac                    
  INTEGER,  INTENT(in)    :: krow                     
  REAL(dp), INTENT(in)    :: paphm1   (kbdim,klevp1)  
  REAL(dp), INTENT(in)    :: papm1    (kbdim,klev)    
  REAL(dp), INTENT(in)    :: paphp1   (kbdim,klevp1)  
  REAL(dp), INTENT(in)    :: papp1    (kbdim,klev)    
  REAL(dp), INTENT(in)    :: ptm1     (kbdim,klev)    
  REAL(dp), INTENT(inout) :: ptte     (kbdim,klev)    
  REAL(dp), INTENT(in)    :: ptsurf   (kbdim)         
  REAL(dp), INTENT(in)    :: pqm1     (kbdim,klev)    
  REAL(dp), INTENT(inout) :: pqte     (kbdim,klev)    
  REAL(dp), INTENT(in)    :: pxlm1    (kbdim,klev)    
  REAL(dp), INTENT(inout) :: pxlte    (kbdim,klev)    
  REAL(dp), INTENT(in)    :: pxim1    (kbdim,klev)   
  REAL(dp), INTENT(inout) :: pxite    (kbdim,klev)   
  REAL(dp), INTENT(inout) :: pxtm1    (kbdim,klev,ktrac)
  REAL(dp), INTENT(inout) :: pxtte    (kbdim,klev,ktrac)
  REAL(dp), INTENT(in)    :: pgeom1   (kbdim,klev)      
  REAL(dp), INTENT(in)    :: pgeohm1  (kbdim,klevp1)    
  REAL(dp), INTENT(in)    :: paclc    (kbdim,klev)      
  REAL(dp), INTENT(in)    :: ppbl     (kbdim)           
  REAL(dp), INTENT(in)    :: pvervel  (kbdim,klev)      
  LOGICAL,  INTENT(in)    :: loland   (kbdim)           
  LOGICAL,  INTENT(in)    :: loglac   (kbdim)           
\end{lstlisting}

\begin{longtable}{l@{\extracolsep\fill}llp{7.0cm}}
\hline\hline\caption[Parameters of {\tt physc\_subm\_3}]{Parameter list
  of arguments passed to {\tt physc\_subm\_3}}\\\hline\label{tabphysc_subm_3}
\endfirsthead
\caption[]{{\tt Parameters of {\tt physc\_subm\_3}} --- continued}\\\hline
\endhead
\hline\multicolumn{4}{r}{\slshape table continued on next page}\\
\endfoot
\hline %\multicolumn{4}{r}{end of table}
\endlastfoot
name & type & intent & description \\\hline
{\tt kproma} & integer & in & actual length of
block of geographical longitudes (one longitude block can contain grid
cells of various geographical latitudes)\\
{\tt kbdim} &  integer & in & maximum length of block of
geographical longitudes (one longitude block can contain grid cells of
various geographical latitudes)\\
{\tt klev} &  integer & in & number of model levels (layers)\\
{\tt klevp1} &  integer & in & number of layers plus one \\
{\tt ktrac} &  integer & in & number of tracers \\
{\tt krow} &  integer & in & index number of block of geographical
longitudes\\ 
{\tt paphm1(kbdim,klevp1)} & double prec. & in & pressure of dry air at
interfaces between model layers at time step $t-\Delta t$\\
{\tt papm1(kbdim,klev)} & double prec. & in & pressure of dry air at center of model
layers at time step $t-\Delta t$\\
{\tt paphp1(kbdim,klevp1)} & double prec. & in & pressure of dry air at
interfaces between model layers at prognostic time step $t+\Delta t$\\
{\tt papp1(kbdim,klev)} & double prec. & in & pressure of dry air at center of model
layers at time step $t+\Delta t$\\
{\tt ptm1(kbdim,klev)} & double prec. & in & temperature at center of
model layers at time step $t-\Delta t$\\
{\tt ptte(kbdim,klev)}& double prec. & inout & temperature tendency at
center of model layers accumulated over all processes of actual time
step until call of this subroutine\\
{\tt ptsurf(kbdim)} & double prec. & in & surface temperature at time
step $t$\\
{\tt pqm1(kbdim,klev)} & double prec. & in & specific humidity (with
respect to dry air) at
center of model layers at time step $t-\Delta t$ \\
{\tt pqte(kbdim,klev)} & double prec. & inout & tendency of specific
humidity (with respect to dry air) at center of model layers
accumulated over all processes of actual time 
step until call of this subroutine \\
{\tt pxlm1} & double prec. & in & cloud liquid water content (mass of
liquid water per mass of dry air) at center
of model layers at time step $t-\Delta t$ \\
{\tt pxlte} & double prec. & inout & cloud liquid water tendency
(rate of change of mass of liquid water per mass of dry air) at
center of model layers accumulated over all processes of actual time
step until call of this subroutine \\
{\tt pxim1} & double prec. & in & cloud water ice content (mass of
water ice per mass of dry air) at center
of model layers at time step $t-\Delta t$ \\
{\tt pxite} & double prec. & inout & cloud water ice tendency (rate of
change of mass of ice water per mass of dry air) at
center of model layers accumulated over all processes of actual time
step until call of this subroutine \\
{\tt pxtm1(kbdim,klev,ktrac)} & double prec. & inout & tracer mass or
molar mixing ratio with respect to dry air at center of model layers at
time step $t-\Delta t$\\
{\tt pxtte(kbdim,klev,ktrac)} & double prec. & inout & tendency of tracer
mass or molar mixing ratio with respect to dry air at center of model
layers accumulated over all processes of actual time
step until call of this subroutine \\
{\tt pgeom1(kbdim,klev)} & double prec. & in & geopotential at center
of model layers at time step $t-\Delta t$\\
{\tt pgeohm1(kbdim,klevp1)} & double prec. & in & geopotential at
interfaces between model layers at time step $t-\Delta t$\\
{\tt paclc(kbdim,klev)} & double prec. & in & cloud cover at center of
model layer at time step $t$\\
{\tt ppbl(kbdim)} & double prec. & in & model layer index of geometrically
highest model layer of planetary boundary layer converted to a real
number at time $t$\\
{\tt pvervel(kbdim,klev)} & double prec. & in & large scale vertical
velocity at model center at time step $t$ \\
{\tt loland(kbdim)} & double prec. & in & logical land mask including
glaciers \\
{\tt loglac(kbdim)} & double prec. & in & logical glacier mask \\
\end{longtable}

\subsubsection{Interface of {\tt physc\_subm\_4}}

\begin{lstlisting}[caption=physc\_subm\_4]
SUBROUTINE physc_subm_4 (kproma,  kbdim,   klev,            &
                         klevp1,  ktrac,   krow,            &
                         paphm1,  pfrl,    pfrw,            &
                         pfri,    loland,  pxtm1,           &
                         pxtte)
  INTEGER, INTENT(in)   :: kproma, kbdim, klev, klevp1, ktrac, krow
  REAL(dp), INTENT(in)  :: paphm1(kbdim,klevp1), &
                           pfrl(kproma),         &
                           pfrw(kproma),         &
                           pfri(kproma),         &
                           pxtm1(kbdim,klev,ktrac)
  REAL(dp), INTENT(inout)::pxtte(kbdim,klev,ktrac)
  LOGICAL, INTENT(in)   :: loland(kproma)         
\end{lstlisting}

\begin{longtable}{l@{\extracolsep\fill}llp{7.0cm}}
\hline\hline\caption[Parameters of {\tt physc\_subm\_4}]{Parameter list
  of arguments passed to {\tt
    physc\_subm\_4}}\\\hline\label{tabphysc_subm_4}
\endfirsthead
\caption[]{{\tt Parameters of {\tt physc\_subm\_4}} --- continued}\\\hline
\endhead
\hline\multicolumn{4}{r}{\slshape table continued on next page}\\
\endfoot
\hline %\multicolumn{4}{r}{end of table}
\endlastfoot
name & type & intent & description \\\hline
{\tt kproma} & integer & in & actual length of
block of geographical longitudes (one longitude block can contain grid
cells of various geographical latitudes)\\
{\tt kbdim} &  integer & in & maximum length of block of
geographical longitudes (one longitude block can contain grid cells of
various geographical latitudes)\\
{\tt klev} &  integer & in & number of model levels (layers)\\
{\tt klevp1} &  integer & in & number of layers plus one \\
{\tt ktrac} &  integer & in & number of tracers \\
{\tt krow} &  integer & in & index number of block of geographical
longitudes\\ 
{\tt paphm1(kbdim,klevp1)} & double prec. & in & pressure of dry air at
interfaces between model layers at time step $t-\Delta t$\\
{\tt pfrl(kbdim)} & double prec. & in & land fraction \\
{\tt pfrw(kbdim)} & double prec. & in & surface water fraction \\
{\tt pfri(kbdim)} & double prec. & in & surface ice fraction \\
{\tt loland(kbdim)} & double prec. & in & logical land mask including
glaciers \\
{\tt pxtm1(kbdim,klev,ktrac)} & double prec. & in & tracer mass or
molar mixing ratio with respect to dry air at center of model layers at
time step $t-\Delta t$\\
{\tt pxtte(kbdim,klev,ktrac)} & double prec. & inout & tendency of tracer
mass or molar mixing ratio with respect to dry air at center of model
layers accumulated over all processes of actual time
step until call of this subroutine \\
\end{longtable}

\subsubsection{Interface of {\tt free\_subm\_memory}}

\begin{lstlisting}[caption=free\_subm\_memory]
  SUBROUTINE free_subm_memory
\end{lstlisting}

This subroutine has no parameter list.

\subsection{Tracer interface}

Tracer fields are constituents transported with the flow of air in the
atmospheric model. In addition to the transport, they are subject to
several processes such as convection, diffusion, emission, deposition
and chemical conversion. Horizontal and vertical transport is carried
out by the atmospheric model and some standard processes can be
performed by the atmospheric model as well. Other processes which are
specific for the tracer must be calculated by the sub--model. The
tracer interface is a collection of subroutines that allow the
definition and handling of a data structure containing information
about tracers. This information comprises the 3--dimensional mass or
volume mixing ratio of the tracers but also variables that determine
the transport and physical properties of each individual tracer.

Tracers within \echam{} are represented by a 4--dimensional array (the
three spatial dimensions are supplemented by the tracer index) but 
pointers to individual
tracers can be obtained so that 
details of implementation of the data structure remains hidden.  A one
dimensional array of a derived data type holds the
meta--information. In the restart file the tracers are identified by
name, so that restarts can be continued with different sets of tracers
if required. Reading and writing of the tracers to the rerun file and
to the output stream is based on the output stream and memory buffer
facilities described in section~\ref{secstreams}.

\subsubsection{Request a new tracer}

A new tracer with name {\tt 'A'} is requested from a module with name
{\tt 'my\_module'} by a call to the routine {\tt
  new\_tracer}\index{tracers!new\_tracer} of {\tt 
  mo\_tracer.f90}\index{tracers!mo\_tracer}: 
 
{\tt call new\_tracer ('A', 'my\_module', idx)}

Tracer properties are specified by optional arguments of the {\tt
  new\_tracer} subroutine. The interface is as
follows: 

{\small
\begin{tabular}{|lllclp{5cm}|}
\hline
\multicolumn{2}{|l}
{\tt SUBROUTINE new\_tracer}&
\multicolumn{4}{p{10cm}|}
{\tt name, modulename [,spid] [,subname] [,trtype] [,idx]
[,nwrite] [,longname] [,units] [,moleweight]
[,code] [,table] [,bits]  [,nbudget] [,burdenid]
[,ninit] [,vini]  [,nrerun] [,nint]
[,ntran] [,nfixtyp] [,nvdiff] [,nconv]
[,nwetdep] [,ndrydep]  [,nsedi] [,nemis]
[,tdecay]  [,nphase]  [,nsoluble] [,mode]  [,myflag]
[,ierr])}\\
\hline
name&type&intent&default&function$^*$&description\\
\hline
\multicolumn{6}{|l|}{\bf identification of the tracer :}\\
name          &character(len=*) &in & &es & name of the tracer\\
modulename    &character(len=*) &in & &es & name of the module 
                                          requesting the tracer\\
{[spid]}      &integer          &in & &es & species index \\
{[subname]}   &character(len=*) &in & &es & optional for 'colored' tracers\\
{[trtype]}    &integer          &in & &es & tracer type \\
{[idx]}       &integer          &out& &es & index of the tracer\\
\hline
\multicolumn{6}{|l|}{\bf postprocessing output :}\\
{[nwrite]}    &integer          &in & {\tt ON}& p & flag to print the tracer\\
{[longname]}  &character(len=*) &in &     " " & p & long name\\
{[units]}     &character(len=*) &in &     " " & p & physical units\\
{[moleweight]}&real             &in &      0. & p & molecular weight\\
{[code]}      &integer          &in &       0 & p & GRIB code\\
{[table]}     &integer          &in &     131 & p & GRIB table\\
{[bits]}      &integer          &in &      16 & p & number of GRIB
                                                    encoding bits\\
{[nbudget]}   &integer          &in &{\tt OFF}&ep\%& budget flag\\
{[burdenid]}  &integer          &in &         &e\%& burden diagnostics number\\
\hline
\multicolumn{6}{|l|}{\bf initialization and rerun :}\\
{[ninit]}     &integer          &in &{\tt RESTART+}&e& initialization flag\\
              &                 &   &{\tt CONSTANT}& & \\
{[vini]}      &real             &in &            0.&e& initialization value\\
{[nrerun]}    &integer          &in &     {\tt ON} &e& restart flag\\
\hline
\multicolumn{6}{|l|}{\bf transport and other processes :}\\ 
{[nint]}      &integer          &in &     &e& integration flag \\
{[ntran]}     &integer          &in &{\tt TRANSPORT}&e\%& transport switch\\
{[nfixtyp]}   &integer          &in & 1 & e\% & type of mass fixer\\
{[nvdiff]}    &integer          &in &{\tt ON}& e & vertical diffusion flag\\
{[nconv]}     &integer          &in &{\tt ON}& e & convection flag\\
{[nwetdep]}   &integer          &in &{\tt OFF}& e & wet deposition flag\\
{[ndrydep]}   &integer          &in &{\tt OFF}& e\% & dry deposition flag\\
{[nsedi]}     &integer          &in &{\tt OFF}& e\% & sedimentation flag\\
{[nemis]}     &integer          &in &{\tt OFF}& e\% & surface emission flag\\
{[tdecay]}    &real             &in & 0.& e   & exponential decay time\\
\hline
\multicolumn{6}{|l|}{\bf attributes interpreted by the submodel :}\\
{[nphase]}    &integer          &in & 0         &s& phase indicator\\
{[nsoluble]}  &integer          &in &           &s& solubility flag \\
{[mode]}      &integer          &in & 0         &s& mode indicator\\
{[myflag(:)]} &type(t\_flag)    &in &('',0.)    &s& user defined flags\\
\hline
\multicolumn{6}{|l|}{\bf miscellaneous arguments :}\\
{[ierr]}      &integer          &out&{\tt OK=0} &s& error return value\\
\hline
\multicolumn{6}{p{15cm}}
{$^*$ attributes interpreted by ECHAM (e), by the submodel (s),
by the postprocessing module (p), not yet implemented (\%).}\\
\end{tabular}}

In general, integer values are chosen to represent the flags in order
to allow different choices:

\hspace*{2ex} 0: {\tt OFF}\\
\hspace*{2ex} 1: {\tt ON}, standard action\\
\hspace*{2ex} 2: {\tt\dots}, alternative action\\
\hspace*{2ex} \dots\\
\hspace*{2ex} {\tt tag}: specific action performed by the sub-model.

Small numbers indicate that some kind of standard action shall be
performed by ECHAM. Higher {\tt tag} values indicate that the process
will be handled by the submodel.
For the following actual arguments, valid values are defined by
parameter statements (see {\tt mo\_tracdef.f90}\index{tracers!mo\_tracdef}):

{\small
\begin{tabular*}{\textwidth}{|l@{\extracolsep\fill}p{10cm}p{4cm}|}
\hline
argument&values&description\\
\hline
        &{\tt OK, OFF, ON}                              & universal values\\
ntran   &{\tt NO\_ADVECTION, SEMI\_LAGRANGIAN, SPITFIRE, TPCORE}    & transport flag\\
ninit   &{\tt INITIAL, RESTART, CONSTANT, PERIODIC}            & initialization flag\\
nsoluble&{\tt SOLUBLE, INSOLUBLE}                   & soluble flag\\
nphase  &{\tt GAS, AEROSOL, GAS\_OR\_AEROSOL, AEROSOLMASS,
  AEROSOLNUMBER, UNDEFINED}                & phase indicator\\
code    &{\tt AUTO}                                & automatically
                                                     chose unique GRIB code\\
ierr    &{\tt OK,NAME\_USED,NAME\_MISS,TABLE\_FULL}& error return
value (cannot be used currently)\\
\hline
\end{tabular*}}

\paragraph{Tracer properties: Identification of the tracer and sub-model.}

Each tracer is identified by a unique {\tt name} and optionally by a
{\tt subname} in case of colored tracers. In the postprocessing file
colored tracers appear with the name {\tt name\_subname}. Values of
optional arguments provided for the corresponding non--colored tracer
(without argument {\tt subname}) are used for the colored tracer
as well (despite the GRIB code number).

The sub--model identifies itself by a unique character string {\tt
modulename}. {\tt idx} is the index of the new tracer in the global
arrays {\tt XT}, {\tt XTM1}, {\tt trlist}.

\paragraph{Tracer properties: Postprocessing flags.}

The flag {\tt nwrite} (default {\tt ON}) determines, whether the
tracer is written to the standard output stream. A separate file with
name {\tt STANDARDFILENAME\_tracer} for GRIB, or {\tt
STANDARDFILENAME\_tracer.nc} for NetCDF format, is written. The default
file format GRIB can be changed to NetCDF by setting {\tt
trac\_filetype=2} in the namelist {\tt runctl} (see
Tab.~\ref{tabrunctl} of section~\ref{secrunctl}).

If present, the attributes {\tt longname}, {\tt units} and {\tt
moleweight} are written to the NetCDF file.

Within GRIB files, fields are identified by a GRIB code number which
must be given as argument {\tt code}. Note that codes 129 and 152
should not be used because they are attributed to surface pressure and
geopotential height. A
predefined value {\tt AUTO} is accepted indicating automatic
generation of unique GRIB code numbers.  For GRIB files, a code file
{\tt STANDARDFILENAME\_tracer.codes} is written to associate code
numbers with tracer names.  For the tracers, a default GRIB table
number 131 is chosen for tracer output. By default, 16 bits are used
for encoding in GRIB format.
 
\paragraph{Tracer properties: Initialization and rerun.}

The {\tt nrerun} flag (default={\tt ON}) indicates, whether the tracer
variable shall be read and written from/to the rerun file. The tracers
are identified by name in the rerun (NetCDF) file, so that they can be
read selectively.  The initialization flag {\tt ninit} is used to
specify the initialization procedure in more detail: Valid values are
one of INITIAL (read from initial file, this must be done by the
submodel), RESTART (read from restart file), CONSTANT (set to the
initial value {\tt vini}) or a combination (e.g. RESTART+CONSTANT) to
indicate that the quantity is read from the restart file in case of a
rerun but set to a predefined value otherwise.

\paragraph{Tracer properties: Transport and other processes.}

Tracer transport and the impact of certain other processes is
calculated by ECHAM. The flags {\tt nint}, {\tt ntran}, {\tt nfixtyp}, {\tt
nvdiff}, {\tt nconv}, {\tt nwetdep}, {\tt nsedi}, {\tt ndrydep}, {\tt nemis}, {\tt
tdecay} are meant to switch {\tt ON} or {\tt OFF} the
respective processes (not fully implemented currently).

A value of {\tt tdecay}$\ne 0$ leads to an exponential decay of the
tracer with time.

\paragraph{Tracer properties: Attributes interpreted by the submodel.}

%{[henry]}     &real             &in &$10.^{-10}$&s& Henry coefficient.\\
%{[dryreac]}   &real             &in & 0.        &s& reactivity coefficient.\\
%{[nsoluble]}  &integer          &in &{\tt INSOLUBLE}&s& soluble flag.\\
%{[nphase]}    &integer          &in & 0         &s& phase indicator.\\
%{[mode]}      &integer          &in & 0         &s& mode indicator.\\
%{[myflag(:)]} &type(t\_flag)    &in &('',0.)    &s& user defined flags.\\


The following flags are not used by ECHAM. They are reserved to be
used by the sub-models: {\tt nphase}, {\tt nsoluble}, {\tt mode} and {\tt
  myflag}.  
{\tt myflag} is
an array of pairs of character strings and real values.

\subsubsection{Access to tracers with {\tt get\_tracer}}

The routine {\tt get\_tracer}\index{tracers!get\_tracer} returns the
references to tracers 
already defined.\\

Example:
\begin{lstlisting}[caption=get\_tracer]
  CALL get_tracer ('SO2',idx=index,modulename=modulename)
  IF (ierr==0) THEN
    PRINT *, 'Using tracer SO2 from module',modulename
  ELSE
    ! eg. read constant tracer field
    ...
  ENDIF
\end{lstlisting}

The interface of subroutine {\tt get\_tracer} is:

{\small
\begin{tabular}{|lllp{6cm}|}
\hline
\multicolumn{2}{|l}
{\tt SUBROUTINE get\_tracer}&
\multicolumn{2}{p{10cm}|}{\tt 
(name [,subname] [,modulename] [,idx] [,pxt] [,pxtm1] [,ierr])}\\
\hline
name&type&intent&description\\
name            &character(len=*) &in      & name of the tracer\\
{[subname]}     &character(len=*) &in      & subname of the tracer\\
{[modulename]}  &character(len=*) &out     & name of requesting module\\
{[idx]}         &integer          &out     & index of the tracer\\
{[pxt(:,:,:)]}  &real             &pointer & pointer to the tracer field\\
{[pxtm1(:,:,:)]}&real             &pointer & pointer to the tracer field 
                                             at previous time step\\
{[ierr]}        &integer          &out     & error return value (0=OK,
                                             1=tracer not defined)\\
\hline
\end{tabular}}

If the optional parameter {\tt ierr} is not given and the tracer is
not defined the program will abort.  Note that references ({\tt pxt,
pxtm1}) to the allocated memory cannot be obtained before all tracers
are defined and the respective memory is allocated in the last step of
tracer definition.

\subsubsection {Tracer list data type}

Summary information on the tracers is stored in a global variable {\tt
trlist}\index{tracers!trlist}. 
Attributes of individual tracers are stored in the component
array {\tt trlist\% ti(:)}. The definitions of the respective data
types {\tt t\_trlist}\index{tracers!t\_trlist} 
and {\tt t\_trinfo}\index{tracers!t\_trinfo} are given below:
\begin{lstlisting}[caption=t\_trlist]
!
! Basic data type definition for tracer info list
!
TYPE t_trlist
  !
  ! global tracer list information
  !
  INTEGER         :: ntrac        ! number of tracers specified
  INTEGER         :: anyfixtyp    ! mass fixer types used
  INTEGER         :: anywetdep    ! wet deposition requested
                                  ! for any tracer
  INTEGER         :: anydrydep    ! wet deposition requested 
                                  ! for any tracer
  INTEGER         :: anysedi      ! sedimentation  requested 
                                  ! for any tracer
  INTEGER         :: anysemis     ! surface emission flag 
                                  ! for any tracer
  INTEGER         :: anyconv      ! convection flag
  INTEGER         :: anyvdiff     ! vertical diffusion flag
  INTEGER         :: anyconvmassfix  ! 
  INTEGER         :: nadvec       ! number of advected tracers
  LOGICAL         :: oldrestart   ! true to read old restart format
  !
  ! individual information for each tracer
  !
  TYPE (t_trinfo) :: ti  (jptrac) ! Individual settings 
                                  !for each tracer
  !
  ! reference to memory buffer info
  !
  TYPE (t_p_mi)   :: mi  (jptrac) ! memory buffer information 
                                  !for each tracer
  TYPE (memory_info), POINTER :: mixt   ! memory buffer 
                                        ! information for XT
  TYPE (memory_info), POINTER :: mixtm1 ! memory buffer 
                                        ! information for XTM1
END TYPE t_trlist
\end{lstlisting}
%
The component {\tt ntrac}\index{tracers!ntrac}\index{ntrac} 
gives the total number of tracers handled by
the model. The components {\tt any\dots} are derived by a bitwise {\tt
OR} of the corresponding individual tracer flags. Individual flags are
stored in component {\tt ti} of type {\tt
  t\_trinfo}\index{tracers!t\_trinfo}. 
They reflect the
values of the arguments passed to subroutine {\tt
  new\_tracer}\index{tracers!new\_tracer}.
%
\begin{lstlisting}[caption={\tt t\_trinfo}]
TYPE t_trinfo
  !
  ! identification of transported quantity
  !
  CHARACTER(len=ln) :: basename   ! name (instead of xt..)
  CHARACTER(len=ln) :: subname    ! optional for 
                                  !'colored' tracer
  CHARACTER(len=ln) :: fullname   ! name_subname
  CHARACTER(len=ln) :: modulename ! name of requesting 
                                  ! sub-model
  CHARACTER(len=ln) :: units      ! units
  CHARACTER(len=ll) :: longname   ! long name
  CHARACTER(len=ll) :: standardname   ! CF standard name
  INTEGER           :: trtype ! type of tracer: 
                              ! 0=undef., 1=prescribed, 
                              ! 2=diagnostic (no transport),
                              ! 3=prognostic (transported)
  INTEGER           :: spid   ! species id (index in 
                              ! speclist) where physical/chem.
                              ! properties are defined 
  INTEGER           :: nphase ! phase (1=GAS, 2=AEROSOLMASS, 
                              ! 3=AEROSOLNUMBER,...)
  INTEGER           :: mode   ! aerosol mode or bin number
  REAL(dp)          :: moleweight ! molecular mass (copied 
                        ! from species upon initialisation)
  ! Requested resources ...
  ! 
  INTEGER    :: burdenid   ! index in burden diagnostics
  !
  ! Requested resources ...
  !
  INTEGER  :: nbudget    ! calculate budgets (default 0)
  INTEGER  :: ntran      ! perform transport (default 1)
  INTEGER  :: nfixtyp    ! type of mass fixer (default 1)
  INTEGER  :: nconvmassfix ! use xt_conv_massfix in cumastr
  INTEGER  :: nvdiff     ! vertical diffusion flag  
                         ! (default 1)
  INTEGER  :: nconv      ! convection flag (default 1)
  INTEGER  :: ndrydep    ! dry deposition flag: 
                         ! 0=no drydep, 
                         ! 1=prescribed vd,
                         ! 2=Ganzeveld 
  INTEGER  :: nwetdep    ! wet deposition flag (default 0)
  INTEGER  :: nsedi      ! sedimentation flag (default 0)
  REAL     :: tdecay     ! decay time (exponential) 
                         ! (default 0.sec)
  INTEGER  :: nemis  ! surface emission flag (default 0)
  !
  ! initialization and restart
  !
  INTEGER  :: ninit  ! initialization request flag
  INTEGER  :: nrerun ! rerun flag
  REAL     :: vini   ! initialization value (default 0.)
  INTEGER  :: init   ! initialisation method actually used 
  !
  ! Flags used for postprocessing
  !
  INTEGER  :: nwrite   ! write flag (default 1)
  INTEGER  :: code     ! tracer code (default 235...)
  INTEGER  :: table    ! tracer code table (default 0)
  INTEGER  :: gribbits ! bits for encoding (default 16)
  INTEGER  :: nint     ! integration (accumulation) 
                       ! flag  (default 1)  
  !
  ! Flags to be used by chemistry or tracer modules
  !
  INTEGER         :: nsoluble   ! soluble flag (default 0)
  TYPE(t_flag)    :: myflag (nf)! user defined flag
  type(time_days) :: tupdatel   ! last update time
  type(time_days) :: tupdaten   ! next update time
  !
END TYPE t_trinfo
\end{lstlisting}
%

The data type {\tt t\_flag} is defined as follows:
%
\begin{lstlisting}[caption=data type {\tt t\_flag}]
  TYPE t_flag
    CHARACTER(len=lf) :: c      ! character string
    REAL              :: v      ! value
  END TYPE t_flag
\end{lstlisting}
%
The lengths of the character string components are\index{string lengths}:
%
\begin{lstlisting}[caption=Length of strings]
  INTEGER, PARAMETER :: ln =  24 ! length of name 
                                 ! (char) components
  INTEGER, PARAMETER :: ll = 256 ! length of 
                                 ! longname component
  INTEGER, PARAMETER :: lf =   8 ! length of flag 
                                 ! character string
  INTEGER, PARAMETER :: nf =  10 ! number of user 
                                 ! defined flags
  INTEGER, PARAMETER :: ns =  20 ! max number of submodels
\end{lstlisting}
