\subsection{Output Streams and Memory Buffer}
\label{sec.output_stream}
%------------------------------------------------------------------------------
\subsubsection{Functionality}

The {\bf Output Stream} interface maintains a list of output
streams. Generally one ore more streams are associated to an output
file. Each stream has attributes specifying the file name, file type,
etc.{}. It further holds a linked list of {\bf Memory Buffer elements},
of 2 to 4 dimensional arrays and associated meta information.

%------------------------------------------------------------------------------
\subsubsection{Usage}

First, a new output stream must be created by calling subroutine {\tt
new\_stream}\index{streams!new\_stream}. 
Afterwards fields may be allocated by calling {\tt
add\_stream\_element}\index{streams!add\_stream\_element}.

\subsubsection*{Create a new output stream}

The access to the output stream interface is provided by module {\tt
mo\_memory\_base}\index{streams!mo\_memory\_base}
\index{memory!mo\_memory\_base}:
%
{\small
\begin{verbatim}
  USE mo_memory_base, ONLY: t_stream,                                    &
                            new_stream, delete_stream,                   &
                            default_stream_setting, add_stream_element,  &
                            get_stream_element, set_stream_element_info, &
                            memory_info,                                 &
                            ABOVESUR2, ...
\end{verbatim}}
%
To create a new output stream the routine {\tt
new\_stream}\index{streams!new\_stream} has to be called:
%
{\small
\begin{verbatim}
  TYPE (t_stream) ,pointer :: mystream
  ...
  CALL new_stream (mystream ,'mystream')
\end{verbatim}}
%
{\tt mystream} is a pointer holding a reference to the output stream
returned by subroutine {\tt new\_stream}. {\tt 'mystream'} is the
identification name
of the output stream.

By default, the output and rerun filenames are derived from the name
of the output stream (here 'mystream') by appending a respective
suffix (here {\tt '\_mystream'}) to the standard filenames.  The
content of the output stream is written to the rerun file and to the
output file. To change the defaults, optional parameters may be provided
(cf.{} section \ref{sec.new_stream}).

\subsubsection*{Add a field to an output stream}

To include items in the output stream {\tt mystream} the routine {\tt
add\_stream\_element} has to be called. A unique name must be given to
identify the quantity and a pointer associated to the field is
returned. For example, to add a surface field {\tt a} and an
atmospheric field {\tt b} with names {\tt 'A'} and {\tt 'B'}, the
following sequence of subroutine calls is 
required\index{streams!add\_stream\_element}:
%
{\small
\begin{verbatim}
  REAL, POINTER :: a (:,:)
  REAL, POINTER :: b (:,:,:)
  REAL, POINTER :: c (:,:)
  ...
  CALL add_stream_element (mystream, 'A' ,a )
  CALL add_stream_element (mystream, 'B' ,b )
\end{verbatim}}
%
By default suitable sizes are assumed for surface (2-d pointer {\tt
a}) or atmospheric fields (3-d pointer {\tt b}).  To choose
other sizes (e.g.{} spectral fields or a non-standard number of
vertical layers) optional parameters must be specified.  The
specification of the optional parameters is given in section
{\ref{sec.add_stream_element}}

A routine is available to associate a pointer (here {\tt c}) with an
item (here {\tt 'A'}) already included in the list (previously by another
sub-model for example):
%
{\small
\begin{verbatim}
  CALL get_stream_element (mystream, 'A', c)
\end{verbatim}}
%
If stream element {\tt 'A'} has not been created beforehand, a null
pointer is returned for {\tt c}.

%------------------------------------------------------------------------------
\subsubsection{Create an output stream}
\label{sec.new_stream}

Optional parameters may be passed to subroutines
{\tt new\_stream} and {\tt add\_stream\_element}
in order to specify the attributes of output streams
and memory buffers. Furthermore, routines are available to
change default values for optional parameters.

The interface of the routine to create an output stream 
is\index{streams!new\_stream}:

{\small
\begin{tabular}{|llllp{7cm}|}
\hline
\multicolumn{2}{|l}
{\tt SUBROUTINE new\_stream}&
\multicolumn{3}{p{11cm}|}
                         {\tt (stream ,name [,filetype]
                         [,post\_suf] [,rest\_suf] [,init\_suf]
                         [,lpost] [,lpout] [,lrerun] [,lcontnorest] 
                         [,linit] [,interval])}\\
\hline
name&type&intent&default&description\\
\hline
\,stream     &type(t\_stream)&pointer&       & Returned reference to
                                               the new output stream.\\
\,name       &character(len=*) &in  &        & Name of the new output stream.\\
{[filetype]} &integer      &in      &out\_filetype & Type of output
                                               file. The default
                                               (GRIB) may be changed
                                               in namelist 
                                               /SDSCTL/. Alternatively 
                                               NETCDF may be passed.\\
{[post\_suf]}&character(len=*) &in      &'\_'//name& Suffix of the
                                               output file associated
                                               with the stream. The
                                               default is derived from
                                               the name of the output
                                               stream.\\
{[rest\_suf]}&character(len=*) &in      &'\_'//name& Suffix of the
                                               rerun file.\\
{[init\_suf]}&character(len=*) &in      &'\_'//name& Suffix of initial file.\\
{[lpost]}    &logical      &in      &.true.  & Postprocessing flag. If
                                               .true.{}
                                               an output file is
                                               created for this stream.\\
{[lpout]}    &logical      &in      &.true.  & Output flag. The stream
                                               is written to the
                                               output file if {\tt
                                                 lpout=.true} \\
{[lrerun]}   &logical      &in      &.true.  & If .true.{} the stream
                                               is read/written from/to the
                                               rerun file.\\
{[lcontnorest]} &logical   &in      &---     & Continue a restart even
if this stream is not present in any rerun file. \\
{[linit]}    &logical      &in      &.true.  & Write to initial file
(does not work?)\\
{[interval]}&type(io\_time\_event)&in&putdata& Postprocessing output
                                               interval. Default:
                                               12 hours.\\
\hline
\multicolumn{5}{p{16cm}}{Optional parameters are given in brackets [ ].
They should always be passed by keyword because the number and
ordering of optional parameters may change.}\\
\end{tabular}}

Valid values for the argument {\tt out\_filetype} are defined within
module {\tt mo\_memory\_base}\index{streams!mo\_memory\_base}:
{\small
\begin{verbatim}
  INTEGER ,PARAMETER :: GRIB      = 1
  INTEGER ,PARAMETER :: NETCDF    = 2
\end{verbatim}}

For specification of a non-standard output time interval data type
{\tt io\_time\_event}\index{data type!io\_time\_event} 
(defined in module {\tt mo\_time\_event}
\index{time manager!mo\_time\_event}) 
has to
be passed as argument {\tt interval}. For example, in order to write
every time step or in 6 hourly intervals, specify: {\tt
interval=io\_time\_event(1,'steps','first',0)} or {\tt
(6,'hours','first',0)}, respectively.

Once a stream has been created, a reference can be obtained by calling
subroutine {\bf get\_stream}\index{streams!get\_stream}:

{\small
\begin{tabular}{|llllp{7cm}|}
\hline
\multicolumn{5}{|l|}
{\tt SUBROUTINE get\_stream (stream ,name)}\\
\hline
name&type&intent&default&description\\
\hline
\,stream     &type(t\_stream)&pointer&       & Returned reference to the
                                               output stream.\\
\,name       &character(len=*) &in      &        & Name of the output stream.\\
\hline
\end{tabular}}


%------------------------------------------------------------------------------
\subsubsection{Add a field to the output stream}
\label{sec.add_stream_element}

The routine to add new elements to the output stream 
is\index{streams!add\_stream\_element}:

{\small
\begin{tabular}{|llllp{6cm}|}
\hline
\multicolumn{3}{|l}
{\tt SUBROUTINE add\_stream\_element}&
\multicolumn{2}{p{9cm}|}
 {\tt (stream ,name ,ptr [,ldims] [,gdims]
 [,klev] [,ktrac] [,units] [,longname] [,repr] 
 [,lpost] [,laccu] [lmiss,] [missval,] [,reset] [,lrerun] [,contnorest]
 [,table] [,code] [,bits] [,leveltype] 
 [,dimnames] [,mem\_info] [,p4] [,no\_default] [,verbose])}\\
\hline
name&type&intent&default&description\\
\hline
%------------------------------------------------------------------------------
\multicolumn{5}{|l|}{\bf mandatory arguments :}\\
stream     &type(t\_stream) &inout    &     & Output stream.\\
name       &character(len=*)&in       &     & Name of the field to add
                                              to the output stream.\\
ptr        &real(:,:[,:][,:])&pointer &     & Returned reference to
                                              the memory of the
                                              2- or 3- or 4-dimensional
                                              field.\\
\hline
%------------------------------------------------------------------------------
\multicolumn{5}{|l|}{\bf specification of dimensions :}\\
{[ldims(:)]}   &integer     &in &cf.\,text & Local size on actual processor.\\
{[gdims(:)]}   &integer     &in &cf.\,text & Global size of the
field.\\
{[klev]}       &integer     &in &cf.\,text & Number of vertical levels.\\
{[ktrac]}      &integer     &in &0         & Number of tracers. \\
{[repr]}       &integer     &in &GRIDPOINT & Representation.\\
{[leveltype]}  &integer     &in &cf.\,text & Dimension index of the
                                             vertical coordinate.\\
\hline
%------------------------------------------------------------------------------
\multicolumn{5}{|l|}{\bf postprocessing flags :}\\
{[lpost]}      &logical &in&.false.& Write the field to the
                                     postprocessing file.\\
{[laccu]}      &logical &in&.false.& ``Accumulation'' flag: Does no
accumulation but divides variable by the number of seconds of the
output interval and resets it to {\tt reset} after output.\\
{[reset]}      &real    &in&0.     & Reset field to this value at initialization,
and after output if {\tt laccu=.true.} or {\tt reset} is not zero.\\
\hline
%------------------------------------------------------------------------------
\multicolumn{5}{|l|}{\bf rerun flags :}\\
{[lrerun]}     &logical &in&.false.& Flag to read/write field from/to
                                     the rerun file.\\
{[contnorest]} &logical &in&.false.& If {\tt contnorest=.true.},
continue restart, stop otherwise.\\
\hline
%------------------------------------------------------------------------------
\multicolumn{5}{|l|}{\bf attributes for NetCDF output :}\\
{[units]}      &character(len=*)&in&' '                & Physical units.\\
{[longname]}   &character(len=*)&in&' '                & Long name.\\
{[dimnames(:)]}&character(len=*)&in&'lon'[,'lev'],'lat'& Dimension names.\\ 
\hline
%------------------------------------------------------------------------------
\multicolumn{5}{|l|}{\bf attributes for GRIB output :}\\
{[table]}      &integer     &in& 0  & table number.\\
{[code]}       &integer     &in& 0  & code number.\\
{[bits]}       &integer     &in& 16 & number of bits used for encoding.\\
\hline
%------------------------------------------------------------------------------
\multicolumn{5}{|l|}{\bf Missing values :} \\
{[lmiss]}      &logical     &in&.false.&If {\tt lmiss=.true.}, missing
values are set to {\tt missval}, not set at all otherwise.\\
{[missval]}    &real        &in& $-9\times10^{33}$& missing value.\\
%------------------------------------------------------------------------------
\multicolumn{5}{|l|}{\bf miscellaneous arguments :}\\
{[mem\_info]}  &type(memory\_info)&pointer&&Reference to meta data information.\\
{[p4(:,:,:,:)]}&real              &pointer&& Pointer to allocated memory provided.\\
{[no\_default]}&logical         &in&.false.&Default values usage flag.\\
{[verbose]}    &logical         &in&.false.&Produce diagnostic printout.\\ 
\hline
\end{tabular}}

Most arguments of the routine are optional. They may be given for the
following purposes:
\begin{description}
\item{\bf specification of dimensions:}\\
  The total size of the field is specified by the parameter {\tt
  gdims}. In a parallel environment, the part allocated on a processor
  element is specified by the parameter {\tt ldims}. The order of
  dimensions is (lon,lat) for 2--d, (lon,lev,lat) for 3--d and
  (lon,lev,any,lat) for 4--dimensional gridpoint fields. The number of
  size of {\tt gdims} and {\tt ldims} corresponds to the rank of {\tt
  ptr(:,:)}.

  Generally, it is not necessary to give dimension information. The
  sizes of the 
  fields are derived from the model field sizes.  If a 2--dimensional
  pointer {\tt ptr(:,:)} is provided for {\tt ptr}, a SURFACE field is
  assumed. If a 3-dimensional pointer {\tt ptr(:,:,:)} is provided, a
  HYBRID field (lon,lev,lat) is assumed.

  For the following cases optional arguments must be specified to
  overwrite the defaults:
  \begin{description}
  \item{\bf The number of vertical levels differs from the number of
  model levels}\\

    To specify a number of levels different from the standard
    $\sigma$--hybrid co--ordinate system used in the model, the
    parameter {\tt klev} 
    may be specified. A HYBRID coordinate system is assumed in this
    case. However if the field is written to the postprocessing file
    {\tt (lpost=.true.)}, it is recommended to either pass a dimension
    index to parameter {\tt leveltype} or the name of the dimensions
    to {\tt dimnames} in order to pass proper attributes to the
    NetCDF and GRIB writing routines.

    For the usual cases, dimension indices are predefined (cf.~table
    \ref{tab.dims}) and may be accessed from module {\tt mo\_netcdf}.
    New dimensions may be defined by the use of the subroutine {\tt
    add\_dim} as described in section \ref{sec.dim}.

  \item{\bf The field is not a gridpoint field}\\
    For non Gaussian gridpoint fields appropriate values should be
    passed as parameter {\tt repr}. Predefined values ({\tt
      mo\_linked\_list}) are:
  {\small
\begin{verbatim}
     INTEGER ,PARAMETER :: UNKNOWN   = -huge(0)
     INTEGER ,PARAMETER :: GAUSSIAN  = 1
     INTEGER ,PARAMETER :: FOURIER   = 2
     INTEGER ,PARAMETER :: SPECTRAL  = 3
     INTEGER ,PARAMETER :: HEXAGONAL = 4
     INTEGER ,PARAMETER :: LAND      = 5
     INTEGER ,PARAMETER :: GRIDPOINT = GAUSSIAN
  \end{verbatim}}
    In all other cases, {\tt gdims} and {\tt ldims} have to be defined
    explicitly.
  \end{description}  
\item{\bf postprocessing flags:}\\\index{postprocessing flags}
  In order to write a field to an output file, {\tt
  lpost=.true.} must be specified. Generally the actual values of the
  field are written.  However, if {\tt laccu=.true.} is specified, the
  values are divided by the number of seconds of the output interval
  before output and set to the value of the variable {\tt reset}
  afterwards. The default is 0.
  In this case the fields should be incremented at
  each time step with values multiplied by the time step length in
  order to write temporarily averaged values to the output
  file. If the field is set to the maximum or
  minimum value during the output time period, values of
  {\tt reset=-huge(0.)} or {\tt reset=huge(0.)} shall be passed.
\item{\bf rerun flags:}\\\index{rerun flags}
  To include the field in the rerun files, {\tt lrerun=.true.} must be
  specified.
\item{\bf attributes for NetCDF output:}\\\index{attributes NetCDF}
  For NetCDF output, the physical units, long name, and dimension
  names of the field should be provided.
\item{\bf attributes for GRIB output:}\\\index{attributes GRIB}
  For GRIB output, a table number and code number is
  required. A predefined value {\tt AUTO} may be passed
  as parameter {\tt code} in order to automatically generate unique
  GRIB code numbers. The number of bits used for encoding may be
  changed by argument {\tt bits}.
\item{\bf miscellaneous arguments:}\\
  If {\tt verbose=.true.} is specified, a printout is generated.

  The default values of the optional parameters may be changed by
  calling the subroutine\\ {\tt default\_stream\_setting} as described
  below. However if {\tt no\_defaults=.true.} is specified, these
  changed default values will not be used.

  Generally memory is allocated for the argument {\tt ptr} when
  calling {\tt add\_stream\_element}, but memory may be provided
  externally by passing it via the argument {\tt p4}. Even if
  2--dimensional or 
  3--dimensional arrays are accessed via {\tt ptr}, 4--dimensional
  fields are used internally and must be passed for {\tt p4} (with
  dimension sizes (lon,lat,1,1) or (lon,lev,lat,1), respectively).

  Meta data information about memory may be accessed by the argument {\tt
  mem\_info}.

\end{description}

%------------------------------------------------------------------------------
\subsubsection{Change of default values for optional arguments}

The default values for the optional arguments of subroutine {\tt
add\_stream\_entry}
\index{streams!add\_stream\_entry} 
may be changed for all subsequent calls related to
an output stream by calling the subroutine\\ {\tt
default\_stream\_setting}. 
\index{streams!default\_stream\_setting}
This subroutine accepts the same arguments
as subroutine {\tt add\_stream\_entry}:

{\small
\begin{tabular}{|llllp{6cm}|}
\hline
\multicolumn{3}{|l}
{\tt SUBROUTINE default\_stream\_setting}&
\multicolumn{2}{p{9cm}|}
   {\tt (stream [,units] [,ldims] [,gdims] [,repr]
         [,lpost] [,laccu] [,reset] [,lrerun] [,contnorest]
         [,table] [,code] [,bits] [,leveltype]
         [,dimnames] [,no\_default])}\\
\hline
\end{tabular}}

If {\tt no\_default=.true.} is not given, previously changed
default values are kept.

Properties and attributes of an existing stream element may be changed
by calling\\ {\tt set\_stream\_element\_info}
\index{streams!set\_stream\_element\_info}. Again, the arguments are
similar to those of {\tt add\_stream\_element\_info}:

{\small
\begin{tabular}{|llllp{6cm}|}
\hline
\multicolumn{3}{|l}
{\tt set\_stream\_element\_info}&
\multicolumn{2}{p{9cm}|}
   {\tt (stream ,name ,longname [,units] 
         [,ldims] [,gdims] [,ndim] [,klev] [,ktrac] [,alloc] [,repr]
         [,lpost] [,laccu] [,lmiss] [,missval] [,reset] [,lrerun]
         [,contnorest] [,table] [,code] [,bits] [,leveltype]
         [,dimnames] [,no\_default])}\\
\hline
\end{tabular}}

%------------------------------------------------------------------------------
\subsubsection{Access to stream elements}

References to previously defined stream elements or to their meta data
can be obtained by calling the subroutine {\tt get\_stream\_element}
\index{streams!get\_stream\_element} or
{\tt
  get\_stream\_element\_info}\index{streams!get\_stream\_element\_info}, 
respectively:

{\small
\begin{tabular}{|lllp{6cm}|}
\hline
\multicolumn{2}{|l}
{\tt get\_stream\_element\_info}&
\multicolumn{2}{p{9cm}|}
 { (stream, name, info)}\\
\hline
name&type&intent&description\\
\hline
stream & type(t\_stream)   & in  & output stream to which reference
has to be added.\\
name   & character(len=*)  & in  & name of stream element.\\
info   & type(memory\_info)& out & copy of meta data type content.\\
\hline
\end{tabular}}

{\small
\begin{tabular}{|lllp{6cm}|}
\hline
\multicolumn{2}{|l}
{\tt get\_stream\_element}&
\multicolumn{2}{p{9cm}|}
 { (stream, name, ptr)}\\
\hline
name&type&intent&description\\
\hline
stream & type(t\_stream)  & in      & output stream list.\\
name   & character(len=*) & in      & name of stream element.\\
ptr    & real(:,:[,:][,:])& pointer & returned reference to stream
                                      element memory.\\
\hline
\end{tabular}}

%------------------------------------------------------------------------------
\subsubsection{Doubling of stream element entries}

It is possible to add a reference to an output stream element to
another output stream. By calling the subroutine {\tt
add\_stream\_reference}. 
\index{streams!add\_stream\_reference}
This is useful when the same field shall be
written to different output files. 

{\small\begin{tabular}{|lllp{6cm}|}
\hline
\multicolumn{2}{|l}
{\tt add\_stream\_reference}&
\multicolumn{2}{p{9cm}|}
 { (stream ,name [,fromstream] [,lpost] [,kprec)]}\\
\hline
name&type&intent&description\\
\hline
stream     & type(t\_stream) & inout & output stream list to extend.\\
name       & character(len=*)& in    & name of stream element to add.\\
{[fromstream]} & character(len=*)& in    & name of output stream to take
                                       the element from.\\
{[lpost]}      & logical         & in    & postprocessing flag of the 
                                       output stream reference.\\
{[kprec]}      & integer         & in    & precision of GRIB format in
bits (default: 16). \\
\hline     
\end{tabular}}

%------------------------------------------------------------------------------
\subsubsection{Definition of new dimensions}
\label{sec.dim}

  \begin{table}[htb]
  {\small\begin{tabular}{|llllllp{2cm}|}
  \hline
  dimension index & name & klev & GRIB      & values & units & longname\\
                  &      &      & leveltype &        &       &\\
  \hline
  HYBRID     & "lev"     & nlev   & 109 & 1,\dots,nlev    &    & hybrid level at layer midpoints\\
  HYBRID\_H  & "ilev"    & nlev+1 & 109 & 1,\dots,nlev+1  &    & hybrid level at layer interfaces\\
  SURFACE    & "surface" & 1      &   1 & 0               &    & surface field\\
  ABOVESUR2  & "2m"      & 1      & 105 & 0               & m  & level 2m above the surface\\
  ABOVESUR10 & "10m"     & 1      & 105 & 0               & m  & level 10m above the surface\\
  BELOWSUR   & "jpgrnd"  & 5      & 111 & 3,19,78,268,698 & cm &
  levels below the surface\\
  TILES      & "tiles"   & ntiles & 70  & 1,\dots,ntiles &    & land
  surface tile\\ 
  SOILLEV    & "soil\_layer"& nsoil& 71  & 1              & cm & soil
  levels (water)\\ 
  ROOTZONE   & "root\_zones"& nroot\_zones& 72  & 1,\dots,nroot\_zones
  & & root zone \\
  CANOPY     & "canopy\_layer"& ncanopy & 73  & 1,\dots,ncanopy &    & layers in canopy \\
  \hline
  \end{tabular}}
  \caption{Predefined dimensions}
  \label{tab.dims}
  \end{table}

If other dimensions are required than those defined in Table
\ref{tab.dims}, new dimensions can be defined by calling the subroutine
{\tt add\_dim} defined in module {\tt mo\_netcdf}.

\index{streams!add\_dim}
{\small
\begin{tabular}{|llllp{6cm}|}
\hline
\multicolumn{2}{|l}
{\tt SUBROUTINE add\_dim}&
\multicolumn{3}{p{10cm}|}
                         {\tt (name ,len [,longname] [,units]
                               [,levtyp] [,single] [,value] 
                               [,indx]) }\\
\hline
name&type&intent&default&description\\
\hline
name         & character(len=*) & in &           & name of dimension.\\
len          & integer          & in &           & size of dimension.\\
{[longname]} & character(len=*) & in & ' '       & long name of dimension.\\
{[units]}    & character(len=*) & in & ' '       & physical units of dimension.\\
{[levtyp]}   & integer          & in & 0         & GRIB level type.\\
{[single]}   & logical          & in & .false.   & flag indicating
                                                   single level fields.\\
{[value]}    & real             & in & 1,2,\dots & values of dimension field.\\
{[indx]}     & integer          & out&           & index to be passed 
                                                   as argument {\tt leveltype} to subroutine
                                                   {\tt add\_stream\_element}.\\
\hline
\end{tabular}}

\pagebreak
%==============================================================================
