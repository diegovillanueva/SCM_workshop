The \echam{} output is not directly suitable for visualization since
some of the output fields are in the spectral space (3d--temperature,
vorticity, divergence and the logarithm of the surface
pressure). Furthermore, monthly or yearly mean values are more
suitable for a first analysis of a simulation than instantaneous values at a
certain time step. There is a standard postprocessing tool with which standard
plots can be generated. This postprocessing tool also
produces tables of key quantities. The postprocessing consists of two
steps: (1)~preparation of the \echam{} output data, (2)~generation of the plots
and tables.

\subsubsection{Software requirements}

The postprocessing scripts require the installation of the so--called
``afterburner'' that performs the transformation of spectral
variables into grid point space and the interpolation to pressure
levels, the installation of the cdo climate data operator package for
mean value 
calculations and general manipulation of the data, the installation of
the ncl NCAR
graphics tool to generate the plots, and of the \LaTeX{} program package in
order to arrange the viewgraphs in one document.

\subsubsection[Preparation of the \echambw output data]{Preparation of the \echam{} output data}

In general, \code{\tvar{experiment\_name}.job2} processes the raw \echam{}
output such that it is suited for the postprocessing tool (see
section \ref{runscripts:generation}). If \code{\tvar{experiment\_name}.job2}
was not applied, the output data of an \echam{} simulation can be prepared for
the postprocessing tool by the use of the {\tt after.sh} script. The
prerequisite is to have a simulation that was conducted over a time
period of at least one complete year. The output has to be stored in
monthly files. These files can contain either monthly mean values or
(mean) values over smaller time intervals. It is assumed that the
arithmetic mean of the output variables over the time steps in these
monthly files is a 
good estimate of the monthly mean value. Several variables have to be
modified by the user in the {\tt after.sh} script
(see~Tab.~\ref{tabafter}).

\setlength{\LTcapwidth}{\textwidth}
\setlength{\LTleft}{0pt}\setlength{\LTright}{0pt}

\begin{longtable}{l@{\extracolsep\fill}p{12.0cm}}
\hline\hline\caption[Variables of {\tt after.sh}]{Variables of 
  {\tt after.sh} in alphabetical order \newline (e.g. in {\tt /pool/data/ECHAM6/post/quickplots/ncl/})}\\\hline\label{tabafter}
\endfirsthead
\caption[]{{\tt after.sh} --- continued}\\\hline
\endhead
\hline\multicolumn{2}{r}{\slshape table continued on next page}\\
\endfoot
\hline %\multicolumn{2}{r}{end of table}
\endlastfoot
Variable &  Explanation \\\hline
{\tt after} & Location and name of the executable of the afterburner,
e.g.: {\tt /client/bin/after}\\
{\tt cdo} & Location and name of the executable of the climate data operators,
e.g.: {\tt cdo} if no search path is needed\\
{\tt datdir} & Absolute path to the folder in which the original \echam{}
simulation output files are stored\\
{\tt exp} & Experiment name as defined in the variable {\tt
  out\_expname} of the {\tt runctl} namelist (see
Tab.~\ref{tabrunctl})\\
{\tt filename\_suffix} & The extension of the monthly \echam{} (standard)
output files after the number of the months (including leading
dots), e.g.: {\tt .01\_echam.nc}. The output files can be in either
GRIB format (no extension) or netcdf format (including the
extension {\tt .nc}). \\
{\tt first\_year} & First year of simulation data \\
{\tt last\_year} & Last year of simulation data \\
{\tt out\_format} & should be set to 1 for GRIB output format of {\tt
  after.sh} (standard)\\
{\tt workdir} & Absolute path to which the output files of {\tt
  after.sh} are written \\
\end{longtable}

The output files contain monthly mean values over all simulated years
as given by the {\tt first\_year} and {\tt last\_year} variable. There
are 12 output files for the 3--d variables with names {\tt
  ATM\_\$\{exp\}\_\$\{first\_year\}-\$\{last\_year\}\_MMM} with {\tt MMM}
describing the month and 12 output files for the 2--d surface
variables with names {\tt
  BOT\_\$\{exp\}\_\$\{first\_year\}-\$\{last\_year\}\_MMM}.
These files are the input to the program that actually generates the
tables and view graphs.

\subsubsection{Generation of plots and tables}
\label{postprocessing:quickplots}

The plots and tables are generated by the
script \code{\tvar{experiment\_name}.plot} (in the script directory, see
section \ref{runscripts:generation}) in the case of a comparison of one model
simulation with era40 data, or by the
script \code{\tvar{experiment\_name}.plot\_diff} in the case of the comparison
of two different experiments. Again, some variables have to be set by the user
directly in the scripts. In the case of the
script \code{\tvar{experiment\_name}.plot} the variables are listed in
Tab.~\ref{tabpostjob}, in the case
of \code{\tvar{experiment\_name}.plot\_diff}, the variables are listed in
Tab.~\ref{tabpostjobdiff}.

\setlength{\LTcapwidth}{\textwidth}
\setlength{\LTleft}{0pt}\setlength{\LTright}{0pt}

\begin{longtable}{l@{\extracolsep\fill}p{13.5cm}}
\hline\hline\caption[Variables of {\tt \tvar{experiment\_name}.plot}]{Variables
  of {\tt \tvar{experiment\_name}.plot} in alphabetical order \newline(e.g. in
  {\tt echam-\tvar{tag\_number}/experiments/\tvar{experiment\_name}/scripts/})
  }\\\hline\label{tabpostjob}
\endfirsthead
\caption[]{{\tt \tvar{experiment\_name}.plot} --- continued}\\\hline
\endhead
\hline\multicolumn{2}{r}{\slshape table continued on next page}\\
\endfoot
\hline %\multicolumn{2}{r}{end of table}
\endlastfoot
Variable &  Explanation \\\hline
{\tt ATM} & $=1$ if viewgraphs of atmosphere fields are desired, $=0$
otherwise\\ 
{\tt atm\_RES} & Spectral resolution of the model, e.g. 31 for the T31
spectral resolution\\
{\tt BOT} & $=1$ if viewgraphs of surface fields are desired, $=0$ otherwise\\
{\tt COMMENT} & Any comment that describes your experiment (will
appear on the plots)\\
{\tt EXP} & Experiment name as defined in the variable {\tt
  out\_expname} of the {\tt runctl} namelist (see
Tab.~\ref{tabrunctl})\\
{\tt LEV} & Number of levels\\
{\tt oce\_RES} & Resolution of the ocean, e.g.~{\tt GR30} for the GROB
30 resolution.\\
{\tt LOG} & only if {\tt LOG\_}$\ast$ files exist, currently not
implemented in {\tt after.sh}\\
\code{LONG} & $=1$ prints all bottom codes, $=0$ prints only a selection of
codes (4, 97, 142, 143, 150, 164, 167, 178, 179, 210, 211, 230, 231, 191, 192)\\
\code{NAME} & name of data files (without the \code{ATM\_}, \code{BOT\_}, or
\code{LOG\_} prefix). Defaults to \verb|${EXP}_${YY1}-${YY2}_${TYP}|\\
{\tt PRINTER} & name of black and white printer $=0$ if printing is not
desired (results will be shown on screen only). CAUTION: If the
printer {\tt PRINTER} exists, printing 
is automatic without asking the user again!\\
{\tt PRINTERC} & name of color printer, $=0$ if printing is not
desired (results will be shown on screen only). CAUTION: If the
printer {\tt PRINTERC} exists, printing 
is automatic without asking the user again!\\
{\tt TAB} & $=1$ if tables are desired, $=0$ otherwise\\
\code{TO} & select end of ERAinterim reference period (2008 or 1999)\\
{\tt TYP} & type of plots. There are 17 possible types: {\tt ANN}:
annual mean values (they will be calculated from the monthly means by
weighting with the length of the respective months). Seasonal mean
values for the seasons {\tt DJF} (December, January, February), {\tt
  MAM} (March, April, May), {\tt JJA} (June, July, August), {\tt SON}
(September, October, November). In the case of the seasonal mean
values, the length of the respective months is not taken into account when
the mean values over the corresponding three months are calculated. One
of the twelve months of a year ({\tt JAN}, {\tt FEB}, {\tt MAR}, {\tt
  APR}, {\tt MAY}, {\tt JUN}, {\tt JUL}, {\tt AUG}, {\tt SEP}, {\tt
  OCT}, {\tt NOV}, {\tt DEC}). The seasonal and monthly (and also
annual) mean values are 
``climatological'' mean values over possibly several years.\\
{\tt WORKDIR} & Path to the directory where the monthly means prepared
by the {\tt after.sh} script are stored\\
{\tt YY1} & First simulated year \\
{\tt YY2} & Last simulated year \\
\end{longtable}

\setlength{\LTcapwidth}{\textwidth}
\setlength{\LTleft}{0pt}\setlength{\LTright}{0pt}

\begin{longtable}{l@{\extracolsep\fill}p{13.5cm}}
\hline\hline\caption[Variables of {\tt \tvar{experiment\_name}.plot\_diff}]{
  Variables of {\tt \tvar{experiment\_name}.plot\_diff} in alphabetical order
  for comparison of simulation~1 with simulation~2 \newline(e.g. in {\tt
  echam-\tvar{tag\_number}/experiments/\tvar{experiment\_name}/scripts/}
  }\\\hline\label{tabpostjobdiff}
\endfirsthead
\caption[]{{\tt \tvar{experiment\_name}.plot\_diff} --- continued}\\\hline
\endhead
\hline\multicolumn{2}{r}{\slshape table continued on next page}\\
\endfoot
\hline %\multicolumn{2}{r}{end of table}
\endlastfoot
Variable &  Explanation \\\hline
{\tt ATM} & $=1$ if viewgraphs of atmosphere fields are desired, $=0$
otherwise\\ 
{\tt atm\_RES} & Spectral resolution of the model, e.g. 31 for the T31
spectral resolution\\
{\tt BOT} & $=1$ if viewgraphs of surface fields are desired, $=0$ otherwise\\
{\tt COMMENT} & Any comment that describes your experiment (will
appear on the plots)\\
{\tt AEXP} & Experiment name as defined in the variable {\tt
  out\_expname} of the {\tt runctl} namelist (see
Tab.~\ref{tabrunctl}) for simulation~1\\
\code{ANAME} & name of data files for simulation~1.
Defaults to \verb|${AEXP}_${AYY1}-${AYY2}_${TYP}|\\
{\tt AYY1} & First simulated year of simulation~1\\
{\tt AYY2} & Last simulated year of simulation~1\\
{\tt BEXP} & Experiment name as defined in the variable {\tt
  out\_expname} of the {\tt runctl} namelist (see
Tab.~\ref{tabrunctl}) for simulation~2\\
\code{BNAME} & name of data files for simulation~1.
Defaults to \verb|${BEXP}_${BYY1}-${BYY2}_${TYP}|\\
{\tt BYY1} & First simulated year of simulation~2\\
{\tt BYY2} & Last simulated year of simulation~2\\
{\tt LEV} & Number of levels\\
{\tt oce\_RES} & Resolution of the ocean, e.g.~{\tt GR30} for the GROB
30 resolution.\\
{\tt LOG} & only if {\tt LOG\_}$\ast$ files exist, currently not
implemented in {\tt after.sh}\\
{\tt PRINTER} & name of black and white printer, $=0$ if printing is not
desired (results will be shown on screen only). CAUTION: If the
printer {\tt PRINTER} exists, printing 
is automatic without asking the user again!\\
{\tt PRINTERC} & name of color printer, $=0$ if printing is not
desired (results will be shown on screen only). CAUTION: If the
printer {\tt PRINTERC} exists, printing 
is automatic without asking the user again!\\
{\tt TAB} & $=1$ if tables are desired, $=0$ otherwise\\
{\tt TYP} & type of plots. There are 17 possible types: {\tt ANN}:
annual mean values (they will be calculated from the monthly means by
weighting with the length of the respective months). Seasonal mean
values for the seasons {\tt DJF} (December, January, February), {\tt
  MAM} (March, April, May), {\tt JJA} (June, July, August), {\tt SON}
(September, October, November). In the case of the seasonal mean
values, the length of the respective months is not taken into account when
the mean values over the corresponding three months are calculated. One
of the twelve months of a year ({\tt JAN}, {\tt FEB}, {\tt MAR}, {\tt
  APR}, {\tt MAY}, {\tt JUN}, {\tt JUL}, {\tt AUG}, {\tt SEP}, {\tt
  OCT}, {\tt NOV}, {\tt DEC}). The seasonal and monthly (and also
annual) mean values are
``climatological'' mean values over possibly several years.\\
{\tt WORKDIR} & Path to the directory where the monthly means prepared
by the {\tt after.sh} script are stored\\
\end{longtable}

The results are stored in several files in the directory {\tt
  \$\{WORKDIR\}\_\$\{TYP\}}. The tables are in the files {\tt
tabelle\_\$\{EXP\}\_\$\{YY1\}-\$\{YY2\}\_\$\{TYP\}[.ps]} in either
ASCII or postscript (ending {\tt .ps}) format. The viewgraphs are
stored in the files {\tt ATM\_\$\{TYP\}\_\$\{EXP\}.[tex,ps]}, {\tt
  ATMlola\_\$\{TYP\}\_\$\{EXP\}.[tex,ps]}, and {\tt
  BOT\_\$\{TYP\}\_\$\{EXP\}.[tex,ps]}. The \LaTeX{} files
$\ast${\tt.tex} combine several encapsulated postscript format
viewgraphs in one document.
