\subsection{Single column model (SCM)}\label{secscm}
\echam{} is a general circulation model that simulates the
transport of air masses, energy, and trace gases like water vapour
inside these air 
masses by advection, convection, and small scale turbulence 
(eddies) represented by diffusion equations. Furthermore, all
relevant physics like radiation, cloud and precipitation formation, and
surface processes are included. In some cases, it is difficult to
separate local effects from large scale dynamics, e.g. the direct influence
of radiation on cloud formation may be obscured by advection of energy
from neighbouring columns.
In these cases, the analysis of physics processes in one
single isolated column of the model can shed light on the mutual
relationships of these processes. The analysis of the behaviour of
model physics in one column can help us to develop new
parameterisations and is the natural test bed for physics
parameterisations. Furthermore, a single column may be
considered as a very primitive model of the atmosphere of the earth
represented by the processes in one single ``average'' column. It may be
instructive to investigate extreme scenarios like a very hot climate
and the behaviour of the physics implemented in \echam{} under such
conditions in a ``single column version'' of \echam. 

\subsubsection{Initial conditions and forcing data for the single column
  model}\label{secforcing} 

Similar to a general circulation model, the single column model 
needs initial 
conditions as starting point of time integration. Furthermore, it is
possible to relax the trajectory of certain variables towards a given
trajectory of these variables or to prescribe tendencies for certain
variables. All input data i.e.~initial conditions and externally
prescribed trajectory and tendency data are read from one single ``forcing''
file the name of 
which can be set in the {\tt columnctl} namelist file. 

The geographical location of the column on the globe is given by its
geographical longitude and latitude described by the variables {\tt lon}, {\tt 
  lat} in the forcing file. The single column model reads the
longitude and latitude from this file, they cannot be set in the namelist.
Since the single column model applies the 2d land sea mask and surface
properties to the geographical location of the column, the surface
properties are implicitly determined by the longitude and latitude of
the column. Furthermore, all geographically dependent quantities like
the diurnal cycle, solar irradiation, greenhouse 
gas or aerosol mixing ratios, and sea surface temperature are
automatically calculated for this special geographical location or
extracted from the respective \echam{} input files.

Examples for forcing files can be found in {\tt /pool/data/ECHAM6/SCM}.

\paragraph{Initial condition variables, trajectory variables, and
  tendency variables in the forcing file}

The forcing file contains the variables listed in the first column of
Tab.~\ref{tabiniscm} 
describing at the same time the
initial state and a trajectory of that state. The first time step of
these variables is used as the initial state.
The first column gives the names under which the
variables appear in 
the forcing file. Furthermore, the corresponding tendencies of these
variables may also be present. The names of the corresponding
tendencies are listed 
in the second column of Tab.~\ref{tabiniscm}. 
All variables depend on the dimensions time [and levels] {\tt
  (time[,nlev])}. 

\setlength{\LTcapwidth}{\textwidth}
\setlength{\LTleft}{0pt}\setlength{\LTright}{0pt}

\begin{longtable}{l@{\extracolsep\fill}llp{5.0cm}p{3.0cm}}
\hline\hline\caption[Initial conditions SCM]
{Variables describing the initial state, its trajectory, and
  tendencies in the forcing file. As initial conditions, 
  the first time step of the variables listed in the first column of
  the table are used. The
dimensions of each variable are reported in the third column of the
table. The mode in the last column of the table is
marked ``essential'' if the 
variable must be present as initial condition or optional if the
variable can be set to zero at the initial state.}\\\hline\label{tabiniscm}
\endfirsthead
\caption[]{Initial conditions SCM --- continued}\\\hline
\endhead
\hline\multicolumn{5}{r}{\slshape table continued on next page}\\
\endfoot
\hline %\multicolumn{4}{r}{end of table}
\endlastfoot
variable & tendency & dimension & explanation & mode \\\hline
{\tt t} & {\tt ddt\_t} & {\tt (time,lev)} & temperature in the column
& essential \\ 
{\tt u} & {\tt ddt\_u} & {\tt (time,lev)} & wind in $\vec{u}$
direction & essential \\ 
{\tt v} & {\tt ddt\_v} & {\tt (time,lev)} & wind in $\vec{v}$
direction & essential \\ 
{\tt q} & {\tt ddt\_q} & {\tt (time,lev)} & specific humidity & essential \\
{\tt ps} & --- & {\tt (time)} & surface pressure & essential \\
{\tt xl} & {\tt ddt\_ql} & {\tt (time,lev)} & liquid water content & optional \\
{\tt xi} & {\tt ddt\_qi} & {\tt (time,lev)} & ice water content & optional \\
\hline
\end{longtable}

As mentioned above, it is possible to relax the state variables listed
in Tab.~\ref{tabiniscm} 
towards some
given trajectory.
The relaxation is performed
in the following way: Let $X^{\rm (f)}_t$ be the value of a quantity
$X$ at time $t$ to which the original prediction $X_t$ of this quantity for
time $t$ has to be relaxed. Let $\tau>0$ be a relaxation time and
$\Delta t>0$ the integration time step. Then,
the new prediction $\tilde{X}_t$ at time $t$ is given by:

\begin{equation}\label{eqrelaxation}
\tilde{X}_t:= \begin{cases}
X_t+(X^{\rm (f)}_t-X_t)\frac{\Delta
  t}{\tau} & \mbox{for}\quad \tau>\Delta t \\
X^{\rm (f)}_t\ &\mbox{for}\quad \tau\le\Delta t
\end{cases}
\end{equation}

In addition to the application of a trajectory until it ends,
the same given trajectory may be repetitively applied (``cycled''),
e.g.~a diurnal cycle may be applied over and over again. The
prescribed trajectory can be given at any regular time intervals and
is interpolated to the actual model time steps. 

When one applies the relaxation method to certain variables, the trajectory of
the respective variables will be 
restricted to a neighbourhood of the given trajectory. There is a second
method to influence the trajectory: Instead of the internally
produced tendencies (internal tendencies) resulting from the physics
processes in the respective 
column, tendencies originating from 3d large scale dynamics (external
tendencies) may be
used or added to the internally produced tendency. In general, if any
external tendencies are provided, the single column
model simply replaces the internal tendencies by the external
tendencies with one exeption: If vertical pressure velocity or divergence is
prescribed from an external data set (see Sec.~\ref{secvarfor}), the external
tendencies of {\tt t}, {\tt u}, {\tt v}, {\tt q}, {\tt ql}, {\tt qi}
are added to the internal tendencies.
Tendencies can be
used for all variables of Tab.~\ref{tabiniscm} except for the surface
pressure. Since the mass of dry air in the column is considered to be
constant in time, the surface pressure can not change.

The various forcing options described above for the variables of
Tab.~\ref{tabiniscm} are coded in an ``option'' array of three integer numbers
$\{i_\Delta,\tau,i_{\rm cycle}\}$. To each variable such an option
array is assigned. The first element $i_\Delta$ is
equal to~0 if no external tendencies are used for the respective
variable, i.e.~the variable is only changed due to physics processes in the
column. If $i_\Delta=1$, the external tendencies are applied according
to the rule above. The second element $\tau$ of the option array is
the relaxation time in seconds. The third element $i_{\rm cycle}$ has
to be set to~1 if cycling of the external trajectory is desired, it
has to be set to~0 if the trajectory is not cycled.

\paragraph{Forcing by prescribing values of certain
  variables}\label{secvarfor}
Up to now, we described how to influence the trajectory of the state
variables listed in Tab.~\ref{tabiniscm}.
Furthermore, there is a set of variables the values of which can or can not be
externally prescribed. These variables are listed in
Tab.~\ref{tabextvar}.

\setlength{\LTcapwidth}{\textwidth}
\setlength{\LTleft}{0pt}\setlength{\LTright}{0pt}

\begin{longtable}{l@{\extracolsep\fill}lp{5.0cm}}
\hline\hline\caption[Boundary condition variables]{Boundary condition
  variables}\\\hline\label{tabextvar} 
\endfirsthead
\caption[]{{\tt boundary condition variables} --- continued}\\\hline
\endhead
\hline\multicolumn{3}{r}{\slshape table continued on next page}\\
\endfoot
\hline %\multicolumn{3}{r}{end of table}
\endlastfoot
variable & dimension & explanation \\\hline
{\tt ts} & {\tt (time)} & surface temperature \\
{\tt div} & {\tt (time,lev)} & divergence of the wind field \\
{\tt omega} & {\tt (time,lev)} & vertical pressure velocity \\
\hline
\end{longtable}

For the variables listed in Tab.~\ref{tabextvar} the ``option'' array
consists of two elements $\{i_{\rm set},i_{\rm cycle}\}$. If the first
element $i_{\rm set}=0$, the variable is allowed to change freely,
whereas $i_{\rm set}=1$ means that the corresponding variable is set
to the value given by the external data set. The second element
$i_{\rm cycle}$ determines whether ($i_{\rm cycle}=1$) or not ($i_{\rm
  cycle}=0$) cyclic interpolation with respect to time of the external
data set is required.

\subsubsection{Namelist columnctl}

The namelist is described in Sec.~\ref{seccolumnctl}.
