
\section[cr2014\_07\_14\_rjs: Radiative convective equilibrium]{cr2014\_07\_14\_rjs:
  Simulation of radiative--convective equilibrium using \echam}\label{cr20140715rjs}

\subsection{Introduction}

The radiative--convective equilibrium (RCE) offers a possibility
to improve our fundamental understanding of processes in the
atmosphere and their impact on climate change
(e.g.~\cite{man641}). The idea behind this simplified modeling
of the atmosphere 
is that the basic 
atmospheric structure, especially in the tropics, is determined by the
balance between cooling of the atmosphere through radiative processes
and a commensurate heating through convection, mainly by the net
release of latent heat through precipitation. 

The RCE has been investigated in models of different complexity,
ranging from simple energy balance, 1--dimensional column models to
high resolution LES simulations. The RCE is also implemented into the 
general circulation model \echam{} by creating a model configuration,
where the resulting climate is given merely through the balance of
radiative processes and convection. Columns can interact with each
other and thus create a mean three--dimensional circulation which
develops interactively, although it is very different from the general
circulation we know from the real Earth. E.g., the RCE results in slowly moving
convective clusters of sometimes continental extension (\cite{pop131}).  

To inhibit net energy transport from the tropics to the poles,
homogeneous boundary conditions are specified, where every gridpoint
of the sphere receives the same incoming solar radiation
(e.g.~$340\,{\rm W/m^2}$). A diurnal
cycle may be switched on but is kept exactly the same for each column
representing a pulsating light source shining from all directions
equally. The Earth's rotation velocity is set to zero. In the standard
RCE configuration, land--sea contrasts are removed by specifiying an
underlying mixed--layer ocean with a constant ocean albedo, but can
easily be included in idealized form for land--sea contrast
studies~(\cite{bec14x}). Besides the modifications mentioned 
above and technical details listed below, the results of the
ECHAM--RCE 
well resemble the tropics of a control simulation with the full
\echam{} model in the mean
state~\cite{pop131}.
However, this model version has not been tested
for possible equilibria dependence on the initial boundary conditions
yet, nor for complete isotropy of variables expected from the
homogeneous boundary conditions. 

\subsection{Namelist settings for radiative--convective equilibrium}

The simulation of the radiative--convective equilibrium needs some
special namelist settings in order to switch off the diurnal cycle for
example. Table~\ref{tab_cr20140715_rcenamelist} gives an
overview of the necessary settings.

\setlength{\LTcapwidth}{\textwidth}
\setlength{\LTleft}{0pt}\setlength{\LTright}{0pt}

\begin{longtable}{l@{\extracolsep\fill}p{6.0cm}p{2.5cm}}\hline\hline
\caption[RCE namelist]{Namelist setting for radiative--convective
  equilibrium simulations with \echam}\\\hline
\label{tab_cr20140715_rcenamelist}
\endfirsthead
\caption[]{RCE namelist --- continued}\\\hline
\endhead
\hline\multicolumn{3}{r}{\slshape table continued on next page}\\
\endfoot
\hline %\multicolumn{2}{r}{end of table}
\endlastfoot
\multicolumn{1}{c}{Variable} & \multicolumn{1}{c}{Explanation} & \multicolumn{1}{c}{default} \\\hline
\multicolumn{3}{c}{{\tt runctl} namelist}\\\hline
$ {\tt earth\_angular\_velocity} = {\tt 0.0}$ &  switch off Coriolis force: no rotation &
{\tt 7.29212e-5} \\
${\tt lrce} = {\tt .true.} $ & turn on radiative--convective equilibrium mode:\newline
                                same zenith angle at all grid points
                                with either a perpetual day or (${\tt
                                  ldiur}={\tt .true.}$) a diurnal
                                cycle that is equal for all grid
                                points (the irradiation is dimmed and
                                brightened independently of the
                                geographic position).\newline  
                                use constant ocean surface albedo (0.07)\newline
                                ignore dynamical planetary boundary
                                layer height in
                      planetary boundary layer calculation. & {\tt .false.}\\
${\tt ly360} = {\tt .true.}$ & use a 360 days calendar (this is not
a prerequisite for an RCE simulation) & {\tt
  .false.}\\
${\tt l\_orbvsop87}  = {\tt .false.}$ &  use PCMDI (AMIP) orbit that
does not change with time and corresponds to a Kepler orbit. & {\tt
  .true.}\\
\hline
\multicolumn{3}{c}{{\tt radctl} namelist}\\\hline
$ {\tt cecc}  = {\tt 0.0}$ &  eccentricity of Kepler orbit is set to
zero meaning that the orbit is circular & {\tt 0.016715} \\
$ {\tt cobld} = {\tt 0.0}$ & obliquity in degrees is set to zero &
{\tt 23.441} \\ 
$ {\tt iaero} = {\tt 0}  $ & simulate an aerosol free atmosphere &
{\tt 2} \\
$ {\tt isolrad} = {\tt 4\,{\rm or}\,5}$ & solar irradiation\newline
${\tt isolrad} = {\tt 4}$: solar irradiation for RCE including a
diurnal cycle. In this case, {\tt ldiur} must be set to {\tt .true.}\newline
${\tt isolrad} = {\tt 5}$: time constant solar irradiation for RCE. In
this case, {\tt ldiur} must be set to {\tt .false.}& {\tt 3}\\
$ {\tt icfc}    = {\tt 0}$ & switch off all effects of
chlorofluorocarbons &  {\tt 2} \\ 
\hline
\end{longtable}

\subsection{Initial and boundary conditions}

The initial and boundary conditions are different from the usual model
set--up since they are isotropic except for the initial conditions into
which a small perturbation of the isotropy is introduced.

The initial and boundary conditions can be generated in various
resolutions from initial and boundary condition files that exist
already. The script {\tt create\_initial\_files\_rce\_aqua.sh} is
provided at {\tt /pool/data/ECHAM6/input/r0004/rce/bin}. In order to
change the resolution, you have to modify the values of the variables
{\tt RES}, {\tt ILEV}, {\tt OCERES} according to the desired spectral
resolution, vertical levels, and ocean resolution, respectively. In
that case, initial and boundary condition files of standard \echam{}
have to exist in the new resolution {\tt RES}, {\tt ILEV}, {\tt
  OCERES} in {\tt /pool/data/ECHAM6/input/r0001}. The following
tables~\ref{tab_cr20140715_janspec}--\ref{tab_cr20140715_surf} give an
overview of the values of the variables modified for
radiative--convective equilibrium simulations with \echam.

\setlength{\LTcapwidth}{\textwidth}
\setlength{\LTleft}{0pt}\setlength{\LTright}{0pt}

\begin{longtable}{l@{\extracolsep\fill}p{11cm}}\hline\hline
\caption[Spectral initial data for RCE]{Spectral initial data in {\tt
    \{RES\}\{LEV\}\_jan\_spec\_rce.nc} for radiative--convective
  equilibrium simulations}\\\hline
\label{tab_cr20140715_janspec}
\endfirsthead
\caption[]{Spectral initial data for RCE --- continued}\\\hline
\endhead
\hline\multicolumn{2}{r}{\slshape table continued on next page}\\
\endfoot
\hline %\multicolumn{2}{r}{end of table}
\endlastfoot
Variable  & Description \\\hline
${\tt SVO} = 10^{-8}{\rm 1/s}$ & vorticity of the wind field, is
transformed to spectral space in the file\\
${\tt SD} = 10^{-8}{\rm 1/s}$ & divergence of the wind
field, is transformed to spectral space in the file\\
${\tt STP} = (300 {\rm K},11.5261)$ & temperature is set to $300
\rm{K}$ at all model levels globally and the logarithm of the surface
pressure $\ln(p_{\rm surf}/(1 {\rm Pa}))$. Both are transformed to
spectral space in the file and stored in the variable {\tt STP}, the
pressure is stored in the ``level'' ${\tt nlev} + 1$. \\
${\tt Q} = 10^{-8}$ & specific humidity stored in grid point space.\\\hline  
\hline
\end{longtable}

The vorticity and divergence of the wind field are set to a small
value in order to initiate dynamics in the atmosphere. Finally, this leads to
spatial inhomogeneities and triggers 
regional dynamics. 

The surface variables collected in {\tt
  \{RES\}\{OCR\}\_jan\_surf\_rce.nc} are mostly set to zero with only
a few exceptions. Table~\ref{tab_cr20140715_jansurf} lists the
respective variables.

\setlength{\LTcapwidth}{\textwidth}
\setlength{\LTleft}{0pt}\setlength{\LTright}{0pt}

\begin{longtable}{l@{\extracolsep\fill}p{9cm}}\hline\hline
\caption[Surface initial data for RCE]{Surface (initial) data in {\tt 
  \{RES\}\{OCR\}\_jan\_surf\_rce.nc} for radiative--convective
  equilibrium simulations}\\\hline
\label{tab_cr20140715_jansurf}
\endfirsthead
\caption[]{Surface initial data for RCE --- continued}\\\hline
\endhead
\hline\multicolumn{2}{r}{\slshape table continued on next page}\\
\endfoot
\hline %\multicolumn{2}{r}{end of table}
\endlastfoot
Variable  & Description \\\hline
${\tt SLM}=0$ & land--sea mask set to zero to indicate that there is ocean
everywhere \\
${\tt GEOSP}=0{\rm m^2/s^2}$ & The surface geopotential is set to zero (no
mountains)\\
${\tt WS}=0{\rm m}$ & soil wetness \\
${\tt SN}=0{\rm m}$ & snow depth \\
${\tt SLF}=0$ & fractional land--sea mask \\
${\tt AZ0}=0.001{\rm m}$ & surface roughness \\
${\tt ALB}=0.07$ & surface background albedo\\
${\tt FOREST}=0$ & vegetation type\\
${\tt WSMX}=10^{-13}{\rm m}$ & field capacity of soil\\
${\tt FAO}=0$ & FAO data set\\
${\tt GLAC}=0$ & glacier mask\\
${\tt ALAKE}=0$ & lake mask\\
${\tt OROMEA}=0{\rm m^2/s^2}$ & mean orography \\
${\tt OROSTD}=0{\rm m^2/s^2}$ & standard deviation of orography \\
${\tt OROSIG}=0^\circ$ & orographic slope\\
${\tt OROGAM}=0^\circ$ & orographic anisotropy\\
${\tt OROTHE}=0^\circ$ & orographic angle\\
${\tt OROPIC}=0{\rm m}$ & elevation of orographic peaks\\
${\tt OROVAL}=0{\rm m}$ & elevation of orographic valleys\\
\hline
\end{longtable}

The following files listed in Table~\ref{tab_cr20140715_surf} contain
variables related to boundary conditions 
at the surface:

\setlength{\LTcapwidth}{\textwidth}
\setlength{\LTleft}{0pt}\setlength{\LTright}{0pt}

\begin{longtable}{l@{\extracolsep\fill}lp{4cm}}\hline\hline
\caption[Surface initial data for RCE]{Surface boundary condition data
  for radiative--convective 
  equilibrium simulations}\\\hline
\label{tab_cr20140715_surf}
\endfirsthead
\caption[]{Surface boundary condition data for RCE --- continued}\\\hline
\endhead
\hline\multicolumn{3}{r}{\slshape table continued on next page}\\
\endfoot
\hline %\multicolumn{2}{r}{end of table}
\endlastfoot
Variable  & Description \\\hline
{\tt \{RES\}\_amip2sst\_rce.nc} & ${\tt sst}=300{\rm K}$ & sea surface temperature \\
{\tt \{RES\}\_amip2sic\_rce.nc} & ${\tt sic}=0\%$ & sea ice coverage\\
{\tt \{RES\}\_qflux\_rce.nc} & ${\tt aflux}=0$ & heat flux in the
ocean for simulations including mixed--layer ocean\\
\hline
\end{longtable}

As ozone profile, an equatorial column from an ozone file from the
AC\&C/SPARC ozone data base for CMIP5 for the year 1870 is taken and
stored in {\tt  \{RES\}\_ozone\_CMIP5\_rce.nc}.

For all other land--surface files, the usual \echam--files can be linked
since they do not have any influence on the results as long as the
planet has a pure ocean surface.
