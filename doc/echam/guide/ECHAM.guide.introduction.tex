The \echam{} model is a program for the
interactive calculation of the general circulation.
This manual contains a user guide of \echam{}
(chapter~\ref{capuserguide}) including a 
description of the compilation procedure on the 
supercomputer platform \blizzard{} at DKRZ Hamburg
(section~\ref{seccompiling}), a description of the input namelists
(section~\ref{secnamelist}), input files
(section~\ref{secinputfiles}), and ouput files
(section~\ref{secoutputfiles}), a description of example run scripts
(section~\ref{secrunscripts}), and postprocessing scripts
(section~\ref{secpostprocessing}). 
We restrict our description to the supercomputer platform 
\blizzard{} at DKRZ in Hamburg. Performing a simulation on other
computer platforms requires the same input data, but the compiling
procedure and the 
directory structure for output in particular, will be
different.

Chapter~\ref{captechguide} contains a short description of the code of
\echam{} and is intended to be a guide for people who work with the
source code of the atmosphere part of \echam. An introduction to the
\echam--code 
with explanations will become available in form of a lecture soon (``Using
and programming \echam{} --- a first introduction''). 

This description is valid for version echam--\echamversion.
