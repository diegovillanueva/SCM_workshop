\subsection[Systematic technical testing of \echambw]{Systematic technical testing of \echam}
In many cases, scientists wish to modify the \echam{} code for their
special applications. Before any ``production'' simulation can be
started, the modified \echam{} version has to be tested thoroughly. 
The purpose of this collection of korn shell scripts is to provide a
systematic and easy to use test bed of the \echam{} code on a technical
level. These test scripts perform very short simulations in the T31L47
resolution over 12~time
steps in different model configurations in order to trap errors in
the code that cause technical malfunctions. However, this kind of tests can
not detect any 
scientific failure or evaluate the scientific quality of the
results. The tests rely on a
comparison of the output of 12~time steps using the
{\tt cdo diff} tool. We apply the term that the results of two
simulations are ``bit
identical'' if the {\tt cdo diff} command does not find differences
between all netcdf or GRIB output files of these two simulations. This means
that the output on the standard output device of these two simulations
is allowed to be different, e.g.~by new messages for a newly built in
submodel facility. Futhermore, it is only checked whether the netcdf
representation of the output of the two simulations is bit--identical
but not whether all variables during the run of the \echam{} program have
bit identical values in both simulations. 
In addition to tests on one model version that will be called the
test version, such a test version of the model can be compared to a
reference version in the so--called update test.
 
The package of scripts performing these tests can be used on various
computers without queueing system and can be modified in such a
way that individual namelists and input data can be provided to the
test and reference model.

The following tests and combinations of them can be performed by the
test tool (including checkout and compilation of the model which is
always performed): 

\begin{description}
\item[compile:] This is not a real test. The respective test version is
  checked out from the svn version control system if necessary and
  compiled, but no run is performed. 
\item[single test:] In this test, the test version is (checked out,
  compiled, and) run for 12 time
  steps. The test is successful if the program does not crash.
\item[debug test:] This runs the \emph{single test}, but on a single processor
  only to allow for debugging.
\item[parallel test:] For this test, a simulation of the test \echam{}
  version over
  12~time steps is performed on 1~and 2~processors, respectively, and
  the result is compared by the {\tt cdo diff}
  tool for every time step. The test simulation on a single processor
  is also performed using the 
  parallel mode of the program. It is therefore not a test for the
  version of \echam{} without message passing interface (mpi). With this
  kind of test, possible parallelization errors can be detected like
  the usage of variables or fields which were not sent to all
  processors. The result of these two simulations should be bit
  identical. On massive parallel machines, using a lot of processors
  distributed over several nodes further problems may occur
  even if this test is passed. Such problems are often either subtle
  errors in the usage of mpi or compiler problems. Supplemental tests
  have to be performed on a later stage when the program is ported to
  such a platform.
\item[nproma test:] The section of the globe that is present on a
  processor after distribution of the data onto the processors, is
  vectorized by blocks of maximum length {\tt nproma}. This means that ---
  even if only one processor is used --- surface fields of the earth do not
  simply have two dimensions of the size of longitudes $n_{\rm lon}$
  and latitudes $n_{\rm lat}$
  but are reshaped to {\tt ngpblks} blocks of maximum length {\tt
    nproma}. Since {\tt nproma} may not be a divisor of $n_{\rm
    lon}\times n_{\rm lat}$, there may be a last block that contains fewer than
  {\tt nproma} elements. This may lead to problems in the code, if
  such non--initialized elements of the last block are
  used accidentally. The nproma test traps such errors by using two different
  nproma lengths of 17 and 23 which are both not divisors of $n_{\rm
    lon}$ in the T31
  resolution in the test simulations and comparing the results of
  12~time steps. The results should be bit--identical.
\item[rerun test:] \echam{} has the possibility to split up a long term
  simulation into several runs of a shorter time period and to restart
  the model at a certain date. The results after restart are bit
  identical with those of a simulation without restart. There is a
  large variety of errors associated with a failure of the restart
  facility which can not all be trapped by this test like wrong
  scripting of the use of transient boundary conditions, but to pass this
  test is a minimum requirement. The base simulation starts at
  1999-12-31, 22:00:00h, writes a
  restart file at 23:45:00h. It stops after a total of 12~time
  steps. The rerun files are used to restart the program and 
  to complete the 12~times steps. The five time steps after restart
  are then compared with the simulation that was not interrupted. The
  results should be bit--identical.  
\item[update test:] This test compares the results of two simulations
  with different model versions (test version versus reference
  version). Under certain circumstances, 
  bit--identical results may be required in this test.
\item[submodel off test:] The above standard tests are all run in a
  model configuration that comprises submodels (configuration similar
  to the CMIP--5 simulations). In some cases, one may be interested in
  a configuration without any submodel. This test tries to run \echam{}
  without any submodel. If two revisions are compared, the
  results of this model configuration are also compared for the test
  and reference 
  revision.
\end{description}

\subsubsection{System requirements}

The \echam{} test scripts can be adapted to UNIX~computers without queuing
system. The automatic configure procedure for the model compilation
has to work and the environment has to provide the possibility to run
programs using message passing interface (mpi). The initial and
boundary condition data of \echam{} have to be 
directly accessible in some directory. If there is no direct access to
the version control system of echam (svn), individual model versions
on the computer may be used in the 
tests, but the path name of the location of these model versions has to
follow the below described conventions.

\subsubsection{Description of the scripts}\label{sususecdescr}

In figure~\ref{figflow}, we present the flow chart of the scripts
performing the test simulations of \echam{} and the comparison of the
results. The scripts need some additional variables that are written
to files by the master script {\tt test\_echam6.sh} and read from these
files by the dependent scripts. The variables can be set in the master
script as described in Tab.~\ref{tabvar1}. The corresponding
files must not be modified by hand. The file {\tt
  c.dat} contains the module name of the C compiler, the file {\tt
  fortran.dat} contains the module name of the fortran compiler, the
file {\tt mpirun.dat} contains the absolute path and name of the
command to start programs using message passing interface (mpi), the
file {\tt outfiletype.dat} contains a number associated with the type
of the output files (1~for GRIB format and 2~for netcdf format).

\begin{figure}[h]
\tikzstyle{blockred} = [rectangle, draw, fill=red!20,
    text width=12em, text centered, minimum height=2em]
\tikzstyle{blockgreen} = [rectangle, draw, fill=green!20,
    text width=12em, text centered, minimum height=2em]
\tikzstyle{blockblue} = [rectangle, draw, fill=blue!20,
    text width=12em, text centered, minimum height=2em]
\tikzstyle{block} = [rectangle, draw,
    text width=13em, text centered, minimum height=2em]
\tikzstyle{joint} = [draw, ]
\tikzstyle{line} = [draw, -]
\begin{tikzpicture}[node distance = 2cm, auto]
\node [blockred] (init) {\tt test\_echam6.sh};
\node [joint, below of=init] (j0) {};
\node [block, right of=j0, node distance=3.3cm] (dir) {\tt test\_checkdirectories.sh};
\node [joint, below of=j0] (j1) {};
\node [block, right of=j1, node distance=3.3cm] (comp) {\tt compile\_echam6.sh};
\node [joint, below of=j1] (j2) {};
\node [blockblue, right of=j2, node distance=3cm] (mode) {\tt test\_{\it mode}.sh};
\node [joint, below of=mode] (j21) {} ;
\node [block, right of=j21, node distance=3.3cm] (run) {\tt
  test\_echam6\_run.sh};
\node [joint, below of=run] (j31) {};
\node [blockgreen, right of=j31, node distance=5cm, text width=20em] (links) {\tt
  test\_echam6\_$\left\{\stackrel{\displaystyle\rm
      reference}{\displaystyle\rm test}\right\}$\_links.sh};
\node [block, below of=links] (link) {\tt link.sh};
\node [joint, below of=j31, node distance=4cm] (j32) {};
\node [blockgreen, right of=j32, node distance=5cm, text width=20em] (namelists) {\tt
  test\_echam6\_$\left\{\stackrel{\displaystyle\rm 
      reference}{\displaystyle\rm test}\right\}$\_namelists.sh};
\node [joint, below of=j32] (j33) {};
\node [blockgreen, right of=j33, node distance=5cm] (mpi) {\tt mpirun.dat};
\node [joint, below of=j21, node distance=10cm] (j22) {};
\node [block, right of=j22, node distance=3.3cm] (diff) {\tt test\_diff.sh};
\path [line] (init)--(j0);
\path [line] (j0)--(dir);
\path [line] (j0)--(j1);
\path [line] (j1)--(comp);
\path [line] (j1)--(j2);
\path [line] (j2)--(mode);
\path [line] (mode)--(j21);
\path [line] (j21)--(j22);
\path [line] (j21)--(run);
\path [line] (run)--(j31);
\path [line] (j31)--(links);
\path [line] (links)--(link);
\path [line] (j31)--(j32);
\path [line] (j32)--(namelists);
\path [line] (j32)--(j33);
\path [line] (j33)--(mpi);
\path [line] (j22)--(diff);
\end{tikzpicture}
\caption{Flow chart of test scripts. 
  The main script in the red box has to be modified by the user.
  The scripts in the green boxes can be modified in order to use
  different model settings than the standard ones for test or
  reference model, respectively.
  The script in the blue box
  depends on the test mode and is one of {\tt\it mode}$=${\tt single},
  {\tt
    parallel}, {\tt nproma}, {\tt rerun}, {\tt submodeloff}, {\tt
    parallelnproma}, {\tt 
    parallelnpromarerun}, {\tt parallelnpromarerunsubmodelloff}, 
    {\tt update}, {\tt
    all}.}\label{figflow}   
\end{figure}

\begin{description}
\item[test\_echam6.sh:] This script contains a definition part
  where all the path names and the model version for the test and
  reference model must be set. It is also the place at which the key
  word for the kind of test is defined. It calls the scripts for
  downloading the respective model versions from svn if they are not
  yet present on your computer and calls the compile and test run scripts.
\item[test\_directories.sh:] Checks the existance of the directories
  and the svn URL of the test and reference model. You may enter your
  special settings on the command 
  line if one of the items is not found. If it is found, it is used
  without further notice. Only the relevant items are checked. 
\item[compile\_echam6.sh:] This script downloads the respective
  model version from the revision administration system svn if it is
  not yet present on your computer and compiles the model. Compilation
  can be forced. Note that the compiler options depend
  on the settings in the input scripts of the configure procedure
  and may be different from revision to revision. Different compiler
  options may lead to numerically different results although the
  algorithms in the code are identical!
\item[test\_{\textsl{mode}}.sh:] This family of scripts performs the various
  simulations and the comparison of the results. The {\tt\it mode} is
  one of {\tt single}, {\tt parallel}, {\tt nproma}, {\tt rerun}, {\tt update},
  {\tt submodeloff},
  {\tt parallelnproma}, {\tt 
    parallelnpromarerun}, {\tt parallelnpromarerunsubmodeloff}, {\tt all}.
\item[test\_echam6\_run.sh:] General run script for echam. 
\item[test\_echam6\_\{test,reference\}\_links.sh:] Script that
  provides the links to all input and boundary condition files needed
  for simulations with \echam. In the standard version, the two scripts
  are identical but allow the user to apply different files for the
  reference and test model, respectively.
\item[test\_echam6\_\{test,reference\}\_namelists.sh:] These scripts
  generate the namelists for the reference and test model
  separately. In the standard version, these two scripts are
  identical. They are useful if the introduction of a new submodel
  requires a namelist for the test model that is different from the
  namelist used for the reference model.
\item[test\_diff.sh:] This script performs a comparison of all
  output files that are common to two test simulations. It also gives
  a list of outputfiles that are not common to the two test
  simulations. If there are no results written into an output file
  during the 12~time steps of the test simulations, the comparison of
  the files with the {\tt cdo diff} command leads to an error message
  that the respective file structure is unsupported. 
\end{description}

\subsubsection{Usage}

The scripts should be copied into a directory
that is different form the original \echam{} directory so that
you can savely change them without overwriting the original.
The files {\tt $\ast$.dat} must not be changed but contain values of
``global'' variables to all scripts. They are described in
section~\ref{sususecdescr}. 
The variables
that have to be modified in {\tt test\_echam6.sh} are listed in
table~\ref{tabvar1}. 
Note that the revision specific
  path of the \echam{} 
  model will be automatically composed as {\tt
    \$\{REF\_DIR\}/\$\{REF\_BRANCH\}\_rev\$\{REF\_REVISION\}} for the reference
  model and as {\tt \$\{TEST\_DIR\}/\$\{TEST\_BRANCH\}\_rev\$\{TEST\_REVISION\}}
  for the test model, respectively. Inside these directories, the
  echam model sources are expected to be in a revision independent
  directory {\tt \$\{REF\_BRANCH\}} and {\tt \$\{TEST\_BRANCH\}},
  respectively.
The simulation results will be in directories {\tt
  \$\{REF\_ODIR\}/0000nrev\$\{REF\_REVISION\}} and {\tt
  \$\{TEST\_ODIR\}/0000nrev\$\{TEST\_REVISION\}} for the reference and
test model, respectively. The number {\tt n} is the number of the
experiment. If in such a directory, an outputfile $\ast${\tt .err}
exists, the test tool assumes that the simulation already exists and
does not perform a new simulation. The results are not removed
once a test is performed in order to avoid the repetition of the same
test simulation over and over again (e.g.~for the reference model). If
experiments have to be repeated, the corresponding directory or at
least the $\ast${\tt .err} file inside this directory has to be
removed by hand.

The test is then started by typing {\tt ./test\_echam6.sh} in the directory
of the test scripts. 

The test script {\tt test\_echam6.sh} can be started by adding three
arguments {\tt MODE} {\tt TEST\_REVISION} {\tt REF\_REVISION} giving
the key word for the test mode, the revision number of the test model
and the revision number of the reference model, respectively. It is
possible to omit {\tt REF\_REVISION} or both {\tt TEST\_REVISION} and
{\tt REF\_REVISION}.

The links to input and boundary condition data and the input namelists
for the model revisions can be modified for the reference and the test
model individually by editing the scripts {\tt
  test\_echam6\_\{reference,test\}\_links.sh} and {\tt
  test\_echam6\_\{reference,test\}\_namelists.sh}, respectively. This
makes this collection of test scripts rather flexible: It may be used
even for models containing extensions of \echam{} like \echam-HAM or
\echam-HAMMOZ. 

\setlength{\LTcapwidth}{\textwidth}
\setlength{\LTleft}{0pt}\setlength{\LTright}{0pt}

\begin{longtable}{l@{\extracolsep\fill}p{10cm}}
\hline\hline
\caption[Variables of {\tt test\_echam6.sh}]{Variables of {\tt
    test\_echam6.sh} that have to be modified 
  by the user of the test scripts. The variables are listed in the
  order of their appearance in {\tt test\_echam6.sh}. Note that the
  revision specific 
  path of the \echam{} 
  model will be automatically composed as {\tt
    \$\{REF\_DIR\}/\$\{REF\_BRANCH\}\_\$\{REF\_REVISION\}} for the reference
  model and as {\tt \$\{TEST\_DIR\}/\$\{TEST\_BRANCH\}\_\$\{TEST\_REVISION\}}
  for the test model, respectively.}\\\hline\label{tabvar1}
\endfirsthead
\caption[]{{\tt test\_echam6.sh} --- continued}\\\hline
\endhead
\hline\multicolumn{2}{r}{\slshape table continued on next page}\\
\endfoot
\hline %\multicolumn{2}{r}{end of table}
\endlastfoot
Variable & Explanation \\\hline
{\tt SCR\_DIR} & Absolute path to diretory where test scripts are
located. \\
{\tt OUTFILETYPE} & File type of output files. Set to 1 for GRIB
format output files and to 2 for netcdf output files. It is
recommended to test \echam{} with both output formats.\\
{\tt FORTRANCOMPILER} & If a module has to be loaded in order to use
the correct fortran compiler version, give the fortran
compiler module here. \\
{\tt CCOMPILER} & If a module has to be loaded in order to use the
correct C compiler version, give the C compiler module here.\\
{\tt MPI\_MODULE} & If a module has to be loaded in order to use the
Message Passing Interface (MPI) runtime environment, specify the module here.\\
{\tt MPIRUN} & command specification to run a program using MPI.
When running, the script will replace \verb|%n| and \verb|%x| by the number of
processes and the name of the executable, respectively.\\
{\tt TEST\_DIR}, {\tt REF\_DIR} & Absolute base path to directory containing
model versions of test and reference model, respectively. Even if
the model source code is loaded from svn, this directory has to exist.\\
{\tt TEST\_BRANCH}, {\tt REF\_BRANCH} & name of branch of test and reference
model in the revision
control system svn a revision of which has to be tested, respectively. \\
{\tt TEST\_REVISION}, {\tt REF\_REVISION} & revision number of
test and reference model revision, respectively. \\
{\tt TEST\_SVN}, {\tt REF\_SVN} & URL address of test and reference model
branch in svn system, respectively. Can be omitted if model source
code is on local disk. \\
{\tt TEST\_ODIR}, {\tt REF\_ODIR} & Absolute path where test scripts
can open directories for simulation results of test and reference
model, respectively. This directory has to exist.\\
{\tt LCOMP} & {\tt LCOMP=.true.} forces compilation, with {\tt
  LCOMP=.false.} compilation is done only if executable is not
existing.\\
{\tt MODE} & One of {\tt compile}, {\tt single}, {\tt debug}, {\tt parallel},
{\tt nproma}, {\tt rerun}, {\tt update}, {\tt submodeloff}, {\tt
  parallelnproma}, {\tt parallelnpromarerun}, {\tt
  parallelnpromarerunsubmodeloff}, {\tt all} in order to perform the
corresponding tests. \\
\end{longtable}

If some step or test was not successful, more information about the
possible error is given in
the protocol files that are written for each step.
If the model was checked out from the svn system, there is a protocol
file {\tt checkout.log} of the checkout procedure in {\tt
    \$\{REF\_DIR\}/\$\{REF\_BRANCH\}\_\$\{REF\_REVISION\}} for the reference
  model and {\tt
    \$\{TEST\_DIR\}/\$\{TEST\_BRANCH\}\_\$\{TEST\_REVISION\}} for the
  test model, respectively. 
The configure procedure and compilation is protocolled inside the {\tt
  \$\{BRANCH\}} directory of the aforementioned paths in the files
{\tt config.log} and {\tt compile.log}, respectively.
Information about each simulation can be found inside the directories
{\tt \$\{REF\_ODIR\}/0000nrev\$\{REF\_REVISION\}} and {\tt
  \$\{TEST\_ODIR\/0000nrev\$\{TEST\_REVISION\}}  
with {\tt n} being the number of the test case indicated during the test
run procedure on the screen, respectively. In these
directories, the standard and standard error output of the \echam{}
program can be found in the {\tt 0000nrev\$\{REF\_REVISION\}.\{log,err\}}
and the {\tt 0000nrev\$\{TEST\_REVISION\}.\{log,err\}} files,
respectively. The detailed result of the cdo comparison for each
output file is also in these output directories in respective files
{\tt diff$\ast$.dat}. On the screen, only the most important steps and
results are displayed. A certain test is successfully passed if the
comparison for each file
results in the message ``0 of $r$ records differ'' where $r$ is the
number of records.


\subsection[Automatic generation of run scripts for \echambw]{Automatic generation of run scripts for \echam}
\label{runscripts:generation}

The \emph{mkexp} tool allows to automatically generate run scripts for \echam{}
experiments. It uses a set of experiment templates to generate these scripts.

To set up an experiment, you have to write a simple configuration file
containing experiment specific information like an experiment type to choose
the appropriate templates, and possibly model settings that override the
default set of options. All information needed to run \echam{} is then written
to the run scripts and may be adjusted as needed for more specialized
experiments.

By default, templates and examples for AMIP and CMIP5's SSTClim style
experiments are provided, for two spatial resolutions, LR and MR.

\begin{quote}
  \colorbox{dgray}{\parbox{\linewidth}{Some part of the input below is variable
      and marked by \code{\tvar{angle\_brackets}}. Remember to replace these
      markers by actual experiment name, project name, version tag, etc.\ before
      trying any of the examples.}}
\end{quote}

\subsubsection{Re-create a reference experiment on \blizzard}
\label{runscripts:generation:reference_run}

After compiling \echam{} on \blizzard{} you may check the model output
against the reference data provided for the release. To allow direct comparison
of data, it is essential to load the exact compiler version given in section
\ref{seccompiling} before compiling.

\begin{enumerate}

\item 

From the \verb|echam-<version_tag>| directory (see section \ref{seccompiling}),
change into the \code{run} directory, load the Python environment and set up
the reference experiment:
%
\begin{quote}
\begin{verbatim}
cd run
module add PYTHON
../util/mkexp/mkexp examples/amiptest.config
\end{verbatim}
\end{quote}

This creates directories for run scripts and output data, and prints their
names. It also prints the 'data directory' needed in step 4.

\item

Change to the script directory created by step 1, and submit your experiment
\texttt{amiptest} to the execution queue:
%
\begin{quote}
\begin{verbatim}
cd ../experiments/amiptest/scripts
llsubmit amiptest.run_start
\end{verbatim}
\end{quote}

\item

To check if your experiment is running, you may use:
%
\begin{quote}
\begin{verbatim}
llqdetail
\end{verbatim}
\end{quote}

The Status column shows e.g. \verb|I| for waiting (Idle), \verb|R| for Running,
or \verb|NQ| for Not Queued. The latter may happen if a user submits too many
jobs (current limit is 10). If a run stops due to errors, you will get a
notification by email.

\item 

When the experiment has finished, it will disappear from the \verb|llqdetail|
list. Now go to the data directory as printed in step 1:
%
\begin{quote}
\begin{verbatim}
cd <data_directory>
\end{verbatim}
\end{quote}

Check your output files against the reference data listed in
\url{https://code.zmaw.de/projects/echam/wiki/ECHAM6_reference_experiments}
using \verb|cdo diff|.

\end{enumerate}

\subsubsection{Getting started: create experiment setups}

This section gives instructions for three different computer systems that are
used at the \mpimet: \blizzard, \thunder, and \emph{CIS desktops}.

For any system, start from the directory where you have placed
the \echam{} source code - as described in section \ref{seccompiling} - and
change into the \code{run} directory:
\begin{quote}
\begin{verbatim}
cd run
\end{verbatim}
\end{quote}

By convention, experiments are named using a three-letter acronym plus a unique
4-digit experiment number (e.g. \verb|jus0001|). For \mpimet{} users, acronyms
are defined in
\url{https://code.zmaw.de/projects/mpi-intern/wiki/List_of_Experimenter_IDs}.

\paragraph*{Roadmap for \blizzard}
\label{runscripts:generation:blizzard}

\begin{enumerate}

\item

Make sure that the Python environment is loaded:
\begin{quote}
\begin{verbatim}
module add PYTHON
\end{verbatim}
\end{quote}

\item

Create a copy of the \texttt{amiptest.config} example:
%
\begin{quote}
\begin{verbatim}
cp examples/amiptest.config <experiment_name>.config
\end{verbatim}
\end{quote}

\item

Edit this file and complete the configuration as described in
\ref{runscripts:generation:example_configuration}.

\item

Create scripts and experiment directories:
\begin{quote}
\begin{verbatim}
../util/mkexp/mkexp <experiment_name>.config
\end{verbatim}
\end{quote}
This prints the 'script directory' and 'data directory' needed in later steps.

\item

Change to the script directory written by the previous step:
\begin{quote}
\begin{verbatim}
cd <script_directory>
\end{verbatim}
\end{quote}

\item

Submit the first experiment job (see
\ref{runscripts:generation:generated_scripts} for the different options):
%
\begin{quote}
\begin{verbatim}
llsubmit <experiment_name>.run_start ### or <experiment_name>.run_init
\end{verbatim}
\end{quote}

\item 

To check the status of your jobs, use one of
%
\begin{quote}
\begin{verbatim}
llqdetail
llq -u $USER
\end{verbatim}
\end{quote}

\end{enumerate}

\paragraph*{Roadmap for \thunder}
\label{runscripts:generation:thunder}

\begin{enumerate}

\item

Make sure that the Python and MPI environments are loaded:
\begin{quote}
\begin{verbatim}
module add python mvapich2/1.9b-static-intel12
\end{verbatim}
\end{quote}

\item

Create a copy of the \texttt{amiptest.config} example:
%
\begin{quote}
\begin{verbatim}
cp examples/amiptest.config <experiment_name>.config
\end{verbatim}
\end{quote}

\item

Edit this file and complete the configuration as described in
\ref{runscripts:generation:example_configuration}.

\item

Create scripts and experiment directories:
\begin{quote}
\begin{verbatim}
../util/mkexp/mkexp <experiment_name>.config
\end{verbatim}
\end{quote}
This prints the 'script directory' and 'data directory' needed in later steps.

\item

Change to the script directory written by the previous step:
\begin{quote}
\begin{verbatim}
cd <script_directory>
\end{verbatim}
\end{quote}

\item

Submit the first experiment job (see
\ref{runscripts:generation:generated_scripts} for the different options):
%
\begin{quote}
\begin{verbatim}
sbatch <experiment_name>.run_start ### or <experiment_name>.run_init
\end{verbatim}
\end{quote}

\item 

To check the status of your jobs, use:
%
\begin{quote}
\begin{verbatim}
squeue -u $USER ### Add -l for more details
\end{verbatim}
\end{quote}

\end{enumerate}


\paragraph*{Roadmap for \emph{CIS desktops}}
\label{runscripts:generation:desktops}

\begin{enumerate}

\item

Make sure that the Python and MPI environments are loaded:
\begin{quote}
\begin{verbatim}
module add python mpich2
\end{verbatim}
\end{quote}

\item

Create a copy of the \texttt{amiptest.config} example:
%
\begin{quote}
\begin{verbatim}
cp examples/amiptest.config <experiment_name>.config
\end{verbatim}
\end{quote}

\item

Edit this file and complete the configuration as described in
\ref{runscripts:generation:example_configuration}.

\item

Create scripts and experiment directories:
\begin{quote}
\begin{verbatim}
../util/mkexp/mkexp <experiment_name>.config
\end{verbatim}
\end{quote}
This prints the 'script directory' and 'data directory' needed in later steps.

\item

Change to the script directory written by the previous step:
\begin{quote}
\begin{verbatim}
cd <script_directory>
\end{verbatim}
\end{quote}

\item

Run the first experiment job in background (see
\ref{runscripts:generation:generated_scripts} for the different options):
%
\begin{quote}
\begin{verbatim}
( ./<experiment_name>.run_start & ) ### or <experiment_name>.run_init
\end{verbatim}
\end{quote}

\item 

To check the status of your jobs, use:
%
\begin{quote}
\begin{verbatim}
ps -fu $USER
\end{verbatim}
\end{quote}

\end{enumerate}

\subsubsection{Description of generated scripts}
\label{runscripts:generation:generated_scripts}

After running \code{mkexp}, the script directory contains these files:
%
\begin{description}
\item[{\texttt{README}}]\leavevmode\\
contains the experiment description you entered as comment for the config file

\item[\code{\tvar{experiment\_name}.run\_start}
      (submit to start from another experiment)]\leavevmode\\
Restart script. Provides restart files from a previous experiment, and calls
\code{\tvar{experiment\_name}.run} for the first model year.

\item[\code{\tvar{experiment\_name}.run\_init}
      (submit to run from initial conditions)]\leavevmode\\
Initialization script. Performs an initial run for the first model year. Calls
\code{\tvar{experiment\_name}.job*} for post-processing, and
\code{\tvar{experiment\_name}.run} for subsequent model years.

\item[\code{\tvar{experiment\_name}.run}]\leavevmode\\ 
Run script, called by \code{\tvar{experiment\_name}.run\_start} or
\code{\tvar{experiment\_name}.run\_init}. Calls
\code{\tvar{experiment\_name}.job*} for post-processing, and itself for
subsequent model years.

\item[\code{\tvar{experiment\_name}.job1}]\leavevmode\\ 
Packs model restart files into an archive file. Called by
\code{\tvar{experiment\_name}.run} and
\code{\tvar{experiment\_name}.run\_init}.

\item[\code{\tvar{experiment\_name}.job2}]\leavevmode\\ 
Post-processing of model output files. This creates the so-called ATM, BOT, and
LOG files used for \echam{} standard visualization. Called by
\code{\tvar{experiment\_name}.run} and
\code{\tvar{experiment\_name}.run\_init}.

\item[\code{\tvar{experiment\_name}.job3}]\leavevmode\\
Concatenates model output files by year. Files are moved to the data
directory if this is different from the model working directory. Called by
\code{\tvar{experiment\_name}.job2}.

\item[\code{\tvar{experiment\_name}.plot, \tvar{experiment\_name}.plot\_diff}]
\leavevmode\\
Create plots and tables for evaluation of results. See section
\ref{postprocessing:quickplots} for details. These files are not run
automatically.

\end{description}


\subsubsection{Model output and log files}

All jobs redirect their standard output and error to log files, \verb|<experiment_name>_run*_<process_number>.log|, in the script directory.

Output data files are written to the data directory as printed in step 4 above.

\begin{quote}\colorbox{dgray}{\parbox{\linewidth}{ 
  Due to incompatible handling of return codes, \thunder{} complains even
  if the run job completes successfully. The message:

  \begin{quote}{\ttfamily \raggedright \noindent
==========================================================\\
=   BAD TERMINATION OF ONE OF YOUR APPLICATION PROCESSES\\
=   EXIT CODE: 127\\
=   CLEANING UP REMAINING PROCESSES\\
=   YOU CAN IGNORE THE BELOW CLEANUP MESSAGES\\
==========================================================
  }\end{quote}

  may safely be ignored.
}}\end{quote}

\subsubsection{Example configuration details}
\label{runscripts:generation:example_configuration}

The \texttt{examples} sub-directory currently contains only one experiment
configuration, \code{amiptest.config}.
It performs an AMIP type simulation with the LR model for one year, used for
checking the model against reference data (see
\ref{runscripts:generation:reference_run} above). Since \echam{} 6.2, by default
only monthly mean output is created, without the formerly used 6-hourly data.
The selection of variables reflects the standard selection that goes into the
ATM, BOT and LOG files

This configuration contains some basic settings that you will need for most of
your own experiments:

\begin{enumerate}

\item
The header comment text (starting with \texttt{\#}) will be used as experiment
description. When writing your own files, take care to describe important
characteristics of your experiment here.

\item
The name of the configuration file determines the experiment identifier. For
\verb|jus0001.config| the experiment identifier is \verb|jus0001|. Setting
\verb|EXP_ID| overrides the file name:
%
\begin{quote}
\begin{verbatim}
EXP_ID = jus0001 ### ID is 'jus0001', regardless of file name
\end{verbatim}
\end{quote}

Make sure to choose a unique identifier for each experiment.

\item
The experiment type \texttt{EXP\_TYPE} may be set to one of the standard types
which define the AMIP and SSTClimatology atmosphere model experiments, with the
LR or MR model, respectively (see table
\ref{runscripts:generation:type_table}):
%
\begin{quote}
\begin{verbatim}
amip-LR
amip-MR
sstClim-LR
sstClim-MR
\end{verbatim}
\end{quote}

\item
\texttt{ENVIRONMENT} defines the compute host settings. May currently be set to
\texttt{blizzard}, \texttt{thunder}, or \texttt{DEFAULT} (for CIS desktops).

\item 
For running your experiment on \thunder, compute and disk accounting must
be set to your project's name. On \blizzard{} this may also be used; if
unset it defaults to your standard project as set in \verb|$HOME/.acct|:
%
\begin{quote}
\begin{verbatim}
ACCOUNT = <project_name>
\end{verbatim}
\end{quote}

\item
Set the model sub-directory:
%
\begin{quote}
\begin{verbatim}
MODEL_SUBDIR = echam-<version_tag>
\end{verbatim}
\end{quote}

Model source and binaries, experiment scripts, working files, and
post-processed data may reside in different parts of the file system. By
default, in each of these locations a subdirectory \texttt{MODEL\_SUBDIR} is
expected for the model files, or will be created otherwise.

The standard locations depend on your host environment; they're usually one of
\verb|$HOME|, \verb|/work/<project_name>/$USER|,
\verb|/scratch/<user_prefix>/$USER|, or
\verb|/scratch/mpi/<project_name>/$USER|, where \verb|HOME| and \verb|USER|
refer to the UNIX standard environment variables.

For details, refer to section \ref{runscripts:generation:directories}.

\item
The \verb|[namelists]| section contains all namelist settings, with subsections
for each namelist file name, e.g. \verb|[[namelist.echam]]|. Namelist groups
are set with sub-subsections like \verb|[[[runctl]]]|. Use lower case for
namelist groups and their variables. See section \ref{secnamelist} for details.

\end{enumerate}


\subsubsection{Customization of experiment setups}
\label{runscripts:generation:customization}

\paragraph*{Using different directories}
\label{runscripts:generation:directories}

\emph{mkexp} may store model code, scripts, working files, and
output data in different directories. Though they all may be set manually,
most of the time you will only want to change the base directories, e.g. from
\texttt{/work} to \texttt{/scratch} when using \blizzard{} (see
\ref{runscripts:generation:blizzard_dirs}). For this, there is a number of
variables that allows using the same structure within all four directories.
They are pre-set as:
%
\begin{quote}
\begin{verbatim}
MODEL_DIR = $MODEL_ROOT/$MODEL_SUBDIR
SCRIPT_DIR = $SCRIPT_ROOT/$MODEL_SUBDIR/$EXPERIMENTS_SUBDIR/$EXP_ID/
                                                             $SCRIPTS_SUBDIR
DATA_DIR = $DATA_ROOT/$MODEL_SUBDIR/$EXPERIMENTS_SUBDIR/$EXP_ID/$DATA_SUBDIR
WORK_DIR = $WORK_ROOT/$MODEL_SUBDIR/$EXPERIMENTS_SUBDIR/$EXP_ID/$WORK_SUBDIR
\end{verbatim}
\end{quote}

The default settings are:
%
\begin{quote}
\begin{verbatim}
MODEL_ROOT = /scratch/mpi/$ACCOUNT/$USER for thunder,
             $HOME for blizzard and others
SCRIPT_ROOT = $MODEL_ROOT
DATA_ROOT = /work/$ACCOUNT/$USER for blizzard,
            /scratch/mpi/$ACCOUNT/$USER for thunder, and
            $MODEL_ROOT for others
WORK_ROOT = $SCRATCH for blizzard,
            $DATA_ROOT for thunder and others

MODEL_SUBDIR = echam-<version_tag>
EXPERIMENTS_SUBDIR = experiments
WORK_SUBDIR = ### empty string
\end{verbatim}
\end{quote}

Sub-directories may be omitted by setting them to an empty string (see
\texttt{WORK\_SUBDIR}). \texttt{HOME} and \texttt{USER} refer to the UNIX
standard environment variables, and may be used in the configuration file like
any other variable.


\paragraph*{Change global experiment settings}

Besides directory definitions, the top section of a configuration may contain
more pre-defined variables controlling the experiment. The examples below show
their names and respective default values.

\begin{description}
  
\item[\code{ECHAM\_EXE = echam}]\leavevmode\\ 
  Name of model executable

\item[\code{POST\_FILETAG = echamm}]\leavevmode\\ 
  Name of output stream for post-processing

\item[{\parbox[b]{\linewidth}{
  \code{PLOT\_START\_YEAR}\\
  \code{PLOT\_END\_YEAR}
}}]
  Years covered by the plot scripts; values are given as integers

\item[\code{PARENT\_EXP\_ID}]\leavevmode\\ 
  Identifier of experiment from which restart files are taken

\end{description}


\paragraph*{Change job resource settings}

Variables defining job resources go into section:

\begin{quote}
\begin{verbatim}
[jobs]
  [[<job_id>]]
\end{verbatim}
\end{quote}

where \code{\tvar{job\_id}} may be a comma separated list of job identifiers,
e.g. \code{run, run\_start, run\_init}.

\begin{description}

\item[\code{time\_limit}]\leavevmode\\
  Maximum run time for the given jobs; value is given as string (hh:mm:ss)

\item[{\parbox[b]{\linewidth}{
  \code{nodes}\\
  \code{tasks\_per\_node}\\
  \code{threads\_per\_task}
}}]
  Process distribution on the compute host; values are given as integers

\item[\code{tasks}]\leavevmode\\
  Usually the number of processes is computed as
  $\code{nodes}\times\code{tasks\_per\_node}$. For testing reasons or
  over-committing, this number may also be set explicitly.

\end{description}

\paragraph*{Change namelist settings}

Within the \texttt{{[}namelists{]}} section you may modify any namelist
settings that are defined for \echam{}.

For example, to enable writing of 6 hourly output as used for AMIP simulations,
locate or add the \code{namelists/namelist.echam/runctl} section and add the
appropriate settings. In our \code{amiptest.config} example this would look
like:

\begin{quote}
\begin{verbatim}
[namelists]

  [[namelist.echam]]

    [[[runctl]]]
      dt_stop = 1979, 12, 31, 23, 50, 00

      # Additional settings to enable 6 hourly output
      default_output = true
      putdata = 6, hours, first, 0
\end{verbatim}
\end{quote}

Empty lines and text after \code{\#} are treated as a comment and ignored.

To change the parallelization of ECHAM, use:

\begin{quote}
\begin{verbatim}
[[[parctl]]]
  nproca = 16
  nprocb = 8

[[[runctl]]]
  nproma = 72
\end{verbatim}
\end{quote}

Note that in \emph{mkexp} all namelist variables are lower-case. When setting
the variables, quotes around strings and periods around truth values may be
omitted. As in Fortran namelists, truth values may also be set using \texttt{t}
or \texttt{f}. Lists of values use comma as separator.

Thus, to perform an initialisation run with a non-standard output interval and
with explicit stratosphere, use:
%
\begin{quote}
\begin{verbatim}
[[[runctl]]]
  lresume = false
  lmidatm = t
  putdata = 1, days, last, 0
\end{verbatim}
\end{quote}

which will then translate into the Fortran namelist:
%
\begin{quote}
\begin{verbatim}
&runctl
    lresume = .false.
    lmidatm = .true.
    putdata = 1, 'days', 'last', 0
/
\end{verbatim}
\end{quote}

\emph{mkexp} currently does not impose any restrictions on namelist file names
or group names. When you need to use additional namelist files or groups, you
may directly set them in the configuration file, without any further
programming.

Some namelist groups like \verb|mvstreamctl| may be used repeatedly. For each
group, a unique identifier must be added to the group name, separated from
the group name by at least one space character:
%
\begin{quote}
\begin{verbatim}
[[[mvstreamctl spm]]]
  ...

[[[mvstreamctl glm]]]
  ...

[[[mvstreamctl g3bm]]]
  ...
\end{verbatim}
\end{quote}

\paragraph*{Suppress standard namelist groups%
  \label{suppress-standard-namelist-groups}%
}

Sometimes you need to switch off optional namelist groups that are defined by
default. For instance, to remove all \texttt{mvstreamctl} output from
\texttt{amip-LR}, use:
%
\begin{quote}
\begin{verbatim}
[[namelist.echam]]
  remove = mvstreamctl spm, mvstreamctl glm, mvstreamctl g3bm
\end{verbatim}
\end{quote}

or:
%
\begin{quote}
\begin{verbatim}
[[namelist.echam]]
  remove = mvstreamctl *m
\end{verbatim}
\end{quote}

The latter disables \emph{all} namelist groups that begin with 'mvstreamctl '
and end in 'm'. Note that the \texttt{remove} list is applied to \emph{all}
groups in \texttt{namelist.echam}, so use with care.

\begin{quote}\colorbox{dgray}{\parbox{\linewidth}{ 
  Note that when removing all \code{mvstreamctl} output, you need to configure
  postprocessing to use 6h output by adding
  \begin{quote}
    \code{POST\_FILETAG = echam}
  \end{quote}
  to the top section of your configuration. When starting from a restart
  file, you also have to give the correct parent experiment, e.g.::
  \begin{quote}
    \code{PARENT\_EXP\_ID = mbe0495}
  \end{quote}
}}\end{quote}


\paragraph*{Special expressions%
  \label{special-expressions}%
}

\emph{mkexp} supports the usual arithmetic expressions in the config file:
%
\begin{quote}
\begin{verbatim}
hours = 6
seconds = eval($hours * 3600)
\end{verbatim}
\end{quote}

evaluates to
%
\begin{quote}
\begin{verbatim}
hours = 6
seconds = 21600
\end{verbatim}
\end{quote}

Date strings -- as used in many tools, e.g. \verb|cdo| -- may be split into
a date \emph{list} as used in namelists for \echam:
%
\begin{quote}
\begin{verbatim}
date = 2013-12-11 12:34:56
date_list = split_date($date)
\end{verbatim}
\end{quote}

is equivalent to
%
\begin{quote}
\begin{verbatim}
date = 2013-12-11 12:34:56
date_list = 2013, 12, 11, 12, 34, 56
\end{verbatim}
\end{quote}


\subsubsection{Extending the standard configuration}


\paragraph*{Standard experiment types%
  \label{standard-experiment-types}%
}

Standard experiment setups are stored in the \texttt{standard\_experiments}
sub-directory. The experiment type defines a certain \emph{kind} of experiment,
e.g. \code{amip} or \code{sstClim}, and an experiment \emph{quality}, e.g.
\code{LR} or \code{MR}.

All experiments of the same \emph{kind} share one set of template files,
\texttt{kind.*.tmpl}, that define the basic work flow of the experiment.

For each experiment type \emph{kind-quality} there is a file
\texttt{kind-quality.config} with the settings needed for this type of
experiment. There may be different qualities for one experiment kind, allowing
to e.g. run the same experiment in different resolutions. In this case
\texttt{kind-quality1.config} and \texttt{kind-quality2.config} will mainly
differ in the model resolution settings.

The currently supported experiment types are:

\setlength{\DUtablewidth}{\linewidth}
\begin{longtable*}[c]{|p{0.098\DUtablewidth}|p{0.145\DUtablewidth}|p{0.528\DUtablewidth}|}
\hline
\label{runscripts:generation:type_table}
\textbf{%
kind
} & \textbf{%
quality
} & \textbf{%
description
} \\
\hline
\endfirsthead
\hline
\textbf{%
kind
} & \textbf{%
quality
} & \textbf{%
description
} \\
\hline
\endhead
\multicolumn{3}{c}{\hfill ... continued on next page} \\
\endfoot
\endlastfoot
\multirow{3}{0.10\DUtablewidth}{%
amip
} & 
LR
 & 
T63L47, prescribed AMIP SST and sea-ice
 \\
\cline{2-2}
\cline{3-3}
 & 
MR
 & 
T63L95
 \\
\hline
\multirow{2}{0.10\DUtablewidth}{%
sstClim
} & 
LR
 & 
T63L47, prescribed climatological SST
 \\
\cline{2-2}
\cline{3-3}
 & 
MR
 & 
T63L95
 \\
\hline
\end{longtable*}


\paragraph*{Standard host environments%
  \label{standard-host-environments}%
}

Host environments are stored in the \texttt{standard\_environments}
sub-directory. Currently supported hosts are \texttt{blizzard} and
\texttt{thunder}. The \texttt{DEFAULT} settings are for CIS desktops.

The corresponding \texttt{.config} files contain mainly settings for job
queuing, directory structure and MPI environment. The \texttt{.tmpl} files
contain specific code for queuing and submitting.

\subsubsection{Directory structure and file systems on \blizzard}
\label{runscripts:generation:blizzard_dirs}

The supercomputer platform \blizzard{} provides a number
of file systems for different purposes.

\begin{enumerate}

\item

The {\tt \$HOME} file system (located in {\tt /pf}) has a quota per user (8GB)
and provides regular backups. This file system is good for holding the source
code of the echam model and the run scripts that are used to perform a computer
experiment.

\item

The {\tt \$SCRATCH} file system (located in {\tt /scratch}) has very fast I/O
but data will be deleted automatically after a system-defined period (currently
14~days). There is no backup available. This file system is good for the
primary output from a model that will be treated by some postprocessing
immediately after the run. By default, it is not used by the automatically
generated run scripts mentioned above.

\item

The {\tt /work/\{PROJECT\}} file system that also has fast I/O
possibilities. There is no backup available, but data are not deleted
automatically. There is a quota per project and \emph{not} per user. Reasonable
use of this file system requires the coordination of your work with the other
members of this project. Although data are not automatically deleted, it is
\emph{not} an archive. It is meant for frequently accessed data only.

\item

There are two kinds of archive systems: {\tt /hpss/arch} and {\tt /hpss/doku},
both accessible by {\tt pftp}. Be careful to move your results into the archive
as soon as you do not work with them regularly.

\end{enumerate}

\subsection{Runs with parallel I/O}

Runs with parallel I/O need a slight modification of the standard run scripts.
An additional node should be reserved for output, and ${\tt nprocio}$
set to 32 on 
\blizzard, or to 16 on \thunder. On \blizzard, you
need to adjust the total number of tasks accordingly. Thus, the changes
in your \code{.config} file are the following: 

\begin{quote}
\begin{verbatim}
    [[[parctl]]]
      ...
      nprocio = 32 ### 16 on thunder
      iomode = 2 ### or 1 on thunder
...
[jobs]
  [[run]]
    nodes = 5
    tasks = 288 ### nproca * nprocb + nprocio; only for blizzard!
\end{verbatim}
\end{quote}

The order of the variables in the output files can vary from file to file. If
comparisons between outputfiles shall be made, the command
\begin{lstlisting}
cdo sortcode <ifile> <ofile>
\end{lstlisting}
has to be applied to every output file before the ``diff'' command is
used.
