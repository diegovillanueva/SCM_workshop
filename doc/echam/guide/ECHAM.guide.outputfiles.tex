The number and names of outputfiles depend on the model
configuration. Tab.~\ref{taboutputfiles} lists all standard output files
and gives an overview of the kind of variables being in these
files. The names of the outputfiles are composed of the experiment
name {\tt EXPNAME} as it is given by the {\tt out\_expname} variable of the {\tt
  runctl} namelist (see section~\ref{secrunctl}), a date information
{\tt DATE} corresponding to the simulation date at which the output file
was opened and an extension {\tt EXT} that describes the output stream or family
of output streams written to this file. GRIB format output files do
not have further extensions, netcdf format output files have the
additional extension {\tt .nc}. The filename is therefore composed as
{\tt EXPNAME\_DATE\_EXT[.nc]}.

All the variables that are written to an output file are members of
so--called streams, a special data structure that allows for standardized
output. Not all variables of a stream are written to output
files. Detailed information about all streams and variables are
written to the standard error output device when \echam{} is started. 

\setlength{\LTcapwidth}{\textwidth}
\setlength{\LTleft}{0pt}\setlength{\LTright}{0pt}

\begin{longtable}{l@{\extracolsep\fill}p{9cm}}\hline\hline
\caption[Output files]{Output files
  of \echam}\\\hline\label{taboutputfiles}
\endfirsthead
\caption[]{Output files --- continued}\\\hline
\endhead
\hline\multicolumn{2}{r}{\slshape table continued on next page}\\
\endfoot
\hline %\multicolumn{2}{r}{end of table}
\endlastfoot
Extension {\tt EXT} & Content \\\hline
{\tt cfdiag} & diagnostic of 3--dimensional radiation and convective mass flux\\
{\tt co2} & diagnostic of CO$_2$ submodel (carbon cycle)\\
{\tt cosp} & COSP simulator output\\
{\tt echam} & main echam outputfile comprising several echam streams
containing 2-- and 3--d atmospheric gridpoint and spectral variables\\
{\tt forcing} & radiation fluxes and heating rates\\
{\tt surf} & variables from the surface model JSBACH\\
{\tt tdiag} & tendency diagnostic\\
{\tt tracer} & mass mixing ratios of (transported) trace gas species \\
\hline
\end{longtable}

The number of variables in each output stream also depend on the model
configuration. In the case of GRIB output, information about code
numbers and variables can be found in the respective files {\tt
  EXPNAME\_DATE\_EXT.codes}. In the case of netcdf output, the
explanation of the variable can be found inside the netcdf files.
Some of the variables are mean values over the output interval, some
are in spectral space, others in grid point space.
We give tables of outputvariables of the most important output files only.

\subsection{Output file {\tt echam}}

The {\tt echam} output file combines the variables of several output
streams (g3b, gl, and sp) and contains the main prognostic and
diagonstic \echam{} 
output variables describing the dynamic state of the atmosphere.

\setlength{\LTcapwidth}{\textwidth}
\setlength{\LTleft}{0pt}\setlength{\LTright}{0pt}

\begin{longtable}{l@{\extracolsep\fill}rccccp{5cm}}\hline\hline
\caption[Output file {\tt echam}]{Output
    file {\tt echam}. The type of the output fields can be
g (instantaneous grid point variable), \gm{} (mean value over the
output interval of grid point variable), s (spectral space
variable). The dimension is either 2d (variable depends on
longitudes and latitudes only), 3d (variable depends on longitudes,
latitudes, and levels).}\\\hline\label{taboutputecham}
\endfirsthead
\caption[]{Output file {\tt echam} --- continued}\\\hline
\endhead
\hline\multicolumn{7}{r}{\slshape table continued on next page}\\
\endfoot
\hline %\multicolumn{7}{r}{end of table}
\endlastfoot
Name          &    Code & Type & Unit & Dimension & Stream & Explanation \\\hline
{\tt abso4}   &    235  & \gm  & kg/m$^2$&2d   &  g3b   &
anthropogenic sulfur burden\\
{\tt aclcac}  &    223  & \gm  & ---    & 3d   &  g3b   & cloud cover\\
{\tt aclcov}  &    164  & \gm  & ---    & 2d   &  g3b   & total cloud
cover \\
{\tt ahfcon}  &    208  & \gm  & W/m$^2$& 2d   &  g3b   & conductive heat
flux through ice\\
{\tt ahfice}  &    125  & g    & W/m$^2$& 2d   &  g3b   & conductive heat
flux \\
{\tt ahfl}    &    147  & \gm  & W/m$^2$& 2d   &  g3b   & latent heat
flux\\
{\tt ahfliac} &    110  & \gm  & W/m$^2$& 2d   &  g3b   & latent heat
flux over ice\\
{\tt ahfllac} &    112  & \gm  & W/m$^2$& 2d   &  g3b   & latent heat
flux over land\\
{\tt ahflwac} &    111  & \gm & W/m$^2$& 2d   &  g3b   & latent heat
flux over water\\
{\tt ahfres}  &    209  & \gm  & W/m$^2$& 2d   &  g3b   & melting of ice\\
{\tt ahfs}    &    146  & \gm  & W/m$^2$& 2d   &  g3b   & sensible
heat flux\\
{\tt ahfsiac} &    119  & \gm  & W/m$^2$& 2d   &  g3b   & sensible heat
flux over ice\\
{\tt ahfslac} &    121  & \gm  & W/m$^2$& 2d   &  g3b   & sensible heat
flux over land\\
{\tt ahfswac} &    120  & \gm  & W/m$^2$& 2d   &  g3b   & sensible heat
flux over water\\
{\tt albedo}  &    175  & g    & ---    & 2d   &  g3b   & surface albedo\\
{\tt albedo\_nir} & 101 & g    & --- &  2d    &  g3b   & surface
albedo for near infrared radiation range \\
{\tt albedo\_nir\_dif} & 82 & g & --- &  2d    &  g3b   & surface
albedo for near infrared radiation range, diffuse \\
{\tt albedo\_nir\_dir} & 80 & g & --- &  2d    &  g3b   & surface
albedo for near infrared radiation range, direct \\
{\tt albedo\_vis} & 100 & g    & ---  &  2d    &  g3b   & surface
albedo for visible radiation range\\
{\tt albedo\_vis\_dif} & 81 & g & --- &  2d    &  g3b   & surface
albedo for visible radiation range, diffuse \\
{\tt albedo\_vis\_dir} & 79 & g & --- &  2d    &  g3b   & surface
albedo for visible radiation range, direct \\
{\tt alsobs}  & 72      & g    & ---  &  2d    &  g3b   & albedo of
bare ice and snow without melt ponds \\
{\tt alsoi}   & 122     & g    & ---  &  2d    &  g3b   & albedo of
ice \\
{\tt alsol}   & 124     & g    & ---  &  2d    &  g3b   & albedo of
land\\
{\tt alsom}   & 71      & g    & ---  &  2d    &  g3b   & albedo of
melt ponds \\
{\tt alsow}   & 123     & g    & ---  &  2d    &  g3b   & albedo of
water \\
{\tt ameltdepth}&77     & g    & m    &  2d    &  g3b   & total melt
pond depth\\
{\tt ameltfrac}&78      & g    & ---  &  2d    &  g3b   & fractional
area of melt ponds on sea ice\\
{\tt amlcorac}& 89      & \gm  &W/m$^2$& 2d    &  g3b   & mixed layer
flux correction\\
{\tt ao3}     & 236     & g    & ---  &  3d    &  g3b   & mass mixing
ratio of IPCC ozone\\
{\tt apmeb}   & 137     & \gm  &$\rm kg/(m^2s)$&2d& g3b & vertical
integral tendency of water\\
{\tt aprc}    & 143     & \gm  &$\rm kg/(m^2s)$&2d& g3b & convective
precipitation \\
{\tt aprl}    & 142     & \gm  &$\rm kg/(m^2s)$&2d& g3b & large scale
precipitation \\
{\tt aprs}    & 144     & \gm  &$\rm kg/(m^2s)$&2d& g3b & snow fall\\
{\tt aps}     & 134     & g    & Pa   &  2d    & g3b    & surface
pressure \\
{\tt az0i}    & 116     & g    & m    &  2d    & g3b    & roughness
length over ice \\
{\tt az0l}    & 118     & g    & m    &  2d    & g3b    & roughness
length over land \\
{\tt az0w}    & 117     & g    & m    &  2d    & g3b    & roughness
length over water \\

{\tt barefrac}& 70      & g    & ---  &  2d    & g3b    & bare ice
fraction \\
{\tt dew2}    & 168     & g    & K    &   2d   & g3b    & dew point
temperature at 2m above surface\\
{\tt evap}    & 182     & \gm  &$\rm kg/(m^2s)$&2d& g3b & evaporation\\
{\tt evapiac} & 113     & \gm  &$\rm kg/(m^2s)$&2d& g3b & evaporation
over ice\\
{\tt evaplac} & 115     & \gm  &$\rm kg/(m^2s)$&2d& g3b & evaporation
over land\\
{\tt evapwac} & 114     & \gm  &$\rm kg/(m^2s)$&2d& g3b & evaporation
over water\\
{\tt fage}    & 68      & g    & ---  &  2d    & g3b    & aging factor
of snow on ice\\
{\tt friac}   & 97      & \gm  & ---  &  2d    & g3b    & ice cover
fraction of grid box \\
{\tt geosp}   & 129     & g    &$\rm m^2/s^2$&2d& g3b   & surface
geopotential (orography)\\
{\tt glac}    & 232     & g    & ---  &  2d    &  g3b   & fraction of
land covered by glaciers\\
{\tt gld}     & 213     & g    & m    &  2d    &  g3b   & glacier depth\\
{\tt lsp}     & 152     & s    & ---  &  2d    &  sp    & nat.~logarithm of
surface pressure\\
{\tt q}       & 133     & g    & ---  &  3d    &  gl    & specific
humidity \\
{\tt qres}    & 126     & g    & W/m$^2$  & 2d &  g3b   & residual
heat flux for melting sea ice\\
{\tt qvi}     & 230     & \gm  & kg/m$^2$ & 2d &  g3b   & vertically
integrated water vapour \\
{\tt relhum}  & 157     & g    & ---  &  3d    &  g3b   & relative
humidity\\
{\tt sd}      & 155     & s    & 1/s  &  3d    &  sp    & divergence
\\
{\tt seaice}  & 210     & g    & ---  &  2d    &  g3b   & ice cover
(fraction of 1-SLM)\\
{\tt siced}   & 211     & g    & m    &  2d    &  g3b   & ice
depth\\
{\tt sicepdi} & 74      & g    & m    &  2d    &  g3b   & ice
thickness on melt pond\\
{\tt sicepdw} & 73      & g    & m    &  2d    &  g3b   & melt pond
depth on sea ice \\
{\tt sicepres}& 76      & g    & W/m$^2$&2d    &  g3b   & residual
heat flux \\
{\tt slf}     & 194     & g    & ---  &  2d    &  g3b   & sea land fraction\\
{\tt slm}     & 172     & g    & ---  &  2d    &  g3b   & land sea
mask (1=land, 0=sea/lake) \\
{\tt sn}      & 141     & g    & m    &  2d    &  g3b   & snow depth\\
{\tt snc}     & 233     & g    & m    &  2d    &  g3b   & snow depth
at the canopy\\
{\tt sni}     & 214     & g    & m    &  2d    &  g3b   & water
equivalent of snow on ice\\
{\tt snifrac} & 69      & g    & ---  &  2d    & g3b    & fraction of
ice covered with snow \\
{\tt sofliac} & 94      & \gm  & W/m$^2$&2d    &  g3b   & solar
radiation energy flux over ice\\
{\tt sofllac} & 96      & \gm  & W/m$^2$&2d    &  g3b   & solar
radiation energy flux over land\\
{\tt soflwac} & 95      & \gm  & W/m$^2$&2d    &  g3b   & solar
radiation energy flux over water\\
{\tt srad0d}  & 184     & \gm  & W/m$^2$&2d    &  g3b   & incoming solar
radiation energy flux at top of atmosphere\\
{\tt srad0u}  & 203     & \gm  & W/m$^2$&2d    &  g3b   & upward solar
radiation energy flux at top of atmosphere\\
{\tt srad0}   & 178     & \gm  & W/m$^2$&2d    &  g3b   & net solar
radiation energy flux at top of atmosphere\\
{\tt sradl}   & 86      & \gm  & W/m$^2$&2d    &  g3b   & solar
radiation at 200~hPa\\
{\tt srads}   & 176     & \gm  & W/m$^2$&2d    &  g3b   & net solar
radiation energy flux at surface\\
{\tt sradsu}  & 204     & \gm  & W/m$^2$&2d    &  g3b   & upward solar
radiation energy flux at surface\\
{\tt sraf0}   & 187     & \gm  & W/m$^2$&2d    &  g3b   & net solar
radiation energy flux at top of atmosphere for clear sky conditions\\
{\tt srafl}   & 88      & \gm  & W/m$^2$&2d    &  g3b   & solar
radiation energy flux at 200~hPa for clear sky conditions\\
{\tt srafs}   & 185      & \gm  & W/m$^2$&2d    & g3b    & net solar
radiation energy flux at surface for clear sky conditions\\
{\tt st}      & 130     & s    & K    &  3d    &  sp    & temperature \\
{\tt svo}     & 138     & s    & 1/s  &  3d    &  sp    & vorticity
\\
{\tt t2max}   & 201     & g    & K    &  2d    &  g3b   & maximum
temperature at 2m above surface\\
{\tt t2min}   & 202     & g    & K    &  2d    &  g3b   & minimum
temperature at 2m above surface\\
{\tt temp2}   & 167     & g    & K    &  2d    &  g3b   & temperature
at 2m above surface\\
{\tt thvsig}  & 238     & g    & K    &  2d    &  g3b   & standard
deviation of virtual potential temperature at half level klevm1\\
{\tt topmax}  & 217     & g    & Pa   &  2d    &  g3b   & pressure of
height level of convective cloud tops\\
{\tt tpot}    & 239     & g    & K    &  3d    &  g3b   & potential
temperature\\ 
{\tt trad0}   & 179     & \gm  & W/m$^2$&2d    &  g3b   & net thermal
radiation energy flux at top of atmosphere\\
{\tt tradl}   & 85      & \gm  & W/m$^2$&2d    &  g3b   & thermal
radiation energy flux at 200~hPa\\
{\tt trads}   & 177     & \gm  & W/m$^2$&2d    &  g3b   & net thermal
radiation energy flux at surface\\
{\tt tradsu}  & 205     & \gm  & W/m$^2$&2d    &  g3b   & upward thermal
radiation energy flux at surface\\
{\tt traf0}   & 188     & \gm  & W/m$^2$&2d    &  g3b   & net thermal
radiation energy flux at top of atmosphere for clear sky conditions\\
{\tt trafl}   & 87      & \gm  & W/m$^2$&2d    &  g3b   & thermal
radiation energy flux at 200~hPa for clear sky conditions\\
{\tt trafs}   & 186      & \gm  & W/m$^2$&2d    & g3b    & thermal
radiation energy flux at surface for clear sky conditions\\
{\tt trfliac} & 91      & \gm  & W/m$^2$&2d    &  g3b   & thermal
radiation energy flux over ice\\
{\tt trfllac} & 93      & \gm  & W/m$^2$&2d    &  g3b   & thermal
radiation energy flux over land\\
{\tt trflwac} & 92      & \gm  & W/m$^2$&2d    &  g3b   & thermal
radiation energy flux over water\\
{\tt tropo}   & 237     & g    & Pa   &  2d    &  g3b   & pressure of
height level where tropopause is located according to WMO definition\\
{\tt tsi}     & 102     & g    & K    &  2d    &  g3b   & surface
temperature of ice\\
{\tt tsicepdi}& 75      & g    & K    &  2d    &  g3b   & ice
temperature on frozen melt pond\\
{\tt tslm1}   & 139     & g    & K    &  2d    &  g3b   & surface
temperature of land\\
{\tt tsurf}   & 169     & \gm  & K    &  2d    &  g3b   & surface temperature\\
{\tt tsw}     & 103     & g    & K    &  2d    &  g3b   & surface
temperature of water\\
{\tt u10}     & 165     & g    & m/s  &  2d    &  g3b   & zonal wind
velocity at 10m above surface \\
{\tt ustr}    & 180     & \gm  & Pa   &  2d    &  g3b   & zonal wind
stress \\
{\tt ustri}   & 104     & g    & Pa   &  2d    &  g3b   & zonal wind
stress over ice\\
{\tt ustrl}   & 108     & g    & Pa   &  2d    &  g3b   & zonal wind
stress over land\\
{\tt ustrw}   & 106     & g    & Pa   &  2d    &  g3b   & zonal wind
stress over water\\
{\tt v10}     & 166     & g    & m/s  &  2d    &  g3b   & meridional wind
velocity at 10m above surface \\
{\tt vdis}    & 145     & \gm  & W/m$^2$&2d    &  g3b   & boundary
layer dissipation\\
{\tt vdisgw}  & 197     & g    & W/m$^2$&2d    &  g3b   & gravity dissipation\\
{\tt vstr}    & 181     & \gm  & Pa   &  2d    &  g3b   & meridional wind
stress \\
{\tt vstri}   & 105     & g    & Pa   &  2d    &  g3b   & meridional wind
stress over ice\\
{\tt vstrl}   & 109     & g    & Pa   &  2d    &  g3b   & meridional wind
stress over land\\
{\tt vstrw}   & 107     & g    & Pa   &  2d    &  g3b   & meridional wind
stress over water\\
{\tt wimax}   & 216     & g    & m/s  &  2d    &  g3b   & maximum wind
speed at 10m above surface\\
{\tt wind10}  & 171     & \gm    & m/s  &  2d    &  g3b   & wind
velocity at 10m above surface\\
{\tt wl}      & 193     & g    & m    &  2d    &  g3b   & skin
reservoir content\\
{\tt ws}      & 140     & g    & m    &  2d    &  g3b   & soil wetness\\
{\tt wsmx}    & 229     & g    & m    &  2d    &  g3b   & field
capacity of soil\\ 
{\tt xi}      & 154     & g    & ---  &  3d    &  gl    & fractional
cloud ice \\
{\tt xivi}    & 150     & \gm  & kg/m$^2$&2d   &  g3b   & vertically
integrated cloud ice\\
{\tt xl}      & 153     & g    & ---  &  3d    &  gl    & fractional
cloud water \\
{\tt xlvi}    & 231     & \gm  & kg/m$^2$&2d   &  g3b   & vertically
integrated cloud water \\
\hline
\end{longtable}

\subsection{Output file forcing}

The forcing output file contains the instantaneous radiative aerosol
forcing if it was required by the setting of the corresponding
namelist parameters (see also Appendix~\ref{cr20100510}). In the table
of the output variables, we denote 
the net short wave radiation flux under clear sky
conditions by $\ftswc$ at the top of any model layer and by $\fbswc$ at
the bottom of this layer. Similarly, we symbolize the net short wave
radiation flux under all sky condition at the top of any model layer
by $\ftswa$ and by $\fbswa$ at its bottom. The corresponding quantities
for thermal radiation are denoted by $\ftlwc$, $\fblwc$, $\ftlwa$, and
$\fblwa$, respectively. A superscript 0 is added if these quantities
are meant for an atmosphere free of aerosols:
$\ftswco$, $\fbswco$, $\ftswao$, $\fbswao$, 
$\ftlwco$, $\fblwco$, $\ftlwao$, $\fblwao$.
With a certain conversion factor $c_{\rm h}$, the heating rates with
and without aerosols can be obtained from the radiation fluxes. 
The subscript sw indicates
quantities calculated for the solar radiation and lw indicates
quantities calculated for the thermal radiation range:

\begin{eqnarray*}
T'_{\rm sw}:= (\ftswa-\fbswa)c_{\rm h} \\
T'_{\rm lw}:=(\ftlwa-\fblwa)c_{\rm h} \\
{T'}^0_{\rm sw}:= (\ftswao-\fbswao)c_{\rm h} \\
{T'}^0_{\rm lw}:=(\ftlwao-\fblwao)c_{\rm h} \\
\end{eqnarray*}     

From these quantities, we obtain the heating rate forcing or heating
rate anomalies $\Delta
T'_{\rm sw}$ and $\Delta T'_{\rm lw}$ for solar and thermal radiation:

\begin{eqnarray*}
\Delta T'_{\rm sw} & := & T'_{\rm sw}-{T'}^0_{\rm sw} \\
\Delta T'_{\rm lw} & := & T'_{\rm lw}-{T'}^0_{\rm lw}
\end{eqnarray*}


\setlength{\LTcapwidth}{\textwidth}
\setlength{\LTleft}{0pt}\setlength{\LTright}{0pt}

\begin{longtable}{l@{\extracolsep\fill}rccccp{5cm}}\hline\hline
\caption[Output file {\tt forcing}]{Output
    file {\tt forcing}. The type of the output fields can be
g (instantaneous grid point variable), \gm{} (mean value over the
output interval of grid point variable), s (spectral space
variable). The dimension is either 2d (variable depends on
longitudes and latitudes only), 3d (variable depends on longitudes,
latitudes, and levels).}\\\hline\label{taboutputforcing}
\endfirsthead
\caption[]{Output file {\tt forcing} --- continued}\\\hline
\endhead
\hline\multicolumn{7}{r}{\slshape table continued on next page}\\
\endfoot
\hline %\multicolumn{7}{r}{end of table}
\endlastfoot
Name          &    Code & Type & Unit & Dimension & Stream & Explanation \\\hline
{\tt aps} & \multicolumn{6}{c}{see Tab.~\ref{taboutputecham}}\\
{\tt d\_aflx\_lw} & 25 & \gm  & W/m$^2$& 3d   & forcing&
$\ftlwa-\ftlwao$\\
{\tt d\_aflx\_lwc} & 26 & \gm  & W/m$^2$& 3d   & forcing&
$\ftlwc-\ftlwco$\\
{\tt d\_aflx\_sw} & 15 & \gm  & W/m$^2$& 3d   & forcing&
$\ftswa-\ftswao$\\
{\tt d\_aflx\_swc} & 16 & \gm  & W/m$^2$& 3d   & forcing&
$\ftswc-\ftswco$\\
{\tt FLW\_CLEAR\_SUR}&23& \gm  & W/m$^2$& 2d   & forcing&
$\fblwc-\fblwco$ at the surface\\
{\tt FLW\_CLEAR\_TOP}&21& \gm  & W/m$^2$& 2d   & forcing&
$\ftlwc-\ftlwco$ at the top of the atmosphere\\
{\tt FLW\_TOTAL\_SUR}&23& \gm  & W/m$^2$& 2d   & forcing&
$\fblwa-\fblwao$ at the surface\\
{\tt FLW\_TOTAL\_TOP}&22& \gm  & W/m$^2$& 2d   & forcing&
$\ftlwa-\ftlwao$ at the top of the atmosphere\\
{\tt FSW\_CLEAR\_SUR}&13& \gm  & W/m$^2$& 2d   & forcing&
  $\fbswc-\fbswco$ at the surface \\
{\tt FSW\_CLEAR\_TOP}&11& \gm  & W/m$^2$& 2d   & forcing&
  $\ftswc-\ftswco$ at the top of the atmosphere\\
{\tt FSW\_TOTAL\_SUR}&14& \gm  & W/m$^2$& 2d   & forcing&
$\fbswa-\fbswao$ at the surface\\
{\tt FSW\_TOTAL\_TOP}&12& \gm  & W/m$^2$& 2d   & forcing&
  $\ftswa-\ftswao$ at the top of the atmosphere\\
{\tt gboxarea} & \multicolumn{6}{c}{see Tab.~\ref{taboutputecham}}\\
{\tt geosp} & \multicolumn{6}{c}{see Tab.~\ref{taboutputecham}}\\
{\tt lsp} & \multicolumn{6}{c}{see Tab.~\ref{taboutputecham}}\\
{\tt netht\_lw}&      27& \gm  & K/d    & 3d   & forcing& $\Delta
T'_{\rm lw}$\\
{\tt netht\_sw}&      17& \gm  & K/d    & 3d   & forcing& $\Delta
T'_{\rm sw}$\\
\end{longtable}

\subsection{Output files {\tt station}}\label{secdiagoutput}

The output of the CFMIP2 station diagnostic is written to a separate
file for each station (site). The outputfiles have names {\tt
  EXPNAME\_DATE\_cfSites{\it xxxx}.nc} where {\it xxxx} is the
four--digit number of the respective site in the table of K.~Tayler as
they are defined in the file {\tt pointlocations.txt} (see
Tab.~\ref{tabdiaginput}). 

\setlength{\LTcapwidth}{\textwidth}
\setlength{\LTleft}{0pt}\setlength{\LTright}{0pt}

\begin{longtable}{l@{\extracolsep\fill}ccccp{4cm}}\hline\hline
\caption[CFMIP2 output]{Output file for each of the CFMIP2
  sites. Instantaneous variables are of type~g, variables averaged over
  time are of type~\gm. Surface variables are marked by 1ds, variables
  of which the 
  average of the grid box is given are marked as 1d, column
  variables are marked as 2d in the ``dimension'' column. The entry in
  the ``stream''
  column gives information about the internal \echam{} stream from
  which
  the corresponding variable was collected. The
  original name of this variable is given in
  parenthesis.}\\\hline\label{taboutputcfmip}   
\endfirsthead
\caption[]{CFMIP2 output --- continued}\\\hline
\endhead
\hline\multicolumn{6}{r}{\slshape table continued on next page}\\
\endfoot
\hline %\multicolumn{7}{r}{end of table}
\endlastfoot
Name          &  Type & Unit & Dimension & Stream & Explanation \\\hline
{\tt aclcov}  &  \gm  &  --  & 1ds       & g3b ({\tt aclcov}) & total
cloud cover \\
{\tt aprl}    &   \gm &  $\rm kg/(m^2s)$ & 1ds & g3b ({\tt aprl}) &
large scale precipitation \\
{\tt cct} & g & Pa & 1d & g3b ({\tt topmax}) & pressure of altitude
level of convective cloud tops \\
{\tt cl} & g & -- & 2d & g3b ({\tt aclc}) & cloud cover \\
{\tt cli} & g & -- & 2d & gl ({\tt xi}) & fractional cloud ice \\
{\tt clivi}   &   \gm & $\rm kg/m^2$ & 1ds & g3b ({\tt xivi}) &
vertically integrated cloud ice \\
{\tt clw} & g & -- & 2d & gl ({\tt xl}) & fractional cloud water \\
{\tt clwvi}   &   \gm & $\rm kg/m^2$ & 1ds & g3b ({\tt xlvi}) &
vertically integrated cloud water \\
{\tt dqdt\_cloud} & g & 1/day & 2d & tdiag ({\tt dqdt\_cloud}) &
tendency of specific humidity due to cloud scheme\\
{\tt dqdt\_cucall} & g & 1/day & 2d & tdiag ({\tt dqdt\_cucall}) &
tendency of specific humidity due to convection\\
{\tt dqdt\_vdiff} & g & 1/day & 2d & tdiag ({\tt dqdt\_vdiff}) &
tendency of specific humidity due to vertical diffusion\\
{\tt dtdt\_cucall} & g & K/day & 2d & tdiag ({\tt dtdt\_cucall}) &
tendency of temperature due to convection\\
{\tt dtdt\_cloud} & g & K/day & 2d & tdiag ({\tt dtdt\_cloud}) &
tendency of temperature due to cloud scheme\\
{\tt dtdt\_hines} & g & K/day & 2d & tdiag ({\tt dtdt\_hines}) &
tendency of temperature due to gravity waves (Hines parametrization)\\
{\tt dtdt\_rheat\_lw} & g & K/day & 2d & tdiag ({\tt dtdt\_rheat\_lw}) &
tendency of temperature due to radiative heating (thermal wavelength bands)\\
{\tt dtdt\_rheat\_sw} & g & K/day & 2d & tdiag ({\tt dtdt\_rheat\_sw}) &
tendency of temperature due to radiative heating (solar wavelength bands)\\
{\tt dtdt\_sso} & g & K/day & 2d & tdiag ({\tt dtdt\_sso}) &
tendency of temperature due to orographic gravity waves\\
{\tt dtdt\_vdiff} & g & K/day & 2d & tdiag ({\tt dtdt\_vdiff}) &
tendency of temperature due to vertical diffusion\\
{\tt evspsbl} &   \gm &  $\rm kg/(m^2s)$ & 1ds & g3b ({\tt evap}) &
evaporation from the surface\\
{\tt geosp} & g & $\rm m^2/s^2$ & 1ds & g3b ({\tt geosp}) & surface
geopotential (orography) \\
{\tt hur} & g & -- & 2d & g3b ({\tt relhum}) & relative humidity\\
{\tt hus}  & g & -- & 2d & gl ({\tt q}) & specific humidity \\
{\tt mc} & g & $\rm kg/(m^2s)$ & 2d & cfdiag ({\tt imc})& net upward
convective mass flux \\ 
{\tt mhfls}   &   \gm &  $\rm W/m^2$&            1ds & g3b ({\tt ahfl}) &
latent heat flux \\
{\tt mhfss}   &   \gm &  $\rm W/m^2$&            1ds & g3b ({\tt ahfs}) &
sensible heat flux \\
{\tt mrlus}   &   \gm &  $\rm W/m^2$&            1ds & g3b ({\tt tradsu}) &
upward thermal radiation energy flux at surface \\
{\tt mrlut}   &   \gm &  $\rm W/m^2$&            1d & g3b ({\tt trad0}) &
net thermal radiation energy flux at top of atmosphere \\
{\tt mrlutcs}   &   \gm &  $\rm W/m^2$&            1d & g3b ({\tt traf0}) &
net thermal radiation energy flux at top of atmosphere for clear sky
conditions\\ 
{\tt mrsus}   &   \gm &  $\rm W/m^2$&            1ds & g3b ({\tt sradsu}) &
upward solar radiation energy flux at surface \\
{\tt mrsut}   &   \gm &  $\rm W/m^2$&            1d & g3b ({\tt srad0u}) &
upward solar radiation energy flux at top of atmosphere \\
{\tt prc}     &   \gm &  $\rm kg/(m^2s)$ & 1ds & g3b ({\tt aprc}) &
convective precipitation \\
{\tt prsn}    &   \gm &  $\rm kg/(m^2s)$ & 1ds & g3b ({\tt aprs}) &
snow fall \\
{\tt prw}     &   \gm & $\rm kg/m^2$ & 1ds & g3b ({\tt qvi}) &
vertically integrated water vapour \\
{\tt ps}      &   g   &  Pa  &     1ds   &   g3b ({\tt aps}) & surface
pressure\\
{\tt rld} & g & $\rm W/m^2$& 2d & cfdiag ({\tt irld}) & downward
energy flux of radiation integrated over thermal wavelength bands \\
{\tt rldcs} & g & $\rm W/m^2$& 2d & cfdiag ({\tt irldcs}) & downward
energy flux of radiation integrated over thermal wavelength bands
under clear sky conditions\\
{\tt rlu} & g &$\rm W/m^2$& 2d & cfdiag ({\tt irlu}) & upward energy flux of
radiation integrated over thermal wavelength bands \\
{\tt rlucs} & g &$\rm W/m^2$& 2d & cfdiag ({\tt irlucs}) & upward
energy flux of 
radiation integrated over thermal wavelength bands under clear sky conditions\\
{\tt rsd} & g & $\rm W/m^2$& 2d & cfdiag ({\tt irsd}) & downward
energy flux of radiation integrated over solar wavelength bands \\
{\tt rsdcs} & g & $\rm W/m^2$& 2d & cfdiag ({\tt irsdcs}) & downward
energy flux of radiation integrated over solar wavelength bands under
clear sky conditions\\
{\tt rsdt}   &   \gm & $\rm W/m^2$&     1d   &   g3b ({\tt srad0d}) &
incoming solar radiation energy flux at top of atmosphere \\
{\tt rsu} & g &$\rm W/m^2$& 2d & cfdiag ({\tt irsu}) & upward energy flux of
radiation integrated over solar wavelength bands \\
{\tt rsucs} & g &$\rm W/m^2$& 2d & cfdiag ({\tt irsucs}) & upward
energy flux of 
radiation integrated over solar wavelength bands under clear sky conditions\\
{\tt sfcWind} &   \gm & m/s  &     1d    &   g3b ({\tt wind10}) & 10
meter wind\\
{\tt slm}  & g & -- & 1ds & g3b ({\tt slm}) & land sea mask (1=land,
0=sea/lake) \\
{\tt srads}   &   \gm & $\rm W/m^2$&     1ds   &   g3b ({\tt srads}) & net
solar radiation energy flux at surface \\
{\tt srafs}   &   \gm & $\rm W/m^2$&     1ds   &   g3b ({\tt srafs}) & net
solar radiation energy flux at surface for clear sky conditions \\
{\tt sraf0}   &   \gm & $\rm W/m^2$&     1d   &   g3b ({\tt sraf0}) & net
solar radiation energy flux at top of atmosphere for clear sky conditions \\
{\tt ta}      &   g   &  K   &     2d    &   g1a ({\tt tm1})  & temperature
at time step $t-\Delta t$ (not time filtered?) \\
{\tt tas}     &   g   &  K   &     1d    &   g3b ({\tt temp2})  & temperature 2m
above the surface \\
{\tt tauu}    &   \gm &  m/s &     1d    &   g3b ({\tt ustr})  & zonal
wind stress\\
{\tt tauv}    &   \gm &  m/s &     1d    &   g3b ({\tt vstr})  &
meridional wind stress\\
{\tt trads}   &   \gm &  $\rm W/m^2$&    1ds   &   g3b ({\tt trads}) & net
thermal radiation energy flux at surface\\
{\tt trafs}   &   \gm &  $\rm W/m^2$&    1ds   &   g3b ({\tt trafs}) & net
thermal radiation energy flux at surface for clear sky conditions\\
{\tt ts}      &   \gm &  K   &     1ds   &   g3b ({\tt tsurf}) & surface
  temperature \\
{\tt ua}     &   g   & m/s  &     2d    &   g2a ({\tt um1}) & 
zonal wind velocity at time step $t-\Delta t$ (not time filtered?);
caution: this variable is multiplied by the cosine of the latitudes at
some points in \echam, but not here\\
{\tt uas}     &   g   & m/s  &     1d    &   g3b ({\tt u10}) & zonal
wind velocity 10m above the surface \\
{\tt va}     &   g   & m/s  &     2d    &   g2a ({\tt vm1}) & 
meridional wind velocity at time step $t-\Delta t$ (not time filtered?);
caution: this variable is multiplied by the cosine of the latitudes at
some points in \echam, but not here\\
{\tt vas}     &   g   & m/s  &     1d    &   g3b ({\tt v10}) &
meridional wind velocity 10m above the surface \\
{\tt wap}     &   g   & Pa/s & 2d & -- & vertical velocity $\omega$ \\
{\tt zg}      &   g  & $\rm m^2/s^2$ & 2d & -- & geopotential over ground \\
\end{longtable}

\subsection{Output file tdiag}

Wind, temperature, and moisture tendencies due to
various processes are collected in this output file. All the
tendencies are instantaneous 
values the mean values of which may be calculated during a model run
using the mean value stream. The actual content of the tdiag output
file depends on the exact choice of output variables in the {\tt
  tdiagctl} namelist (see Sec.~\ref{sectdiagctl}).

\setlength{\LTcapwidth}{\textwidth}
\setlength{\LTleft}{0pt}\setlength{\LTright}{0pt}

\begin{longtable}{l@{\extracolsep\fill}rccccp{5cm}}\hline\hline
\caption[Output file {\tt tdiag}]{Output file {\tt
    tdiag}. The type of the output fields can be
g (instantaneous grid point variable), \gm{} (mean value over the
output interval of grid point variable), s (spectral space
variable). The dimension is either 2d (variable depends on
longitudes and latitudes only), 3d (variable depends on longitudes,
latitudes, and levels).}\\\hline\label{taboutputtdiag} 
\endfirsthead
\caption[]{Output file {\tt tdiag} --- continued}\\\hline
\endhead
\hline\multicolumn{7}{r}{\slshape table continued on next page}\\
\endfoot
\hline %\multicolumn{7}{r}{end of table}
\endlastfoot
Name          &    Code & Type & Unit & Dimension & Stream & Explanation
\\\hline
{\tt aps} & \multicolumn{6}{c}{see Tab.~\ref{taboutputecham}}\\
{\tt dqdt\_cloud} &36   & g    & 1/d  & 3d     & tdiag  & $dq/dt$ due
to processes computed by the subroutine {\tt cloud}\\
{\tt dqdt\_cucall}&35   & g    & 1/d  & 3d     & tdiag  & $dq/dt$ due
to processes computed by the subroutine {\tt cucall} (convective clouds)\\
{\tt dqdt\_vdiff} &31   & g    & 1/d  & 3d     & tdiag  & $dq/dt$ due
to processes computed by the subroutine {\tt vdiff} (vertical diffusion)\\
{\tt dtdt\_cloud} &6    & g    & K/d  & 3d     & tdiag  & $dT/dt$ due
to processes computed by the subroutine {\tt cloud}\\
{\tt dtdt\_cucall}&5    & g    & K/d  & 3d     & tdiag  & $dT/dt$ due
to processes computed by the subroutine {\tt cucall} (convective clouds)\\
{\tt dtdt\_hines} &3    & g    & K/d  & 3d     & tdiag  & $dT/dt$ due
to processes computed by the Hines gravity wave parametrization\\
{\tt dtdt\_rheat\_lw}&72& g    & K/d  & 3d     & tdiag  & $dT/dt$ due
to radiative heating caused by radiation in the thermal spectral range\\
{\tt dtdt\_rheat\_sw}&62& g    & K/d  & 3d     & tdiag  & $dT/dt$ due
to radiative heating caused by radiation in the solar spectral range\\
{\tt dtdt\_sso}   &4    & g    & K/d  & 3d     & tdiag  & $dT/dt$ due
to gravity wave drag\\
{\tt dtdt\_vdiff} &1    & g    & K/d  & 3d     & tdiag  & $dT/dt$ due
to processes computed by the subroutine {\tt vdiff} (vertical diffusion)\\
{\tt dudt\_cucall}&15   & g    & m/s/d  & 3d     & tdiag  & $du/dt$
(zonal wind component) due to processes computed by the subroutine
{\tt cucall} (convective clouds)\\ 
{\tt dudt\_hines} &13   & g    & m/s/d  & 3d     & tdiag  & $du/dt$
(zonal wind component) due to processes computed by the Hines gravity
wave parametrization\\ 
{\tt dudt\_sso}   &14   & g    & m/s/d  & 3d     & tdiag  & $du/dt$
(zonal wind component) due to gravity wave drag\\
{\tt dudt\_vdiff} &11   & g    & m/s/d  & 3d     & tdiag  & $du/dt$
(zonal wind component) due
to processes computed by the subroutine {\tt vdiff} (vertical diffusion)\\
{\tt dvdt\_cucall}&25   & g    & m/s/d  & 3d     & tdiag  & $dv/dt$
(meridional wind component) due to processes computed by the subroutine
{\tt cucall} (convective clouds)\\
{\tt dvdt\_hines} &23   & g    & m/s/d  & 3d     & tdiag  & $dv/dt$
(zonal wind component) due to processes computed by the Hines gravity
wave parametrization\\
{\tt dvdt\_sso}   &24   & g    & m/s/d  & 3d     & tdiag  & $dv/dt$
(zonal wind component) due to gravity wave drag\\
{\tt dvdt\_vdiff} &21   & g    & m/s/d  & 3d     & tdiag  & $du/dt$
(zonal wind component) due
to processes computed by the subroutine {\tt vdiff} (vertical diffusion)\\
{\tt dxidt\_cloud} &56   & g    & 1/d  & 3d     & tdiag  & $dx_{\rm
  i}/dt$ (cloud water ice) due to processes computed by the subroutine
{\tt cloud}\\
{\tt dxidt\_vdiff} &51   & g    & 1/d  & 3d     & tdiag  & $dx_{\rm
  i}/dt$ (cloud water ice) due to processes computed by the subroutine
{\tt vdiff} (vertical diffusion)\\
{\tt dxldt\_cloud} &46   & g    & 1/d  & 3d     & tdiag  & $dx_{\rm
  l}/dt$ (cloud water) due to processes computed by the subroutine
{\tt cloud}\\
{\tt dxldt\_vdiff} &41   & g    & 1/d  & 3d     & tdiag  & $dx_{\rm
  l}/dt$ (cloud water) due to processes computed by the subroutine
{\tt vdiff} (vertical diffusion)\\
{\tt gboxarea} & \multicolumn{6}{c}{see Tab.~\ref{taboutputecham}}\\
{\tt geosp} & \multicolumn{6}{c}{see Tab.~\ref{taboutputecham}}\\
{\tt lsp} & \multicolumn{6}{c}{see Tab.~\ref{taboutputecham}}\\
{\tt st} & \multicolumn{6}{c}{see Tab.~\ref{taboutputecham}}\\
{\tt tm1} & \multicolumn{6}{c}{see Tab.~\ref{taboutputecham}}\\
\end{longtable}