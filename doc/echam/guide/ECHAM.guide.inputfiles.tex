This section provides a brief description of the input files but does
not describe the input data itself. Such a description can be found in
the scientific part of the documentation.\newline
All input files are stored in the
directory \newline 
\ini \newline
and its subdirectories for the atmospheric part and in the directory
\newline
{\tt /pool/data/JSBACH}/ \newline
for the land--surface model.
The input data of the atmospheric part are released in various
versions. The version {\tt r0001} contains all data needed for the
basic atmosphere--only experiments conducted for CMIP5: amip--LR,
amip--MR, sstClim--LR, sstClim--MR. 
Transient input data files are provided for the maximum possible time
period. Some data depend on the scenario. In this version, the rcp26,
rcp45, and rcp85 scenarios are taken into account.

In \ini, you find the resolution independent data. Furthermore,
it contains
directories {\tt \{RES\}}
where {\tt \{RES\}} has to be replaced by one of the spectral model
resolutions T31, T63, T127, and T255, respectively providing
resolution dependent input files. Similarly, the resolution dependent
land--surface model data are stored in subdirectories {\tt T31}, {\tt
  T63}, etc. of the {\tt /pool/data/JSBACH} directory.
In the following, the vertical resolution
will be denoted by {\tt \{LEV\}} which represents the number of 
vertical $\sigma$--levels preceeded by a capital {\tt L}. 
The most common model resolutions are
T63L47 and T63L95. Currently, \echam{} is tuned for the resolutions
T63L47, T63L95, T127L95 only.
Other resolutions may require a new tuning of the model in order to
adjust the parameters of certain equations to the particular model resolution.
Some of the input data contain information about the land--sea
distribution and therefore are provided for various ocean resolutions
even if the model is not coupled to an interactive ocean. The ocean
resolution will be symbolized by {\tt \{OCR\}}. Currently, the GR15,
GR30, and
TP04 ocean resolutions are considered.

There are three kinds of input data: initial conditions, boundary
conditions, and data of model parameters. The boundary conditions can
be either ``transient boundary 
conditions'' depending on the actual year or ``climatological boundary
conditions'' which do not depend on the year but may contain a
seasonal cycle.
The files containing the initial conditions are listed in 
Tab.~\ref{tabechami}.

\setlength{\LTcapwidth}{\textwidth}
\setlength{\LTleft}{0pt}\setlength{\LTright}{0pt}

\begin{longtable}{l@{\extracolsep\fill}lp{6cm}}\hline\hline
\caption[Initial conditions]{Initial conditions for
  \echam{}}\\\hline\label{tabechami}  
\endfirsthead
\caption[]{Initial conditions for \echam{} 
 continued}\\\hline 
\endhead
\hline\multicolumn{3}{r}{\slshape table continued on next page}\\
\endfoot
\hline %\multicolumn{3}{r}{end of table}
\endlastfoot

\multicolumn{3}{c}{Resolution dependent \echam{} initial data in
  \ini{\tt \{RES\}}} \\\hline
Link target      & Link name & Explanation\\\hline
{\tt\{RES\}\{LEV\}\_jan\_spec.nc}& 
{\tt unit.23}\index{input files!unit.23}
 & Variables
describing the vertical $\sigma$--coordinates, spectral fields like
divergence, vorticity etc.~serving to start the model from some
initial values. These values are very rough estimates only and do not
describe any dynamic state of the atmosphere that occurs with high
probability! 
\\
{\tt\{RES\}\{OCR\}\_jan\_surf.nc}      & 
{\tt unit.24}\index{input files!unit.24}
 & Surface fields like
land sea mask, glacier mask etc.~for a start of the model from initial
values. \\ \hline
\multicolumn{3}{c}{Resolution independent \echam{} initial data in
  \ini} \\\hline
{\tt hdstart.nc} & {hdstart.nc}\index{input files!hdstart.nc} & Initial data for hydrological
discharge model. \\
\hline
\end{longtable}

The climatological boundary condition files are listed in 
Tab.~\ref{tabechamc}. Sea surface temperature and sea ice cover
climatologies for \echam{} 
are based on 500 year--climatologies of our coupled control simulations and
are available for the T63 resolutions only. 
Furthermore, some of the data are formally read by \echam{} but not
used: The leaf area index, vegetation ratio, and albedo e.g.~are calculated
by the surface model JSBACH and it is impossible to use climatological
values read from files. Actually, JSBACH reads these quantities again,
but discards them also, even if dynamic vegetation is switched off:
This just means that the geographical distribution of vegetation types 
is fixed in time, but the leaf area index changes with season
and soil moisture and consequently also the albedo varies with time
according to the vegetation model used in JSBACH, only the vegetation
ratio remains fixed at the value read from file.

The input data for the hydrological discharge model (see
Tab.~\ref{tabechami} and Tab.~\ref{tabechamc}) are not entirely
resolution independent, but the current data can be used for a wide
range of resolutions.

\begin{longtable}{p{5cm}@{\extracolsep\fill}lp{6cm}}\hline\hline\caption
[Climatological boundary conditions]
{Climatological boundary conditions for \echam{}. Some of the
  climatological boundary conditions have to be linked to year
  dependent files. The year is symbolized by {\tt
    yyyy}.}\\\hline\label{tabechamc}  
\endfirsthead
\caption[]{Climatological boundary conditions for \echam{} 
 continued}\\\hline 
\endhead
\hline\multicolumn{3}{r}{\slshape table continued on next page}\\
\endfoot
\hline %\multicolumn{3}{r}{end of table}
\endlastfoot

\multicolumn{3}{c}{Resolution dependent data in
  \ini{\tt \{RES\}}} \\\hline
Link target      & Link name & Explanation\\\hline
{\tt\{RES\}\_ozone\_CMIP5\_y1-y2.nc} & 
{\tt ozonyyyy}\index{input files!ozonyyyy}
 & 3--d ozone
  climatology being a mean value over the years {\tt y1} to {\tt
    y2}. Currently, {\tt y1-y2}=1850-1860 and 1979-1988 is
  available. These files 
  have to be linked to filenames {\tt ozonyyyy} where {\tt yyyy} is the
  actually simulated year. \\
{\tt\{RES\}\{OCR\}\_VLTCLIM.nc}        & 
{\tt unit.90}\index{input files!unit.90}
 & Climatological leaf
area index (monthly data). \\ 
{\tt\{RES\}\{OCR\}\_VGRATCLIM.nc}      & 
{\tt unit.91}\index{input files!unit.21}
 & Climatological
vegetation ratio (monthly data). \\ 
{\tt\{RES\}\_TSLCLIM2.nc}       & 
{\tt unit.92}\index{input files!unit.92}
 & Climatological land
surface temperature (monthly data). \\ 
T{\tt \{RES\}\{OCR\}\_piControl- LR\_sst\_1880-2379.nc}&{\tt
  unit.20}\index{input files!unit.20}
 &
Climatological 
sea surface temperatures (monthly data, only in T63GR15 available). \\ 
{\tt \{RES\}\{OCR\}\_piControl- LR\_sic\_1880-2379.nc}&{\tt
  unit.96}\index{input files!unit.96}
&Climatological 
sea ice data (monthly data, only in T63GR15 available). \\ 
\hline
\multicolumn{3}{c}{Tropospheric aerosols}\\\hline
{\tt aero/\{RES\}\_aeropt\_ kinne\_sw\_b14\_coa.nc} & {\tt
  aero\_coarse\_yyyy.nc}\index{input files!aero\_coarse\_yyyy.nc} 
& Optical properties of coarse mode
aerosols in the solar spectral range. Since these are mostly of natural
origin, climatological 
boundary conditions are sufficient for historic times.\\
{\tt aero/\{RES\}\_aeropt\_ kinne\_lw\_b16\_coa.nc} & {\tt
  aero\_farir\_yyyy.nc}\index{input files!aero\_farir\_yyyy.nc}
 & Aerosol optical properties in the thermal
spectral range. Only coarse mode aerosols play a role. Since these are
mostly of natural 
origin, climatological 
boundary conditions are sufficient for historic times.\\
\hline
\multicolumn{3}{c}{Land surface model JSBACH ({\tt /pool/data/JSBACH})}\\\hline
{\tt jsbach/ jsbach\_\{RES\}\{OCR\}\_\{t\}\_yyyy.nc} &
{\tt jsbach.nc}\index{input files!jsbach.nc}
 & Boundary conditions for land surface model
JSBACH. It also depends on the ocean resolution because the land--sea
mask does. The structure of JSBACH may vary with the number of tiles,
encoded in {\tt \{t\}}={\tt 4tiles}, {\tt 8tiles}, {\tt 11tiles}, or
{\tt 12tiles}. Not all combinations of resolutions are
available. \\
\hline
\multicolumn{3}{c}{Resolution independent data in
  \ini} \\\hline
{\tt hdpara.nc} & {\tt hdpara.nc}\index{input files!hdpara.nc}
 & Data for hydrological discharge
model. \\

\hline
\end{longtable}


Furthermore, various transient boundary conditions are available which
can either replace their climatological counterparts or be used as
supplemental conditions.
Examples for transient boundary conditions are observed sea surface
temperatures and sea ice data, transient greenhouse gas concentrations
or data accounting for interannual variability in
solar radiation, ozone concentration or aerosol optical
properties. The historical sea surface temperature (SST) and sea ice
cover (SIC) data 
are taken from the Program for Climate Model Diagnosis and
Intercomparison (PCMDI, status: November 2009). A list of possible
input data can be found in 
Tab.~\ref{tabechamt}.

\setlength{\LTcapwidth}{\textwidth}
\setlength{\LTleft}{0pt}\setlength{\LTright}{0pt}

\begin{longtable}{p{5cm}@{\extracolsep\fill}lp{6cm}}\hline\hline 
\caption[Transient boundary conditions]{\echam{}
  transient boundary conditions. Specific years are symbolized by {\tt
    yyyy}.}\\\hline\label{tabechamt}  
\endfirsthead
\caption[]{Transient boundary conditions
  continued}\\\hline 
\endhead
\hline\multicolumn{3}{r}{\slshape table continued on next page}\\
\endfoot
\hline %\multicolumn{3}{r}{end of table}
\endlastfoot

\multicolumn{3}{c}{Resolution dependent data
  in \ini{\tt \{RES\}}} \\\hline
Link target      & Link name & Explanation\\\hline
{\tt amip/ \{RES\}\_amip2sst\_yyyy.nc} & 
{\tt sstyyyy}\index{input files!sstyyyy}
 & historical sea surface
temperatures (monthly data).  \\
{\tt amip/ \{RES\}\_amip2sic\_yyyy.nc}&
{\tt iceyyyy}\index{input files!iceyyyy}
& historical
sea ice data (monthly data). \\\hline
\multicolumn{3}{c}{Tropospheric aerosols}\\\hline
{\tt aero/\{RES\}\_aeropt\_ kinne\_sw\_b14\_fin\_yyyy.nc} & {\tt
  aero\_fine\_yyyy.nc}\index{input files!aero\_fine\_yyyy.nc}
 & Optical properties of fine mode aerosols in
the solar spectrum. These aerosols are of anthropogenic origin mainly.
Therefore, they depend on the year. These are the historical
data. \\
{\tt aero/\{RES\}\_aeropt\_ kinne\_sw\_b14\_fin\_\{sc\}\_yyyy.nc} & {\tt
  aero\_fine\_yyyy.nc}\index{input files!aero\_fine\_yyyy.nc}
 & Optical properties of fine mode aerosols in
the solar spectrum. These aerosols are of anthropogenic origin mainly.
Therefore, they depend on the year. They are provided for different
scenarios for the future ({\tt \{sc\}}= {\tt rcp26}, {\tt rcp45}, {\tt
  rcp85}).
\\
\hline
\multicolumn{3}{c}{Volcanic (stratrospheric) aerosols, Stenchikov}\\\hline
{\tt volcano\_aerosols/strat\_ aerosol\_sw\_T\{RES\}\_yyyy.nc} & 
{\tt strat\_aerosol\_sw\_yyyy.nc}
\index{input files!strat\_aerosol\_sw\_yyyy.nc}
 & Aerosol 
optical properties of stratospheric aerosols of volcanic
origin in the solar spectral range. \\
{\tt volcano\_aerosols/strat\_ aerosol\_ir\_T\{RES\}\_yyyy.nc} & 
{\tt strat\_aerosol\_ir\_yyyy.nc}
\index{input files!strat\_aerosol\_ir\_yyyy.nc}
 & Aerosol
optical properties of stratospheric aerosols of volcanic
origin in the thermal spectral range. \\
\hline
\multicolumn{3}{c}{Volcanic (stratrospheric) aerosols, provided by HAM}\\\hline
N.N. & {\tt aoddz\_ham\_yyyy.nc}\index{input files!aodz\_ham\_yyyy.nc}
 & Aerosol optical properties as
provided by the HAM model. These data have to be used together with
the {\tt b30w120}\index{input files!b30w120}
parameter file of Tab.~\ref{tabechamp}. The aerosol
type described by the HAM model has to be compatible with that of the
parameter file. \\
\hline
 \multicolumn{3}{c}{Transient 3d--ozone data in \ini{\tt \{RES\}/ozone}}\\\hline
{\tt \{RES\}\_ozone\_CMIP5\_ yyyy.nc} & 
{\tt ozonyyyy}\index{input files!ozonyyyy}
 & Historic
3d--distribution 
of ozone in the stratosphere and troposphere. \\
{\tt \{RES\}\_ozone\_CMIP5\_ \{sc\}\_yyyy.nc} & {\tt
  ozonyyyy}\index{input files!ozonyyyy}
&3d--distribution 
of ozone in the stratosphere and troposphere for the scenarios
{\tt RCP26}, {\tt RCP45}, and {\tt RCP85}. \\
\hline
\multicolumn{3}{c}{Resolution independent data
  in \ini} \\\hline
\multicolumn{3}{c}{Volcanic (stratrospheric) aerosols,
  T.~Crowley}\\\hline
{\tt volc\_data} & 
{\tt aodreff\_crow.dat}\index{input files!aodreff\_crow.dat}
 & Stratospheric aerosol optical properties of volcanic aerosols compiled
by T. Crowley. All years are in one file. The {\tt b30w120} parameter
file of Tab.~\ref{tabechamp} has to be used together with these
data. \\\hline
\multicolumn{3}{c}{Transient solar irradiance in \ini{\tt
    solar\_irradiance}}\\\hline 
{\tt swflux\_14band\_yyyy.nc} & 
{\tt swflux\_yyyy.nc}\index{input files!swflux\_yyyy.nc}
 & Monthly
spectral solar irradiance for year {\tt yyyy}.  \\
\hline
\multicolumn{3}{c}{Greenhouse gas scenarios in \ini}\\\hline 
{\tt greenhouse\_\{sc\}.nc} & 
{\tt greenhouse\_gases.nc}\index{input files!greenhouse\_gases.nc}
 & Transient
greenhouse gas concentrations (all years in one file) for the
scenarios {\tt \{sc\}}= {\tt rcp26}, {\tt rcp45}, {\tt rcp85}. The
{\tt rcp45}--file contains the historic data also. \\
\hline

\end{longtable}

Some of the equations used in \echam{} need tables of
parameters. E.g.~the radiation needs temperature and pressure
(concentration) dependent absorption coefficients, the calculation of
the aerosol optical properties at all wave lengths from the effective
aerosol radius and 
the aerosol optical depth at a certain wavelength needs conversion
factors. The surface model JSBACH needs further input parameters that
are provided in a kind of a standard input file. A list of the input
files containing model parameters is provided in Tab.~\ref{tabechamp}.


\setlength{\LTcapwidth}{\textwidth}
\setlength{\LTleft}{0pt}\setlength{\LTright}{0pt}

\begin{longtable}{p{5cm}@{\extracolsep\fill}lp{6cm}}\hline\hline 
\caption[Parameter files]{Input files for \echam{} containing
  parameters for various 
  physical processes in \ini}\\\hline\label{tabechamp}  
\endfirsthead
\caption[]{Parameters files
  continued}\\\hline 
\endhead
\hline\multicolumn{3}{r}{\slshape table continued on next page}\\
\endfoot
\hline %\multicolumn{3}{r}{end of table}
\endlastfoot

Link target      & Link name & Explanation\\\hline
{\tt surrta\_data} & {\tt  rrtadata}\index{input files!rrtadata}
 & Tables for RRTM radiation scheme
--- solar radiation.\\
{\tt rrtmg\_lw.nc} & 
{\tt rrtmg\_lw.nc}\index{input files!rrtmg\_lw.nc}
 & Tables for RRTMG radiation scheme
--- thermal radiation.\\
{\tt ECHAM6\_CldOptProps.nc} & {\tt
  ECHAM6\_CldOptProps.nc}\index{input files!ECHAM6\_CldOptProps.nc} 
& Optical
properties of clouds. \\
{\tt ../../b30w120} & 
{\tt aero\_volc\_tables.dat}\index{input files!aero\_volc\_tables.dat}
 & Parametrizations of the
aerosol optical properties in the case of T.~Crowley aerosols and
aerosols provided by HAM. This table has to be compatible with the
aerosol data.\\
{\tt ../../jsbach/ lctlib\_nlct21.def\_rev5793}& {\tt
  lctlib.def}\index{input files!lctlib.def}
 & Parametrization of properties of vegetation and land model
JSBACH. (imported from the cosmos svn)\\\hline
\end{longtable}

In some rare cases, input data for diagnostic subroutines are
needed. This is the case for the station diagnostic that writes values
at different geographic locations (CFMIP2 sites by K.~Taylor). The file
and link name is listed in Tab.~\ref{tabdiaginput}

\setlength{\LTcapwidth}{\textwidth}
\setlength{\LTleft}{0pt}\setlength{\LTright}{0pt}

\begin{longtable}{p{5cm}@{\extracolsep\fill}lp{6cm}}\hline\hline 
\caption[Parameter files]{Input files for diagnostics
 provided in \ini{\tt /CFMIP}}\\\hline\label{tabdiaginput}  
\endfirsthead
\caption[]{Diagnostic input files
  continued}\\\hline 
\endhead
\hline\multicolumn{3}{r}{\slshape table continued on next page}\\
\endfoot
\hline %\multicolumn{3}{r}{end of table}
\endlastfoot

Link target      & Link name & Explanation\\\hline
{\tt pointlocations.txt}\index{input files!pointlocations.txt} & {\tt pointlocations.txt} & List of CFMIP2
sites as given by K.~Taylor. At these locations, surface and column
variables are written to output files.\\\hline
\end{longtable}
