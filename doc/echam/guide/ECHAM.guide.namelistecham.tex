\subsection{Input namelists in file {\tt namelist.echam}}
  In Fortran, you can provide the values of input variables that are
  organized in
  namelists, specifying name and value of each variable. Several
  namelists are used to specify the input of \echam. Some of the
  namelists are for the atmospheric part and have to be written into
  the file {\tt namelist.echam}, others determine input variables of
  the land surface model JSBACH and have to be written into {\tt
    namelist.jsbach}. The atmospheric part can accept the following
  namelists in {\tt namelist.echam} (alphabetical order):

\begin{description}
  \item[{\tt cfdiagctl}:] \index{namelists!cfdiagctl}CFMIP2 station
    diagnostics. 
  \item[{\tt co2ctl}:] \index{namelists!co2ctl}interactive CO$_2$
    budget calculation. 
  \item[{\tt columnctl}:] \index{namelists!columnctl}single column model. 
  \item[{\tt cospctl}:] \index{namelists!cospctl}controls the COSP
    satellite simulator 
  \item[{\tt cospofflctl}:] \index{namelists!cospofflctl}namelist
    group for offline COSP 
    calculation. Offline means that data from files are read and no
    simulation of the general circulation takes place.
  \item[{\tt debugsctl}:] \index{namelists!debugsctl}creates a stream
    for grid point variables 
    that can be written to output easily (for debugging).
  \item[{\tt dyctl}:] \index{namelists!dynctl}parameters for
    atmosphere dynamics. 
  \item[{\tt ensctl}:] \index{namelists!ensctl}generate forecast ensembles.
  \item[{\tt gwsctl}:] \index{namelists!gwsctl}gravity wave parameterisation.
  \item[{\tt hratesctl}:] \index{namelists!hratesctl}diagnostic of
    heating rates. 
  \item[{\tt mvstreamctl}:] \index{namelists!mvstreamctl}variables
    controlling output of mean 
    values.
  \item[{\tt ndgctl}:] \index{namelists!ndgctl}variables which are
    related to the nudging of 
    the model, i.e.~to the relaxation method constraining the
    meteorological variables divergence, 
    vorticity, temperature and pressure to externally given values.
  \item[{\tt new\_tracer}:] \index{namelists!new\_tracer}new tracers
    can be introduced by the use of 
    this namelist group.
  \item[{\tt nmictl}:] \index{namelists!nmictl}normal mode analysis of waves.
  \item[{\tt parctl}:] \index{namelists!parctl}parameters concerning
    the parallel 
    configuration of model. 
  \item[{\tt physctl}:] \index{namelists!physctl}variables related to
    the physics calculation 
    like switching on/off radiation, diffusion,
  convection, surface exchange, \dots
  \item[{\tt radctl}:] \index{namelists!radctl}variables for
    controlling the radiation calculation. 
  \item[{\tt runctl}:] \index{namelists!runctl}contains variables
    concerning the start and the end 
  of a simulation.
  \item[{\tt set\_stream}:] \index{namelists!set\_stream}set the
    properties of an existing stream 
    by this namelist
  \item[{\tt set\_stream\_element}:]
    \index{namelists!set\_stream\_element}set stream element
    properties of 
    an existing stream element by namelist.
  \item[{\tt set\_tracer}:] \index{namelists!set\_tracer}this namelist
    group helps to set tracer 
    properties if they are created by some (sub)model.
  \item[{\tt stationctl}:] \index{namelists!stationctl}high frequency
    output at the location of 
    various sites including profiles.
  \item[{\tt submdiagctl}:] \index{namelists!submdiagctl}submodel diagnostics.
  \item[{\tt submodelctl}:] \index{namelists!submodelctl}namelists for
    registration of submodels in 
    \echam.
  \item[{\tt tdiagctl}:] \index{namelists!tdiagctl}tendency diagnostic.
\end{description}

The syntax for each namelist in {\tt namelist.echam} is
\index{namelist syntax}:
\begin{lstlisting}[caption= namelist syntax]
  & <namelist name>
    <varname> = <value>
  /
\end{lstlisting}

{\bf Remark:} The mere presence of a certain variable in a certain
namelist does not mean that the action associated with this variable
really works properly or works at all.

Variables describing repeated events have a special format (type
``special'' in the 
following tables\index{special format}): \newline
{\tt \{interval\}, \{unit\}, \{adjustment\}, \{offset\}}\newline
where {\tt \{interval\}} is a positive integer number, {\tt \{unit\}}
is one of {\tt 'steps'}, {\tt 'seconds'}, {\tt 'minutes'}, {\tt
  'hours'}, {\tt 'days'}, {\tt 'months'}, {\tt 'years'}, {\tt \{adjustment\}} is
one of {\tt 'first'}, {\tt 'last'}, {\tt 'exact'}, {\tt 'off'}, and {\tt
  \{offset\}} is an integer number giving the offset with respect to
the initial date of the simulation in seconds. A detailed description
of the control of time events can be found in the 
lecture ``Using and programming \echam{} --- a first introduction''
by S.~Rast. The variable list is given in
alphabetical order even if the most important variables are not at the
first place in this case.

\subsubsection{Namelist {\tt cfdiagctl}\index{namelists!cfdiagctl}}\label{seccfdiagctl}

This namelists contains only one parameter to switch on or off the
CFMIP2 diagnostics of 3--dimensional fluxes.

\setlength{\LTcapwidth}{\textwidth}
\setlength{\LTleft}{0pt}\setlength{\LTright}{0pt}

\begin{longtable}{l@{\extracolsep\fill}lp{5.0cm}p{3.0cm}}
\hline\hline\caption[Namelist {\tt cfdiagctl}]{Namelist
  {\tt cfdiagctl}}\\\hline\label{tabcfdiagctl}
\endfirsthead
\caption[]{{\tt cfdiagctl} --- continued}\\\hline
\endhead
\hline\multicolumn{4}{r}{\slshape table continued on next page}\\
\endfoot
\hline %\multicolumn{4}{r}{end of table}
\endlastfoot
Variable & type & Explanation & default \\\hline
{\tt locfdiag}\index{namelist variables!locfdiag} & logical & switches on/off CFMIP2 diagnostic output of
convective mass flux and 3-D radiation fluxes & {\tt .FALSE.} \\\hline
\end{longtable}
\subsubsection{Namelist {\tt co2ctl}\index{namelists!co2ctl}}\label{secco2ctl}

This namelist controls the behaviour of the CO$_2$ submodel. This
submodel is not a simple submodel like the transport of some gas phase
species would be because the CO$_2$ module interacts with the JSBACH
surface and vegetation model. In this namelist, the behaviour of the
CO$_2$ submodel in the atmosphere simulated by \echam{} and the
interaction with the ocean and soil simulated by JSBACH can be controlled.


\setlength{\LTcapwidth}{\textwidth}
\setlength{\LTleft}{0pt}\setlength{\LTright}{0pt}

\begin{longtable}{l@{\extracolsep\fill}lp{5.0cm}p{3.0cm}}
\hline\hline\caption[Namelist {\tt co2ctl}]{Namelist
  {\tt co2ctl}}\\\hline\label{tabco2ctl}
\endfirsthead
\caption[]{{\tt co2ctl} --- continued}\\\hline
\endhead
\hline\multicolumn{4}{r}{\slshape table continued on next page}\\
\endfoot
\hline %\multicolumn{4}{r}{end of table}
\endlastfoot
Variable & type & Explanation & default \\\hline
{\tt lco2\_flxcor}\index{namelist variables!lco2\_flxcor} 
& logical & switches on/off flux correction for
  exact mass balance & {\tt .TRUE.} \\
{\tt lco2\_mixpbl}\index{namelist variables!lco2\_mixpbl}
 & logical & switches on/off CO$_2$ mixing in
  planetary boundary layer & {\tt .TRUE.} \\
{\tt lco2\_2perc}\index{namelist variables!lco2\_2perc} 
& logical & switches on/off limitation of relative CO$_2$
  tendency to 2\% & {\tt .FALSE.} \\
{\tt lco2\_emis}\index{namelist variables!lco2\_emis} 
& logical & switches on/off reading prescribed CO$_2$
  emissions from a file & {\tt .FALSE.} \\
{\tt lco2\_clim}\index{namelist variables!lco2\_clim} 
& logical & switches on/off treating the CO$_2$
  concentration as a climatological quantity not being transported
  & {\tt .FALSE.} \\
{\tt lco2\_scenario}\index{namelist variables!lco2\_scenario} 
& logical & switches on/off reading CO$_2$
  concentrations from a certain greenhouse gas scenario
  & {\tt .FALSE.} but {\tt .TRUE.} if {\tt ighg=1} and {\tt lco2=.FALSE.}\\

\hline 
\end{longtable}


\subsubsection{Namelist {\tt columnctl}\index{namelists!columnctl}}\label{seccolumnctl}

This namelist controls the behaviour of the single column model. A more
detailed description of the single column model can be found in
section~\ref{secscm}. Here, we only present the namelist.

\setlength{\LTcapwidth}{\textwidth}
\setlength{\LTleft}{0pt}\setlength{\LTright}{0pt}

\begin{longtable}{l@{\extracolsep\fill}lp{5.0cm}p{3.0cm}}
\hline\hline\caption[Namelist {\tt columnctl}]{Namelist
  {\tt columnctl}}\\\hline\label{tabcolumnctl}
\endfirsthead
\caption[]{{\tt columnctl} --- continued}\\\hline
\endhead
\hline\multicolumn{4}{r}{\slshape table continued on next page}\\
\endfoot
\hline %\multicolumn{4}{r}{end of table}
\endlastfoot
variable & type & explanation & default \\\hline
{\tt forcingfile(32)}\index{namelist variables!forcingfile}
    &character       &name of the forcing file   &   ---   \\ 
 
{\tt mld}\index{namelist variables!mld} 
 & real & depth of mixed layer in metres & {\tt 10}\\
{\tt ml\_input}\index{namelist variables!ml\_input}
 & logical & {\tt ml\_input=.true.}: initial
temperature of mixed layer ocean is set to the value of the surface
temperature of the forcing file. {\tt ml\_input=.false.}: the sea
surface temperature is set to the value
given in the \echam{} sst file for the respective column & {\tt .false.}\\
{\tt nfor\_div(2)}\index{namelist variables!nfor\_div}
       &integer & option array describing the
treatment of the divergence of the wind field. The option array consists of
$\{i_{\rm set},i_{\rm cycle}\}$ as described in section~\ref{secforcing}. & {\tt (/0,0/)} \\
{\tt nfor\_lhf(2)}\index{namelist variables!nfor\_lhf}
       &integer & option array describing the
treatment of the latent heat flux. The option array consists of
$\{i_{\rm set},i_{\rm cycle}\}$ as described in section~\ref{secforcing}. {\bf This option
  array is not working.} & {\tt (/0,0/)} \\
{\tt nfor\_omega(2)}\index{namelist variables!nfor\_omega}
       &integer & option array describing the
treatment of the pressure velocity. The option array consists of
$\{i_{\rm set},i_{\rm cycle}\}$ as described in section~\ref{secforcing}. & {\tt (/0,0/)} \\
{\tt nfor\_q(3)}\index{namelist variables!nfor\_q}
        &integer   &option array describing the
treatment of the specific humidity in the column. The option array consists of
$\{i_\Delta,\tau,i_{\rm cycle}\}$ as described in section~\ref{secforcing}.
                                 &   {\tt (/0,0,0/)}  \\  
{\tt nfor\_shf(2)}\index{namelist variables!nfor\_shf}
       &integer & option array describing the
treatment of the sensible heat flux. The option array consists of
$\{i_{\rm set},i_{\rm cycle}\}$ as described in section~\ref{secforcing}. {\bf This option
  array is not working.} & {\tt (/0,0/)} \\
{\tt nfor\_t(3)}\index{namelist variables!nfor\_t}
        &integer   & option array describing the
treatment of the column temperature. The option array consists of
$\{i_\Delta,\tau,i_{\rm cycle}\}$ as described in section~\ref{secforcing}. 
                                 &   {\tt (/0,0,0/)}  \\ 
{\tt nfor\_ts(2)}\index{namelist variables!nfor\_ts}
       &integer & option array describing the
treatment of the surface temperature. The option array consists of
$\{i_{\rm set},i_{\rm cycle}\}$ as described in section~\ref{secforcing}. & {\tt (/0,0/)} \\
{\tt nfor\_uv(3)}\index{namelist variables!nfor\_uv}
       &integer   &option array describing the
treatment of the wind in $\vec{u}$ and $\vec{v}$ direction. The option array consists of
$\{i_\Delta,\tau,i_{\rm cycle}\}$ as described in section~\ref{secforcing}. The $\vec{u}$
and $\vec{v}$ winds can not be treated individually.
                                 &   {\tt (/0,0,0/)}  \\ 
{\tt nfor\_uvgeo(2)}\index{namelist variables!nfor\_uvgeo}
       &integer & option array describing the
treatment of the geostrophic wind. The option array consists of
$\{i_{\rm set},i_{\rm cycle}\}$ as described in section~\ref{secforcing}. & {\tt (/0,0/)} \\
{\tt nfor\_xi(3)}\index{namelist variables!nfor\_xi}
       &integer   & option array describing the
treatment of the ice water content. The option array consists of
$\{i_\Delta,\tau,i_{\rm cycle}\}$ as described in section~\ref{secforcing}.
                                 &   {\tt (/0,0,0/)}  \\ 
{\tt nfor\_xl(3)}\index{namelist variables!nfor\_xl}
       &integer   & option array describing the
treatment of the liquid water content. The option array consists of
$\{i_\Delta,\tau,i_{\rm cycle}\}$ as described in section~\ref{secforcing}.
                                 &   {\tt (/0,0,0/)}  \\ 
\hline 
\end{longtable}


\subsubsection{Namelist {\tt cospctl}\index{namelists!cospctl}}\label{seccospctl}

This namelist group controls the calculations of the COSP satellite
simulator.

\setlength{\LTcapwidth}{\textwidth}
\setlength{\LTleft}{0pt}\setlength{\LTright}{0pt}

\begin{longtable}{l@{\extracolsep\fill}lp{5.0cm}p{3.0cm}}
\hline\hline\caption[Namelist {\tt cospctl}]{Namelist
  {\tt cospctl}}\\\hline\label{tabcospctl}
\endfirsthead
\caption[]{{\tt cospctl} --- continued}\\\hline
\endhead
\hline\multicolumn{4}{r}{\slshape table continued on next page}\\
\endfoot
\hline %\multicolumn{4}{r}{end of table}
\endlastfoot
Variable & type & Explanation & default \\\hline
{\tt extra\_output}\index{namelist variables!extra\_output} 
 & logical & switches on ({\tt extra\_output=.true.}) or off
({\tt extra\_output=.false.}) additional output & {\tt .false.}\\
{\tt Lisccp\_sim}\index{namelist variables!Lisccp\_sim}
 & logical & switches on ({\tt Lisccp\_sim=.true.}) or off
({\tt Lisccp\_sim=.false.}) the ISCCP simulator & {\tt .true.}\\
{\tt Llidar\_sim}\index{namelist variables!Llidar\_sim}
 & logical & switches on ({\tt Llidar\_sim=.true.}) or off
({\tt Llidar\_sim=.false.}) the COSP LIDAR simulator & {\tt .true.}\\
{\tt Llidar\_cfad}\index{namelist variables!Llidar\_cfad}
 & logical & switches on ({\tt Llidar\_cfad=.true.}) or off
({\tt Llidar\_cfad=.false.}) the output of COSP CFAD Lidar Scattering
Ratio at 532nm& {\tt .false.}\\
{\tt locosp}\index{namelist variables!locosp}
 & logical & switches on ({\tt locosp=.true.}) or off
({\tt locosp=.false.}) all COSP satellite simulators & {\tt .false.}\\
{\tt l\_fixed\_ref}\index{namelist variables!l\_fixed\_ref}
 & logical & switches on ({\tt l\_fixed\_ref=.true.}) or
off ({\tt l\_fixed\_ref=.false.}) the use of fixed hydrometeor
diameters for tests & {\tt .false.}\\
{\tt Ncolumns}\index{namelist variables!Ncolumns}
 & integer & number of sub--columns used for each
profile & {\tt 12} \\
{\tt offl2dout}\index{namelist variables!offl2dout}
 & integer & extra output for offline tests if {\tt
  offl2dout$>$0} 
& {\tt -1} \\
{\tt use\_netcdf}\index{namelist variables!use\_netcdf}
 & logical & switches between netcdf format output ({\tt
  use\_netcdf=.true.}) 
or GRIB format output ({\tt use\_netcdf=.false.})
& {\tt .true.}\\
\hline 
\end{longtable}

\subsubsection{Namelist {\tt cospofflctl}\index{namelists!cospofflctl}}\label{seccospofflctl}

This namelist group controls the calculations of the COSP satellite
simulator from echam output without really performing a new simulation
of the general circulation.

\setlength{\LTcapwidth}{\textwidth}
\setlength{\LTleft}{0pt}\setlength{\LTright}{0pt}

\begin{longtable}{l@{\extracolsep\fill}lp{5.0cm}p{3.0cm}}
\hline\hline\caption[Namelist {\tt cospofflctl}]{Namelist
  {\tt cospofflctl}}\\\hline\label{tabcospofflctl}
\endfirsthead
\caption[]{{\tt cospofflctl} --- continued}\\\hline
\endhead
\hline\multicolumn{4}{r}{\slshape table continued on next page}\\
\endfoot
\hline %\multicolumn{4}{r}{end of table}
\endlastfoot
Variable & type & Explanation & default \\\hline
{\tt locospoffl}\index{namelist variables!locospoffl}
 & logical & switches on ({\tt locospoffl=.true.}) or off
({\tt locospoffl=.false.}) the offline COSP satellite simulators &
{\tt .false.}\\ 
{\tt offl2dout}\index{namelist variables!offl2dout}
 & integer & extra output for offline simulators if {\tt
  offl2dout$>$0} 
& {\tt -1} \\
\hline 
\end{longtable}


\subsubsection{Namelist {\tt
    debugsctl}\index{namelists!debugsctl}}\label{secdebugsctl} 

The debug stream is meant to provide a quick and easy tool to the user of
\echam{} that allows him to write any 2d-- or 3d--gridpoint variable
on either layer midpoints or layer interfaces
to an extra stream for debugging. A detailed description can be found
in Appendix~\ref{cr20090331}.


\setlength{\LTcapwidth}{\textwidth}
\setlength{\LTleft}{0pt}\setlength{\LTright}{0pt}

\begin{longtable}{l@{\extracolsep\fill}lp{5.0cm}p{3.0cm}}
\hline\hline\caption[Namelist {\tt debugsctl}]{Namelist
  {\tt debugsctl}}\\\hline\label{tabdebugsctl}
\endfirsthead
\caption[]{{\tt debugsctl} --- continued}\\\hline
\endhead
\hline\multicolumn{4}{r}{\slshape table continued on next page}\\
\endfoot
\hline %\multicolumn{4}{r}{end of table}
\endlastfoot
Variable & type & Explanation & default \\\hline
{\tt nddf}\index{namelist variables!nddf}
 & integer & number of 3d--fields  on full levels (layer midpoints) created in addition to the default
  fields & {\tt 0} \\
{\tt nddfh}\index{namelist variables!nddfh} & integer & number of
3d--fields on half levels (layer interfaces) created in addition to
the default fields & {\tt 0} \\
{\tt nzdf}\index{namelist variables!nzdf}
 & integer & number of 2d--fields created in addition to the default
  fields & {\tt 0} \\
{\tt putdebug\_stream}\index{namelist variables!putdebug\_stream}
 & special & output frequency of debug stream &
{\tt 6, 'hours', 'first', 0}\\
\hline 
\end{longtable}

\subsubsection{Namelist {\tt dynctl}\index{namelists!dynctl}}\label{secdynctl}

With the help of these namelist parameters, the (large
scale) dynamics of the atmosphere can be controlled.

\setlength{\LTcapwidth}{\textwidth}
\setlength{\LTleft}{0pt}\setlength{\LTright}{0pt}

\begin{longtable}{l@{\extracolsep\fill}lp{5.0cm}p{3.0cm}}
\hline\hline\caption[Namelist {\tt dynctl}]{Namelist
  {\tt dynctl}}\\\hline\label{tabdynctl}
\endfirsthead
\caption[]{{\tt dynctl} --- continued}\\\hline
\endhead
\hline\multicolumn{4}{r}{\slshape table continued on next page}\\
\endfoot
\hline %\multicolumn{4}{r}{end of table}
\endlastfoot
Variable & type & Explanation & default \\\hline
{\tt apsurf}\index{namelist variables!apsurf}
 & real & fixed global mean of surface pressure
  in Pa fixing the mass of the dry atmosphere & {\tt 98550.0} \\
{\tt damhih}\index{namelist variables!damhih}
 & real & extra diffusion in the middle atmosphere
& {\tt 1000.} \\
{\tt dampth}\index{namelist variables!dampth}
 & real & damping time in hours for the horizontal
  diffusion of vorticity (linear square Laplacian), divergence, and
  temperature. Depends on the spectral resolution {\tt nn} & 
  {\tt nn=31}: {\tt 12.0} \newline
  {\tt nn=63}: {\tt 7.0} \newline
  {\tt nn=127}: {\tt 1.5} \newline
  {\tt nn=255}: {\tt 0.5} \\
{\tt diagdyn}\index{namelist variables!diagdyn}
 & special & frequency for diagnostic output of
  quantities describing the dynamics of the atmosphere & {\tt 5,
    'days', 'off', 0} \\
{\tt diagvert}\index{namelist variables!diagvert}
 & special & frequency for special (all layers)
  diagnostic output of 
  quantities describing the dynamics of the atmosphere & {\tt 5,
    'days', 'off', 0} \\
{\tt enspodi}\index{namelist variables!enspodi}
 & real & factor by which upper sponge layer
  coefficient is increased from one layer to the adjacent layer above
  & {\tt 1.0}\\ 
{\tt enstdif}\index{namelist variables!enstdif} 
& real & factor by which stratospheric horizontal
  diffusion is increased from one layer to the adjacent layer above &
  {\tt 1.0} \\
{\tt eps}\index{namelist variables!eps} & real & coefficient in the Robert--Asselin time
  filter & {\tt 0.1} \\
{\tt hdamp}\index{namelist variables!hdamp} & real & damping factor
for strong stratospheric 
  damping & {\tt 1.0} \\
{\tt ldiahdf}\index{namelist variables!ldiahdf}
 & logical & switches on/off statistical analysis of
horizontal diffusion & {\tt .FALSE.} \\
{\tt lumax}\index{namelist variables!lumax}
 & logical & switches on/off the printing of information
  on maximum wind speeds & {\tt .FALSE.} \\
{\tt lzondia}\index{namelist variables!lzondia}
 & logical & purpose unknown & {\tt .FALSE.} \\
{\tt nlvspd1}\index{namelist variables!nlvspd1}
 & integer & model layer index of uppermost layer of
  upper sponge & {\tt 1} \\
{\tt nlvspd2}\index{namelist variables!nlvspd2}
 & integer & model layer index of lowest layer of
  upper sponge & {\tt 1} \\
{\tt nlvstd1}\index{namelist variables!nlvstd1}
 & integer & model layer index of uppermost layer at which
  stratospheric horizontal diffusion is enhanced & {\tt 1} \\
{\tt nlvstd2}\index{namelist variables!nlvstd2}
 & integer & model layer index of lowest layer at which
  stratospheric horizontal diffusion is enhanced & {\tt 1} \\
{\tt ntrn(1:nlev)}\index{namelist variables!ntrn}
 & integer & layer and resolution dependent critical
  wave numbers for strong stratospheric damping & see {\tt setdyn.f90}
  \\
{\tt spdrag}\index{namelist variables!spdrag}
 & real & coefficient for upper sponge layer in 1/s
  & {\tt 0.0}, if {\tt lmidatm=.TRUE.}: {$\tt 0.926\times 10^{-4}$} (see
  Tab.~\ref{tabrunctl} \\
{\tt vcheck}\index{namelist variables!vcheck}
 & real & threshold value for check of high
windspeed in m/s & {\tt 200.0} \newline
                   if {\tt lmidatm=.TRUE.} (see Tab.~\ref{tabrunctl}:
                   {\tt 235.0} \\
{\tt vcrit}\index{namelist variables!vcrit}
 & real & critical velocity above which horizontal
  diffusion is enhanced in m/s. Depends on the spectral resolution {\tt
  nn} & {\tt nn=106}: 68.0 \newline
        all other {\tt nn}: 85.0 \\
\hline
\end{longtable}


\subsubsection{Namelist {\tt ensctl}\index{namelists!ensctl}}\label{secensctl}

This namelist controls simulations for ensemble forecasts.


\setlength{\LTcapwidth}{\textwidth}
\setlength{\LTleft}{0pt}\setlength{\LTright}{0pt}

\begin{longtable}{l@{\extracolsep\fill}lp{5.0cm}p{3.0cm}}
\hline\hline\caption[Namelist {\tt ensctl}]{Namelist
  {\tt ensctl}}\\\hline\label{tabensctl}
\endfirsthead
\caption[]{{\tt ensctl} --- continued}\\\hline
\endhead
\hline\multicolumn{4}{r}{\slshape table continued on next page}\\
\endfoot
\hline %\multicolumn{4}{r}{end of table}
\endlastfoot
Variable & type & Explanation & default \\\hline

{\tt ensemble\_member}\index{namelist variables!ensemble\_member}
 & integer & index of ensemble member between 1
and {\tt ensemble\_size} & {\tt -1} \\
{\tt ensemble\_size}\index{namelist variables!ensemble\_size}
   & integer & number of ensemble members & {\tt -1} \\
{\tt forecast\_type}\index{namelist variables!forecast\_type}
   & integer & type of forecast, one of {\tt
  HR\_CONTROL=0}, {\tt LR\_CONTROL=1}, {\tt NEGATIV\_PERTURBED=2}, {\tt
  POSITIV\_PERTURBED=3}, {\tt MULTI\_MODEL=4}& {\tt -1}\\

\end{longtable}

\subsubsection{Namelist {\tt gwsctl}\index{namelists!gwsctl}}\label{secgwsctl}

This namelist controls the settings for the gravity wave drag
parameterization. 

\setlength{\LTcapwidth}{\textwidth}
\setlength{\LTleft}{0pt}\setlength{\LTright}{0pt}

\begin{longtable}{l@{\extracolsep\fill}lp{5.0cm}p{3.0cm}}
\hline\hline\caption[Namelist {\tt gwsctl}]{Namelist
  {\tt gwsctl}}\\\hline\label{tabgwsctl}
\endfirsthead
\caption[]{{\tt gwsctl} --- continued}\\\hline
\endhead
\hline\multicolumn{4}{r}{\slshape table continued on next page}\\
\endfoot
\hline %\multicolumn{4}{r}{end of table}
\endlastfoot
Variable & type & Explanation & default \\\hline

{\tt emiss\_lev}\index{namelist variables!emiss\_lev}
 & integer & model layer index counted from the
  surface at which gravity waves are emitted. This number depends on
  the vertical resolution and corresponds to a model layer that is at
  roughly 600~hPa in the standard atmosphere. &
  {\tt nlev=199}: {\tt 26}\newline
  all other {\tt nlev}: {\tt 10}\\
{\tt front\_thres}\index{namelist variables!front\_thres}
 & real & minimum value of the frontogenesis
  function for which gravity waves are emitted from fronts in $\rm
  (K/m)^2/h$ & {\tt 0.12} \\
{\tt iheatcal}\index{namelist variables!iheatcal}
 & integer & controls upper atmosphere processes
  associated with gravity waves: \newline
  {\tt iheatcal=1}: calculate heating rates and diffusion coefficient
  in addition to momentum flux deposition\newline
  {\tt iheatcal=2}: momentum flux deposition only &
  {\tt 1}\\ 
{\tt kstar}\index{namelist variables!kstar} & real & typical gravity
wave horizontal wave 
  number & {$\tt 5\times10^{-5}$}\\
{\tt lat\_rmscon\_hi}\index{namelist variables!lat\_rmscon\_hi}
 & real & latitude above which extratropical
  gravity wave source is used. Is only relevant if {\tt
    lrmscon\_lat=.TRUE.} & {\tt 10.0} \\
{\tt lat\_rmscon\_lo}\index{namelist variables!lat\_rmscon\_lo}
 & real & latitude below which tropical
  gravity wave source is used. Is only relevant if {\tt
    lrmscon\_lat=.TRUE.}. There is a linear interpolation between
  {\tt lat\_rmscon\_lo} and {\tt lat\_rmscon\_hi} degrees N and S,
  respectively, between the values given by {\tt rmscon\_lo} (associated
  with the tropical gravity wave parameterization) and {\tt rmscon\_hi}
  associated with the extratropical gravity wave parameterization& {\tt 5.0} \\
{\tt lextro}\index{namelist variables!lextro}
 & logical & switches on/off the Doppler spreading
  extrowave parameterization by Hines & {\tt .TRUE.} \\
{\tt lfront}\index{namelist variables!lfront}
 & logical & switches on/off gravity waves emerging from
  fronts and the background. Parameterization by Charron and Manzini &
  {\tt .FALSE.} \\
{\tt lozpr}\index{namelist variables!lozpr}
 & logical & switches on/off the background enhancement of
  gravity waves associated with precipitation by Manzini et al.. Does
  not work with \echam. & 
  {\tt .FALSE.} \\
{\tt lrmscon\_lat}\index{namelist variables!lrmscon}
 & logical & switches on/off latitude dependent {\tt
  rmscon} as defined in {\tt setgws}. Must not be {\tt .TRUE.} if {\tt
  lfront=.TRUE.} or {\tt lozpr=.TRUE.} & {\tt .FALSE.}\\
{\tt m\_min}\index{namelist variables!m\_min} & real & minimum bound in vertical wave number &
  {\tt 0.0}\\
{\tt pcons}\index{namelist variables!pcons}
 & real & factor for  background enhancement 
  associated with precipitation & {\tt 4.75}\\
{\tt pcrit}\index{namelist variables!pcrit}
 & real & critical precipitation value above which
  root mean square gravity wave wind enhancement is applied in mm/d &
  {\tt 5.0} \\
{\tt rms\_front}\index{namelist variables!rms\_front}
 & real & root mean square frontal gravity wave
  horizontal wave number in 1/m & {\tt 2.0}\\
{\tt rmscon}\index{namelist variables!rmscon}
 & real & root mean square gravity wave wind at
  lowest layer in m/s & {\tt 1.0} \\
{\tt rmscon\_hi}\index{namelist variables!rmscon\_hi}
 & real & root mean square gravity wave wind at
  lowest layer in m/s for extratropical gravity wave source. Is only
  relevant if {\tt lrmscon\_lat=.TRUE.}& {\tt 1.0} \\
{\tt rmscon\_lo}\index{namelist variables!rmscon\_lo}
 & real & root mean square gravity wave wind at
  lowest layer in m/s for tropical gravity wave source. Is only
  relevant if {\tt lrmscon\_lat=.TRUE.}. Depends on the spectral
  resolution {\tt nn} & {\tt nn=31}: {\tt 1.0} \newline
                        {\tt nn=63}: {\tt 1.2} \newline
                        {\tt nn=127}: {\tt 1.05} but {\tt 1.1} if {\tt
                          lcouple=.TRUE.} (see
                        Tab.~\ref{tabrunctl})\newline
                        any other {\tt nn}: 1.1 \\ 
\hline 
\end{longtable}




\subsubsection{Namelist {\tt hratesctl}\index{namelists!hratesctl}}\label{sechratesctl}

This namelist is obsolete since its functionality is included in {\tt
  tdiagctl} (see section~\ref{sectdiagctl}).

\subsubsection{Namelist {\tt mvstreamctl}\index{namelists!mvstreamctl}}\label{secmvstreamctl}

Using this namelist, the online calculation of time averages of
non--accumulated grid point, spectral, and land variables
of any output stream is possible. If variables are averaged in the
original stream, they may be referenced in the mean value stream
For each stream, you can ask for one
additional stream containing the mean values of a subset of variables
of this stream. The namelist {\tt mvstreamctl} controls which output
streams will 
be doubled. The output of mean values of trace species concentrations
are written to the output stream {\tt
  tracerm}. In this new implementation, you are more flexible in terms
of names of the outputfiles. Furthermore, all variables are now
collected in the {\tt mvstreamctl} namelist and you do not need to
specify any further variables in the {\tt mvctl} namelist.
However, for backwards compatibility reasons, the old method using
the namelist {\tt mvctl}
described in section~\ref{secmvctl} still works.
A thorough documentation describing the numerical method and some
scientific aspects of the mean value calculation over time is
presented in Appendix~\ref{cr20100728}.

\begin{footnotesize}

\setlength{\LTcapwidth}{\textwidth}
\setlength{\LTleft}{0pt}\setlength{\LTright}{0pt}

\begin{longtable}{l@{\extracolsep\fill}lp{5.2cm}p{3.3cm}}
\hline\hline\caption[Namelist {\tt mvstreamctl}]{Namelist
  {\tt mvstreamctl}}\\\hline\label{tabmvstreamctl}
\endfirsthead
\caption[]{{\tt mvstreamctl} --- continued}\\\hline
\endhead
\hline\multicolumn{4}{r}{\slshape table continued on next page}\\
\endfoot
\hline %\multicolumn{4}{r}{end of table}
\endlastfoot
Variable & Type & Explanation & Default \\\hline
{\tt filetag}\index{namelist variables!filetag}
 & character(len=7) & The averaged variables of each
stream listed in {\tt source} will be written to the same 
outputfile with ending tag {\tt filetag}. If {\tt filetag} is not
present, the names of the streams are used as filetags and possibly
more than one file will be created. & {\tt
  target}  \\
{\tt interval}\index{namelist variables!interval}
 & special & time averaging interval & The
default depends on the setting of {\tt default\_output} in {\tt
  runctl}:
For {\tt default\_output= .false.}: {\tt interval=putdata}; for {\tt
  default\_output= .true.}: 
{\tt interval= 1,'months','first',0}
\\
{\tt meannam(500)}\index{namelist variables!meannam}
 & character(len=64) & variable names of stream elements
of which
time average is desired. If {\tt source} contains more streams than
one, the program stops if the variables are not contained in every 
of these streams. In that case, specify {\tt mvstreamctl} for each
stream separately. Variables that are not either spectral, 2d or 3d
grid point, or land variables are skipped. 
If meannam is not specified or equal to $\star$ or {\tt ''},
all variables of the respective stream(s) are averaged. & 
{\tt ''} \\ 
{\tt source}(50)\index{namelist variables!source} & character(len=16) & 
A mean value stream will be created for each stream listed in {\tt
  source}. Per default, the names of these replicated streams are 
  the original names with appended {\tt 'm'}. Furthermore,
per default corresponding 
outputfiles with these tags in their names will be created.
The default can be changed by the use of the {\tt target} and {\tt
  filetag} namelist 
variables. 
& {\tt ''} \\
{\tt sqrmeannam(500)}\index{namelist variables!sqrmeannam}
 & character(len=64) & variable names of stream
elements of
  which time average of their square is desired. Variables that are
  averaged over the output interval in the original stream and may
  only be referenced are excluded. If {\tt
    sqrmeannam='$\star$'} the mean of the square is calculated of all
  variables in the stream. Does work with several streams in {\tt
    source} & {\tt ''} \\
{\tt target}\index{namelist variables!target} & character(len=16) & 
If {\tt source} contains a single stream only, you can give a name to the
corresponding mean value stream by setting {\tt target} to a name of
your choice. You can also define
a common ending for all streams in {\tt source}
by setting {\tt 
  target=$\ast$<ending>}. In that case, the replicate of each original
stream 
will have the name {\tt <name of original stream><ending>}.
&
{\tt $\ast$m}\\\hline 
\multicolumn{4}{c}{variables for backward compatibility} \\\hline
{\tt m\_stream\_name(1:50)}\index{namelist variables!m\_stream\_name}
 & character(len=256) & List of names of
  streams for the elements of which mean values shall be
  calculated. Note that a maximum of 50 output streams is allowed
  (including the mean value streams). This variable can still be used
  together with the {\tt mvctl} namelist but is included only for
  backward compatibility. Note that you cannot set both variables {\tt
    source} and {\tt m\_stream\_name} at the same time.& {\tt ''} \\\hline
\end{longtable}

\end{footnotesize}

{\bf Remarks:}

\begin{description}


\item[target]

You may use the renaming of the mean value stream 
if you want to calculate monthly and daily means of some variables of
the same source stream in one simulation. If you do not rename at
least one of
these streams, there will be a naming conflict since the default would
be to name both mean value streams after the source stream with an
appended {\tt 'm'}.

Note: you can specify the {\tt mvstreamctl} namelist several times for
different (sets of) streams in the same {\tt namelist.echam} input file.

\item[interval]

Because of the time integration scheme used in \echam,
there is a particular behaviour
in calculating the mean values. Let's assume that you gave 
\texttt{interval = 2,'hours','first',0} and that you have a 40 minutes
time step. 
This means that you have instantaneous 
values at 00:00h, 00:40h, 01:20h, 02:00h, 02:40h and
so forth.
The above setting of \texttt{interval} now causes a mean value over the values
at 00:00h, 00:40h, 01:20h for the tracer stream, over the values at 
00:40h, 01:20h, 02:00h for all other streams. When you specify 
\texttt{interval = 2,'hours','last',0}, the mean values are taken over values at
00:40h, 01:20h, 02:00h for the tracer stream and at 01:20h, 02:00h, 02:40h
for all other streams. This is due to the organization of the time integration
in \echam. In general, this is not very important for calculating mean values
over a month or so.

You should also be careful in changing your mean value calculation interval
in combination with reruns. Assume that you interrupt your model writing 
rerun files every month but that your mean value interval is 2 months. 
Then, between two output intervals of your mean values, the rerun file
for the mean value streams contains the accumulated values of one month,
this means the sum over the instantaneous values multiplied by the
time step length.
If you now decide to change to daily meanvalues for example, the large already
over one month accumulated value of each variable is taken, further 
instantaneous values accumulated until the end of a day and then this value
is devided by the number of seconds of the new mean value calculation interval
of one day. This means that you will end up with a erroneous much too high 
resulting ``mean value''.

\end{description}



{\bf Restrictions:}

\begin{enumerate}

\item In \echam, the current maximum number of streams is 50.
      Each stream for which you require a mean value calculation is doubled,
      so that you have two streams for each one in the above \code{source} list:
      the original one and the mean value stream. Furthermore, only
      30~different (repeated) events are allowed in \echam. 

\item
\label{item:applicable_vars}

Variables all have to be on a Gaussian grid or in spectral space,
either two dimensional or three dimensional, or land (JSBACH) variables.
If the variables have the
{\tt laccu} flag set to {\tt .true.} they are only referenced if the output
interval of the respective mean value stream and the stream of origin
are identical. Otherwise they will be automatically skipped from the list.
For variables that have {\tt laccu=.true.} in their original stream,
no means of the squares can be calculated.

\item

The variable names, full names, and units have to meet
length restrictions that are somewhat more restrictive than the normal \echam{} 
restrictions. This is a consequence of the fact that
new names and units are given to the averaged
variables. The new names are chosen as follows
\begin{description}
\item [name:] The name of the mean value of a variable is the same as
  the name of the original (instantaeous) variable. 
For the mean of the
square {\tt \_s} is 
added at the end of the variable name. Consequently, if the mean of the 
square is desired, the variable name has to be 2 characters shorter than the
allowed maximum specified in \echam.
\item [full name:] Same as for name (relevant for tracer stream only).
\item [unit:] Units of mean values are unchanged of course, but in the case of
mean values of the square {\tt \it unitchar} is replaced by 
{\tt ({\it unitchar})**2} so that units have to be 5 characters shorter than
the maximum allowed by \echam{} if mean values of the square are required.
\end{description}

\item If \texttt{target} is not set,
      the length of \texttt{source} must 
      allow for an additional {\tt 'm'}.

\item If \texttt{filetag} is not set, the length of \texttt{target}
  must not exceed the maximum length of \texttt{filetag(len=7)}. 
\end{enumerate}

{\bf Backwards compatibility:}

Before \echam{} version~1.03, the namelist group \CODE{mvstreamctl} only defined
the source streams, using \code{m\_stream\_name} instead of \code{source}.
Other settings, namely \code{putmean} (same as \code{interval}),
\code{meannam}, and \code{stddev} (replaced by \code{sqrmeannam})
were to be put into a namelist group \CODE{mvctl} stored in a separate
namelist file named
\code{\cvar{streamname}.nml}.
For compatibility reasons, these are still recognised,
so old setups will continue to work.

Note though, that if you additionally use the new variables \code{interval}
or \code{meannam} of \CODE{mvstreamctl}, a warning will appear,
and the  \CODE{mvstreamctl} settings will override any settings from
\code{\cvar{streamname}.nml} to avoid inconsistencies.

{\bf New features and migration hints:}

\begin{itemize}
\item resulting stream may be renamed by setting \code{target}
\item file name suffix may be set using \code{filetag};
      an underscore (\_) is prepended automatically
\item to request all variables of a stream, simply omit the
  \code{meannam} element; 
      setting it to an empty string ('') or '*' has the same effect
\item for \CODE{mvstreamctl}, 
      \code{stddev} has been replaced by \code{sqrmeannam}.
      It takes variable names instead of numeric flags,
      to allow for a more direct and -- if only a few square means are needed --
      a more concise definition of those variables.
      \code{stddev = -1} is now \code{sqrmeannam ='*'}
\end{itemize}

The relation between old and new variables in the namelist group {\tt
  mvstreamctl} and {\tt mvctl} is summarized below.

%\vspace{\baselineskip}
\begin{tabular}{lll} \hline
\textbf{mvstreamctl (new)} & \textbf{mvstreamctl (old)} & \textbf{mvctl} \\\hline
source & m\_stream\_name & + '.nml' \emph{as file names} \\
target & m\_stream\_name(i) + 'm' & \\
interval & & putmean \\
filetag & '\_' + m\_stream\_name(i) + 'm' & \\
meannam & & meannam \\
meannam \emph{not set,} = ''\emph{, or} = '*' & & meannam = 'all' \\
sqrmeannam & & stddev \\
sqrmeannam = 'var1', 'var4', \dots & & stddev = 1, 0, 0, 1, \dots \\
sqrmeannam = '*' & & stddev = -1 \\\hline
\end{tabular}
%\vspace{\baselineskip}



\subsubsection{Namelist {\tt ndgctl}\index{namelists!ndgctl}}

This namelist controls all variables that are relevant for nudging,
i.e.~relevant for a simulation mode in which the spectral
3d--temperature, vorticity, 
divergence, surface pressure, and surface temperature can be constrained
to external fields obtained e.g.~from the assimilation of
observations. It has to be underlined that constraining the surface
temperature may lead to wrong sea ice coverage since the presence of
sea ice is diagnosed from the surface temperature directly without
taking into account any hysteresis effects (see Appendix~\ref{cr20120806}).

\setlength{\LTcapwidth}{\textwidth}
\setlength{\LTleft}{0pt}\setlength{\LTright}{0pt}

\begin{longtable}{l@{\extracolsep\fill}lp{7cm}p{3.5cm}}\hline\hline
\caption[Namelist {\tt ndgctl}]{Namelist 
  {\tt ndgctl}}\\\hline\label{tabndgctl}
\endfirsthead
\caption[]{{\tt ndgctl} --- continued}\\\hline
\endhead
\hline\multicolumn{4}{r}{\slshape table continued on next page}\\
\endfoot
\hline %\multicolumn{4}{r}{end of table}
\endlastfoot
Variable & type & Explanation & default \\\hline
{\tt dt\_nudg\_start(1:6)}\index{namelist variables!dt\_nudg\_start}
 & integer    & defines the beginning of the nudging 
 in the experiment. Is of the form yy,mo,dy,hr,mi,se (year, month,
 day, hour, minute, second) & {\tt 0,0,0,0,0,0} \\
{\tt dt\_nudg\_stop(1:6)}\index{namelist variables!dt\_nudg\_stop}
 & integer     &  defines the
  date at which nudging stops in a simulation. Is of the form
  yy,mo,dy,hr,mi,se (year, month, 
  day, hour, minute, second) & {\tt 0,0,0,0,0,0} \\
{\tt inudgformat}\index{namelist variables!inudgformat}
 & integer & format of nudging input files \newline
${\tt inudgformat}=0$: old CRAY format input files \newline
${\tt inudgformat}=2$: netcdf format input file & 0 \\
{\tt ldamplin}\index{namelist variables!ldamplin}
 & logical & linear damping (${\tt ldamplin}={\tt
 .true.}$) or damping with a parabolic function (${\tt ldamplin}={\tt
 .false.}$) 
 of the nudging efficiency
 between two synoptic times at
 which nudging data sets are given
 & {\tt .true.} \\
{\tt lnudgdbx}\index{namelist variables!lnudgdbx}
 & logical  & {\tt .true.} for additional diagnostic output
 about nudging, {\tt .false.} otherwise& {\tt .false.}\\
{\tt lnudgcli}\index{namelist variables!lnudgcli}
 & logical & ${\tt lnudgcli}={\tt .true.}$: \echam{}
 ignores the information about the year in the 
 nudging data file and reads nudging data in a cyclic way. Consequently,
 for each model year, the same nudging data are read.\newline
 ${\tt lnudgcli}={\tt .false.}$: The information about the year is
 included in the 
 nudging procedure, the data to which the model is constrained depend
 on the year. & {\tt .false.} \\
{\tt lnudgfrd}\index{namelist variables!lnudgfrd}
 & logical  & ${\tt lnudgfrd}={\tt .true.}$: normal
 mode filtering is done 
 at reading the data\newline 
 ${\tt lnudgfrd}={\tt .false.}$: normal mode filtering is done
 elsewhere. Works only together with {\tt lnmi=.true.}& {\tt .false.}\\
{\tt lnudgimp}\index{namelist variables!lnudgimp}
 &  logical & ${\tt lnudgimp}= {\tt .true.}$: implicit
 nudging\newline 
 ${\tt lnudgimp}={\tt .false.}$:
 explicit nudging & {\tt .true.} \\
{\tt lnudgini}\index{namelist variables!lnudgini} 
 & logical & ${\tt lnudgini}={\tt .false.}$: \echam{}
 starts or restarts a simulation for 
 a certain experiment from the date given in the namelist by {\tt
 dt\_start} or the restart date in the restart file\newline
 ${\tt lnudgini}={\tt .true.}$: If ${\tt lresume}={\tt .false.}$, the model
 starts the simulation at the
 date of the first nudging data set being in the nudging files the names of
 which correspond to {\tt dt\_nudge\_start}. There must be nudging
 files having a file name corresponding to {\tt dt\_nudge\_start}.
 If {\tt lresume=.true.}, the model starts its run at the first date
 being in the nudging data files the file names of which correspond to
 the {\tt next\_date} (next time step) of the rerun date. & {\tt
  .false.} \\
{\tt lnudgpat}\index{namelist variables!lnudgpat}
 &  logical & ${\tt lndgpat}={\tt .true.}$: pattern
 nudging. Does not work properly, to be removed\newline
 ${\tt lndgpat}= {\tt .false}$:
 otherwise& {\tt .false.}\\
{\tt lnudgwobs}\index{namelist variables!lnudgwobs}
 & logical & {\tt .true.} for storing additional nudging
 reference fields, {\tt .false.} otherwise& {\tt .false.}\\ 
{\tt lsite}& logical  &  switches on/off the Systematic Initial
Tendency Error diagnostic&{\tt .false.}\\ 
{\tt ltintlin}&logical   & ${\tt ltintlin}={\tt .true.}$: linear  time
 interpolation\newline 
 ${\tt ltintlin}= {\tt .false.}$ for cubic spline
 time interpolation between two  
 synoptic times at which nudging data sets are given& {\tt .true.}\\
{\tt ndg\_file\_div(256)}\index{namelist variables!ndg\_file\_div}
 & character & file name template for the file
 containing the nudging data for the divergence & --- \\
{\tt ndg\_file\_nc(256)}\index{namelist variables!ndg\_file\_nc}
 & character & file name template for netcdf
format file
containing all nudging data (temperature, logarithm of surface
pressure, divergence and vorticity)& --- \\
{\tt ndg\_file\_sst(256)}\index{namelist variables!ndg\_vile\_stt}
 & character & file name template for file
 containing the sea surface temperature& --- \\ 
{\tt ndg\_file\_stp(256)}\index{namelist variables!ndg\_file\_stp}
 & character & file name template for the file
containing the nudging data for the temperature and the logarithm of
the surface pressure & --- \\
{\tt ndg\_file\_vor(256)}\index{namelist variables!ndg\_file\_vor} 
 & character &  file name template for the file
 containing the nudging data for the vorticity & --- \\
{\tt ndg\_freez}\index{namelist variables!ndg\_freez}
 & real & temperature at which sea water is
 assumed to freeze in Kelvin& {\tt 271.65}\\
{\tt nsstinc}\index{namelist variables!nsstinc} 
 & integer            & treatment of the sea surface
 temperature (sst): read new sst data set each
 {\tt nsstinc} hours. A value of 0 means that sst is not used and
 prevents the model to produce too low sea ice coverage when nudging
 since sea ice would be detected only if temperatures drop below {\tt
   ndg\_freez}  & {\tt 0} \\
{\tt nsstoff}\index{namelist variables!nsstoff}
 & integer   & read the first sst data at hour {\tt nsstoff}
 after the beginning of the nudging & {\tt 12}\\
{\tt nudgd(1:80)}\index{namelist variables!ndgd}
 &  real & the relaxation time for each model layer
 for the nudging of the spectral divergence is given by $1/({\tt
   nudgd}\times 10^{-5})s$. Note the maximum of 80 layers!
 & {\tt 0.5787(1:80)}/s
 corresponding to 48 hours\\ 
{\tt nudgdamp}\index{namelist variables!ndgdamp}
 & real  & the nudging between two synoptic times
 will be reduced to {\tt nudgdamp}. Consequently, {\tt nudgdamp}=1.0
 means that nudging will be effective at 100\% at every time step, {\tt
 nudgdamp}=0.0 means that the nudging will be switched off somewhere
 between two synoptic times at which nudging data are available
  & {\tt 1.0}\\
{\tt nudglmax}\index{namelist variables!ndglmax}
 & integer         & highest index of the model layer at
 which nudging 
 is performed. Note the maximum of 80 layers! & {\tt 80}\\
{\tt nudglmin}\index{namelist variables!ndglmin}
 & integer  &  lowest index of the model layer at which nudging is
 performed& {\tt 1}\\
{\tt nudgp}\index{namelist variables!nudgp}
 & real     & the relaxation time for the nudging of
 the logarithm of the surface pressure is given by $1/({\tt
 nudgp}\times10^{-5})\,{\rm s}$& {\tt 1.1574}/s correspondig to 24 hours\\
{\tt nudgdsize}\index{namelist variables!nudgdsize}
 & real & fraction of the synoptic time interval after
 which only the fraction {\tt nudgdamp} is applied in the nudging
 procedure. If ${\tt nudgdsize}<0.5$, the minimum is reached after a
 fraction of {\tt nudgdsize} of the synoptic time interval. This
 minimum nudging level is then maintained until the model time reaches
 the next synoptic time step minus the fraction {\tt nudgdsize} of the
 synoptic time interval. Then, the nudging strength is starting to
 increase again & {\tt 0.5}\\
{\tt nudgsmax}\index{namelist variables!ndgsmax}
 & integer  & highest nudged wavenumber. Note the
restriction to model resolution not higher than T106! & {\tt 106}\\
{\tt nudgsmin}\index{namelist variables!nudgsmin}
 & integer  & Index of lowest nudged wavenumber minus one. This
 means, that with ${\tt nudgsmin}=0$, the spectral coefficient 0 (global
 average) is not nudged& {\tt 0}\\
{\tt nudgt(1:80)}\index{namelist variables!nudgt}
 & real & the relaxation time for each model layer for
 the nudging of the spectral temperature is given by $1/({\tt
   nudgt}\times10^{-5})s$. Note the maximum of 80
 layers! & {\tt 1.1574(1:80)}/s
 corresponding to 24 hours\\ 
{\tt nudgtrun}\index{namelist variables!nudgtrun}
 & integer  & mode of selection of spectral coefficients
  for nudging (see {\tt mo\_nudging\_init.f90}). The spectral
  coefficients of any quantity in spectral space are characterized by
  two indices $(n,m)$ 
  associated with zeros of the spherical harmonics $Y_n^m$ in longitudinal
  direction (index $m$) and latitudinal direction (index $n$). Let $L$
  be the spectral model resolution characterized by the maximum $n$
  and $\tilde{L}$ the maximum spectral resolution to which nudging has
  to be applied ($\tilde{L}$ can be set by the namelist parameter {\tt
    nudgsmax}). If one
  sets ${\tt nudgtrun}={\tt
    NDG\_WINDOW\_ALL}=0$, all spectral coefficients to the
  maximum possible $m=L$ are used even if $\tilde{L}<L$. For ${\tt
    nudgtrun}={\tt NDG\_WINDOW\_CUT}=1$, $m\le n\le\tilde{L}$ is
  chosen, even if $\tilde{L}<L$. If ${\tt nudgtrun}={\tt
    NDG\_WINDOW\_CUT0}=2$, all spectral coefficients are nudged as
  with ${\tt nudgtrun}=1$ but for $m=0$ ALL coefficients up to $n=L$
  are used. & {\tt 0}\\
{\tt nudgv(1:80)}\index{namelist variables!nudgv}
 & real & the relaxation time for each model layer
 for the nudging of the spectral vorticity is given by $1/({\tt
   nudgv}\times 10^{-5})s$. Note the maximum of 80
 layers! & {\tt 4.6296(1:80)}/s
 corresponding to 6 hours\\ \hline
\end{longtable}

\subsubsection{Namelist {\tt new\_tracer}\index{namelists!new\_tracer}}

This namelist allows to declare new transported tracers in \echam{}
without the direct use of the ``tracer interface''.
However, in most cases, tracers belong to submodels and will be declared by
them. Often the processes that modify the (mass) mixing ratios of
tracers other than transport by advection, diffusion, convection
and a constant decay (radioactive decay) in the atmosphere are very
complex and need to be programmed in special subroutines. If
transport and some kind of radioactive decay are the only processes
that are relevant for changes of the (mass) mixing ratio of a tracer,
this namelist is sufficient. Any quantity proportional to the mass
mixing ratio can be transported. This namelist can be specified
several times in the namelist file {\tt namelist.echam} for the
definition of more tracers than one.

\setlength{\LTcapwidth}{\textwidth}
\setlength{\LTleft}{0pt}\setlength{\LTright}{0pt}

\begin{longtable}{l@{\extracolsep\fill}lp{7cm}p{3.5cm}}\hline\hline
\caption[Namelist {\tt new\_tracer}]{Namelist 
  {\tt new\_tracer}}\\\hline\label{tabnewtracer}
\endfirsthead
\caption[]{{\tt new\_tracer} --- continued}\\\hline
\endhead
\hline\multicolumn{4}{r}{\slshape table continued on next page}\\
\endfoot
\hline %\multicolumn{4}{r}{end of table}
\endlastfoot
Variable & type & Explanation & default \\\hline
{\tt bits}\index{namelist variables!bits}
 & integer & number of bits used in GRIB encoding & 16 \\
{\tt code}\index{namelist variables!code}
 & integer & GRIB code number of tracer in GRIB format
output file & 0 \\
{\tt nconv}\index{namelist variables!nconv}
 & integer & transport by convection switched on ({\tt
  nvdiff=ON=1}) or switched off ({\tt nvdiff=OFF=0}) & {\tt ON}\\
{\tt name(24)}\index{namelist variables!name} 
 & character & name of tracer & --- \\
{\tt ninit}\index{namelist variables!ninit}
 & integer & initialization flag, for a detailed
explanation, see the lecture ``Working with \echam{} --- a first
introduction''. {\tt RESTART + CONSTANT = 1 + 2 = 3} means that the
tracer (mass) 
mixing ratio will be read from a restart file if there is any or set
to a constant throughout the atmosphere & {\tt RESTART}+{\tt CONSTANT} \\
{\tt nint}\index{namelist variables!nint}
 & integer & time integration switched on ({\tt
  nint=ON=1}) or switched off ({\tt nint=OFF=0}) & {\tt ON}\\
{\tt nrerun}\index{namelist variables!nrerun}
 & integer & restart flag. {\tt ON=1} means that the tracer
(mass) mixing ratio is written to the restart file, {\tt OFF=0} that it
is not written to the restart file & {\tt ON} \\
{\tt ntran}\index{namelist variables!ntran}
 & integer & transport flag. {\tt ntran=no\_advection=0}:
advective transport switched off, {\tt ntran=semi\_lagrangian=1}:
semi--Lagrangian transport scheme (based on a version of Olson \&
Rosinski), {\tt ntran=tpcore=3}: multi--dimensional 
flux form semi-Lagrangian (FFSL) scheme (Lin and Rood 1996, Monthly
Weather Review) with many unpublished modifications  & {\tt tpcore} \\
{\tt nvdiff}\index{namelist variables!nvdiff}
 & integer & transport by vertical diffusion switched on ({\tt
  nvdiff=ON=1}) or switched off ({\tt nvdiff=OFF=0}) & {\tt ON}\\
{\tt nwrite}\index{namelist variables!nwrite}
 & integer & {\tt ON=1}: tracer (mass) mixing ratio is
written to output file {\tt $\ast$\_tracer}; {\tt OFF=0}: no output &
{\tt ON}\\
{\tt table}\index{namelist variables!table} & integer & GRIB code table number in GRIB format output
file & 131 \\
{\tt tdecay}\index{namelist variables!tdecay}
 & real & decay time in seconds (exponential decay) & {\tt
  0.e0}\\
{\tt units(24)}\index{namelist variables!units}
 & character & units of tracer  & --- \\
{\tt vini}\index{namelist variables!vini}
 & real & Performing an initial run, the tracer (mass) mixing
ratio will be set to this value at the beginning of the time
integration & {\tt 0.e0}\\
\hline
\end{longtable}

\subsubsection{Namelist {\tt nmictl}\index{namelists!nmictl}}

This is the namelist to control the normal mode analysis.

\setlength{\LTcapwidth}{\textwidth}
\setlength{\LTleft}{0pt}\setlength{\LTright}{0pt}

\begin{longtable}{l@{\extracolsep\fill}lp{7cm}p{3.5cm}}\hline\hline
\caption[Namelist {\tt nmictl}]{Namelist 
  {\tt nmictl}}\\\hline\label{tabnmictl}
\endfirsthead
\caption[]{{\tt nmictl} --- continued}\\\hline
\endhead
\hline\multicolumn{4}{r}{\slshape table continued on next page}\\
\endfoot
\hline %\multicolumn{4}{r}{end of table}
\endlastfoot
Variable & type & Explanation & default \\\hline
{\tt dt\_nmi\_start(1:6)}\index{namelist variables!dt\_nmi\_start}
 & integer & start date of the NMI procedure.
  Is of the form yy,mo,dy,hr,mi,se (year, month,
 day, hour, minute, second)& {\tt 0,0,0,0,0,0}\\
{\tt lnmi\_cloud}\index{namelist variables!lnmi\_cloud}
 & logical & run initialization including clouds &
  {\tt .TRUE.} \\
{\tt ntdia}\index{namelist variables!ntdia}
 & integer & number of time steps of accumulation interval
  for diabatic tendencies & {\tt 8}\\
{\tt ntiter}\index{namelist variables!ntiter}
 & integer & number of time steps in an iteration interval
  & {\tt 2}\\
{\tt ntpre}\index{namelist variables!ntpre}
 & integer & number of time steps of pre--integration
  interval & {\tt 2} \\
{\tt pcut}\index{namelist variables!pcut}
 & real & cut off period in hours (used for nudging)
  & {\tt 12.0} \\
{\tt pcutd}\index{namelist variables!pcutd}
 & real & cut off period in hours (used for
  initialization) & {\tt 6.0} \\\hline
\end{longtable}


\subsubsection{Namelist {\tt parctl}\index{namelists!parctl}}

This namelist controls the parallelization of the \echam{} program.

\setlength{\LTcapwidth}{\textwidth}
\setlength{\LTleft}{0pt}\setlength{\LTright}{0pt}

\begin{longtable}{l@{\extracolsep\fill}lp{7cm}p{3.5cm}}\hline\hline
\caption[Namelist {\tt parctl}]{Namelist 
  {\tt parctl}}\\\hline\label{tabparctl}
\endfirsthead
\caption[]{{\tt parctl} --- continued}\\\hline
\endhead
\hline\multicolumn{4}{r}{\slshape table continued on next page}\\
\endfoot
\hline %\multicolumn{4}{r}{end of table}
\endlastfoot
Variable & type & Explanation & default \\\hline
{\tt db\_host(132)}\index{namelist variables!db\_host}
 & character & hostname to display time performance profiles of a
 model run on a special server: {\tt
   http://jobs-mpi.zmaw.de/index.php}
 The value of {\tt db\_host} is machine dependent as follows:
 \blizzard: {\tt 'plogin1'} \newline
 ZMAW computers: {\tt 'jobs-mpi.zmaw.de'}\newline
 In order to use this feature, set ${\tt ltimer}={\tt .true.}$ in the
 {\tt runctl} namelist group (see Tab.~\ref{tabrunctl}) & {\tt ''}\\
{\tt iomode}\index{namelist variables!iomode}& integer & I/O mode for
parallel output. On \blizzard{} ${\tt iomode}=2$ should be used,
on \thunder{} the faster ${\tt iomode}=1$ is also possible & {\tt 0}\\
{\tt lyaxt\_transposition}\index{namelist
  variables!lyaxt\_transposition} & logical & switch on/off the use of
  YAXT library for transpositions. This is currently unsupported. & 
  {\tt .false.}\\
{\tt network\_logger(132)}\index{namelist variables!network\_logger}
 & character & network logger for TCP/IP based debugging
  & {\tt ''} \\ 
{\tt nproca}\index{namelist variables!nproca}
 & integer & number of processors for set A division of earth &
  {\tt 1} \\
{\tt nprocb}\index{namelist variables!nprocb}
 & integer & number of processors for set B division of earth &
  {\tt 1}\\
{\tt nprocio}\index{namelist variables!nprocio}
 & integer & number of processors used for parallel~I/O.
 For serial~I/O, choose ${\tt nprocio}=0$.
 
 On \blizzard{}
 ${\tt nprocio}=32$, on \thunder{} ${\tt nprocio}=16$ is recommended &
 {\tt 0} \\
\hline
\end{longtable}


\subsubsection{Namelist {\tt physctl}\index{namelists!physctl}}

This namelist controls the physics calculations in \echam. These are
mainly calculations in the grid point space with parametrized
equations for convection, diffusion, gravity waves, and the exchange
of energy and mass at the surface of the earth.

\setlength{\LTcapwidth}{\textwidth}
\setlength{\LTleft}{0pt}\setlength{\LTright}{0pt}

\begin{longtable}{l@{\extracolsep\fill}lp{7cm}p{3.5cm}}\hline\hline
\caption[Namelist {\tt physctl}]{Namelist
  {\tt physctl}}\\\hline\label{tabphysctl}
\endfirsthead
\caption[]{{\tt physctl} --- continued}\\\hline
\endhead
\hline\multicolumn{4}{r}{\slshape table continued on next page}\\
\endfoot
\hline %\multicolumn{4}{r}{end of table}
\endlastfoot
Variable & type & Explanation & default \\\hline
{\tt iconv}\index{namelist variables!iconv}
 &   integer  &   switch for convection scheme:\newline
                         ${\tt iconv}=1$: Nordeng\newline
                         ${\tt iconv}=2$: Tiedtke\newline
                         ${\tt iconv}=3$: Hybrid                 &
                         {\tt 1} \\
{\tt icover}\index{namelist variables!icover}
 &   integer   & switch for cloud cover scheme:\newline
                         ${\tt icover}=1$: diagnostic (Sundqvist)
                         \newline
                         ${\tt icover}=2$: prognostic (Tompkins) &
                         {\tt 1} \\
{\tt lcdnc\_progn}\index{namelist variables!lcdnc\_progn}
 & logical & switches on/off prognostic cloud
  droplet number concentration & {\tt .false.} \\
{\tt lcond}\index{namelist variables!lcond}
 &   logical  &   switches on/off large scale condensation scheme &
{\tt .true.} \\    
{\tt lconv}\index{namelist variables!lconv}
 &   logical  &   switches on/off convection & {\tt .true.}\\
{\tt lconvmassfix}\index{namelist variables!lconvmassfix}
 & logical & switches on/off aerosol mass fixer in
  convection (obsolete?) & {\tt .true.} \\
{\tt lgwdrag}\index{namelist variables!lgwdrag}
 & logical  &   switches on/off gravity wave drag scheme of orographic
 gravity waves & {\tt
  .true.} \\  
{\tt lice}\index{namelist variables!lice} 
 &    logical  &   switches on/off sea--ice temperature calculation &
  {\tt .true.} \\
{\tt lice\_supersat}\index{namelist variables!lice\_supersat}
 & logical & switches on/off saturation over ice
for cirrus clouds (former ${\tt icnc} = 2$) & {\tt .false.} \\
{\tt lmfpen}\index{namelist variables!lmfpen}
 &  logical  &   switches on/off penetrative convection & {\tt
  .true.} \\
{\tt lphys}\index{namelist variables!lphys}
 &   logical &   switches on/off the parameterisation of
  diabatic processes   & {\tt .true.} \\
{\tt lrad}\index{namelist variables!lrad}
 &     logical &   switches on/off radiation calculation & {\tt
  .true.} \\
{\tt lsurf}\index{namelist variables!lsurf}
 &   logical  &   switches on/off surface--atmosphere exchanges &
  {\tt .true.}\\
{\tt lvdiff}\index{namelist variables!lvdiff}
 &   logical &   switches on/off vertical diffusion processes &
  {\tt .true.} \\
{\tt nauto}\index{namelist variables!nauto}
 &integer&  autoconversion scheme (not yet implemented) &
{\tt 1} \\
{\tt ncd\_activ}\index{namelist variables!ncd\_activ}
 & integer & type of cloud droplet activation scheme
(not yet implemented) & {\tt 0} \\
\hline
\end{longtable}



\subsubsection{Namelist {\tt radctl}\index{namelists!radctl}}

The namelist {\tt radctl} controls the radiation calculation, in
particular the frequency of the calls of the full radiation scheme,
and which greenhouse gas concentrations and aerosol properties are
taken into account. See the scientific documentation of \echam{} for
futher details. For some namelist variables, special documentation
exists and can be provided by S.~Rast (sebastian.rast@zmaw.de):
3d-ozone climatology (Appendix~\ref{cr20100408}), CO$_2$ submodel 
(Appendix~\ref{cr20091210}), 
stratospheric aerosols by T.~Crowley or HAM
(Appendix~\ref{cr20110323}), 
tropospheric aerosols by S.~Kinne
(Appendix~\ref{cr20090109}),
variable solar irradiance (Appendix~\ref{cr20100401}), volcanic aerosols by
G.~Stenchikov (Appendix~\ref{cr20100315}).

\setlength{\LTcapwidth}{\textwidth}
\setlength{\LTleft}{0pt}\setlength{\LTright}{0pt}

\begin{longtable}{l@{\extracolsep\fill}lp{7cm}p{3.5cm}}\hline\hline
\caption[Namelist {\tt radctl}]{Namelist
  {\tt radctl}}\\\hline\label{tabradctl}
\endfirsthead
\caption[]{{\tt radctl} --- continued}\\\hline
\endhead
\hline\multicolumn{4}{r}{\slshape table continued on next page}\\
\endfoot
\hline %\multicolumn{4}{r}{end of table}
\endlastfoot

Variable & type & Explanation & default \\\hline
{\tt cecc}\index{namelist variables!cecc}
 &real & eccentricity of the orbit of the earth& {\tt
  0.016715}\\ 
{\tt cfcvmr(1:2)}\index{namelist variables!cfcvmr}
 & real &  CFC volume mixing ratios for CFC11 and
  CFC12 if {\tt icfc=2} & ${\tt 252.8\times10^{-12},}$ ${\tt
    466.2\times10^{-12}}$\\ 
{\tt   ch4vmr}\index{namelist variables!ch4vmr}
 & real & CH$_4$ volume mixing ratio (mole fraction)
  for {\tt ich4=2,3} & ${\tt 1693.6\times10^{-9}}$ \\
{\tt clonp}\index{namelist variables!clonp}
 &real &longitude of perihelion measured from vernal
  equinox in degrees&{\tt 282.7}\\
{\tt   co2vmr}\index{namelist variables!co2vmr}
 & real & CO$_2$ volume mixing ratio (mole fraction)
  for {\tt ico2=2} & ${\tt 353.9\times10^{-06}}$ \\
{\tt cobld}\index{namelist variables!cobld}
 &real & obliquity of the orbit of the earth in
  degrees&{\tt 23.441}\\ 
{\tt fco2}\index{namelist variables!fco2}
  &real & if an external co2 scenario ($\tt ighg=1$ and
$\tt ico2=4$) is used, the CO$_2$ concentrations are multiplied by {\tt
  fco2} & {\tt 1.}\\ 
{\tt iaero}\index{namelist variables!iaero}
 & integer & ${\tt iaero}= 0$: the aerosol concentrations
  are set to 
  zero  in the radiation computation \newline
  ${\tt  iaero} = 1$: prognostic aerosol of a submodel (HAM)\newline
  ${\tt  iaero} = 2$: climatological Tanre aerosols\newline
  ${\tt  iaero} = 3$: aerosol climatology compiled by S.~Kinne \newline 
  ${\tt  iaero} = 5$: aerosol climatology compiled by S.~Kinne
  complemented with the volcanic aerosols of G.~Stenchikov \newline
  ${\tt  iaero} = 6$: aerosol climatology compiled by S.~Kinne
  complemented with the volcanic aerosols of G.~Stenchikov plus
  additional (stratospheric) aerosols from submodels like HAM. The
  additional aerosol optical properties are computed from effective
  radii and the aerosol optical depth at 550~nm, both quantities
  provided by external files with the help of a lookup table by
  S.~Kinne (b30w120), see Tab.~\ref{tabechamp} \newline
  ${\tt  iaero} = 7$: aerosol climatology compiled by S.~Kinne
  complemented by the volcanic aerosols by T.~Crowley that are
  computed using the lookup table by S.~Kinne (b20w120), see
  Tab.~\ref{tabechamp} \newline
  There is no ${\tt  iaero} = 4$. & 2\\
{\tt icfc}\index{namelist variables!icfc}
  & integer &   ${\tt icfc} = 0$:  all
  chloro--fluoro--carbon (CFC) concentrations are set to zero
  for the radiation computation\newline
  ${\tt icfc} = 1$: transported CFCs by any submodel (not yet
  implemented) \newline
  ${\tt icfc} = 2$: uniform volume mixing ratios as defined in the 2--element
  vector {\tt cfcvmr(1:2)} are used for CFC11 and CFC12, respectively\newline 
  ${\tt icfc} = 4$: uniform volume mixing ratios for a specific scenario
  defined by {\tt ighg} are used in the radiation computation & {\tt 2} \\
{\tt ich4}\index{namelist variables!ich4}
 &integer&${\tt ich4} = 0$: CH$_4$ concentration is set to
zero for the 
  radiation computation \newline 
  ${\tt ich4} = 1$: transported CH$_4$ by any submodel (not yet
  implemented) \newline
  ${\tt ich4} = 2$: uniform volume mixing ratio {\tt ch4vmr} of methane
  used in radiation 
  computation\newline 
  ${\tt ich4} = 3$: in the troposphere a volume mixing ratio {\tt ch4vmr} with
  a decay in the layers above the troposphere is used in the
  radiation computation\newline
  ${\tt ich4} = 4$: a uniform volume mixing ratio for a certain scenario
  defined by the parameter {\tt ighg} is used in the radiation
  computation & {\tt 3} \\ 
{\tt ico2}\index{namelist variables!ico2}
 & integer &  ${\tt ico2} = 0$: CO$_2$ concentration set to
zero for the
  radiation computation\newline 
  ${\tt ico2} = 1$: interactively calculated CO$_2$ volume mixing
  ratio is used with a start value of {\tt co2vmr} \newline
  ${\tt ico2} = 2$: uniform volume mixing ratio {\tt co2vmr} used in radiation
  computation\newline 
  ${\tt ico2} = 4$: uniform volume mixing ratio for a certain scenario run
  defined by the {\tt ighg} parameter is used & {\tt 2} \\
{\tt ighg}\index{namelist variables!ighg}
 & integer &     {$\tt ighg = 0$}: no specific scenario is
  chosen\newline 
                   {$\tt ighg = 1$}: a certain scenario of greenhouse
                   gas volume mixing ratios is used. {\bf Caution}:
                   the variables {\tt icfc}, {\tt ich4}, {\tt ico2},
                   {\tt in2o} have to be set to the values
                   corresponding to the usage of a scenario in that
                   case & 0\\
{\tt ih2o}\index{namelist variables!ih2o}
 & integer & ${\tt ih2o} = 0$: H$_2$O is not taken into account in the
  radiation computation, i.e. specific humidity, cloud water, cloud ice
  are all set to zero for the radiation computation\newline
  ${\tt ih2o} = 1$: use prognostic specific humidity, cloud water and cloud ice
  in radiation computation & {\tt 1} \\
{\tt in2o}\index{namelist variables!in2o}
  &integer&  ${\tt in2o} = 0$ : the N$_2$O concentration is
  set to zero for 
  the radiation computation \newline
  ${\tt in2o} = 1$: transported N$_2$O by any submodel (not yet
  implemented) \newline
  ${\tt in2o} = 2$: a uniform volume mixing ratio of {\tt n2ovmr} is
  used for
  the radiation computation \newline
  ${\tt in2o} = 3$: a uniform volume mixing ratio of {\tt n2ovmr} is used in
  the troposphere with a decay in the layers above the troposphere for
  the radiation computation\newline
  ${\tt in2o} = 4$: a uniform volume mixing ratio of N$_2$O for a specific
  scenario run defined by {\tt ighg} is used for the radiation
  computation & {\tt 3} \\
{\tt io3}\index{namelist variables!io3}
 & integer & ${\tt io3}  = 0$: the O$_3$ concentration is set
  to zero for
  the radiation computation\newline
  ${\tt io3} = 1$: transported O$_3$ by any submodel (not yet
  implemented) \newline
  ${\tt io3}  = 2$: climatological O$_3$ volume mixing ratios given in spectral
  space are used in the radiation computation as it was done in ECHAM4 \newline
  ${\tt io3}  = 3$: climatological O$_3$ volume mixing ratios given in
  gridpoint space in a NetCDF file are used in the radiation computation\newline
  ${\tt io3}  = 4$: climatological O$_3$ volume mixing ratios provided by the
  IPCC process in NetCDF files are
  used for the radiation calculation & {\tt 3} \\
{\tt io2}\index{namelist variables!io2}
 & integer & ${\tt io2}  = 0$: the O$_2$ concentration is set
  to zero for the radiation computation \newline
                     $ {\tt io2}  = 2$: the O$_2$ volume mixing ratio
                     is set to {\tt o2vmr} for the radiation
                     computation. & {\tt 2} \\
{\tt isolrad}\index{namelist variables!isolrad} & integer & controls choice of solar constant. \newline
  ${\tt isolrad} = 0$: standard rrtm solar constant \newline
  ${\tt isolrad} = 1$: time dependent spectrally resolved solar
   constant read from file \newline
  ${\tt isolrad} = 2$: pre--industrial solar constant \newline
  ${\tt isolrad} = 3$: solar constant for amip runs (fixed in time) &
  3 \newline
  ${\tt isolrad} = 4$: constant solar irradiation for
  radiative--convective equilibrium including a diurnal cycle.\newline
  ${\tt isolrad} = 5$: constant solar irradiation for
  radiative--convective equilibrium without diurnal cycle, so it's
  really constant in time here.\\
{\tt ldiur}\index{namelist variables!ldiur}
 & logical & switches on/off diurnal cycle & {\tt .true.} \\
{\tt lradforcing(2)}\index{namelist variables!lradforcing}
 & logical & switches on/off the diagnostic of
  instantaneous aerosol forcing in the solar spectral range ({\tt
  lradforcing(1)}) and the thermal spectral range ({\tt
  lradforcing(2)}). See Appendix~\ref{cr20100510} & {\tt .false.,.false.}\\
{\tt   n2ovmr}\index{namelist variables!n2ovmr}
 & real & N$_2$O volume mixing ratio (mole fraction)
  for {\tt in2o=2,3} & ${\tt 309.5\times10^{-9}}$ \\
{\tt lw\_gpts\_ts}\index{namelist variables!lw\_gpts\_ts} & integer &
number of g--points in Monte--Carlo 
spectral integration for thermal radiation, see {\tt lw\_spec\_samp} &
{\tt 1}\\
{\tt lw\_spec\_samp}\index{namelist variables!lw\_spec\_samp} &
integer & sampling of spectral bands in 
radiation calculation for thermal radiation\newline
  ${\tt lw\_spec\_samp} = 1$: standard broad band sampling \newline
  ${\tt lw\_spec\_samp} = 2$: Monte--Carlo spectral integration
  (MSCI); {\tt lw\_gpts\_ts} randomly chosen g--points per column and
  radiation call\newline
  ${\tt lw\_spec\_samp} = 3$: choose g--points not completely randomly
  in order to reduce errors in the surface radiative fluxes\\
  &{\tt 1}\\
{\tt nmonth}\index{namelist variables!nmonth}
 & integer & ${\tt nmonth}=0$: execute full annual cycles\newline
  ${\tt nmonth}=1,2,\dots,12$: perpetual repetition of the month
  corresponding to the number to which {\tt nmonth} is set. The
  perpetual month works with a 360--day orbit only ({\tt
  l\_orbvsop87=.false.} must be set in {\tt runctl}).& {\tt 0}\\
{\tt o2vmr}\index{namelist variables!o2vmr}
 & real & O$_2$ volume mixing ratio & {\tt 0.20946}\\
{\tt rad\_perm}\index{namelist variables!rad\_perm} & integer &
number that influences the perturbation of the random seed from column
to column & {\tt 0} \\
{\tt sw\_gpts\_ts}\index{namelist variables!sw\_gpts\_ts} & integer &
number of g--points in Monte--Carlo 
spectral integration for solar radiation, see {\tt sw\_spec\_samp} &
{\tt 1}\\
{\tt sw\_spec\_samp}\index{namelist variables!sw\_spec\_samp} &
integer & sampling of spectral bands in 
radiation calculation for thermal radiation\newline
  ${\tt sw\_spec\_samp} = 1$: standard broad band sampling \newline
  ${\tt sw\_spec\_samp} = 2$: Monte--Carlo spectral integration
  (MSCI); {\tt lw\_gpts\_ts} randomly chosen g--points per column and
  radiation call\newline
  ${\tt sw\_spec\_samp} = 3$: choose g--points not completely randomly
  in order to reduce errors in the surface radiative fluxes\\
  &{\tt 1}\\
{\tt trigrad}\index{namelist variables!trigrad}
 & special& time interval for radiation calculation &
  {\tt 2,'hours','first',0}\\ 
{\tt yr\_perp}\index{namelist variables!yr\_perp}
 & integer & year in the Julian calendar for perpetual
year simulations. Works with {\tt l\_orbvsop87=.true.} only. & {\tt
  -99999} \\\hline
\end{longtable}



\subsubsection{Namelist {\tt runctl}\index{namelists!runctl}}\label{secrunctl}

This namelist contains variables which control the start and end of a
simulation and general properties of the output. For some namelist
variables, special documentation 
exists and can be provided by S.~Rast (sebastian.rast@zmaw.de):
debug stream (Appendix~\ref{cr20090331}) and tendency diagnostic 
(Appendix~\ref{cr20110118}).

\setlength{\LTcapwidth}{\textwidth}
\setlength{\LTleft}{0pt}\setlength{\LTright}{0pt}

\begin{longtable}{l@{\extracolsep\fill}lp{7.cm}p{3.5cm}}\hline\hline
\caption[Namelist {\tt runctl}]{Namelist
  {\tt runctl}}\\\hline\label{tabrunctl}
\endfirsthead
\caption[]{{\tt runctl} --- continued}\\\hline
\endhead
\hline\multicolumn{4}{r}{\slshape table continued on next page}\\
\endfoot
\hline %\multicolumn{4}{r}{end of table}
\endlastfoot

Variable & type & Explanation & default \\\hline
{\tt default\_output}\index{namelist variables!default\_output}
 & logical & this variable sets the default value
of {\tt lpost} of any stream. If {\tt lpost} of a certain stream is
{\tt .TRUE.}, the respective variables of the stream will be written
to the respective output file. It can be used to switch off all
default output & {\tt .TRUE.}\\
{\tt delta\_time}\index{namelist variables!delta\_time}
 & integer & time step length in seconds & default depends
on model resolution, e.g.: T63L47: 600~s, T63L95: 450~s, T127L95: 240~s\\
{\tt dt\_resume(1:6)}\index{namelist variables!dt\_resume}
 & integer  & reset restart date to a user defined
 value. Is of the form yy,mo,dy,hr,mi,se (year, month,
 day, hour, minute, second) &
{\tt 0,0,0,0,0,0} \\
{\tt dt\_start(1:6)}\index{namelist variables!dt\_start}
 & integer  & vector of 6 integer numbers defining the start
date of the experiment of the form yy,mo,dy,hr,mi,se (year, month,
day, hour, minute, second) & {\tt 0,0,0,0,0,0} \\
{\tt dt\_stop(1:6)}\index{namelist variables!dt\_stop}
 & integer    & stop date of experiment. Is of the form
yy,mo,dy,hr,mi,se (year, month, 
 day, hour, minute, second) & {\tt 0,0,0,0,0,0} \\
{\tt earth\_angular\_velocity}\index{namelist
  variables!earth\_angular\_velocity} & real & value for solid earth angular
velocity in ${\rm rad/s}$ & {\tt 7.29212e-5}${\rm rad/s}$\\  
{\tt gethd}\index{namelist variables!gethd}
 & special      & time interval for getting data from
  hydrological discharge model & {\tt 1,'days','off',0}\\
{\tt getocean}\index{namelist variables!getocean}
 &special   & time interval for sending atmospheric data to an
  ocean program coupled to ECHAM5 & {\tt 1,'days','off',0}\\
{\tt iadvec}\index{namelist variables!iadvec}
 & integer     & selection of the advection scheme:\newline 
             ${\tt iadvec}=0$: no advection of trace species and water
             vapour\newline
             ${\tt iadvec}=1$: semi Lagrangian transport
             algorithm\newline
             ${\tt iadvec}=2$: spitfire advection scheme\newline
             ${\tt iadvec}=3$:
             flux form semi Lagrangian transport (Lin and Rood) &
             {\tt 3} -- flux form semi Lagrangian transport \\
{\tt l\_orbvsop87}\index{namelist variables!l\_orbvsop87}
 & logical & ${\tt l\_orbvsop87}={\tt .true.}$: use
orbit functions from 
  vsop87 (real orbit); ${\tt l\_orbvsop87}={\tt .false.}$:
  ``climatological'' pcmdi 
  (AMIP) orbit & {\tt .true.} \\
{\tt lfractional\_mask}\index{namelist variables!lfractional\_mask}
& logical & switches on/off the usage of a fractional land--sea mask &
{\tt .false.} \\
{\tt l\_volc}\index{namelist variables!l\_volc}
 & logical & switches on/off volcanic aerosols. This
variable is obsolete and has to be removed. Use {\tt iaero} of the
{\tt radctl} namelist instead. & {\tt .false.} \\
{\tt lamip}\index{namelist variables!lamip}
 & logical   & switches on/off the use of a timeseries of sea
  surface temperatures (AMIP style simulation) & {\tt .false.} \\
{\tt lcollective\_write} & logical & switch on/off parallel writing of restart
files & {\tt .false.} \\
{\tt lcouple}\index{namelist variables!lcouple} 
 & logical    & switches on/off coupling with ocean & {\tt
    .false.} \\
{\tt lcouple\_co2}\index{namelist variables!lcouple\_co2}
 & logical & switches on/off the interactive CO$_2$
budget calculation in a coupled atmosphere/ocean run & {\tt .false.}
\\
{\tt lcouple\_parallel}\index{namelist variables!lcouple\_parallel}
 & logical & Only if model was compiled with
{\tt --prism}: switches on/off communication by OASIS via all or one
processor & {\tt .false.}\\
{\tt ldailysst}\index{namelist variables!ldailysst}
 & logical   & switches on/off daily varying sea surface
temperature and sea ice & {\tt .false.}\\
{\tt ldebug}\index{namelist variables!ldebug}
 & logical     & switches on/off mass fixer diagnostics &
     {\tt .false.}\\
{\tt ldebugcpl}\index{namelist variables!ldebugcpl}
 & logical & switches on/off debugging of OASIS
coupling (only if model was compiled using {\tt --prism}) & {\tt
  .false.} \\
{\tt ldebugev}\index{namelist variables!ldebugev}
 & logical    & switches on/off the output of debugging
information about events & {\tt .false.}\\
{\tt ldebughd}\index{namelist variables!ldebughd}
 & logical     & switches on/off the output of
debugging information 
about the hydrological discharge model & {\tt .false.}\\
{\tt ldebugio}\index{namelist variables!ldebugio}
 & logical      & switches on/off the output of debugging
information about input and output & {\tt .false.}\\
{\tt ldebugmem}\index{namelist variables!ldebugmem}
 & logical     & switches on/off the output of debugging
information about memory use & {\tt .false.}\\
{\tt ldebugs}\index{namelist variables!ldebugs}
 & logical    & switches on/off the debug stream & {\tt
  .false.}\\
{\tt ldiagamip}\index{namelist variables!ldiagamip}
 & logical  & switches on/off AMIP diagnostics & {\tt
    .false. }\\
{\tt lforcererun}\index{namelist variables!lforcererun}
 & logical  & switches on/off forced re--initialization of the
 \echam{} run when restart files were written. Forced
 re--initialization makes the model behave like starting from restart
 files with {\tt lresume=.true.} & {\tt .true.} \\ 
{\tt lhd}\index{namelist variables!lhd}
 & logical        & switches on/off the coupling to the hydrologic
  discharge model (HD model) & {\tt .false.}\\
{\tt lhd\_highres}\index{namelist variables!lhd\_highres}
 & logical & switches on/off high resolution
(0.5$^\circ$) output of hydrological discharge model & {\tt .false.} \\
{\tt lhd\_que}\index{namelist variables!lhd\_que}
 & logical    &switches on/off additional output from
  hydrological discharge model & {\tt .false.}\\
{\tt lindependent\_read}\index{namelist variables!lindependent\_read}
 & logical & switches on/off reading initial or restart
data by each MPI rank separately & {\tt .false.}\\
{\tt lmeltpond}\index{namelist variables!lmeltpond}
 & logical & switches on/off the presence of meltponds
in albedo calculation & {\tt .true.} \\
{\tt lmidatm}\index{namelist variables!lmidatm}
 & logical    &  switches on/off middle atmosphere model
  version & {\tt .true.} \\
{\tt lmlo}\index{namelist variables!lmlo}
 & locical       & switches on/off mixed layer ocean & {\tt
    .false.} \\
{\tt lmlo\_ice}\index{namelist variables!lmlo\_ice} & logical & switch
on/off sea ice calculation for mixed layer ocean & {\tt .true.}\\
{\tt lnmi}\index{namelist variables!lnmi}
 & logical       & switches on/off normal mode initialisation &
  {\tt .false.}\\
{\tt lnudge}\index{namelist variables!lnudge}
 & logical     & switches on/off the ``nudging''
  i.e. constraining the dynamic variables divergence, vorticity,
  temperature, and surface pressure towards given external values by relaxation
  & {\tt .false.} \\
{\tt lnwp}\index{namelist variables!lnwp}
 & logical       & switches on/off Numerical Weather Prediction
  mode & {\tt .false.} \\
{\tt lport}\index{namelist variables!lport}
 & logical      & switches on/off the introduction of a random
  perturbation for portability  tests & {\tt .false.}\\
{\tt lprint\_m0}\index{namelist variables!lprint\_m0}
 & logical & switches on/off measuring and printing
the cpu time for every time step & {\tt .false.}\\
{\tt lrce}\index{namelist variables!lrce} & logical & switch on/off
radiation calculation for radiative--convective equilibrium (same
zenith angle at all grid points), constant
ocean albedo of 0.07, ignore dynamical planetary boundary layer height
in planetary boundary layer calculation & {\tt .false.} \\
{\tt lresume}\index{namelist variables!lresume}
 & logical & ${\tt lresume}={\tt .true.}$: perform a rerun
\newline ${\tt lresume}={\tt .false.}$: perform an initial run & {\tt
  .false.} \\
{\tt ltctest}\index{namelist variables!ltctest}
 & logical & switches on/off a test of time control
 without performing a true simulation& {\tt .false.}\\
{\tt ltdiag}\index{namelist variables!ltdiag}
 & logical     & switches on/off an additional detailed tendency
  diagnostic & {\tt .false.}\\
{\tt ltimer}\index{namelist variables!ltimer}
 & logical & switches on/off the output of some
performance related information (run time) & {\tt .false.}\\
{\tt ly360}\index{namelist variables!ly360}
 & logical& switches on/off the use of a 360--day year &
  {\tt .false.}\\ 
{\tt ndiahdf}\index{namelist variables!ndiahdf}
  & integer & logical unit number for file containing
horizontal diffusion diagnostics. & {\tt 10} \\
{\tt nhd\_diag}\index{namelist variables!ndg\_diag}
 & integer   & number of region for which hydrological
  discharge  model diagnostics is required & {\tt 0}\\
{\tt no\_cycles}\index{namelist variables!no\_cycles}
 & integer  & stop after {\tt no\_cycles} of reruns &
{\tt 1} \\
{\tt no\_days}\index{namelist variables!no\_days}
 & integer  & stop after {\tt no\_days} days after {\tt
    dt\_start}& {\tt -1}\\
{\tt no\_steps}\index{namelist variables!no\_steps}
 & integer  & stop after the integration of {\tt no\_steps}
  of time steps after {\tt dt\_start} & {\tt -1} \\
{\tt nproma}\index{namelist variables!nproma}
 &integer     & vector length of calculations in grid point
  space & number of longitudes \\
{\tt nsub}\index{namelist variables!nsub}
 & integer       & number of subjobs & {\tt 0} \\ 
{\tt out\_datapath}(256)\index{namelist variables!out\_datapath}
 & character   & name of path to which output
files are written. Must have a
``/'' at the end & {\tt ' '} \\
{\tt out\_expname}(19)\index{namelist variables!out\_expname}
 &character & prefix of output file names & {\tt ' '} \\
{\tt out\_filetype}\index{namelist variables!out\_filetype}
 &integer  & format of meteorological output files \newline
                        ${\tt out\_filetype}=1$:  GRIB format \newline 
                        ${\tt out\_filetype}=2$:  NetCDF format
                        \newline 
                        ${\tt out\_filetype}=6$:  NetCDF4 format &
                        {\tt 1} \\
{\tt out\_ztype}\index{namelist variables!out\_ztype}
 & integer & compression type of outputfiles\newline
                        ${\tt out\_ztype}=0$: no compression\newline
                        ${\tt out\_ztype}=1$: szip only for GRIB
                        output\newline
                        ${\tt out\_ztype}=2$: zip only for NetCDF4
                        output                                    &
                        {\tt 0} \\
{\tt putdata}\index{namelist variables!putdata}
 & special    & time interval at which output data are written
  to output files & {\tt 12,'hours','first',0} \\
{\tt puthd}\index{namelist variables!puthd}
 & special      & time interval for putting data to the
  hydrological discharge model & {\tt 1,'days','off',0}\\
{\tt putocean}\index{namelist variables!putocean}
 & special   & time interval for receiving ocean data in the
  atmospheric part if \echam{} is coupled to an ocean model & {\tt
    1,'days','off',0} \\
{\tt putrerun}\index{namelist variables!putrerun}
 & special  & time interval for writing rerun files &
  {\tt 1,'months','last',0}\\
{\tt rerun\_filetype}\index{namelist variables!rerun\_filetype}
 &integer & format of rerun files\newline
                        ${\tt rerun\_filetype}=2$: NetCDF format
                        \newline
                        ${\tt rerun\_filetype}=4$: NetCDF2 format &
                        {\tt 2}\\
{\tt rmlo\_depth}\index{namelist variables!rmlo\_depth} & real & depth of
mixed layer ocean in metres & {\tt 50}m \\
{\tt subflag(1:9)}\index{namelist variables!subflag}
 & logical &vector of nine switches for switching on/off the
  binding of subjob output to output streams & {\tt .false.}\\
{\tt trac\_filetype}\index{namelist variables!trac\_filetype}
 &integer  & format of tracer output files \newline
                        ${\tt trac\_filetype}=1$:  GRIB format \newline 
                        ${\tt trac\_filetype}=2$:  NetCDF format &
                        {\tt 1} \\
{\tt trigfiles}\index{namelist variables!trigfiles}
 & special  & time interval at which new output files are
  opened & {\tt 1,'months','first',0} \\
{\tt trigjob}(1:9)\index{namelist variables!trigjob}
 & special    & time interval for the automatic submission of
  up to nine subjobs &{\tt 1,'months','off',0} \\
\hline
\end{longtable}

\subsubsection{Namelist {\tt set\_stream}\index{namelists!set\_stream}}

This namelist allows to modify the properties of an existing stream. You
may overwrite output properties of an existing stream here. If a namelist
variable is not present or set to certain values, the original values
of these descriptor variables remain unchanged. In that case the
default is marked by ``original state''.
For
specifying these properties for several streams, this namelist group
has to be specified for each single 
stream.

\setlength{\LTcapwidth}{\textwidth}
\setlength{\LTleft}{0pt}\setlength{\LTright}{0pt}

\begin{longtable}{l@{\extracolsep\fill}lp{7cm}p{3.5cm}}\hline\hline
\caption[Namelist {\tt set\_stream}]{Namelist 
  {\tt set\_stream}}\\\hline\label{tabset_stream}
\endfirsthead
\caption[]{{\tt set\_stream} --- continued}\\\hline
\endhead
\hline\multicolumn{4}{r}{\slshape table continued on next page}\\
\endfoot
\hline %\multicolumn{4}{r}{end of table}
\endlastfoot
Variable & type & Explanation & default \\\hline
{\tt filetype}\index{namelist variables!filetype} 
& integer & file format of output file associated
with stream.  \newline
                        ${\tt out\_filetype}=1$:  GRIB format \newline 
                        ${\tt out\_filetype}=2$:  NetCDF format
                        \newline 
                        ${\tt out\_filetype}=6$:  NetCDF4 format &
                        original filetype \\
{\tt init\_suf(8)}\index{namelist variables!init\_suf}
 & character & suffix of file with initial data for
this stream & original suffix \\
{\tt interval}\index{namelist variables!interval}
 & special & output frequency & original state (also if
{\tt counter=0})\\
{\tt linit}\index{namelist variables!linit}
 & integer & switch on ({\tt linit=1}) or off ({\tt
  linit$\neq$1,-1}) writing stream to initial file (does not work?) &
original state \\
{\tt lpost}\index{namelist variables!lpost}
 & integer & switch on ({\tt lpost=1}) or off ({\tt
  lpost$\neq$1,-1}) output of stream. & original state  \\
{\tt lrerun}\index{namelist variables!lrerun}
 & integer & switch on ({\tt lrerun=1}) or off ({\tt
  lrerun$\neq$1,-1}) output of stream to restart file. & original state \\
{\tt post\_suf(8)} & character & suffix of output file associated with
this stream & original suffix \\
{\tt rest\_suf(8)}\index{namelist variables!rest\_suf}
 & character & suffix of restart file associated
with this stream & original suffix \\
{\tt stream(16)}\index{namelist variables!stream}
 & character & name of the stream the properties of
which shall be changed & --- \\
\hline
\end{longtable}


\subsubsection{Namelist {\tt
    set\_stream\_element}\index{namelists!set\_stream\_element}} 

This namelist allows to modify the properties of an existing stream
element. You may overwrite output properties of an existing stream
element here. If a namelist
variable is not present or set to certain values, the original values
of these descriptor variables remain unchanged.
For specifying these properties for several stream
elements, this namelist group has to be specified for each single
stream element.

\setlength{\LTcapwidth}{\textwidth}
\setlength{\LTleft}{0pt}\setlength{\LTright}{0pt}

\begin{longtable}{l@{\extracolsep\fill}lp{7cm}p{3.5cm}}\hline\hline
\caption[Namelist {\tt set\_stream\_element}]{Namelist 
  {\tt set\_stream\_element}}\\\hline\label{tabset_stream_element}
\endfirsthead
\caption[]{{\tt set\_stream\_element} --- continued}\\\hline
\endhead
\hline\multicolumn{4}{r}{\slshape table continued on next page}\\
\endfoot
\hline %\multicolumn{4}{r}{end of table}
\endlastfoot
Variable & type & Explanation & default \\\hline
{\tt bits}\index{namelist variables!bits}
 & integer & number of bits used in GRIB encoding & original
value\\
{\tt code}\index{namelist variables!code}
 & integer & GRIB code number of stream element in GRIB format
output file & original value \\
{\tt longname(64)}\index{namelist variables!longname}
 & character & long name of stream element
containing some explanation & original value \\ 
{\tt lpost}\index{namelist variables!lpost}
 & integer & switch on ({\tt lpost=1}) or off ({\tt
  lpost$\neq$1,-HUGE(lpost)}) output of stream element.  & original state  \\
{\tt lrerun}\index{namelist variables!lrerun}
 & integer & switch on ({\tt lrerun=1}) or off ({\tt
  lrerun$\neq$1,-HUGE(lrerun)}) output of stream element to restart
file. If {\tt 
  lrerun=1}, the {\tt 
  lrerun} element of a variable of type {\tt memory\_info} associated
with this stream element will be set
to {\tt .true.} & original state  \\
{\tt name(64)}\index{namelist variables!name}
 & character & name of stream element as it was used in
its declaration & --- \\ 
{\tt reset}\index{namelist variables!reset}
 & real & value to which stream element is set after output
if and only if {\tt laccu=.true.} for this stream element & original value \\
{\tt stream(64)}\index{namelist variables!stream}
 & character &name of stream to which the stream
element belongs & --- \\
{\tt table}\index{namelist variables!table}
 & integer & GRIB code table number in GRIB format output
file & original value \\
{\tt units(64)}\index{namelist variables!units}
 & character & units of the quantity described by stream
element  & original value \\ 
\hline
\end{longtable}


\subsubsection{Namelist {\tt set\_tracer}\index{namelists!set\_tracer}}

This namelist allows to modify the properties of a tracer that is defined
in some submodel or subroutine of \echam{}
without the direct use of the ``tracer interface''.
If a namelist
variable is not present or set to certain values, the original values
of these descriptor variables remain unchanged.
In order to set the properties of several tracers, you can specify this
namelist several times in the namelist file {\tt namelist.echam}.

\setlength{\LTcapwidth}{\textwidth}
\setlength{\LTleft}{0pt}\setlength{\LTright}{0pt}

\begin{longtable}{l@{\extracolsep\fill}lp{7cm}p{3.5cm}}\hline\hline
\caption[Namelist {\tt set\_tracer}]{Namelist 
  {\tt set\_tracer}}\\\hline\label{tabset_tracer}
\endfirsthead
\caption[]{{\tt set\_tracer} --- continued}\\\hline
\endhead
\hline\multicolumn{4}{r}{\slshape table continued on next page}\\
\endfoot
\hline %\multicolumn{4}{r}{end of table}
\endlastfoot
Variable & type & Explanation & default \\\hline
{\tt bits}\index{namelist variables!bits}
 & integer & number of bits used in GRIB encoding & original
value\\
{\tt code}\index{namelist variables!code}
 & integer & GRIB code number of tracer in GRIB format
output file & original value \\
{\tt nconv}\index{namelist variables!nconv}
 & integer & transport by convection switched on ({\tt
  nvdiff=ON=1}) or switched off ({\tt nvdiff=OFF=0}) & original value\\
{\tt name(24)}\index{namelist variables!name}
 & character & name of tracer as it was used in the
tracer declaration & --- \\
{\tt ninit}\index{namelist variables!ninit}
 & integer & initialization flag, for a detailed
explanation, see the lecture ``Working with \echam{} --- a first
introduction''. {\tt RESTART + CONSTANT = 1 + 2 = 3} means that the
tracer (mass) 
mixing ratio will be read from a restart file if there is any or set
to a constant throughout the atmosphere & original value \\
{\tt nint}\index{namelist variables!nint}
 & integer & time integration switched on ({\tt
  nint=ON=1}) or switched off ({\tt nint=OFF=0}) & original value\\
{\tt nrerun}\index{namelist variables!nrerun}
 & integer & restart flag. {\tt ON=1} means that the tracer
(mass) mixing ratio is written to the restart file, {\tt OFF=0} that it
is not written to the restart file & original value \\
{\tt ntran}\index{namelist variables!ntran}
 & integer & transport flag. {\tt ntran=no\_advection=0}:
advective transport switched off, {\tt ntran=semi\_lagrangian=1}:
semi--Lagrangian transport scheme (based on a version of Olson \&
Rosinski), {\tt ntran=tpcore=3}: multi--dimensional 
flux form semi-Lagrangian (FFSL) scheme (Lin and Rood 1996, Monthly
Weather Review) with many unpublished modifications  & original value \\
{\tt nvdiff}\index{namelist variables!nvdiff}
 & integer & transport by vertical diffusion switched on ({\tt
  nvdiff=ON=1}) or switched off ({\tt nvdiff=OFF=0}) & original value\\
{\tt nwrite}\index{namelist variables!nwrite}
 & integer & {\tt ON=1}: tracer (mass) mixing ratio is
written to output file {\tt $\ast$\_tracer}; {\tt OFF=0}: no output &
original value\\
{\tt table}\index{namelist variables!table}
 & integer & GRIB code table number in GRIB format output
file & original value \\
{\tt tdecay}\index{namelist variables!tdecay}
 & real & decay time in seconds (exponential decay) &
original value\\
{\tt units(24)}\index{namelist variables!units}
 & character & units of tracer  & original value \\
{\tt vini}\index{namelist variables!vini}
 & real & Performing an initial run, the tracer (mass) mixing
ratio will be set to this value at the beginning of the time
integration & original value \\
\hline
\end{longtable}

\subsubsection{Namelist {\tt stationctl}\index{namelists!stationctl}}

This namelist switches on/off the high frequency output of \echam{}
variables at the CFMIP2--sites including profiles. The collection of sites
is that of K.~Taylor that were used in the CMIP5 simulations and are
listed in the file {\tt pointlocations.txt} (see
Tab.~\ref{tabdiaginput}). A description of the output can be found in
section~\ref{secdiagoutput}. A more detailed description of the
station diagnostic can be found in section~\ref{cr20130913}.

\setlength{\LTcapwidth}{\textwidth}
\setlength{\LTleft}{0pt}\setlength{\LTright}{0pt}

\begin{longtable}{l@{\extracolsep\fill}lp{5.0cm}p{3.0cm}}\hline\hline
\caption[Namelist {\tt stationctl}]{Namelist
  {\tt stationctl}}\\\hline\label{tabstationctl}
\endfirsthead
\caption[]{{\tt stationctl} --- continued}\\\hline
\endhead
\hline\multicolumn{4}{r}{\slshape table continued on next page}\\
\endfoot
\hline %\multicolumn{4}{r}{end of table}
\endlastfoot
Variable & type & Explanation & default \\\hline
{\tt lostation}\index{namelist variables!lostation} & logical & logical that switches on ({\tt
  lostation=.true.} or off {\tt lostation=.false.} the output
of certain \echam{} variables at a collection of sites. At these
sites, profiles are also written to the output. & {\tt .false.}\\
\hline
\end{longtable}

\subsubsection{Namelist {\tt
    submdiagctl}\index{namelists!submdiagctl}}\label{secsubmdiagctl} 

This namelist controls diagnostic output of generic submodel variables
and streams. In the ``pure'' \echam{} version, these switches do not
have any functionality. 

\setlength{\LTcapwidth}{\textwidth}
\setlength{\LTleft}{0pt}\setlength{\LTright}{0pt}

\begin{longtable}{l@{\extracolsep\fill}lp{5.0cm}p{3.0cm}}\hline\hline
\caption[Namelist {\tt submdiagctl}]{Namelist
  {\tt submdiagctl}}\\\hline\label{tabsubmdiagctl}
\endfirsthead
\caption[]{{\tt submdiagctl} --- continued}\\\hline
\endhead
\hline\multicolumn{4}{r}{\slshape table continued on next page}\\
\endfoot
\hline %\multicolumn{4}{r}{end of table}
\endlastfoot
Variable & type & Explanation & default \\\hline
{\tt drydep\_gastrac(24,1:200)}
\index{namelist variables!drydep\_gastrac}
 & character & names of gas phase
  tracers to be included in
  dry deposition stream. Special name {\tt 'default'} is possible 
  & {\tt drydep\_gastrac(1) = 'default'}, {\tt drydep\_gastrac (2:200) = ''}\\
{\tt drydep\_keytype}\index{namelist variables!drydep\_keytype}
 & integer & aggregation level of
  output of dry deposition stream \newline
  {\tt drydep\_keytype=1}: output by tracer\newline
  {\tt drydep\_keytype=2}: output by (chemical) species\newline
  {\tt drydep\_keytype=3}: output by (aerosol) mode\newline
  {\tt drydep\_keytype=4}: user defined
  & {\tt 2}\\
{\tt drydep\_lpost}\index{namelist variables!drydep\_lpost}
 & logical & switches on/off output of wet deposition stream
  & {\tt .true.} \\
{\tt drydep\_tinterval}\index{namelist variables!drydep\_tinterval}
 & special & output frequency of wet deposition stream
  & {\tt putdata} (see {\tt runctl} namelist) \\
{\tt drydepnam(32,1:50)}\index{namelist variables!drydepnam}
 & character & list of tracer names of
  dry deposition output stream. There are the special names {\tt 'all' =
  'detail'},  and {\tt 
  'default'}
  & {\tt drydepnam(1) = 'default'}, {\tt drydepnam(2:50) = ''}\\
{\tt emi\_gastrac(24,1:200)}\index{namelist variables!emi\_gastrac}
 & character & names of gas phase
  tracers to be included in
  emission stream to diagnose emissions. Special name {\tt 'default'}
  is possible  
  & {\tt emi\_gastrac(1) = 'default'}, {\tt emi\_gastrac (2:200) = ''}\\
{\tt emi\_keytype}\index{namelist variables!emi\_keytype}
 & integer & aggregation level of
  output of emission diagnostic stream \newline
  {\tt emi\_keytype=1}: output by tracer\newline
  {\tt emi\_keytype=2}: output by (chemical) species\newline
  {\tt emi\_keytype=3}: output by (aerosol) mode\newline
  {\tt emi\_keytype=4}: user defined
  & {\tt 2}\\
{\tt emi\_lpost}\index{namelist variables!emi\_lpost}
 & logical & switches on/off output of emission
  diagnostic stream 
  & {\tt .true.} \\
{\tt emi\_lpost\_detail}\index{namelist variables!emi\_lpost\_detail}
 & logical & switches on/off detailed
(emissions by sector) output of emission diagnostic stream
  & {\tt .true.} \\
{\tt emi\_tinterval}\index{namelist variables!emi\_tinterval}
  & special & output frequency of emission diagnostic stream
  & {\tt putdata} (see {\tt runctl} namelist) \\
{\tt eminam(32,1:50)}\index{namelist variables!eminam}
 & character & list of tracer names of
  emission diagnostic stream. There are the special names {\tt 'all' =
  'detail'},  and {\tt 
  'default'}
  & {\tt eminam(1) = 'default'}, {\tt eminam(2:50) = ''}\\
  {\tt sedi\_keytype}\index{namelist variables!sedi\_keytype}
 & integer & aggregation level of
  output of sedimentation stream \newline
  {\tt sedi\_keytype=1}: output by tracer\newline
  {\tt sedi\_keytype=2}: output by (chemical) species\newline
  {\tt sedi\_keytype=3}: output by (aerosol) mode\newline
  {\tt sedi\_keytype=4}: user defined
  & {\tt 2}\\
{\tt sedi\_lpost}\index{namelist variables!sedi\_lpost}
 & logical & switches on/off output of sedimentation stream
  & {\tt .true.} \\
{\tt sedi\_tinterval}\index{namelist variables!sedi\_interval}
 & special & output frequency of sedimentation stream
  & {\tt putdata} (see {\tt runctl} namelist) \\
{\tt sedinam(32,1:50)}\index{namelist variables!sedinam}
 & character & list of tracer names of
  sedimentation diagnostic stream. There are the special names ${\tt 'all'}=
  {\tt 'detail'}$,  and {\tt 
  'default'}
  & {\tt sedinam(1) = 'default'}, {\tt sedinam(2:50) = ''}\\
{\tt vphysc\_lpost}\index{namelist variables!vphysc\_lpost}
 & logical & switches on/off output of vphysc stream
  & {\tt .true.} \\
{\tt vphysc\_tinterval}\index{namelist variables!vphysc\_tinterval}
 & special & output frequency of vphysc stream
  & {\tt putdata} (see {\tt runctl} namelist) \\
{\tt vphyscnam(32,1:50)}\index{namelist variables!vphyscnam}
 & character & list of variable names of
  vphysc stream. There are the special names {\tt 'all'} and {\tt
  'default'}
  & {\tt vphyscnam(1) = 'default'}, {\tt vphyscnam(2:50) = ''}\\
{\tt wetdep\_gastrac(24,1:200)}
\index{namelist variables!wetdep\_gastrac}
 & character & names of gas phase
  tracers to be included in
  wet deposition stream. Special name {\tt 'default'} is possible 
  & {\tt wetdep\_gastrac(1) = 'default'}, {\tt wetdep\_gastrac (2:200) = ''}\\
{\tt wetdep\_keytype}\index{namelist variables!wetdep\_keytype}
 & integer & aggregation level of
  output of wet deposition stream \newline
  {\tt wetdep\_keytype=1}: output by tracer\newline
  {\tt wetdep\_keytype=2}: output by (chemical) species\newline
  {\tt wetdep\_keytype=3}: output by (aerosol) mode\newline
  {\tt wetdep\_keytype=4}: user defined
  & {\tt 2}\\
{\tt wetdep\_lpost}\index{namelist variables!wetdep\_lpost}
 & logical & switches on/off output of wet deposition stream
  & {\tt .true.} \\
{\tt wetdep\_tinterval}\index{namelist variables!wetdep\_tinterval}
 & special & output frequency of wet deposition stream
  & {\tt putdata} (see {\tt runctl} namelist) \\
{\tt wetdepnam(32,1:50)}\index{namelist variables!wetdepnam}
 & character & list of tracer names of
wet deposition output stream. There are the special names ${\tt 'all'}={\tt
  'detail'}$,  and {\tt
  'default'}
  & {\tt wetdepnam(1) = 'default'}, {\tt wetdepnam(2:50) = ''}\\
\hline
\end{longtable}


\subsubsection{Namelist {\tt
    submodelctl}\index{namelists!submodelctl}}\label{secsubmodelctl} 

This namelist contains general submodel switches of
``proper submodels'' including switches that control the coupling
among submodels. 

\setlength{\LTcapwidth}{\textwidth}
\setlength{\LTleft}{0pt}\setlength{\LTright}{0pt}

\begin{longtable}{l@{\extracolsep\fill}lp{5.0cm}p{3.0cm}}\hline\hline
\caption[Namelist {\tt submodelctl}]{Namelist
  {\tt submodelctl}}\\\hline\label{tabsubmodelctl}
\endfirsthead
\caption[]{{\tt submodelctl} --- continued}\\\hline
\endhead
\hline\multicolumn{4}{r}{\slshape table continued on next page}\\
\endfoot
\hline %\multicolumn{4}{r}{end of table}
\endlastfoot
Variable & type & Explanation & default \\\hline
{\tt laircraft}\index{namelist variables!laircraft}
 & logical & switches on/off aircraft emissions
  & {\tt .false.} \\
{\tt lburden}\index{namelist variables!lburden}
 & logical & switches on/off burden calculation (column
integrals)
  & {\tt .false.} \\
{\tt lco2}\index{namelist variables!lco2}
 & logical & switches on/off CO$_2$ submodel (interacting
  with JSBACH) & {\tt .false.} \\
{\tt lchemheat}\index{namelist variables!lchemheat}
 & logical & switches on/off chemical heating
  & {\tt .false.} \\
{\tt lchemistry}\index{namelist variables!lchemistry}
 & logical & switches on/off chemistry
  & {\tt .false.} \\
{\tt ldrydep}\index{namelist variables!ldrydep}
 & logical & switches on/off dry deposition
  & {\tt .false.} \\
{\tt lham}\index{namelist variables!lham}
 & logical & switches on/off HAM aerosol submodel &
  {\tt .false.} \\
{\tt lemissions}\index{namelist variables!lemissions}
 & logical & switches on/off emissions
  & {\tt .false.} \\
{\tt lhammonia}\index{namelist variables!lhammonia}
 & logical & switches on/off HAMMONIA submodel (middle
  and upper atmosphere submodel) & {\tt .false.} \\
{\tt lhammoz}\index{namelist variables!lhammoz}
 & logical & switches on/off HAM aerosol submodel and
  MOZART chemistry submodel and the coupling between the two &
  {\tt .false.} \\
{\tt lhmzhet}\index{namelist variables!lhmzhet}
 & logical & switches on/off hammoz heterogeneous chemistry &
  {\tt .false.} \\
{\tt lhmzphoto}\index{namelist variables!lhmzphoto}
 & logical & switches on/off hammoz photolysis &
  {\tt .false.} \\
{\tt lhmzoxi}\index{namelist variables!lhmzoxi}
 & logical & switches on/off hammoz oxidant fields &
  {\tt .false.} \\
{\tt linterchem}\index{namelist variables!linterchem}
 & logical & switches on/off coupling of chemistry
  with radiation & {\tt .false.} \\
{\tt linteram}\index{namelist variables!linteram}
 & logical &
  switches on/off interactive airmass calculation (HAMMONIA)
  & {\tt .false.} \\
{\tt lintercp}\index{namelist variables!lintercp}
 & logical & switches on/off interactive $c_p$
  calculation (HAMMONIA) 
  & {\tt .false.} \\
{\tt llght}\index{namelist variables!llght}
 & logical & switches on/off interactive computation of
  lightning emissions & {\tt .false.} \\
{\tt lmethox}\index{namelist variables!lmethox}
 & logical & switches on/off methane oxidation in stratosphere &
  {\tt .false.} \\
{\tt lmegan}\index{namelist variables!lmegan}
 & logical & switches on/off biogenic vegetation emissions &
  {\tt .false.} \\
{\tt lmicrophysics}\index{namelist variables!lmicrophysics}
 & logical & switches on/off microsphysics calculations &
  {\tt .false.} \\
{\tt lmoz}\index{namelist variables!lmoz}
 & logical & switches on/off MOZART chemistry submodel &
  {\tt .false.} \\
{\tt loisccp}\index{namelist variables!loisccp}
 & logical & switches on/off isccp
  simulator. Currently, the isccp simulator is implemented outside the
  submodel interface &
  {\tt .false.} \\
{\tt losat}\index{namelist variables!losat}
 & logical & switches on/off satellite
  simulator. Currently, the {\tt locosp} switch for the cosp satellite
  simulator is implemented outside the submodel interface.&
  {\tt .false.} \\
{\tt lsalsa}\index{namelist variables!lsalsa}
 & logical & switches on/off SALSA aerosol submodel &
  {\tt .false.} \\
{\tt lsedimentation}\index{namelist variables!lsedimentation}
 & logical & switches on/off sedimentation &
  {\tt .false.} \\
{\tt ltransdiag}\index{namelist variables!ltransdiag}
 & logical & switches on/off atmospheric energy
  transport diagnostic & {\tt .false.} \\
{\tt lwetdep}\index{namelist variables!lwetdep}
 & logical & switches on/off drydeposition &
  {\tt .false.} \\
{\tt lxt}\index{namelist variables!lxt}
 & logical & switches on/off generic test of tracer submodel &
  {\tt .false.} \\\hline 
\end{longtable}


\subsubsection{Namelist {\tt tdiagctl}\index{namelists!tdiagctl}}\label{sectdiagctl}

This namelist determines the output of the tendency diagnostic.
The tendencies of Tab.~\ref{tab_var} can be diagnosed.
The following variables are contained in the diagnostic stream {\tt tdiag}.
         The top row describes the variables, the first column
         gives the routine names (processes) producing the tendencies
         saved under the names in the corresponding rows.
  The units of the variables and code numbers are given in
  parenthesis.

\begin{scriptsize}

\setlength{\LTcapwidth}{\textwidth}
\setlength{\LTleft}{0pt}\setlength{\LTright}{0pt}

\begin{longtable}{c@{\extracolsep\fill}cccccc}\hline\hline
\caption[Variables of {\tt tdiagctl}]{Variables of the diagnostic stream
  {\tt tdiagctl}}\\\hline\label{tab_var}
\endfirsthead
\caption[]{{\tt variables of tdiagctl} --- continued}\\\hline
\endhead
\hline\multicolumn{7}{r}{\slshape table continued on next page}\\
\endfoot
\hline %\multicolumn{4}{r}{end of table}
\endlastfoot
\rule{0cm}{2.5ex}{\bf variable} & $du/dt$ & $dv/dt$ & $dT/dt$ &
$dq/dt$ & $dx_{\rm l}/dt$ & $dx_{\rm i}/dt$ \\
& (m/s/day) & (m/s/day) & (K/day) & (1/day) & (1/day) & (1/day)\\
{\bf routine} &&&&&& \\
{\bf (process)} &&&&&& \\\hline\hline
\rule{0cm}{2.5ex}\raisebox{-1.5ex}[1.5ex]{\bf vdiff} & {\tt dudt\_vdiff} &
{\tt dvdt\_vdiff} & {\tt dtdt\_vdiff} & {\tt dqdt\_vdiff} & 
{\tt dxldt\_vdiff} & {\tt dxidt\_vdiff} \\
 & (code 11) & (code 21) & (code 1) & (code 31) & (code 41) & (code 51) \\\hline
\rule{0cm}{2.5ex}\raisebox{-1.5ex}[1.5ex]{\bf radheat} & --- & --- &
{\tt dtdt\_rheat\_sw} (code 62) & --- & --- & ---\\ 
& --- & --- & {\tt dtdt\_rheat\_lw} (code 72) & --- & --- & --- \\\hline
\rule{0cm}{2.5ex}\raisebox{-1.5ex}[1.5ex]{\bf gwspectrum} &
{\tt dudt\_hines} & {\tt dvdt\_hines} & {\tt dtdt\_hines} & \om &\om  &\om \\
&(code 13) & (code 23) & (code 3) & & & \\\hline
\rule{0cm}{2.5ex}\raisebox{-1.5ex}[1.5ex]{\bf ssodrag} &
{\tt dudt\_sso} & {\tt dvdt\_sso} & {\tt dtdt\_sso} &\om &\om & \om\\
&(code 14) & (code 24) & (code 4) & & & \\\hline
\rule{0cm}{2.5ex}\raisebox{-1.5ex}[1.5ex]{\bf cucall} & {\tt dudt\_cucall} &
{\tt dvdt\_cucall} & {\tt dtdt\_cucall} & {\tt dqdt\_cucall} &\om & \om\\
& (code 15) & (code 25) & (code 5) & (code 35) && \\\hline
\rule{0cm}{2.5ex}\raisebox{-1.5ex}[1.5ex]{\bf cloud} &
\om&\om & {\tt dtdt\_cloud} & {\tt dqdt\_cloud} & {\tt dxldt\_cloud} & {\tt dxidt\_cloud} \\
&&& (code 6)    & (code 36)   & (code 46)    & (code 56) \\\hline
\multicolumn{7}{c}{\rule{0cm}{2.5ex}\bf spectral variables}\\\hline
\rule{0cm}{2.5ex}{\bf variable} & $d\hat{\xi}/dt$ & $d\hat{D}/dt$ &
$d\hat{T}/dt$ & & & \\
& (1/s/day) & (1/s/day) & (K/day) & & & \\
{\bf routine} &&&&&& \\
{\bf (process)} &&&&&& \\\hline
\rule{0cm}{2.5ex}\raisebox{-1.5ex}[1.5ex]{\bf hdiff} &
{\tt dsvodt\_hdiff} & {\tt dsddt\_hdiff} & {\tt
  dstdt\_hdiff} & & & \\
& (code 87) & (code 97) & (code 7) & & & \\\hline
\multicolumn{7}{c}{\rule{0cm}{2.5ex}\bf atmospheric variables}\\\hline
\multicolumn{7}{c}{\rule{0cm}{2.5ex}
\begin{tabular*}{\textwidth}{c@{\extracolsep\fill}ccccc}
Box area & surface geopotential & $\ln(p_{\rm s}/p^\ominus)$ & $p_s$ & 
$T(t)$ & $T(t-\Delta t)$ \\
$\rm m^2$ & $\rm m^2/s^2$ & spectral & Pa & spectral & K \\
\end{tabular*}}\\\hline\hline
\end{longtable}

\end{scriptsize}

Additional documentation can be found in Appendix~\ref{cr20110118}.

\setlength{\LTcapwidth}{\textwidth}
\setlength{\LTleft}{0pt}\setlength{\LTright}{0pt}

\begin{longtable}{l@{\extracolsep\fill}lp{7.0cm}p{3.5cm}}\hline\hline
\caption[Namelist {\tt tdiagctl}]{Namelist
  {\tt tdiagctl}}\\\hline\label{tabtdiagctl}
\endfirsthead
\caption[]{{\tt tdiagctl} --- continued}\\\hline
\endhead
\hline\multicolumn{4}{r}{\slshape table continued on next page}\\
\endfoot
\hline %\multicolumn{4}{r}{end of table}
\endlastfoot
Variable & type & Explanation & default \\\hline
{\tt puttdiag}\index{namelist variables!puttdiag}
& special & Output frequency of tendency stream & {\tt
  6, 'hours', 'first', 0}\\
{\tt tdiagnam(32,1:22)}\index{namelist variables!tdiagnam}
 & character & determines the choice of tendencies
that are written to the output stream {\tt \_tdiag} \newline
\begin{tabular*}{7cm}{p{3cm}|p{3.7cm}}\\\hline
keyword & explanation \\\hline
{\tt 'all'} & output all tendencies of {\tt tdiag} stream\\\hline
one of \rule{2cm}{0cm} 
\rule{2cm}{0cm}\newline
{\tt 'vdiff'}, \newline 
{\tt 'hdiff'}, \newline
{\tt 'radheat'},\newline
{\tt 'gwspectrum'},\newline
{\tt 'ssodrag'},\newline
{\tt 'cucall'},\newline
{\tt 'cloud'}
&
output all tendencies
associated with \newline\newline
{\bf vdiff},\newline
{\bf hdiff},\newline
{\bf radheat},\newline
{\bf gwspectrum},\newline
{\bf ssodrag},\newline 
{\bf cucall}, \newline
{\bf cloud}
\\\hline
one of \rule{2cm}{0cm}
\rule{2cm}{0cm}\newline
{\tt 'uwind'}\newline
{\tt 'vwind'}\newline
{\tt 'temp'}\newline
{\tt 'qhum'}\newline
{\tt 'xl'}\newline
{\tt 'xi'}\newline
&
of all processes, output the tendencies\newline\newline
$du/dt$, $d\hat{\xi}/dt$, $d\hat{D}/dt$\newline
$dv/dt$, $d\hat{\xi}/dt$, $d\hat{D}/dt$\newline
$dT/dt$, $d\hat{T}/dt$\newline
$dq/dt$\newline
$dx_{\rm l}$\newline
$dx_{\rm i}$
\\\hline
one of the variable names of the tendencies listed in
table~\ref{tab_var}, e.g.~{\tt dudt\_hines} & output this tendency,
e.g.~$du/dt$ due to {\tt gwspectrum}\\\hline
\end{tabular*}\newline

&${\tt
  tdiagnam}(1)={\tt 'all'}$, ${\tt tdiagnam}(2:22)={\tt 'end'}$\\
\hline 
\end{longtable}
